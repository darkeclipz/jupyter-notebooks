% Default to the notebook output style

    


% Inherit from the specified cell style.




    
\documentclass[11pt]{article}

    
    
    \usepackage[T1]{fontenc}
    % Nicer default font (+ math font) than Computer Modern for most use cases
    \usepackage{mathpazo}

    % Basic figure setup, for now with no caption control since it's done
    % automatically by Pandoc (which extracts ![](path) syntax from Markdown).
    \usepackage{graphicx}
    % We will generate all images so they have a width \maxwidth. This means
    % that they will get their normal width if they fit onto the page, but
    % are scaled down if they would overflow the margins.
    \makeatletter
    \def\maxwidth{\ifdim\Gin@nat@width>\linewidth\linewidth
    \else\Gin@nat@width\fi}
    \makeatother
    \let\Oldincludegraphics\includegraphics
    % Set max figure width to be 80% of text width, for now hardcoded.
    \renewcommand{\includegraphics}[1]{\Oldincludegraphics[width=.8\maxwidth]{#1}}
    % Ensure that by default, figures have no caption (until we provide a
    % proper Figure object with a Caption API and a way to capture that
    % in the conversion process - todo).
    \usepackage{caption}
    \DeclareCaptionLabelFormat{nolabel}{}
    \captionsetup{labelformat=nolabel}

    \usepackage{adjustbox} % Used to constrain images to a maximum size 
    \usepackage{xcolor} % Allow colors to be defined
    \usepackage{enumerate} % Needed for markdown enumerations to work
    \usepackage{geometry} % Used to adjust the document margins
    \usepackage{amsmath} % Equations
    \usepackage{amssymb} % Equations
    \usepackage{textcomp} % defines textquotesingle
    % Hack from http://tex.stackexchange.com/a/47451/13684:
    \AtBeginDocument{%
        \def\PYZsq{\textquotesingle}% Upright quotes in Pygmentized code
    }
    \usepackage{upquote} % Upright quotes for verbatim code
    \usepackage{eurosym} % defines \euro
    \usepackage[mathletters]{ucs} % Extended unicode (utf-8) support
    \usepackage[utf8x]{inputenc} % Allow utf-8 characters in the tex document
    \usepackage{fancyvrb} % verbatim replacement that allows latex
    \usepackage{grffile} % extends the file name processing of package graphics 
                         % to support a larger range 
    % The hyperref package gives us a pdf with properly built
    % internal navigation ('pdf bookmarks' for the table of contents,
    % internal cross-reference links, web links for URLs, etc.)
    \usepackage{hyperref}
    \usepackage{longtable} % longtable support required by pandoc >1.10
    \usepackage{booktabs}  % table support for pandoc > 1.12.2
    \usepackage[inline]{enumitem} % IRkernel/repr support (it uses the enumerate* environment)
    \usepackage[normalem]{ulem} % ulem is needed to support strikethroughs (\sout)
                                % normalem makes italics be italics, not underlines
    

    \usepackage{float}
    
    % Colors for the hyperref package
    \definecolor{urlcolor}{rgb}{0,.145,.698}
    \definecolor{linkcolor}{rgb}{.71,0.21,0.01}
    \definecolor{citecolor}{rgb}{.12,.54,.11}

    % ANSI colors
    \definecolor{ansi-black}{HTML}{3E424D}
    \definecolor{ansi-black-intense}{HTML}{282C36}
    \definecolor{ansi-red}{HTML}{E75C58}
    \definecolor{ansi-red-intense}{HTML}{B22B31}
    \definecolor{ansi-green}{HTML}{00A250}
    \definecolor{ansi-green-intense}{HTML}{007427}
    \definecolor{ansi-yellow}{HTML}{DDB62B}
    \definecolor{ansi-yellow-intense}{HTML}{B27D12}
    \definecolor{ansi-blue}{HTML}{208FFB}
    \definecolor{ansi-blue-intense}{HTML}{0065CA}
    \definecolor{ansi-magenta}{HTML}{D160C4}
    \definecolor{ansi-magenta-intense}{HTML}{A03196}
    \definecolor{ansi-cyan}{HTML}{60C6C8}
    \definecolor{ansi-cyan-intense}{HTML}{258F8F}
    \definecolor{ansi-white}{HTML}{C5C1B4}
    \definecolor{ansi-white-intense}{HTML}{A1A6B2}

    % commands and environments needed by pandoc snippets
    % extracted from the output of `pandoc -s`
    \providecommand{\tightlist}{%
      \setlength{\itemsep}{0pt}\setlength{\parskip}{0pt}}
    \DefineVerbatimEnvironment{Highlighting}{Verbatim}{commandchars=\\\{\}}
    % Add ',fontsize=\small' for more characters per line
    \newenvironment{Shaded}{}{}
    \newcommand{\KeywordTok}[1]{\textcolor[rgb]{0.00,0.44,0.13}{\textbf{{#1}}}}
    \newcommand{\DataTypeTok}[1]{\textcolor[rgb]{0.56,0.13,0.00}{{#1}}}
    \newcommand{\DecValTok}[1]{\textcolor[rgb]{0.25,0.63,0.44}{{#1}}}
    \newcommand{\BaseNTok}[1]{\textcolor[rgb]{0.25,0.63,0.44}{{#1}}}
    \newcommand{\FloatTok}[1]{\textcolor[rgb]{0.25,0.63,0.44}{{#1}}}
    \newcommand{\CharTok}[1]{\textcolor[rgb]{0.25,0.44,0.63}{{#1}}}
    \newcommand{\StringTok}[1]{\textcolor[rgb]{0.25,0.44,0.63}{{#1}}}
    \newcommand{\CommentTok}[1]{\textcolor[rgb]{0.38,0.63,0.69}{\textit{{#1}}}}
    \newcommand{\OtherTok}[1]{\textcolor[rgb]{0.00,0.44,0.13}{{#1}}}
    \newcommand{\AlertTok}[1]{\textcolor[rgb]{1.00,0.00,0.00}{\textbf{{#1}}}}
    \newcommand{\FunctionTok}[1]{\textcolor[rgb]{0.02,0.16,0.49}{{#1}}}
    \newcommand{\RegionMarkerTok}[1]{{#1}}
    \newcommand{\ErrorTok}[1]{\textcolor[rgb]{1.00,0.00,0.00}{\textbf{{#1}}}}
    \newcommand{\NormalTok}[1]{{#1}}
    
    % Additional commands for more recent versions of Pandoc
    \newcommand{\ConstantTok}[1]{\textcolor[rgb]{0.53,0.00,0.00}{{#1}}}
    \newcommand{\SpecialCharTok}[1]{\textcolor[rgb]{0.25,0.44,0.63}{{#1}}}
    \newcommand{\VerbatimStringTok}[1]{\textcolor[rgb]{0.25,0.44,0.63}{{#1}}}
    \newcommand{\SpecialStringTok}[1]{\textcolor[rgb]{0.73,0.40,0.53}{{#1}}}
    \newcommand{\ImportTok}[1]{{#1}}
    \newcommand{\DocumentationTok}[1]{\textcolor[rgb]{0.73,0.13,0.13}{\textit{{#1}}}}
    \newcommand{\AnnotationTok}[1]{\textcolor[rgb]{0.38,0.63,0.69}{\textbf{\textit{{#1}}}}}
    \newcommand{\CommentVarTok}[1]{\textcolor[rgb]{0.38,0.63,0.69}{\textbf{\textit{{#1}}}}}
    \newcommand{\VariableTok}[1]{\textcolor[rgb]{0.10,0.09,0.49}{{#1}}}
    \newcommand{\ControlFlowTok}[1]{\textcolor[rgb]{0.00,0.44,0.13}{\textbf{{#1}}}}
    \newcommand{\OperatorTok}[1]{\textcolor[rgb]{0.40,0.40,0.40}{{#1}}}
    \newcommand{\BuiltInTok}[1]{{#1}}
    \newcommand{\ExtensionTok}[1]{{#1}}
    \newcommand{\PreprocessorTok}[1]{\textcolor[rgb]{0.74,0.48,0.00}{{#1}}}
    \newcommand{\AttributeTok}[1]{\textcolor[rgb]{0.49,0.56,0.16}{{#1}}}
    \newcommand{\InformationTok}[1]{\textcolor[rgb]{0.38,0.63,0.69}{\textbf{\textit{{#1}}}}}
    \newcommand{\WarningTok}[1]{\textcolor[rgb]{0.38,0.63,0.69}{\textbf{\textit{{#1}}}}}
    
    
    % Define a nice break command that doesn't care if a line doesn't already
    % exist.
    \def\br{\hspace*{\fill} \\* }
    % Math Jax compatability definitions
    \def\gt{>}
    \def\lt{<}
    % Document parameters
    \title{Probability Analysis for Monopoly}
    \author{L. Rotgers}
    
    

    % Pygments definitions
    
\makeatletter
\def\PY@reset{\let\PY@it=\relax \let\PY@bf=\relax%
    \let\PY@ul=\relax \let\PY@tc=\relax%
    \let\PY@bc=\relax \let\PY@ff=\relax}
\def\PY@tok#1{\csname PY@tok@#1\endcsname}
\def\PY@toks#1+{\ifx\relax#1\empty\else%
    \PY@tok{#1}\expandafter\PY@toks\fi}
\def\PY@do#1{\PY@bc{\PY@tc{\PY@ul{%
    \PY@it{\PY@bf{\PY@ff{#1}}}}}}}
\def\PY#1#2{\PY@reset\PY@toks#1+\relax+\PY@do{#2}}

\expandafter\def\csname PY@tok@w\endcsname{\def\PY@tc##1{\textcolor[rgb]{0.73,0.73,0.73}{##1}}}
\expandafter\def\csname PY@tok@c\endcsname{\let\PY@it=\textit\def\PY@tc##1{\textcolor[rgb]{0.25,0.50,0.50}{##1}}}
\expandafter\def\csname PY@tok@cp\endcsname{\def\PY@tc##1{\textcolor[rgb]{0.74,0.48,0.00}{##1}}}
\expandafter\def\csname PY@tok@k\endcsname{\let\PY@bf=\textbf\def\PY@tc##1{\textcolor[rgb]{0.00,0.50,0.00}{##1}}}
\expandafter\def\csname PY@tok@kp\endcsname{\def\PY@tc##1{\textcolor[rgb]{0.00,0.50,0.00}{##1}}}
\expandafter\def\csname PY@tok@kt\endcsname{\def\PY@tc##1{\textcolor[rgb]{0.69,0.00,0.25}{##1}}}
\expandafter\def\csname PY@tok@o\endcsname{\def\PY@tc##1{\textcolor[rgb]{0.40,0.40,0.40}{##1}}}
\expandafter\def\csname PY@tok@ow\endcsname{\let\PY@bf=\textbf\def\PY@tc##1{\textcolor[rgb]{0.67,0.13,1.00}{##1}}}
\expandafter\def\csname PY@tok@nb\endcsname{\def\PY@tc##1{\textcolor[rgb]{0.00,0.50,0.00}{##1}}}
\expandafter\def\csname PY@tok@nf\endcsname{\def\PY@tc##1{\textcolor[rgb]{0.00,0.00,1.00}{##1}}}
\expandafter\def\csname PY@tok@nc\endcsname{\let\PY@bf=\textbf\def\PY@tc##1{\textcolor[rgb]{0.00,0.00,1.00}{##1}}}
\expandafter\def\csname PY@tok@nn\endcsname{\let\PY@bf=\textbf\def\PY@tc##1{\textcolor[rgb]{0.00,0.00,1.00}{##1}}}
\expandafter\def\csname PY@tok@ne\endcsname{\let\PY@bf=\textbf\def\PY@tc##1{\textcolor[rgb]{0.82,0.25,0.23}{##1}}}
\expandafter\def\csname PY@tok@nv\endcsname{\def\PY@tc##1{\textcolor[rgb]{0.10,0.09,0.49}{##1}}}
\expandafter\def\csname PY@tok@no\endcsname{\def\PY@tc##1{\textcolor[rgb]{0.53,0.00,0.00}{##1}}}
\expandafter\def\csname PY@tok@nl\endcsname{\def\PY@tc##1{\textcolor[rgb]{0.63,0.63,0.00}{##1}}}
\expandafter\def\csname PY@tok@ni\endcsname{\let\PY@bf=\textbf\def\PY@tc##1{\textcolor[rgb]{0.60,0.60,0.60}{##1}}}
\expandafter\def\csname PY@tok@na\endcsname{\def\PY@tc##1{\textcolor[rgb]{0.49,0.56,0.16}{##1}}}
\expandafter\def\csname PY@tok@nt\endcsname{\let\PY@bf=\textbf\def\PY@tc##1{\textcolor[rgb]{0.00,0.50,0.00}{##1}}}
\expandafter\def\csname PY@tok@nd\endcsname{\def\PY@tc##1{\textcolor[rgb]{0.67,0.13,1.00}{##1}}}
\expandafter\def\csname PY@tok@s\endcsname{\def\PY@tc##1{\textcolor[rgb]{0.73,0.13,0.13}{##1}}}
\expandafter\def\csname PY@tok@sd\endcsname{\let\PY@it=\textit\def\PY@tc##1{\textcolor[rgb]{0.73,0.13,0.13}{##1}}}
\expandafter\def\csname PY@tok@si\endcsname{\let\PY@bf=\textbf\def\PY@tc##1{\textcolor[rgb]{0.73,0.40,0.53}{##1}}}
\expandafter\def\csname PY@tok@se\endcsname{\let\PY@bf=\textbf\def\PY@tc##1{\textcolor[rgb]{0.73,0.40,0.13}{##1}}}
\expandafter\def\csname PY@tok@sr\endcsname{\def\PY@tc##1{\textcolor[rgb]{0.73,0.40,0.53}{##1}}}
\expandafter\def\csname PY@tok@ss\endcsname{\def\PY@tc##1{\textcolor[rgb]{0.10,0.09,0.49}{##1}}}
\expandafter\def\csname PY@tok@sx\endcsname{\def\PY@tc##1{\textcolor[rgb]{0.00,0.50,0.00}{##1}}}
\expandafter\def\csname PY@tok@m\endcsname{\def\PY@tc##1{\textcolor[rgb]{0.40,0.40,0.40}{##1}}}
\expandafter\def\csname PY@tok@gh\endcsname{\let\PY@bf=\textbf\def\PY@tc##1{\textcolor[rgb]{0.00,0.00,0.50}{##1}}}
\expandafter\def\csname PY@tok@gu\endcsname{\let\PY@bf=\textbf\def\PY@tc##1{\textcolor[rgb]{0.50,0.00,0.50}{##1}}}
\expandafter\def\csname PY@tok@gd\endcsname{\def\PY@tc##1{\textcolor[rgb]{0.63,0.00,0.00}{##1}}}
\expandafter\def\csname PY@tok@gi\endcsname{\def\PY@tc##1{\textcolor[rgb]{0.00,0.63,0.00}{##1}}}
\expandafter\def\csname PY@tok@gr\endcsname{\def\PY@tc##1{\textcolor[rgb]{1.00,0.00,0.00}{##1}}}
\expandafter\def\csname PY@tok@ge\endcsname{\let\PY@it=\textit}
\expandafter\def\csname PY@tok@gs\endcsname{\let\PY@bf=\textbf}
\expandafter\def\csname PY@tok@gp\endcsname{\let\PY@bf=\textbf\def\PY@tc##1{\textcolor[rgb]{0.00,0.00,0.50}{##1}}}
\expandafter\def\csname PY@tok@go\endcsname{\def\PY@tc##1{\textcolor[rgb]{0.53,0.53,0.53}{##1}}}
\expandafter\def\csname PY@tok@gt\endcsname{\def\PY@tc##1{\textcolor[rgb]{0.00,0.27,0.87}{##1}}}
\expandafter\def\csname PY@tok@err\endcsname{\def\PY@bc##1{\setlength{\fboxsep}{0pt}\fcolorbox[rgb]{1.00,0.00,0.00}{1,1,1}{\strut ##1}}}
\expandafter\def\csname PY@tok@kc\endcsname{\let\PY@bf=\textbf\def\PY@tc##1{\textcolor[rgb]{0.00,0.50,0.00}{##1}}}
\expandafter\def\csname PY@tok@kd\endcsname{\let\PY@bf=\textbf\def\PY@tc##1{\textcolor[rgb]{0.00,0.50,0.00}{##1}}}
\expandafter\def\csname PY@tok@kn\endcsname{\let\PY@bf=\textbf\def\PY@tc##1{\textcolor[rgb]{0.00,0.50,0.00}{##1}}}
\expandafter\def\csname PY@tok@kr\endcsname{\let\PY@bf=\textbf\def\PY@tc##1{\textcolor[rgb]{0.00,0.50,0.00}{##1}}}
\expandafter\def\csname PY@tok@bp\endcsname{\def\PY@tc##1{\textcolor[rgb]{0.00,0.50,0.00}{##1}}}
\expandafter\def\csname PY@tok@fm\endcsname{\def\PY@tc##1{\textcolor[rgb]{0.00,0.00,1.00}{##1}}}
\expandafter\def\csname PY@tok@vc\endcsname{\def\PY@tc##1{\textcolor[rgb]{0.10,0.09,0.49}{##1}}}
\expandafter\def\csname PY@tok@vg\endcsname{\def\PY@tc##1{\textcolor[rgb]{0.10,0.09,0.49}{##1}}}
\expandafter\def\csname PY@tok@vi\endcsname{\def\PY@tc##1{\textcolor[rgb]{0.10,0.09,0.49}{##1}}}
\expandafter\def\csname PY@tok@vm\endcsname{\def\PY@tc##1{\textcolor[rgb]{0.10,0.09,0.49}{##1}}}
\expandafter\def\csname PY@tok@sa\endcsname{\def\PY@tc##1{\textcolor[rgb]{0.73,0.13,0.13}{##1}}}
\expandafter\def\csname PY@tok@sb\endcsname{\def\PY@tc##1{\textcolor[rgb]{0.73,0.13,0.13}{##1}}}
\expandafter\def\csname PY@tok@sc\endcsname{\def\PY@tc##1{\textcolor[rgb]{0.73,0.13,0.13}{##1}}}
\expandafter\def\csname PY@tok@dl\endcsname{\def\PY@tc##1{\textcolor[rgb]{0.73,0.13,0.13}{##1}}}
\expandafter\def\csname PY@tok@s2\endcsname{\def\PY@tc##1{\textcolor[rgb]{0.73,0.13,0.13}{##1}}}
\expandafter\def\csname PY@tok@sh\endcsname{\def\PY@tc##1{\textcolor[rgb]{0.73,0.13,0.13}{##1}}}
\expandafter\def\csname PY@tok@s1\endcsname{\def\PY@tc##1{\textcolor[rgb]{0.73,0.13,0.13}{##1}}}
\expandafter\def\csname PY@tok@mb\endcsname{\def\PY@tc##1{\textcolor[rgb]{0.40,0.40,0.40}{##1}}}
\expandafter\def\csname PY@tok@mf\endcsname{\def\PY@tc##1{\textcolor[rgb]{0.40,0.40,0.40}{##1}}}
\expandafter\def\csname PY@tok@mh\endcsname{\def\PY@tc##1{\textcolor[rgb]{0.40,0.40,0.40}{##1}}}
\expandafter\def\csname PY@tok@mi\endcsname{\def\PY@tc##1{\textcolor[rgb]{0.40,0.40,0.40}{##1}}}
\expandafter\def\csname PY@tok@il\endcsname{\def\PY@tc##1{\textcolor[rgb]{0.40,0.40,0.40}{##1}}}
\expandafter\def\csname PY@tok@mo\endcsname{\def\PY@tc##1{\textcolor[rgb]{0.40,0.40,0.40}{##1}}}
\expandafter\def\csname PY@tok@ch\endcsname{\let\PY@it=\textit\def\PY@tc##1{\textcolor[rgb]{0.25,0.50,0.50}{##1}}}
\expandafter\def\csname PY@tok@cm\endcsname{\let\PY@it=\textit\def\PY@tc##1{\textcolor[rgb]{0.25,0.50,0.50}{##1}}}
\expandafter\def\csname PY@tok@cpf\endcsname{\let\PY@it=\textit\def\PY@tc##1{\textcolor[rgb]{0.25,0.50,0.50}{##1}}}
\expandafter\def\csname PY@tok@c1\endcsname{\let\PY@it=\textit\def\PY@tc##1{\textcolor[rgb]{0.25,0.50,0.50}{##1}}}
\expandafter\def\csname PY@tok@cs\endcsname{\let\PY@it=\textit\def\PY@tc##1{\textcolor[rgb]{0.25,0.50,0.50}{##1}}}

\def\PYZbs{\char`\\}
\def\PYZus{\char`\_}
\def\PYZob{\char`\{}
\def\PYZcb{\char`\}}
\def\PYZca{\char`\^}
\def\PYZam{\char`\&}
\def\PYZlt{\char`\<}
\def\PYZgt{\char`\>}
\def\PYZsh{\char`\#}
\def\PYZpc{\char`\%}
\def\PYZdl{\char`\$}
\def\PYZhy{\char`\-}
\def\PYZsq{\char`\'}
\def\PYZdq{\char`\"}
\def\PYZti{\char`\~}
% for compatibility with earlier versions
\def\PYZat{@}
\def\PYZlb{[}
\def\PYZrb{]}
\makeatother


    % Exact colors from NB
    \definecolor{incolor}{rgb}{0.0, 0.0, 0.5}
    \definecolor{outcolor}{rgb}{0.545, 0.0, 0.0}



    
    % Prevent overflowing lines due to hard-to-break entities
    \sloppy 
    % Setup hyperref package
    \hypersetup{
      breaklinks=true,  % so long urls are correctly broken across lines
      colorlinks=true,
      urlcolor=urlcolor,
      linkcolor=linkcolor,
      citecolor=citecolor,
      }
    % Slightly bigger margins than the latex defaults
    
    \geometry{verbose,tmargin=1in,bmargin=1in,lmargin=1in,rmargin=1in}
    
    

    \begin{document}
    
    
    \maketitle
    
    \newpage
    
    

    
    \begin{Verbatim}[commandchars=\\\{\}]
{\color{incolor}In [{\color{incolor}1}]:} \PY{o}{\PYZpc{}}\PY{k}{pylab} inline
\end{Verbatim}


    \begin{Verbatim}[commandchars=\\\{\}]
Populating the interactive namespace from numpy and matplotlib

    \end{Verbatim}
    
    \tableofcontents
    
    \newpage

    \hypertarget{monopoly}{%
\newpage
\section{Monopoly}\label{monopoly}}

    Analysis for the probabilities with Monopoly to answer the question:
which are the best houses to buy?

To answer this question we will create a simulated version of Monopoly
and determine the probabilities to land on each square. The square with
the highest probability will be the best square to have.

    \begin{figure}[H]
\centering
\includegraphics{monopoly_1.jpg}
\caption{monopoly}
\end{figure}

    \hypertarget{squares}{%
\newpage
\subsection{Squares}\label{squares}}

    Each edge has 9 positions and there are 4 edges. There are 4 corners.
Which gives a total of 40 positions where a player can land. The labels
are numbered starting at GO.

    \hypertarget{labels}{%
\subsubsection{Labels}\label{labels}}

    \begin{Verbatim}[commandchars=\\\{\}]
{\color{incolor}In [{\color{incolor}2}]:} \PY{n}{squares\PYZus{}labels} \PY{o}{=} \PY{p}{[}\PY{l+s+s1}{\PYZsq{}}\PY{l+s+s1}{start}\PY{l+s+s1}{\PYZsq{}}\PY{p}{,} \PY{l+s+s1}{\PYZsq{}}\PY{l+s+s1}{b1}\PY{l+s+s1}{\PYZsq{}}\PY{p}{,} \PY{l+s+s1}{\PYZsq{}}\PY{l+s+s1}{cc1}\PY{l+s+s1}{\PYZsq{}}\PY{p}{,} \PY{l+s+s1}{\PYZsq{}}\PY{l+s+s1}{b2}\PY{l+s+s1}{\PYZsq{}}\PY{p}{,} \PY{l+s+s1}{\PYZsq{}}\PY{l+s+s1}{it}\PY{l+s+s1}{\PYZsq{}}\PY{p}{,} \PY{l+s+s1}{\PYZsq{}}\PY{l+s+s1}{t1}\PY{l+s+s1}{\PYZsq{}}\PY{p}{,} 
                          \PY{l+s+s1}{\PYZsq{}}\PY{l+s+s1}{lb1}\PY{l+s+s1}{\PYZsq{}}\PY{p}{,} \PY{l+s+s1}{\PYZsq{}}\PY{l+s+s1}{c1}\PY{l+s+s1}{\PYZsq{}}\PY{p}{,} \PY{l+s+s1}{\PYZsq{}}\PY{l+s+s1}{lb2}\PY{l+s+s1}{\PYZsq{}}\PY{p}{,} \PY{l+s+s1}{\PYZsq{}}\PY{l+s+s1}{lb3}\PY{l+s+s1}{\PYZsq{}}\PY{p}{,} \PY{l+s+s1}{\PYZsq{}}\PY{l+s+s1}{jail}\PY{l+s+s1}{\PYZsq{}}\PY{p}{,} \PY{l+s+s1}{\PYZsq{}}\PY{l+s+s1}{p1}\PY{l+s+s1}{\PYZsq{}}\PY{p}{,} 
                          \PY{l+s+s1}{\PYZsq{}}\PY{l+s+s1}{ec}\PY{l+s+s1}{\PYZsq{}}\PY{p}{,} \PY{l+s+s1}{\PYZsq{}}\PY{l+s+s1}{p2}\PY{l+s+s1}{\PYZsq{}}\PY{p}{,} \PY{l+s+s1}{\PYZsq{}}\PY{l+s+s1}{p3}\PY{l+s+s1}{\PYZsq{}}\PY{p}{,} \PY{l+s+s1}{\PYZsq{}}\PY{l+s+s1}{ts2}\PY{l+s+s1}{\PYZsq{}}\PY{p}{,} \PY{l+s+s1}{\PYZsq{}}\PY{l+s+s1}{o1}\PY{l+s+s1}{\PYZsq{}}\PY{p}{,} \PY{l+s+s1}{\PYZsq{}}\PY{l+s+s1}{cc2}\PY{l+s+s1}{\PYZsq{}}\PY{p}{,} 
                          \PY{l+s+s1}{\PYZsq{}}\PY{l+s+s1}{o2}\PY{l+s+s1}{\PYZsq{}}\PY{p}{,} \PY{l+s+s1}{\PYZsq{}}\PY{l+s+s1}{o3}\PY{l+s+s1}{\PYZsq{}}\PY{p}{,} \PY{l+s+s1}{\PYZsq{}}\PY{l+s+s1}{p}\PY{l+s+s1}{\PYZsq{}}\PY{p}{,} \PY{l+s+s1}{\PYZsq{}}\PY{l+s+s1}{r1}\PY{l+s+s1}{\PYZsq{}}\PY{p}{,} \PY{l+s+s1}{\PYZsq{}}\PY{l+s+s1}{c2}\PY{l+s+s1}{\PYZsq{}}\PY{p}{,} \PY{l+s+s1}{\PYZsq{}}\PY{l+s+s1}{r2}\PY{l+s+s1}{\PYZsq{}}\PY{p}{,} \PY{l+s+s1}{\PYZsq{}}\PY{l+s+s1}{r3}\PY{l+s+s1}{\PYZsq{}}\PY{p}{,} 
                          \PY{l+s+s1}{\PYZsq{}}\PY{l+s+s1}{ts3}\PY{l+s+s1}{\PYZsq{}}\PY{p}{,} \PY{l+s+s1}{\PYZsq{}}\PY{l+s+s1}{y1}\PY{l+s+s1}{\PYZsq{}}\PY{p}{,} \PY{l+s+s1}{\PYZsq{}}\PY{l+s+s1}{y2}\PY{l+s+s1}{\PYZsq{}}\PY{p}{,} \PY{l+s+s1}{\PYZsq{}}\PY{l+s+s1}{ww}\PY{l+s+s1}{\PYZsq{}}\PY{p}{,} \PY{l+s+s1}{\PYZsq{}}\PY{l+s+s1}{y3}\PY{l+s+s1}{\PYZsq{}}\PY{p}{,} \PY{l+s+s1}{\PYZsq{}}\PY{l+s+s1}{gtj}\PY{l+s+s1}{\PYZsq{}}\PY{p}{,} 
                          \PY{l+s+s1}{\PYZsq{}}\PY{l+s+s1}{g1}\PY{l+s+s1}{\PYZsq{}}\PY{p}{,} \PY{l+s+s1}{\PYZsq{}}\PY{l+s+s1}{g2}\PY{l+s+s1}{\PYZsq{}}\PY{p}{,} \PY{l+s+s1}{\PYZsq{}}\PY{l+s+s1}{cc3}\PY{l+s+s1}{\PYZsq{}}\PY{p}{,} \PY{l+s+s1}{\PYZsq{}}\PY{l+s+s1}{g3}\PY{l+s+s1}{\PYZsq{}}\PY{p}{,} \PY{l+s+s1}{\PYZsq{}}\PY{l+s+s1}{ts4}\PY{l+s+s1}{\PYZsq{}}\PY{p}{,} \PY{l+s+s1}{\PYZsq{}}\PY{l+s+s1}{c3}\PY{l+s+s1}{\PYZsq{}}\PY{p}{,} 
                          \PY{l+s+s1}{\PYZsq{}}\PY{l+s+s1}{db1}\PY{l+s+s1}{\PYZsq{}}\PY{p}{,} \PY{l+s+s1}{\PYZsq{}}\PY{l+s+s1}{st}\PY{l+s+s1}{\PYZsq{}}\PY{p}{,} \PY{l+s+s1}{\PYZsq{}}\PY{l+s+s1}{db2}\PY{l+s+s1}{\PYZsq{}}\PY{p}{]}
        
        \PY{n}{squares\PYZus{}total} \PY{o}{=} \PY{n+nb}{len}\PY{p}{(}\PY{n}{squares\PYZus{}labels}\PY{p}{)}
        \PY{n+nb}{print}\PY{p}{(}\PY{l+s+s1}{\PYZsq{}}\PY{l+s+s1}{There are }\PY{l+s+si}{\PYZob{}\PYZcb{}}\PY{l+s+s1}{ squares.}\PY{l+s+s1}{\PYZsq{}}\PY{o}{.}\PY{n}{format}\PY{p}{(}\PY{n}{squares\PYZus{}total}\PY{p}{)}\PY{p}{)}
\end{Verbatim}


    \begin{Verbatim}[commandchars=\\\{\}]
There are 40 squares.

    \end{Verbatim}

    \hypertarget{descriptions}{%
\subsubsection{Descriptions}\label{descriptions}}

    We also want to know the proper names, so we don't have to look up the
labels.

    \begin{Verbatim}[commandchars=\\\{\}]
{\color{incolor}In [{\color{incolor}3}]:} \PY{n}{squares\PYZus{}description} \PY{o}{=} \PY{p}{[}\PY{l+s+s1}{\PYZsq{}}\PY{l+s+s1}{Start}\PY{l+s+s1}{\PYZsq{}}\PY{p}{,} \PY{l+s+s1}{\PYZsq{}}\PY{l+s+s1}{Brown 1}\PY{l+s+s1}{\PYZsq{}}\PY{p}{,} \PY{l+s+s1}{\PYZsq{}}\PY{l+s+s1}{Community Chest 1}\PY{l+s+s1}{\PYZsq{}}\PY{p}{,} \PY{l+s+s1}{\PYZsq{}}\PY{l+s+s1}{Brown 2}\PY{l+s+s1}{\PYZsq{}}\PY{p}{,} 
                               \PY{l+s+s1}{\PYZsq{}}\PY{l+s+s1}{Income Tax}\PY{l+s+s1}{\PYZsq{}}\PY{p}{,} \PY{l+s+s1}{\PYZsq{}}\PY{l+s+s1}{Train Station 1}\PY{l+s+s1}{\PYZsq{}}\PY{p}{,} \PY{l+s+s1}{\PYZsq{}}\PY{l+s+s1}{Light Blue 1}\PY{l+s+s1}{\PYZsq{}}\PY{p}{,} 
                               \PY{l+s+s1}{\PYZsq{}}\PY{l+s+s1}{Chance 1}\PY{l+s+s1}{\PYZsq{}}\PY{p}{,} \PY{l+s+s1}{\PYZsq{}}\PY{l+s+s1}{Light Blue 2}\PY{l+s+s1}{\PYZsq{}}\PY{p}{,} \PY{l+s+s1}{\PYZsq{}}\PY{l+s+s1}{Light Blue 3}\PY{l+s+s1}{\PYZsq{}}\PY{p}{,} \PY{l+s+s1}{\PYZsq{}}\PY{l+s+s1}{Jail}\PY{l+s+s1}{\PYZsq{}}\PY{p}{,} 
                               \PY{l+s+s1}{\PYZsq{}}\PY{l+s+s1}{Purple 1}\PY{l+s+s1}{\PYZsq{}}\PY{p}{,} \PY{l+s+s1}{\PYZsq{}}\PY{l+s+s1}{Electric Company}\PY{l+s+s1}{\PYZsq{}}\PY{p}{,} \PY{l+s+s1}{\PYZsq{}}\PY{l+s+s1}{Purple 2}\PY{l+s+s1}{\PYZsq{}}\PY{p}{,} 
                               \PY{l+s+s1}{\PYZsq{}}\PY{l+s+s1}{Purple 3}\PY{l+s+s1}{\PYZsq{}}\PY{p}{,} \PY{l+s+s1}{\PYZsq{}}\PY{l+s+s1}{Train Station 2}\PY{l+s+s1}{\PYZsq{}}\PY{p}{,} \PY{l+s+s1}{\PYZsq{}}\PY{l+s+s1}{Orange 1}\PY{l+s+s1}{\PYZsq{}}\PY{p}{,} 
                               \PY{l+s+s1}{\PYZsq{}}\PY{l+s+s1}{Community Chest 2}\PY{l+s+s1}{\PYZsq{}}\PY{p}{,} \PY{l+s+s1}{\PYZsq{}}\PY{l+s+s1}{Orange 2}\PY{l+s+s1}{\PYZsq{}}\PY{p}{,} \PY{l+s+s1}{\PYZsq{}}\PY{l+s+s1}{Orange 3}\PY{l+s+s1}{\PYZsq{}}\PY{p}{,} 
                               \PY{l+s+s1}{\PYZsq{}}\PY{l+s+s1}{Free Parking}\PY{l+s+s1}{\PYZsq{}}\PY{p}{,} \PY{l+s+s1}{\PYZsq{}}\PY{l+s+s1}{Red 1}\PY{l+s+s1}{\PYZsq{}}\PY{p}{,} \PY{l+s+s1}{\PYZsq{}}\PY{l+s+s1}{Chance 2}\PY{l+s+s1}{\PYZsq{}}\PY{p}{,} \PY{l+s+s1}{\PYZsq{}}\PY{l+s+s1}{Red 2}\PY{l+s+s1}{\PYZsq{}}\PY{p}{,} 
                               \PY{l+s+s1}{\PYZsq{}}\PY{l+s+s1}{Red 3}\PY{l+s+s1}{\PYZsq{}}\PY{p}{,} \PY{l+s+s1}{\PYZsq{}}\PY{l+s+s1}{Train Station 3}\PY{l+s+s1}{\PYZsq{}}\PY{p}{,} \PY{l+s+s1}{\PYZsq{}}\PY{l+s+s1}{Yellow 1}\PY{l+s+s1}{\PYZsq{}}\PY{p}{,} \PY{l+s+s1}{\PYZsq{}}\PY{l+s+s1}{Yellow 2}\PY{l+s+s1}{\PYZsq{}}\PY{p}{,} 
                               \PY{l+s+s1}{\PYZsq{}}\PY{l+s+s1}{Water Works}\PY{l+s+s1}{\PYZsq{}}\PY{p}{,} \PY{l+s+s1}{\PYZsq{}}\PY{l+s+s1}{Yellow 3}\PY{l+s+s1}{\PYZsq{}}\PY{p}{,} \PY{l+s+s1}{\PYZsq{}}\PY{l+s+s1}{Go to Jail}\PY{l+s+s1}{\PYZsq{}}\PY{p}{,} \PY{l+s+s1}{\PYZsq{}}\PY{l+s+s1}{Green 1}\PY{l+s+s1}{\PYZsq{}}\PY{p}{,} 
                               \PY{l+s+s1}{\PYZsq{}}\PY{l+s+s1}{Green 2}\PY{l+s+s1}{\PYZsq{}}\PY{p}{,} \PY{l+s+s1}{\PYZsq{}}\PY{l+s+s1}{Community Chest 3}\PY{l+s+s1}{\PYZsq{}}\PY{p}{,} \PY{l+s+s1}{\PYZsq{}}\PY{l+s+s1}{Green 3}\PY{l+s+s1}{\PYZsq{}}\PY{p}{,} 
                               \PY{l+s+s1}{\PYZsq{}}\PY{l+s+s1}{Train Station 4}\PY{l+s+s1}{\PYZsq{}}\PY{p}{,} \PY{l+s+s1}{\PYZsq{}}\PY{l+s+s1}{Chance 3}\PY{l+s+s1}{\PYZsq{}}\PY{p}{,} \PY{l+s+s1}{\PYZsq{}}\PY{l+s+s1}{Dark Blue 1}\PY{l+s+s1}{\PYZsq{}}\PY{p}{,} 
                               \PY{l+s+s1}{\PYZsq{}}\PY{l+s+s1}{Super Tax}\PY{l+s+s1}{\PYZsq{}}\PY{p}{,} \PY{l+s+s1}{\PYZsq{}}\PY{l+s+s1}{Dark Blue 2}\PY{l+s+s1}{\PYZsq{}}\PY{p}{]}
\end{Verbatim}


    \begin{Verbatim}[commandchars=\\\{\}]
{\color{incolor}In [{\color{incolor}4}]:} \PY{n+nb}{print}\PY{p}{(}\PY{l+s+s1}{\PYZsq{}}\PY{l+s+s1}{There are }\PY{l+s+si}{\PYZob{}\PYZcb{}}\PY{l+s+s1}{ descriptions.}\PY{l+s+s1}{\PYZsq{}}\PY{o}{.}\PY{n}{format}\PY{p}{(}\PY{n+nb}{len}\PY{p}{(}\PY{n}{squares\PYZus{}description}\PY{p}{)}\PY{p}{)}\PY{p}{)}
\end{Verbatim}


    \begin{Verbatim}[commandchars=\\\{\}]
There are 40 descriptions.

    \end{Verbatim}

    \hypertarget{purchasable}{%
\newpage
\subsubsection{Purchasable}\label{purchasable}}

    We want to know if they are purchasable so we can sort on that later.

    \begin{Verbatim}[commandchars=\\\{\}]
{\color{incolor}In [{\color{incolor}5}]:} \PY{n}{squares\PYZus{}purchasable} \PY{o}{=} \PY{p}{[}\PY{k+kc}{False}\PY{p}{,} \PY{k+kc}{True}\PY{p}{,} \PY{k+kc}{False}\PY{p}{,} \PY{k+kc}{True}\PY{p}{,} \PY{k+kc}{False}\PY{p}{,} 
                               \PY{k+kc}{True}\PY{p}{,} \PY{k+kc}{True}\PY{p}{,} \PY{k+kc}{False}\PY{p}{,} \PY{k+kc}{True}\PY{p}{,} \PY{k+kc}{True}\PY{p}{,} 
                               \PY{k+kc}{False}\PY{p}{,} \PY{k+kc}{True}\PY{p}{,} \PY{k+kc}{True}\PY{p}{,} \PY{k+kc}{True}\PY{p}{,} \PY{k+kc}{True}\PY{p}{,} 
                               \PY{k+kc}{True}\PY{p}{,} \PY{k+kc}{True}\PY{p}{,} \PY{k+kc}{False}\PY{p}{,} \PY{k+kc}{True}\PY{p}{,} \PY{k+kc}{True}\PY{p}{,} 
                               \PY{k+kc}{False}\PY{p}{,} \PY{k+kc}{True}\PY{p}{,} \PY{k+kc}{False}\PY{p}{,} \PY{k+kc}{True}\PY{p}{,} \PY{k+kc}{True}\PY{p}{,} 
                               \PY{k+kc}{True}\PY{p}{,} \PY{k+kc}{True}\PY{p}{,} \PY{k+kc}{True}\PY{p}{,} \PY{k+kc}{True}\PY{p}{,} \PY{k+kc}{True}\PY{p}{,} 
                               \PY{k+kc}{False}\PY{p}{,} \PY{k+kc}{True}\PY{p}{,} \PY{k+kc}{True}\PY{p}{,} \PY{k+kc}{False}\PY{p}{,} \PY{k+kc}{True}\PY{p}{,} 
                               \PY{k+kc}{True}\PY{p}{,} \PY{k+kc}{False}\PY{p}{,} \PY{k+kc}{True}\PY{p}{,} \PY{k+kc}{False}\PY{p}{,} \PY{k+kc}{True}\PY{p}{]}
\end{Verbatim}


    \hypertarget{grouping}{%
\subsubsection{Grouping}\label{grouping}}

    We want to know in what group they are so we can aggregate our data
later.

    \begin{Verbatim}[commandchars=\\\{\}]
{\color{incolor}In [{\color{incolor}6}]:} \PY{n}{squares\PYZus{}aggregate} \PY{o}{=} \PY{p}{[}\PY{l+s+s1}{\PYZsq{}}\PY{l+s+s1}{Start}\PY{l+s+s1}{\PYZsq{}}\PY{p}{,} \PY{l+s+s1}{\PYZsq{}}\PY{l+s+s1}{Brown}\PY{l+s+s1}{\PYZsq{}}\PY{p}{,} \PY{l+s+s1}{\PYZsq{}}\PY{l+s+s1}{Community Chest}\PY{l+s+s1}{\PYZsq{}}\PY{p}{,} \PY{l+s+s1}{\PYZsq{}}\PY{l+s+s1}{Brown}\PY{l+s+s1}{\PYZsq{}}\PY{p}{,} 
                             \PY{l+s+s1}{\PYZsq{}}\PY{l+s+s1}{Income Tax}\PY{l+s+s1}{\PYZsq{}}\PY{p}{,} \PY{l+s+s1}{\PYZsq{}}\PY{l+s+s1}{Train Station}\PY{l+s+s1}{\PYZsq{}}\PY{p}{,} \PY{l+s+s1}{\PYZsq{}}\PY{l+s+s1}{Light Blue}\PY{l+s+s1}{\PYZsq{}}\PY{p}{,} 
                             \PY{l+s+s1}{\PYZsq{}}\PY{l+s+s1}{Chance}\PY{l+s+s1}{\PYZsq{}}\PY{p}{,} \PY{l+s+s1}{\PYZsq{}}\PY{l+s+s1}{Light Blue}\PY{l+s+s1}{\PYZsq{}}\PY{p}{,} \PY{l+s+s1}{\PYZsq{}}\PY{l+s+s1}{Light Blue}\PY{l+s+s1}{\PYZsq{}}\PY{p}{,} \PY{l+s+s1}{\PYZsq{}}\PY{l+s+s1}{Jail}\PY{l+s+s1}{\PYZsq{}}\PY{p}{,} 
                             \PY{l+s+s1}{\PYZsq{}}\PY{l+s+s1}{Purple}\PY{l+s+s1}{\PYZsq{}}\PY{p}{,} \PY{l+s+s1}{\PYZsq{}}\PY{l+s+s1}{Electric Company}\PY{l+s+s1}{\PYZsq{}}\PY{p}{,} \PY{l+s+s1}{\PYZsq{}}\PY{l+s+s1}{Purple}\PY{l+s+s1}{\PYZsq{}}\PY{p}{,} \PY{l+s+s1}{\PYZsq{}}\PY{l+s+s1}{Purple}\PY{l+s+s1}{\PYZsq{}}\PY{p}{,} 
                             \PY{l+s+s1}{\PYZsq{}}\PY{l+s+s1}{Train Station}\PY{l+s+s1}{\PYZsq{}}\PY{p}{,} \PY{l+s+s1}{\PYZsq{}}\PY{l+s+s1}{Orange}\PY{l+s+s1}{\PYZsq{}}\PY{p}{,} \PY{l+s+s1}{\PYZsq{}}\PY{l+s+s1}{Community Chest}\PY{l+s+s1}{\PYZsq{}}\PY{p}{,} 
                             \PY{l+s+s1}{\PYZsq{}}\PY{l+s+s1}{Orange}\PY{l+s+s1}{\PYZsq{}}\PY{p}{,} \PY{l+s+s1}{\PYZsq{}}\PY{l+s+s1}{Orange}\PY{l+s+s1}{\PYZsq{}}\PY{p}{,} \PY{l+s+s1}{\PYZsq{}}\PY{l+s+s1}{Free Parking}\PY{l+s+s1}{\PYZsq{}}\PY{p}{,} \PY{l+s+s1}{\PYZsq{}}\PY{l+s+s1}{Red}\PY{l+s+s1}{\PYZsq{}}\PY{p}{,} 
                             \PY{l+s+s1}{\PYZsq{}}\PY{l+s+s1}{Chance}\PY{l+s+s1}{\PYZsq{}}\PY{p}{,} \PY{l+s+s1}{\PYZsq{}}\PY{l+s+s1}{Red}\PY{l+s+s1}{\PYZsq{}}\PY{p}{,} \PY{l+s+s1}{\PYZsq{}}\PY{l+s+s1}{Red}\PY{l+s+s1}{\PYZsq{}}\PY{p}{,} \PY{l+s+s1}{\PYZsq{}}\PY{l+s+s1}{Train Station}\PY{l+s+s1}{\PYZsq{}}\PY{p}{,} \PY{l+s+s1}{\PYZsq{}}\PY{l+s+s1}{Yellow}\PY{l+s+s1}{\PYZsq{}}\PY{p}{,} 
                             \PY{l+s+s1}{\PYZsq{}}\PY{l+s+s1}{Yellow}\PY{l+s+s1}{\PYZsq{}}\PY{p}{,} \PY{l+s+s1}{\PYZsq{}}\PY{l+s+s1}{Water Works}\PY{l+s+s1}{\PYZsq{}}\PY{p}{,} \PY{l+s+s1}{\PYZsq{}}\PY{l+s+s1}{Yellow}\PY{l+s+s1}{\PYZsq{}}\PY{p}{,} \PY{l+s+s1}{\PYZsq{}}\PY{l+s+s1}{Go to Jail}\PY{l+s+s1}{\PYZsq{}}\PY{p}{,} 
                             \PY{l+s+s1}{\PYZsq{}}\PY{l+s+s1}{Green}\PY{l+s+s1}{\PYZsq{}}\PY{p}{,} \PY{l+s+s1}{\PYZsq{}}\PY{l+s+s1}{Green}\PY{l+s+s1}{\PYZsq{}}\PY{p}{,} \PY{l+s+s1}{\PYZsq{}}\PY{l+s+s1}{Community Chest}\PY{l+s+s1}{\PYZsq{}}\PY{p}{,} \PY{l+s+s1}{\PYZsq{}}\PY{l+s+s1}{Green}\PY{l+s+s1}{\PYZsq{}}\PY{p}{,} 
                             \PY{l+s+s1}{\PYZsq{}}\PY{l+s+s1}{Train Station}\PY{l+s+s1}{\PYZsq{}}\PY{p}{,} \PY{l+s+s1}{\PYZsq{}}\PY{l+s+s1}{Chance}\PY{l+s+s1}{\PYZsq{}}\PY{p}{,} \PY{l+s+s1}{\PYZsq{}}\PY{l+s+s1}{Dark Blue}\PY{l+s+s1}{\PYZsq{}}\PY{p}{,} 
                             \PY{l+s+s1}{\PYZsq{}}\PY{l+s+s1}{Super Tax}\PY{l+s+s1}{\PYZsq{}}\PY{p}{,} \PY{l+s+s1}{\PYZsq{}}\PY{l+s+s1}{Dark Blue}\PY{l+s+s1}{\PYZsq{}}\PY{p}{]}
\end{Verbatim}


    \hypertarget{cards}{%
\newpage
\subsection{Cards}\label{cards}}

    There are two decks of cards.

\begin{itemize}
\tightlist
\item
  Community Cards
\item
  Chance Cards
\end{itemize}

Each deck contains 16 cards.

    \hypertarget{community-cards}{%
\subsubsection{Community cards}\label{community-cards}}

    Monopoly has \(16\) community cards.

    \begin{figure}[H]
\centering
\includegraphics{monopoly_2.jpg}
\caption{cc}
\end{figure}

    Because we are only determining the probabilities, we are only
interested in the following cards:

\begin{itemize}
\tightlist
\item
  advance to go
\item
  go to jail
\item
  get out of jail, free
\item
  go back 2 spaces
\end{itemize}

    \hypertarget{community-deck-implementation}{%
\newpage
\subsubsection{Community deck
implementation}\label{community-deck-implementation}}

    We implement the community deck in a class. The class keeps track of a
list with \(16\) cards. An index points to the next card. When we are
out of cards, we reset the index and reshuffle the cards.

    \begin{Verbatim}[commandchars=\\\{\}]
{\color{incolor}In [{\color{incolor}28}]:} \PY{k+kn}{from} \PY{n+nn}{random} \PY{k}{import} \PY{n}{shuffle}
         
         \PY{k}{class} \PY{n+nc}{CommunityDeck}\PY{p}{(}\PY{p}{)}\PY{p}{:}
             \PY{k}{def} \PY{n+nf}{\PYZus{}\PYZus{}init\PYZus{}\PYZus{}}\PY{p}{(}\PY{n+nb+bp}{self}\PY{p}{)}\PY{p}{:}
                 \PY{n+nb+bp}{self}\PY{o}{.}\PY{n}{deck} \PY{o}{=} \PY{p}{[}\PY{l+m+mi}{0}\PY{p}{]} \PY{o}{*} \PY{l+m+mi}{16}
                 \PY{n+nb+bp}{self}\PY{o}{.}\PY{n}{deck}\PY{p}{[}\PY{l+m+mi}{0}\PY{p}{]} \PY{o}{=} \PY{l+s+s1}{\PYZsq{}}\PY{l+s+s1}{gtg}\PY{l+s+s1}{\PYZsq{}} \PY{c+c1}{\PYZsh{} go to go}
                 \PY{n+nb+bp}{self}\PY{o}{.}\PY{n}{deck}\PY{p}{[}\PY{l+m+mi}{1}\PY{p}{]} \PY{o}{=} \PY{l+s+s1}{\PYZsq{}}\PY{l+s+s1}{gtj}\PY{l+s+s1}{\PYZsq{}} \PY{c+c1}{\PYZsh{} go to jail}
                 \PY{n+nb+bp}{self}\PY{o}{.}\PY{n}{deck}\PY{p}{[}\PY{l+m+mi}{2}\PY{p}{]} \PY{o}{=} \PY{l+s+s1}{\PYZsq{}}\PY{l+s+s1}{goj}\PY{l+s+s1}{\PYZsq{}} \PY{c+c1}{\PYZsh{} get out of jail }
                 \PY{n+nb+bp}{self}\PY{o}{.}\PY{n}{deck}\PY{p}{[}\PY{l+m+mi}{3}\PY{p}{]} \PY{o}{=} \PY{l+s+s1}{\PYZsq{}}\PY{l+s+s1}{gb2}\PY{l+s+s1}{\PYZsq{}} \PY{c+c1}{\PYZsh{} go back 2 steps}
                 \PY{n+nb+bp}{self}\PY{o}{.}\PY{n}{index} \PY{o}{=} \PY{l+m+mi}{16}
             
             \PY{k}{def} \PY{n+nf}{draw\PYZus{}card}\PY{p}{(}\PY{n+nb+bp}{self}\PY{p}{)}\PY{p}{:}
                 \PY{k}{if} \PY{n+nb+bp}{self}\PY{o}{.}\PY{n}{index} \PY{o}{\PYZgt{}}\PY{o}{=} \PY{n+nb}{len}\PY{p}{(}\PY{n+nb+bp}{self}\PY{o}{.}\PY{n}{deck}\PY{p}{)}\PY{p}{:}
                     \PY{n+nb+bp}{self}\PY{o}{.}\PY{n}{index} \PY{o}{=} \PY{l+m+mi}{0}
                     \PY{n}{shuffle}\PY{p}{(}\PY{n+nb+bp}{self}\PY{o}{.}\PY{n}{deck}\PY{p}{)}
                 \PY{n}{card} \PY{o}{=} \PY{n+nb+bp}{self}\PY{o}{.}\PY{n}{deck}\PY{p}{[}\PY{n+nb+bp}{self}\PY{o}{.}\PY{n}{index}\PY{p}{]}
                 \PY{n+nb+bp}{self}\PY{o}{.}\PY{n}{index} \PY{o}{+}\PY{o}{=} \PY{l+m+mi}{1}
                 \PY{k}{return} \PY{n}{card}
\end{Verbatim}


    Now we test it:

    \begin{Verbatim}[commandchars=\\\{\}]
{\color{incolor}In [{\color{incolor}34}]:} \PY{n}{deck} \PY{o}{=} \PY{n}{CommunityDeck}\PY{p}{(}\PY{p}{)}
         \PY{n}{deck}\PY{o}{.}\PY{n}{deck}
\end{Verbatim}


\begin{Verbatim}[commandchars=\\\{\}]
{\color{outcolor}Out[{\color{outcolor}34}]:} ['gtg', 'gtj', 'goj', 'gb2', 0, 0, 0, 0, 0, 0, 0, 0, 0, 0, 0, 0]
\end{Verbatim}
            
    \begin{Verbatim}[commandchars=\\\{\}]
{\color{incolor}In [{\color{incolor}35}]:} \PY{n}{deck}\PY{o}{.}\PY{n}{draw\PYZus{}card}\PY{p}{(}\PY{p}{)}
\end{Verbatim}


\begin{Verbatim}[commandchars=\\\{\}]
{\color{outcolor}Out[{\color{outcolor}35}]:} 0
\end{Verbatim}
            
    \begin{Verbatim}[commandchars=\\\{\}]
{\color{incolor}In [{\color{incolor}36}]:} \PY{n}{deck}\PY{o}{.}\PY{n}{deck}
\end{Verbatim}


\begin{Verbatim}[commandchars=\\\{\}]
{\color{outcolor}Out[{\color{outcolor}36}]:} [0, 0, 0, 0, 'gb2', 0, 0, 'gtg', 0, 'gtj', 0, 'goj', 0, 0, 0, 0]
\end{Verbatim}
            
    \begin{Verbatim}[commandchars=\\\{\}]
{\color{incolor}In [{\color{incolor}37}]:} \PY{n}{deck}\PY{o}{.}\PY{n}{index}
\end{Verbatim}


\begin{Verbatim}[commandchars=\\\{\}]
{\color{outcolor}Out[{\color{outcolor}37}]:} 1
\end{Verbatim}
            
    \hypertarget{chance-cards}{%
\newpage
\subsubsection{Chance cards}\label{chance-cards}}

    Monopoly has \(16\) chance cards.

    \begin{figure}[H]
\centering
\includegraphics{monopoly_3.jpg}
\caption{chance}
\end{figure}

    Because we are only determining the probabilities, we are only
interested in the following cards:

\begin{itemize}
\tightlist
\item
  go back three spaces
\item
  get out of jail free
\item
  advance to go
\item
  advance to illinois avenue (R3)
\item
  go to jail
\end{itemize}

    \hypertarget{chance-deck-implementation}{%
\newpage
\subsubsection{Chance deck
implementation}\label{chance-deck-implementation}}

    We implement the chance deck in a class. The class keeps track of a list
with \(16\) cards. An index points to the next card. When we are out of
cards, we reset the index and reshuffle the cards.

    \begin{Verbatim}[commandchars=\\\{\}]
{\color{incolor}In [{\color{incolor}8}]:} \PY{k+kn}{from} \PY{n+nn}{random} \PY{k}{import} \PY{n}{shuffle}
        
        \PY{k}{class} \PY{n+nc}{ChanceDeck}\PY{p}{(}\PY{p}{)}\PY{p}{:}
            \PY{k}{def} \PY{n+nf}{\PYZus{}\PYZus{}init\PYZus{}\PYZus{}}\PY{p}{(}\PY{n+nb+bp}{self}\PY{p}{)}\PY{p}{:}
                \PY{n+nb+bp}{self}\PY{o}{.}\PY{n}{deck} \PY{o}{=} \PY{p}{[}\PY{l+m+mi}{0}\PY{p}{]} \PY{o}{*} \PY{l+m+mi}{16}
                \PY{n+nb+bp}{self}\PY{o}{.}\PY{n}{deck}\PY{p}{[}\PY{l+m+mi}{0}\PY{p}{]} \PY{o}{=} \PY{l+s+s1}{\PYZsq{}}\PY{l+s+s1}{gtg}\PY{l+s+s1}{\PYZsq{}} \PY{c+c1}{\PYZsh{} go to go}
                \PY{n+nb+bp}{self}\PY{o}{.}\PY{n}{deck}\PY{p}{[}\PY{l+m+mi}{1}\PY{p}{]} \PY{o}{=} \PY{l+s+s1}{\PYZsq{}}\PY{l+s+s1}{gtj}\PY{l+s+s1}{\PYZsq{}} \PY{c+c1}{\PYZsh{} go to jail}
                \PY{n+nb+bp}{self}\PY{o}{.}\PY{n}{deck}\PY{p}{[}\PY{l+m+mi}{2}\PY{p}{]} \PY{o}{=} \PY{l+s+s1}{\PYZsq{}}\PY{l+s+s1}{goj}\PY{l+s+s1}{\PYZsq{}} \PY{c+c1}{\PYZsh{} get out of jail}
                \PY{n+nb+bp}{self}\PY{o}{.}\PY{n}{deck}\PY{p}{[}\PY{l+m+mi}{3}\PY{p}{]} \PY{o}{=} \PY{l+s+s1}{\PYZsq{}}\PY{l+s+s1}{gb3}\PY{l+s+s1}{\PYZsq{}} \PY{c+c1}{\PYZsh{} go back 3}
                \PY{n+nb+bp}{self}\PY{o}{.}\PY{n}{deck}\PY{p}{[}\PY{l+m+mi}{4}\PY{p}{]} \PY{o}{=} \PY{l+s+s1}{\PYZsq{}}\PY{l+s+s1}{r3}\PY{l+s+s1}{\PYZsq{}}  \PY{c+c1}{\PYZsh{} go to red 3 (r3)}
                \PY{n+nb+bp}{self}\PY{o}{.}\PY{n}{index} \PY{o}{=} \PY{l+m+mi}{16}
            
            \PY{k}{def} \PY{n+nf}{draw\PYZus{}card}\PY{p}{(}\PY{n+nb+bp}{self}\PY{p}{)}\PY{p}{:}
                \PY{k}{if} \PY{n+nb+bp}{self}\PY{o}{.}\PY{n}{index} \PY{o}{\PYZgt{}}\PY{o}{=} \PY{n+nb}{len}\PY{p}{(}\PY{n+nb+bp}{self}\PY{o}{.}\PY{n}{deck}\PY{p}{)}\PY{p}{:}
                    \PY{n+nb+bp}{self}\PY{o}{.}\PY{n}{index} \PY{o}{=} \PY{l+m+mi}{0}
                    \PY{n}{shuffle}\PY{p}{(}\PY{n+nb+bp}{self}\PY{o}{.}\PY{n}{deck}\PY{p}{)}
                \PY{n}{card} \PY{o}{=} \PY{n+nb+bp}{self}\PY{o}{.}\PY{n}{deck}\PY{p}{[}\PY{n+nb+bp}{self}\PY{o}{.}\PY{n}{index}\PY{p}{]}
                \PY{n+nb+bp}{self}\PY{o}{.}\PY{n}{index} \PY{o}{+}\PY{o}{=} \PY{l+m+mi}{1}
                \PY{k}{return} \PY{n}{card}
\end{Verbatim}


    Now we test it:

    \begin{Verbatim}[commandchars=\\\{\}]
{\color{incolor}In [{\color{incolor}30}]:} \PY{n}{deck} \PY{o}{=} \PY{n}{ChanceDeck}\PY{p}{(}\PY{p}{)}
         \PY{n}{deck}\PY{o}{.}\PY{n}{deck}
\end{Verbatim}


\begin{Verbatim}[commandchars=\\\{\}]
{\color{outcolor}Out[{\color{outcolor}30}]:} ['gtg', 'gtj', 'goj', 'gb3', 'r3', 0, 0, 0, 0, 0, 0, 0, 0, 0, 0, 0]
\end{Verbatim}
            
    \begin{Verbatim}[commandchars=\\\{\}]
{\color{incolor}In [{\color{incolor}31}]:} \PY{n}{deck}\PY{o}{.}\PY{n}{draw\PYZus{}card}\PY{p}{(}\PY{p}{)}
\end{Verbatim}


\begin{Verbatim}[commandchars=\\\{\}]
{\color{outcolor}Out[{\color{outcolor}31}]:} 0
\end{Verbatim}
            
    \begin{Verbatim}[commandchars=\\\{\}]
{\color{incolor}In [{\color{incolor}32}]:} \PY{n}{deck}\PY{o}{.}\PY{n}{deck}
\end{Verbatim}


\begin{Verbatim}[commandchars=\\\{\}]
{\color{outcolor}Out[{\color{outcolor}32}]:} [0, 'gtj', 0, 'gb3', 0, 0, 0, 0, 0, 0, 'goj', 0, 0, 'gtg', 0, 'r3']
\end{Verbatim}
            
    \begin{Verbatim}[commandchars=\\\{\}]
{\color{incolor}In [{\color{incolor}33}]:} \PY{n}{deck}\PY{o}{.}\PY{n}{index}
\end{Verbatim}


\begin{Verbatim}[commandchars=\\\{\}]
{\color{outcolor}Out[{\color{outcolor}33}]:} 1
\end{Verbatim}
            
    \hypertarget{dice}{%
\newpage
\subsection{Dice}\label{dice}}

    We will be implementing the dice as a class. This allows us to
encapsulate how the result is determined. It makes it easier to
implements other scenarios such as throwing with multiple dices.

    \begin{Verbatim}[commandchars=\\\{\}]
{\color{incolor}In [{\color{incolor}9}]:} \PY{k+kn}{from} \PY{n+nn}{random} \PY{k}{import} \PY{n}{randint}
        
        \PY{k}{class} \PY{n+nc}{Dice}\PY{p}{(}\PY{p}{)}\PY{p}{:}
            \PY{k}{def} \PY{n+nf}{\PYZus{}\PYZus{}init\PYZus{}\PYZus{}}\PY{p}{(}\PY{n+nb+bp}{self}\PY{p}{,} \PY{n}{dices} \PY{o}{=} \PY{l+m+mi}{1}\PY{p}{,} \PY{n}{sides} \PY{o}{=} \PY{l+m+mi}{6}\PY{p}{)}\PY{p}{:}
                \PY{n+nb+bp}{self}\PY{o}{.}\PY{n}{dices} \PY{o}{=} \PY{n}{dices}
                \PY{n+nb+bp}{self}\PY{o}{.}\PY{n}{sides} \PY{o}{=} \PY{l+m+mi}{6}
            
            \PY{k}{def} \PY{n+nf}{throw}\PY{p}{(}\PY{n+nb+bp}{self}\PY{p}{)}\PY{p}{:}
                \PY{n}{total} \PY{o}{=} \PY{l+m+mi}{0} 
                \PY{k}{for} \PY{n}{i} \PY{o+ow}{in} \PY{n+nb}{range}\PY{p}{(}\PY{n+nb+bp}{self}\PY{o}{.}\PY{n}{dices}\PY{p}{)}\PY{p}{:}
                    \PY{n}{total} \PY{o}{+}\PY{o}{=} \PY{n}{randint}\PY{p}{(}\PY{l+m+mi}{1}\PY{p}{,} \PY{n+nb+bp}{self}\PY{o}{.}\PY{n}{sides}\PY{p}{)}
                \PY{k}{return} \PY{n}{total}
\end{Verbatim}


    Rolling one time:

    \begin{Verbatim}[commandchars=\\\{\}]
{\color{incolor}In [{\color{incolor}39}]:} \PY{n}{dice} \PY{o}{=} \PY{n}{Dice}\PY{p}{(}\PY{p}{)}
         \PY{n}{dice}\PY{o}{.}\PY{n}{throw}\PY{p}{(}\PY{p}{)}
\end{Verbatim}


\begin{Verbatim}[commandchars=\\\{\}]
{\color{outcolor}Out[{\color{outcolor}39}]:} 5
\end{Verbatim}
            
    \hypertarget{simple-one-dice-with-six-sides}{%

\subsubsection{Simple: one dice with six sides}\label{simple-one-dice-with-six-sides}}

    A simple setup would be one dice with six sides. This will give
uniformly distributed probabilities.

    \begin{Verbatim}[commandchars=\\\{\}]
{\color{incolor}In [{\color{incolor}10}]:} \PY{n}{dice} \PY{o}{=} \PY{n}{Dice}\PY{p}{(}\PY{p}{)}
         \PY{n}{sides} \PY{o}{=} \PY{p}{[}\PY{l+m+mi}{0}\PY{p}{]} \PY{o}{*} \PY{n}{dice}\PY{o}{.}\PY{n}{dices} \PY{o}{*} \PY{n}{dice}\PY{o}{.}\PY{n}{sides}
         \PY{n}{N} \PY{o}{=} \PY{l+m+mi}{10000}
         \PY{k}{for} \PY{n}{i} \PY{o+ow}{in} \PY{n+nb}{range}\PY{p}{(}\PY{n}{N}\PY{p}{)}\PY{p}{:} \PY{n}{sides}\PY{p}{[}\PY{n}{dice}\PY{o}{.}\PY{n}{throw}\PY{p}{(}\PY{p}{)}\PY{o}{\PYZhy{}}\PY{l+m+mi}{1}\PY{p}{]} \PY{o}{+}\PY{o}{=} \PY{l+m+mi}{1}
         \PY{n}{sides} \PY{o}{=} \PY{n}{np}\PY{o}{.}\PY{n}{array}\PY{p}{(}\PY{n}{sides}\PY{p}{)} \PY{o}{/} \PY{n}{N}
         \PY{n}{bar}\PY{p}{(}\PY{n+nb}{range}\PY{p}{(}\PY{l+m+mi}{1}\PY{p}{,}\PY{n+nb}{len}\PY{p}{(}\PY{n}{sides}\PY{p}{)}\PY{o}{+}\PY{l+m+mi}{1}\PY{p}{)}\PY{p}{,} \PY{n}{sides}\PY{p}{)}\PY{p}{;}
\end{Verbatim}


    \begin{center}
    \adjustimage{max size={0.9\linewidth}{0.9\paperheight}}{output_51_0.png}
    \end{center}
    { \hspace*{\fill} \\}
    
    \hypertarget{advanced-two-dice-with-four-sides}{%

\subsubsection{Advanced: two dice with four
sides}\label{advanced-two-dice-with-four-sides}}

    A more advanced set up would be to play with two dices and four sides.
This will give normally distributed probabilities.

    \begin{Verbatim}[commandchars=\\\{\}]
{\color{incolor}In [{\color{incolor}11}]:} \PY{n}{dice} \PY{o}{=} \PY{n}{Dice}\PY{p}{(}\PY{l+m+mi}{2}\PY{p}{,}\PY{l+m+mi}{4}\PY{p}{)}
         \PY{n}{sides} \PY{o}{=} \PY{p}{[}\PY{l+m+mi}{0}\PY{p}{]} \PY{o}{*} \PY{n}{dice}\PY{o}{.}\PY{n}{dices} \PY{o}{*} \PY{n}{dice}\PY{o}{.}\PY{n}{sides}
         \PY{n}{N} \PY{o}{=} \PY{l+m+mi}{10000}
         \PY{k}{for} \PY{n}{i} \PY{o+ow}{in} \PY{n+nb}{range}\PY{p}{(}\PY{n}{N}\PY{p}{)}\PY{p}{:} \PY{n}{sides}\PY{p}{[}\PY{n}{dice}\PY{o}{.}\PY{n}{throw}\PY{p}{(}\PY{p}{)}\PY{o}{\PYZhy{}}\PY{l+m+mi}{1}\PY{p}{]} \PY{o}{+}\PY{o}{=} \PY{l+m+mi}{1}
         \PY{n}{sides} \PY{o}{=} \PY{n}{np}\PY{o}{.}\PY{n}{array}\PY{p}{(}\PY{n}{sides}\PY{p}{)} \PY{o}{/} \PY{n}{N}
         \PY{n}{bar}\PY{p}{(}\PY{n+nb}{range}\PY{p}{(}\PY{l+m+mi}{1}\PY{p}{,}\PY{n+nb}{len}\PY{p}{(}\PY{n}{sides}\PY{p}{)}\PY{o}{+}\PY{l+m+mi}{1}\PY{p}{)}\PY{p}{,} \PY{n}{sides}\PY{p}{)}\PY{p}{;}
\end{Verbatim}


    \begin{center}
    \adjustimage{max size={0.9\linewidth}{0.9\paperheight}}{output_54_0.png}
    \end{center}
    { \hspace*{\fill} \\}
    
    \hypertarget{monopoly-simulation}{%
\subsection{Monopoly simulation}\label{monopoly-simulation}}

    \hypertarget{algorithm}{%
\subsubsection{Algorithm}\label{algorithm}}

    Here we are going to simulate a game for \(N\) amount of rounds. The
game algorithm is simple:

\begin{enumerate}
\def\labelenumi{\arabic{enumi}.}
\tightlist
\item
  Roll the dice
\item
  Move to the new position
\item
  Increment the square counter for that position
\item
  Check and handle go to jail
\item
  Check and handle community chest
\item
  Check and handle chance
\end{enumerate}

    \hypertarget{modulo-arithmetic-for-position-tracking}{%
\subsubsection{Modulo arithmetic for position
tracking}\label{modulo-arithmetic-for-position-tracking}}

    We can easily keep track of our position with modulo arithemetic. Let
\(C\) be our position (or \texttt{index}), \(d\) the result from
throwing the dice, and \(n\) the current round. To determine our new
position we calculate:

\[ C_{n+1} \equiv C_n+d \pmod{40}\]

    The modulo is \(40\) because that are the total amount of squares.

    \hypertarget{implementation}{%
\subsubsection{Implementation}\label{implementation}}

    Below is the implementation for the Monopoly simulation.

    \begin{Verbatim}[commandchars=\\\{\}]
{\color{incolor}In [{\color{incolor}12}]:} \PY{n}{dice} \PY{o}{=} \PY{n}{Dice}\PY{p}{(}\PY{p}{)}
         \PY{n}{community\PYZus{}deck} \PY{o}{=} \PY{n}{CommunityDeck}\PY{p}{(}\PY{p}{)}
         \PY{n}{chance\PYZus{}deck} \PY{o}{=} \PY{n}{ChanceDeck}\PY{p}{(}\PY{p}{)}
         
         \PY{n}{index} \PY{o}{=} \PY{l+m+mi}{0} \PY{c+c1}{\PYZsh{} position}
         \PY{n}{total\PYZus{}squares} \PY{o}{=} \PY{n+nb}{len}\PY{p}{(}\PY{n}{squares\PYZus{}labels}\PY{p}{)}
         \PY{n}{squares} \PY{o}{=} \PY{p}{[}\PY{l+m+mi}{0}\PY{p}{]} \PY{o}{*} \PY{n}{total\PYZus{}squares}
         
         \PY{n}{rounds} \PY{o}{=} \PY{l+m+mi}{1000000}
         
         \PY{k}{for} \PY{n}{i} \PY{o+ow}{in} \PY{n+nb}{range}\PY{p}{(}\PY{n}{rounds}\PY{p}{)}\PY{p}{:}
             
             \PY{c+c1}{\PYZsh{} Throw the dice and move our position on the board.}
             \PY{n}{steps} \PY{o}{=} \PY{n}{dice}\PY{o}{.}\PY{n}{throw}\PY{p}{(}\PY{p}{)}
             \PY{n}{index} \PY{o}{=} \PY{p}{(}\PY{n}{index} \PY{o}{+} \PY{n}{steps}\PY{p}{)} \PY{o}{\PYZpc{}} \PY{n}{total\PYZus{}squares}
             \PY{n}{squares}\PY{p}{[}\PY{n}{index}\PY{p}{]} \PY{o}{+}\PY{o}{=} \PY{l+m+mi}{1}
             
             \PY{c+c1}{\PYZsh{} We landed on go to jail.}
             \PY{k}{if} \PY{n}{squares\PYZus{}labels}\PY{p}{[}\PY{n}{index}\PY{p}{]} \PY{o+ow}{is} \PY{l+s+s1}{\PYZsq{}}\PY{l+s+s1}{gtj}\PY{l+s+s1}{\PYZsq{}}\PY{p}{:} 
                 \PY{n}{index} \PY{o}{=} \PY{n}{squares\PYZus{}labels}\PY{o}{.}\PY{n}{index}\PY{p}{(}\PY{l+s+s1}{\PYZsq{}}\PY{l+s+s1}{jail}\PY{l+s+s1}{\PYZsq{}}\PY{p}{)}
             
             \PY{c+c1}{\PYZsh{} We landed on the community card.}
             \PY{k}{if} \PY{n}{squares\PYZus{}labels}\PY{p}{[}\PY{n}{index}\PY{p}{]} \PY{o+ow}{in} \PY{p}{[}\PY{l+s+s1}{\PYZsq{}}\PY{l+s+s1}{cc1}\PY{l+s+s1}{\PYZsq{}}\PY{p}{,} \PY{l+s+s1}{\PYZsq{}}\PY{l+s+s1}{cc2}\PY{l+s+s1}{\PYZsq{}}\PY{p}{,} \PY{l+s+s1}{\PYZsq{}}\PY{l+s+s1}{cc3}\PY{l+s+s1}{\PYZsq{}}\PY{p}{]}\PY{p}{:}
                 \PY{n}{card} \PY{o}{=} \PY{n}{community\PYZus{}deck}\PY{o}{.}\PY{n}{draw\PYZus{}card}\PY{p}{(}\PY{p}{)}
                 \PY{k}{if} \PY{n}{card} \PY{o+ow}{is} \PY{l+s+s1}{\PYZsq{}}\PY{l+s+s1}{gtg}\PY{l+s+s1}{\PYZsq{}}\PY{p}{:} \PY{n}{index} \PY{o}{=} \PY{n}{squares\PYZus{}labels}\PY{o}{.}\PY{n}{index}\PY{p}{(}\PY{l+s+s1}{\PYZsq{}}\PY{l+s+s1}{start}\PY{l+s+s1}{\PYZsq{}}\PY{p}{)}
                 \PY{k}{if} \PY{n}{card} \PY{o+ow}{is} \PY{l+s+s1}{\PYZsq{}}\PY{l+s+s1}{gtj}\PY{l+s+s1}{\PYZsq{}}\PY{p}{:} \PY{n}{index} \PY{o}{=} \PY{n}{squares\PYZus{}labels}\PY{o}{.}\PY{n}{index}\PY{p}{(}\PY{l+s+s1}{\PYZsq{}}\PY{l+s+s1}{jail}\PY{l+s+s1}{\PYZsq{}}\PY{p}{)}
                 \PY{k}{if} \PY{n}{card} \PY{o+ow}{is} \PY{l+s+s1}{\PYZsq{}}\PY{l+s+s1}{gb2}\PY{l+s+s1}{\PYZsq{}}\PY{p}{:} 
                     \PY{k}{if} \PY{n}{index} \PY{o}{\PYZgt{}}\PY{o}{=} \PY{l+m+mi}{2}\PY{p}{:} \PY{n}{index} \PY{o}{\PYZhy{}}\PY{o}{=} \PY{l+m+mi}{2}
                     \PY{k}{if} \PY{n}{index} \PY{o}{\PYZlt{}} \PY{l+m+mi}{2}\PY{p}{:} \PY{n}{index} \PY{o}{=} \PY{n}{total\PYZus{}squares}\PY{o}{\PYZhy{}}\PY{n+nb}{abs}\PY{p}{(}\PY{n}{index}\PY{o}{\PYZhy{}}\PY{l+m+mi}{2}\PY{p}{)}\PY{o}{\PYZhy{}}\PY{l+m+mi}{1}
             
             \PY{c+c1}{\PYZsh{} We landed on the chance card.}
             \PY{k}{if} \PY{n}{squares\PYZus{}labels}\PY{p}{[}\PY{n}{index}\PY{p}{]} \PY{o+ow}{in} \PY{p}{[}\PY{l+s+s1}{\PYZsq{}}\PY{l+s+s1}{c1}\PY{l+s+s1}{\PYZsq{}}\PY{p}{,} \PY{l+s+s1}{\PYZsq{}}\PY{l+s+s1}{c2}\PY{l+s+s1}{\PYZsq{}}\PY{p}{,} \PY{l+s+s1}{\PYZsq{}}\PY{l+s+s1}{c3}\PY{l+s+s1}{\PYZsq{}}\PY{p}{]}\PY{p}{:}
                 \PY{n}{card} \PY{o}{=} \PY{n}{chance\PYZus{}deck}\PY{o}{.}\PY{n}{draw\PYZus{}card}\PY{p}{(}\PY{p}{)}
                 \PY{k}{if} \PY{n}{card} \PY{o+ow}{is} \PY{l+s+s1}{\PYZsq{}}\PY{l+s+s1}{gtg}\PY{l+s+s1}{\PYZsq{}}\PY{p}{:} \PY{n}{index} \PY{o}{=} \PY{n}{squares\PYZus{}labels}\PY{o}{.}\PY{n}{index}\PY{p}{(}\PY{l+s+s1}{\PYZsq{}}\PY{l+s+s1}{start}\PY{l+s+s1}{\PYZsq{}}\PY{p}{)}
                 \PY{k}{if} \PY{n}{card} \PY{o+ow}{is} \PY{l+s+s1}{\PYZsq{}}\PY{l+s+s1}{gtj}\PY{l+s+s1}{\PYZsq{}}\PY{p}{:} \PY{n}{index} \PY{o}{=} \PY{n}{squares\PYZus{}labels}\PY{o}{.}\PY{n}{index}\PY{p}{(}\PY{l+s+s1}{\PYZsq{}}\PY{l+s+s1}{jail}\PY{l+s+s1}{\PYZsq{}}\PY{p}{)}
                 \PY{k}{if} \PY{n}{card} \PY{o+ow}{is} \PY{l+s+s1}{\PYZsq{}}\PY{l+s+s1}{r3}\PY{l+s+s1}{\PYZsq{}}\PY{p}{:} \PY{n}{index} \PY{o}{=} \PY{n}{squares\PYZus{}labels}\PY{o}{.}\PY{n}{index}\PY{p}{(}\PY{l+s+s1}{\PYZsq{}}\PY{l+s+s1}{r3}\PY{l+s+s1}{\PYZsq{}}\PY{p}{)}
                 \PY{k}{if} \PY{n}{card} \PY{o+ow}{is} \PY{l+s+s1}{\PYZsq{}}\PY{l+s+s1}{gb3}\PY{l+s+s1}{\PYZsq{}}\PY{p}{:}
                     \PY{k}{if} \PY{n}{index} \PY{o}{\PYZgt{}}\PY{o}{=} \PY{l+m+mi}{3}\PY{p}{:} \PY{n}{index} \PY{o}{\PYZhy{}}\PY{o}{=} \PY{l+m+mi}{3}
                     \PY{k}{if} \PY{n}{index} \PY{o}{\PYZlt{}} \PY{l+m+mi}{3}\PY{p}{:} \PY{n}{index} \PY{o}{=} \PY{n}{total\PYZus{}squares}\PY{o}{\PYZhy{}}\PY{n+nb}{abs}\PY{p}{(}\PY{n}{index}\PY{o}{\PYZhy{}}\PY{l+m+mi}{3}\PY{p}{)}\PY{o}{\PYZhy{}}\PY{l+m+mi}{1}
\end{Verbatim}


    It takes around \(2.7\) seconds to run a game when \(N=1,000,000\).

    \hypertarget{detailed-probability-analysis}{%
\newpage
\subsection{Detailed probability
analysis}\label{detailed-probability-analysis}}

    Now we can proceed to analyze our results.

    \hypertarget{determining-probabilities}{%
\subsubsection{Determining
probabilities}\label{determining-probabilities}}

    With the number of times that each square is visited we can calculate
the probabilities. The probability that a square is visited is:

\[ P(\bar{x}=x) = \dfrac{\text{Times visited}}{\text{\# of rounds}} \]

    We also want to create a \texttt{DataFrame} in Python to easily keep
track of everything.

    \begin{Verbatim}[commandchars=\\\{\}]
{\color{incolor}In [{\color{incolor}13}]:} \PY{k+kn}{import} \PY{n+nn}{pandas} \PY{k}{as} \PY{n+nn}{pd}
         \PY{n}{df} \PY{o}{=} \PY{n}{pd}\PY{o}{.}\PY{n}{DataFrame}\PY{p}{(}\PY{n}{index}\PY{o}{=}\PY{n+nb}{range}\PY{p}{(}\PY{n}{total\PYZus{}squares}\PY{p}{)}\PY{p}{)}
         \PY{n}{df}\PY{p}{[}\PY{l+s+s1}{\PYZsq{}}\PY{l+s+s1}{Square}\PY{l+s+s1}{\PYZsq{}}\PY{p}{]} \PY{o}{=} \PY{n}{squares\PYZus{}labels}
         \PY{n}{df}\PY{p}{[}\PY{l+s+s1}{\PYZsq{}}\PY{l+s+s1}{Description}\PY{l+s+s1}{\PYZsq{}}\PY{p}{]} \PY{o}{=} \PY{n}{squares\PYZus{}description}
         \PY{n}{df}\PY{p}{[}\PY{l+s+s1}{\PYZsq{}}\PY{l+s+s1}{Purchasable}\PY{l+s+s1}{\PYZsq{}}\PY{p}{]} \PY{o}{=} \PY{n}{squares\PYZus{}purchasable}
         \PY{n}{df}\PY{p}{[}\PY{l+s+s1}{\PYZsq{}}\PY{l+s+s1}{Visited}\PY{l+s+s1}{\PYZsq{}}\PY{p}{]} \PY{o}{=} \PY{n}{squares}
         \PY{n}{df}\PY{p}{[}\PY{l+s+s1}{\PYZsq{}}\PY{l+s+s1}{Probability}\PY{l+s+s1}{\PYZsq{}}\PY{p}{]} \PY{o}{=} \PY{n}{df}\PY{p}{[}\PY{l+s+s1}{\PYZsq{}}\PY{l+s+s1}{Visited}\PY{l+s+s1}{\PYZsq{}}\PY{p}{]} \PY{o}{/} \PY{n}{rounds}
         \PY{n}{df}\PY{p}{[}\PY{l+s+s1}{\PYZsq{}}\PY{l+s+s1}{Aggregate}\PY{l+s+s1}{\PYZsq{}}\PY{p}{]} \PY{o}{=} \PY{n}{squares\PYZus{}aggregate}
\end{Verbatim}


    We can calculate a quick summary about the data:

    \begin{Verbatim}[commandchars=\\\{\}]
{\color{incolor}In [{\color{incolor}14}]:} \PY{n+nb}{print}\PY{p}{(}\PY{l+s+s1}{\PYZsq{}}\PY{l+s+s1}{Total rounds: }\PY{l+s+si}{\PYZob{}\PYZcb{}}\PY{l+s+s1}{\PYZsq{}}\PY{o}{.}\PY{n}{format}\PY{p}{(}\PY{n}{rounds}\PY{p}{)}\PY{p}{)}
         \PY{n+nb}{print}\PY{p}{(}\PY{l+s+s1}{\PYZsq{}}\PY{l+s+s1}{Visited avg:  }\PY{l+s+si}{\PYZob{}\PYZcb{}}\PY{l+s+s1}{\PYZsq{}}\PY{o}{.}\PY{n}{format}\PY{p}{(}\PY{n}{df}\PY{p}{[}\PY{l+s+s1}{\PYZsq{}}\PY{l+s+s1}{Visited}\PY{l+s+s1}{\PYZsq{}}\PY{p}{]}\PY{o}{.}\PY{n}{mean}\PY{p}{(}\PY{p}{)}\PY{p}{)}\PY{p}{)}
         \PY{n+nb}{print}\PY{p}{(}\PY{l+s+s1}{\PYZsq{}}\PY{l+s+s1}{Visited min:  }\PY{l+s+si}{\PYZob{}\PYZcb{}}\PY{l+s+s1}{\PYZsq{}}\PY{o}{.}\PY{n}{format}\PY{p}{(}\PY{n}{df}\PY{p}{[}\PY{l+s+s1}{\PYZsq{}}\PY{l+s+s1}{Visited}\PY{l+s+s1}{\PYZsq{}}\PY{p}{]}\PY{o}{.}\PY{n}{min}\PY{p}{(}\PY{p}{)}\PY{p}{)}\PY{p}{)}
         \PY{n+nb}{print}\PY{p}{(}\PY{l+s+s1}{\PYZsq{}}\PY{l+s+s1}{Visited max:  }\PY{l+s+si}{\PYZob{}\PYZcb{}}\PY{l+s+s1}{\PYZsq{}}\PY{o}{.}\PY{n}{format}\PY{p}{(}\PY{n}{df}\PY{p}{[}\PY{l+s+s1}{\PYZsq{}}\PY{l+s+s1}{Visited}\PY{l+s+s1}{\PYZsq{}}\PY{p}{]}\PY{o}{.}\PY{n}{max}\PY{p}{(}\PY{p}{)}\PY{p}{)}\PY{p}{)}
         \PY{n+nb}{print}\PY{p}{(}\PY{l+s+s1}{\PYZsq{}}\PY{l+s+s1}{Visited std:  }\PY{l+s+si}{\PYZob{}:.2f\PYZcb{}}\PY{l+s+s1}{\PYZsq{}}\PY{o}{.}\PY{n}{format}\PY{p}{(}\PY{n}{df}\PY{p}{[}\PY{l+s+s1}{\PYZsq{}}\PY{l+s+s1}{Visited}\PY{l+s+s1}{\PYZsq{}}\PY{p}{]}\PY{o}{.}\PY{n}{std}\PY{p}{(}\PY{p}{)}\PY{p}{)}\PY{p}{)}
\end{Verbatim}


    \begin{Verbatim}[commandchars=\\\{\}]
Total rounds: 1000000
Visited avg:  25000.0
Visited min:  18140
Visited max:  33538
Visited std:  4644.91

    \end{Verbatim}

    \hypertarget{plot-of-probabilities-by-square}{%

\subsubsection{Plot of probabilities by
square}\label{plot-of-probabilities-by-square}}

    If we sort these values in descending order on the probability, we can easily see
which squares have the highest probability to be visited.

    \begin{Verbatim}[commandchars=\\\{\}]
{\color{incolor}In [{\color{incolor}45}]:} \PY{n}{plt}\PY{o}{.}\PY{n}{rc}\PY{p}{(}\PY{l+s+s1}{\PYZsq{}}\PY{l+s+s1}{xtick}\PY{l+s+s1}{\PYZsq{}}\PY{p}{,} \PY{n}{labelsize}\PY{o}{=}\PY{l+m+mi}{16}\PY{p}{)} 
         \PY{n}{plt}\PY{o}{.}\PY{n}{rc}\PY{p}{(}\PY{l+s+s1}{\PYZsq{}}\PY{l+s+s1}{ytick}\PY{l+s+s1}{\PYZsq{}}\PY{p}{,} \PY{n}{labelsize}\PY{o}{=}\PY{l+m+mi}{14}\PY{p}{)} 
         \PY{n}{df}\PY{p}{[}\PY{p}{[}\PY{l+s+s1}{\PYZsq{}}\PY{l+s+s1}{Description}\PY{l+s+s1}{\PYZsq{}}\PY{p}{,} \PY{l+s+s1}{\PYZsq{}}\PY{l+s+s1}{Probability}\PY{l+s+s1}{\PYZsq{}}\PY{p}{]}\PY{p}{]}\PY{o}{.}\PY{n}{sort\PYZus{}values}\PY{p}{(}\PY{n}{by}\PY{o}{=}\PY{l+s+s1}{\PYZsq{}}\PY{l+s+s1}{Probability}\PY{l+s+s1}{\PYZsq{}}\PY{p}{,} \PY{n}{ascending}\PY{o}{=}\PY{k+kc}{False}\PY{p}{)}\PYZbs{}
             \PY{o}{.}\PY{n}{plot}\PY{p}{(}\PY{n}{kind}\PY{o}{=}\PY{l+s+s1}{\PYZsq{}}\PY{l+s+s1}{bar}\PY{l+s+s1}{\PYZsq{}}\PY{p}{,} \PY{n}{figsize}\PY{o}{=}\PY{p}{(}\PY{l+m+mi}{20}\PY{p}{,}\PY{l+m+mi}{5}\PY{p}{)}\PY{p}{)}
         \PY{n}{plt}\PY{o}{.}\PY{n}{xticks}\PY{p}{(}\PY{n+nb}{range}\PY{p}{(}\PY{n}{total\PYZus{}squares}\PY{p}{)}\PY{p}{,} \PY{n}{df}\PY{p}{[}\PY{p}{[}\PY{l+s+s1}{\PYZsq{}}\PY{l+s+s1}{Description}\PY{l+s+s1}{\PYZsq{}}\PY{p}{,} \PY{l+s+s1}{\PYZsq{}}\PY{l+s+s1}{Probability}\PY{l+s+s1}{\PYZsq{}}\PY{p}{]}\PY{p}{]}
             \PY{o}{.}\PY{n}{sort\PYZus{}values}\PY{p}{(}\PY{n}{by}\PY{o}{=}\PY{l+s+s1}{\PYZsq{}}\PY{l+s+s1}{Probability}\PY{l+s+s1}{\PYZsq{}}\PY{p}{,} \PY{n}{ascending}\PY{o}{=}\PY{k+kc}{False}\PY{p}{)}\PY{p}{[}\PY{l+s+s1}{\PYZsq{}}\PY{l+s+s1}{Description}\PY{l+s+s1}{\PYZsq{}}\PY{p}{]}\PY{p}{)}
         \PY{n}{plt}\PY{o}{.}\PY{n}{ylabel}\PY{p}{(}\PY{l+s+s1}{\PYZsq{}}\PY{l+s+s1}{Probability}\PY{l+s+s1}{\PYZsq{}}\PY{p}{)}
         \PY{n}{plt}\PY{o}{.}\PY{n}{title}\PY{p}{(}\PY{l+s+s1}{\PYZsq{}}\PY{l+s+s1}{Probability by label}\PY{l+s+s1}{\PYZsq{}}\PY{p}{)}\PY{p}{;}
\end{Verbatim}


    \begin{center}
    \adjustimage{max size={0.9\linewidth}{0.9\paperheight}}{output_75_0.png}
    \end{center}
    { \hspace*{\fill} \\}
    
    Here we can conclude that \texttt{Orange\ 1} is the most visited square.
Also notice that \texttt{Orange\ 2} and \texttt{Orange\ 3} are pretty
high. It seems that \texttt{Orange} is the best street to have.

\hypertarget{table-of-probabilities-by-square}{%
\subsubsection{Table of probabilities by
square}\label{table-of-probabilities-by-square}}

Below is the full table with all the squares and their corresponding
values.

    \begin{Verbatim}[commandchars=\\\{\}]
{\color{incolor}In [{\color{incolor}25}]:} \PY{n}{df}\PY{o}{.}\PY{n}{loc}\PY{p}{[}\PY{p}{:}\PY{p}{,} \PY{l+s+s1}{\PYZsq{}}\PY{l+s+s1}{Square}\PY{l+s+s1}{\PYZsq{}}\PY{p}{:}\PY{l+s+s1}{\PYZsq{}}\PY{l+s+s1}{Probability}\PY{l+s+s1}{\PYZsq{}}\PY{p}{]}\PY{o}{.}\PY{n}{sort\PYZus{}values}\PY{p}{(}\PY{n}{by}\PY{o}{=}\PY{l+s+s1}{\PYZsq{}}\PY{l+s+s1}{Probability}\PY{l+s+s1}{\PYZsq{}}\PY{p}{,} \PY{n}{ascending}\PY{o}{=}\PY{k+kc}{False}\PY{p}{)}
\end{Verbatim}


\begin{Verbatim}[commandchars=\\\{\}]
{\color{outcolor}Out[{\color{outcolor}25}]:}    Square        Description  Purchasable  Visited  Probability
         16     o1           Orange 1         True    33538     0.033538
         15    ts2    Train Station 2         True    31676     0.031676
         21     r1              Red 1         True    30453     0.030453
         20      p       Free Parking        False    30063     0.030063
         14     p3           Purple 3         True    30037     0.030037
         22     c2           Chance 2        False    29980     0.029980
         19     o3           Orange 3         True    29722     0.029722
         17    cc2  Community Chest 2        False    29637     0.029637
         30    gtj         Go to Jail        False    29563     0.029563
         29     y3           Yellow 3         True    29470     0.029470
         18     o2           Orange 2         True    29360     0.029360
         25    ts3    Train Station 3         True    29204     0.029204
         26     y1           Yellow 1         True    29178     0.029178
         27     y2           Yellow 2         True    28880     0.028880
         24     r3              Red 3         True    28651     0.028651
         28     ww        Water Works         True    28474     0.028474
         23     r2              Red 2         True    28010     0.028010
         13     p2           Purple 2         True    27773     0.027773
         12     ec   Electric Company         True    26916     0.026916
         11     p1           Purple 1         True    26309     0.026309
         31     g1            Green 1         True    24060     0.024060
         32     g2            Green 2         True    23654     0.023654
         33    cc3  Community Chest 3        False    22656     0.022656
         6     lb1       Light Blue 1         True    21486     0.021486
         4      it         Income Tax        False    20836     0.020836
         5      t1    Train Station 1         True    20824     0.020824
         3      b2            Brown 2         True    20812     0.020812
         34     g3            Green 3         True    20763     0.020763
         1      b1            Brown 1         True    20650     0.020650
         7      c1           Chance 1        False    20565     0.020565
         2     cc1  Community Chest 1        False    20450     0.020450
         37    db1        Dark Blue 1         True    20430     0.020430
         9     lb3       Light Blue 3         True    20238     0.020238
         38     st          Super Tax        False    20028     0.020028
         10   jail               Jail        False    19892     0.019892
         8     lb2       Light Blue 2         True    19715     0.019715
         35    ts4    Train Station 4         True    19699     0.019699
         39    db2        Dark Blue 2         True    19163     0.019163
         0   start              Start        False    19045     0.019045
         36     c3           Chance 3        False    18140     0.018140
\end{Verbatim}
            
    \hypertarget{top-10-highest-probability-squares}{%
\subsubsection{Top 10 highest probability
squares}\label{top-10-highest-probability-squares}}

    The top 10 squares that have the highest probability for a player to
land on are:

    \begin{Verbatim}[commandchars=\\\{\}]
{\color{incolor}In [{\color{incolor}46}]:} \PY{n}{df}\PY{o}{.}\PY{n}{loc}\PY{p}{[}\PY{n}{df}\PY{p}{[}\PY{l+s+s1}{\PYZsq{}}\PY{l+s+s1}{Purchasable}\PY{l+s+s1}{\PYZsq{}}\PY{p}{]} \PY{o}{==} \PY{k+kc}{True}\PY{p}{,} \PY{l+s+s1}{\PYZsq{}}\PY{l+s+s1}{Square}\PY{l+s+s1}{\PYZsq{}}\PY{p}{:}\PY{l+s+s1}{\PYZsq{}}\PY{l+s+s1}{Probability}\PY{l+s+s1}{\PYZsq{}}\PY{p}{]}\PYZbs{}
             \PY{o}{.}\PY{n}{sort\PYZus{}values}\PY{p}{(}\PY{l+s+s1}{\PYZsq{}}\PY{l+s+s1}{Probability}\PY{l+s+s1}{\PYZsq{}}\PY{p}{,} \PY{n}{ascending}\PY{o}{=}\PY{k+kc}{False}\PY{p}{)}\PY{o}{.}\PY{n}{head}\PY{p}{(}\PY{l+m+mi}{10}\PY{p}{)}
\end{Verbatim}


\begin{Verbatim}[commandchars=\\\{\}]
{\color{outcolor}Out[{\color{outcolor}46}]:}    Square      Description  Purchasable  Visited  Probability
         16     o1         Orange 1         True    33538     0.033538
         15    ts2  Train Station 2         True    31676     0.031676
         21     r1            Red 1         True    30453     0.030453
         14     p3         Purple 3         True    30037     0.030037
         19     o3         Orange 3         True    29722     0.029722
         29     y3         Yellow 3         True    29470     0.029470
         18     o2         Orange 2         True    29360     0.029360
         25    ts3  Train Station 3         True    29204     0.029204
         26     y1         Yellow 1         True    29178     0.029178
         27     y2         Yellow 2         True    28880     0.028880
\end{Verbatim}
            
    The total probability for all 10 squares is:

    \begin{Verbatim}[commandchars=\\\{\}]
{\color{incolor}In [{\color{incolor}47}]:} \PY{n}{df}\PY{o}{.}\PY{n}{loc}\PY{p}{[}\PY{n}{df}\PY{p}{[}\PY{l+s+s1}{\PYZsq{}}\PY{l+s+s1}{Purchasable}\PY{l+s+s1}{\PYZsq{}}\PY{p}{]} \PY{o}{==} \PY{k+kc}{True}\PY{p}{]}\PY{o}{.}\PY{n}{sort\PYZus{}values}\PY{p}{(}\PY{l+s+s1}{\PYZsq{}}\PY{l+s+s1}{Probability}\PY{l+s+s1}{\PYZsq{}}\PY{p}{,} \PY{n}{ascending}\PY{o}{=}\PY{k+kc}{False}\PY{p}{)}\PYZbs{}
             \PY{o}{.}\PY{n}{head}\PY{p}{(}\PY{l+m+mi}{10}\PY{p}{)}\PY{p}{[}\PY{l+s+s1}{\PYZsq{}}\PY{l+s+s1}{Probability}\PY{l+s+s1}{\PYZsq{}}\PY{p}{]}\PY{o}{.}\PY{n}{sum}\PY{p}{(}\PY{p}{)}
\end{Verbatim}


\begin{Verbatim}[commandchars=\\\{\}]
{\color{outcolor}Out[{\color{outcolor}47}]:} 0.30151800000000001
\end{Verbatim}
            
    \hypertarget{high-level-probability-analysis}{%
\newpage
\subsection{High-level probability
analysis}\label{high-level-probability-analysis}}

    We want to answer the following questions:

\begin{enumerate}
\def\labelenumi{\arabic{enumi}.}
\tightlist
\item
  What are the best streets to have?
\item
  What is the probability to be in jail?
\item
  What is the probability to draw a card?
\end{enumerate}

    \hypertarget{aggregating-grouping-by}{%
\subsubsection{Aggregating (grouping
by)}\label{aggregating-grouping-by}}

    To find what the probabilities are per street, chance, community chest,
etc., we are going to aggregate the possibilities.

    \begin{Verbatim}[commandchars=\\\{\}]
{\color{incolor}In [{\color{incolor}19}]:} \PY{n}{aggregated\PYZus{}df} \PY{o}{=} \PY{n}{pd}\PY{o}{.}\PY{n}{DataFrame}\PY{p}{(}\PY{n}{df}\PY{o}{.}\PY{n}{groupby}\PY{p}{(}\PY{p}{[}\PY{l+s+s1}{\PYZsq{}}\PY{l+s+s1}{Aggregate}\PY{l+s+s1}{\PYZsq{}}\PY{p}{]}\PY{p}{)}\PY{p}{[}\PY{l+s+s1}{\PYZsq{}}\PY{l+s+s1}{Probability}\PY{l+s+s1}{\PYZsq{}}\PY{p}{]}\PY{o}{.}\PY{n}{sum}\PY{p}{(}\PY{p}{)}\PY{p}{)}\PY{o}{.}\PY{n}{reset\PYZus{}index}\PY{p}{(}\PY{p}{)}
\end{Verbatim}


    Now we plot the aggregated probabilities.

    \begin{Verbatim}[commandchars=\\\{\}]
{\color{incolor}In [{\color{incolor}48}]:} \PY{n}{plt}\PY{o}{.}\PY{n}{rc}\PY{p}{(}\PY{l+s+s1}{\PYZsq{}}\PY{l+s+s1}{xtick}\PY{l+s+s1}{\PYZsq{}}\PY{p}{,} \PY{n}{labelsize}\PY{o}{=}\PY{l+m+mi}{16}\PY{p}{)} 
         \PY{n}{plt}\PY{o}{.}\PY{n}{rc}\PY{p}{(}\PY{l+s+s1}{\PYZsq{}}\PY{l+s+s1}{ytick}\PY{l+s+s1}{\PYZsq{}}\PY{p}{,} \PY{n}{labelsize}\PY{o}{=}\PY{l+m+mi}{14}\PY{p}{)} 
         \PY{n}{aggregated\PYZus{}df}\PY{p}{[}\PY{p}{[}\PY{l+s+s1}{\PYZsq{}}\PY{l+s+s1}{Aggregate}\PY{l+s+s1}{\PYZsq{}}\PY{p}{,} \PY{l+s+s1}{\PYZsq{}}\PY{l+s+s1}{Probability}\PY{l+s+s1}{\PYZsq{}}\PY{p}{]}\PY{p}{]}\PY{o}{.}\PY{n}{sort\PYZus{}values}\PY{p}{(}\PY{n}{by}\PY{o}{=}\PY{l+s+s1}{\PYZsq{}}\PY{l+s+s1}{Probability}\PY{l+s+s1}{\PYZsq{}}\PY{p}{,} \PY{n}{ascending}\PY{o}{=}\PY{k+kc}{False}\PY{p}{)}\PYZbs{}
             \PY{o}{.}\PY{n}{plot}\PY{p}{(}\PY{n}{kind}\PY{o}{=}\PY{l+s+s1}{\PYZsq{}}\PY{l+s+s1}{bar}\PY{l+s+s1}{\PYZsq{}}\PY{p}{,} \PY{n}{figsize}\PY{o}{=}\PY{p}{(}\PY{l+m+mi}{10}\PY{p}{,}\PY{l+m+mi}{5}\PY{p}{)}\PY{p}{)}
         \PY{n}{plt}\PY{o}{.}\PY{n}{xticks}\PY{p}{(}\PY{n+nb}{range}\PY{p}{(}\PY{n+nb}{len}\PY{p}{(}\PY{n}{aggregated\PYZus{}df}\PY{o}{.}\PY{n}{index}\PY{p}{)}\PY{p}{)}\PY{p}{,} \PY{n}{aggregated\PYZus{}df}\PY{p}{[}\PY{p}{[}\PY{l+s+s1}{\PYZsq{}}\PY{l+s+s1}{Aggregate}\PY{l+s+s1}{\PYZsq{}}\PY{p}{,} \PY{l+s+s1}{\PYZsq{}}\PY{l+s+s1}{Probability}\PY{l+s+s1}{\PYZsq{}}\PY{p}{]}\PY{p}{]}
             \PY{o}{.}\PY{n}{sort\PYZus{}values}\PY{p}{(}\PY{n}{by}\PY{o}{=}\PY{l+s+s1}{\PYZsq{}}\PY{l+s+s1}{Probability}\PY{l+s+s1}{\PYZsq{}}\PY{p}{,} \PY{n}{ascending}\PY{o}{=}\PY{k+kc}{False}\PY{p}{)}\PY{p}{[}\PY{l+s+s1}{\PYZsq{}}\PY{l+s+s1}{Aggregate}\PY{l+s+s1}{\PYZsq{}}\PY{p}{]}\PY{p}{)}
         \PY{n}{plt}\PY{o}{.}\PY{n}{ylabel}\PY{p}{(}\PY{l+s+s1}{\PYZsq{}}\PY{l+s+s1}{Probability}\PY{l+s+s1}{\PYZsq{}}\PY{p}{)}
         \PY{n}{plt}\PY{o}{.}\PY{n}{title}\PY{p}{(}\PY{l+s+s1}{\PYZsq{}}\PY{l+s+s1}{Probability by label}\PY{l+s+s1}{\PYZsq{}}\PY{p}{)}\PY{p}{;}
\end{Verbatim}


    \begin{center}
    \adjustimage{max size={0.9\linewidth}{0.9\paperheight}}{output_89_0.png}
    \end{center}
    { \hspace*{\fill} \\}
    
    \hypertarget{overview-of-all-the-probabilities-by-aggregate}{%
\subsubsection{Overview of all the probabilities by
aggregate}\label{overview-of-all-the-probabilities-by-aggregate}}

    A total overview of all the probabilities can be found in the table
below:

    \begin{Verbatim}[commandchars=\\\{\}]
{\color{incolor}In [{\color{incolor}21}]:} \PY{n}{aggregated\PYZus{}df}\PY{o}{.}\PY{n}{sort\PYZus{}values}\PY{p}{(}\PY{l+s+s1}{\PYZsq{}}\PY{l+s+s1}{Probability}\PY{l+s+s1}{\PYZsq{}}\PY{p}{,} \PY{n}{ascending}\PY{o}{=}\PY{k+kc}{False}\PY{p}{)}
\end{Verbatim}


\begin{Verbatim}[commandchars=\\\{\}]
{\color{outcolor}Out[{\color{outcolor}21}]:}            Aggregate  Probability
         16     Train Station     0.101403
         11            Orange     0.092620
         18            Yellow     0.087528
         13               Red     0.087114
         12            Purple     0.084119
         2    Community Chest     0.072743
         1             Chance     0.068685
         7              Green     0.068477
         10        Light Blue     0.061439
         0              Brown     0.041462
         3          Dark Blue     0.039593
         5       Free Parking     0.030063
         6         Go to Jail     0.029563
         17       Water Works     0.028474
         4   Electric Company     0.026916
         8         Income Tax     0.020836
         15         Super Tax     0.020028
         9               Jail     0.019892
         14             Start     0.019045
\end{Verbatim}
            
    \hypertarget{train-station-probabilities}{%
\subsubsection{Train station
probabilities}\label{train-station-probabilities}}

    We can conclude that \texttt{Train\ Station} has the highest probability
to land on. However, we need to take into account that there are four
squares to land on.

    \begin{Verbatim}[commandchars=\\\{\}]
{\color{incolor}In [{\color{incolor}27}]:} \PY{n}{df}\PY{o}{.}\PY{n}{loc}\PY{p}{[}\PY{n}{df}\PY{p}{[}\PY{l+s+s1}{\PYZsq{}}\PY{l+s+s1}{Aggregate}\PY{l+s+s1}{\PYZsq{}}\PY{p}{]} \PY{o}{==} \PY{l+s+s1}{\PYZsq{}}\PY{l+s+s1}{Train Station}\PY{l+s+s1}{\PYZsq{}}\PY{p}{,} \PY{l+s+s1}{\PYZsq{}}\PY{l+s+s1}{Square}\PY{l+s+s1}{\PYZsq{}}\PY{p}{:}\PY{l+s+s1}{\PYZsq{}}\PY{l+s+s1}{Probability}\PY{l+s+s1}{\PYZsq{}}\PY{p}{]}\PY{o}{.}\PY{n}{sort\PYZus{}values}\PY{p}{(}\PY{l+s+s1}{\PYZsq{}}\PY{l+s+s1}{Probability}\PY{l+s+s1}{\PYZsq{}}\PY{p}{,} \PY{n}{ascending}\PY{o}{=}\PY{k+kc}{False}\PY{p}{)}
\end{Verbatim}


\begin{Verbatim}[commandchars=\\\{\}]
{\color{outcolor}Out[{\color{outcolor}27}]:}    Square      Description  Purchasable  Visited  Probability
         15    ts2  Train Station 2         True    31676     0.031676
         25    ts3  Train Station 3         True    29204     0.029204
         5      t1  Train Station 1         True    20824     0.020824
         35    ts4  Train Station 4         True    19699     0.019699
\end{Verbatim}
            
    \hypertarget{probability-to-be-in-jail}{%
\subsubsection{Probability to be in
jail}\label{probability-to-be-in-jail}}

    To find the total probability to be in jail, we need to take into
account that:

\begin{itemize}
\tightlist
\item
  We can land on jail.
\item
  We can land on go to jail.
\item
  There is one community card which sends you to jail.
\item
  There is one chance card which sends you to jail.
\end{itemize}

Each deck has \(16\) cards, therefore the probability to draw go to jail
is \(P(\bar{x}=\text{go to jail})=\dfrac{1}{16}\).

\[P(\bar{x}=\text{in jail}) = P(\bar{x}=\text{jail}) + P(\bar{x}=\text{go to jail}) + \dfrac{1}{16}\left[ P(\bar{x}=\text{community chest}) + P(\bar{x}=\text{chance})\right] \]

    \begin{Verbatim}[commandchars=\\\{\}]
{\color{incolor}In [{\color{incolor}50}]:} \PY{n}{P\PYZus{}jail}           \PY{o}{=} \PY{n+nb}{sum}\PY{p}{(}\PY{n}{aggregated\PYZus{}df}\PY{o}{.}\PY{n}{loc}\PY{p}{[}\PY{n}{aggregated\PYZus{}df}\PY{p}{[}\PY{l+s+s1}{\PYZsq{}}\PY{l+s+s1}{Aggregate}\PY{l+s+s1}{\PYZsq{}}\PY{p}{]} \PY{o}{==} \PY{l+s+s1}{\PYZsq{}}\PY{l+s+s1}{Jail}\PY{l+s+s1}{\PYZsq{}}           \PY{p}{]}\PY{p}{[}\PY{l+s+s1}{\PYZsq{}}\PY{l+s+s1}{Probability}\PY{l+s+s1}{\PYZsq{}}\PY{p}{]}\PY{p}{)}
         \PY{n}{P\PYZus{}go\PYZus{}to\PYZus{}jail}     \PY{o}{=} \PY{n+nb}{sum}\PY{p}{(}\PY{n}{aggregated\PYZus{}df}\PY{o}{.}\PY{n}{loc}\PY{p}{[}\PY{n}{aggregated\PYZus{}df}\PY{p}{[}\PY{l+s+s1}{\PYZsq{}}\PY{l+s+s1}{Aggregate}\PY{l+s+s1}{\PYZsq{}}\PY{p}{]} \PY{o}{==} \PY{l+s+s1}{\PYZsq{}}\PY{l+s+s1}{Go to Jail}\PY{l+s+s1}{\PYZsq{}}     \PY{p}{]}\PY{p}{[}\PY{l+s+s1}{\PYZsq{}}\PY{l+s+s1}{Probability}\PY{l+s+s1}{\PYZsq{}}\PY{p}{]}\PY{p}{)}
         \PY{n}{P\PYZus{}community\PYZus{}card} \PY{o}{=} \PY{n+nb}{sum}\PY{p}{(}\PY{n}{aggregated\PYZus{}df}\PY{o}{.}\PY{n}{loc}\PY{p}{[}\PY{n}{aggregated\PYZus{}df}\PY{p}{[}\PY{l+s+s1}{\PYZsq{}}\PY{l+s+s1}{Aggregate}\PY{l+s+s1}{\PYZsq{}}\PY{p}{]} \PY{o}{==} \PY{l+s+s1}{\PYZsq{}}\PY{l+s+s1}{Community Chest}\PY{l+s+s1}{\PYZsq{}}\PY{p}{]}\PY{p}{[}\PY{l+s+s1}{\PYZsq{}}\PY{l+s+s1}{Probability}\PY{l+s+s1}{\PYZsq{}}\PY{p}{]}\PY{p}{)}
         \PY{n}{P\PYZus{}chance\PYZus{}card}    \PY{o}{=} \PY{n+nb}{sum}\PY{p}{(}\PY{n}{aggregated\PYZus{}df}\PY{o}{.}\PY{n}{loc}\PY{p}{[}\PY{n}{aggregated\PYZus{}df}\PY{p}{[}\PY{l+s+s1}{\PYZsq{}}\PY{l+s+s1}{Aggregate}\PY{l+s+s1}{\PYZsq{}}\PY{p}{]} \PY{o}{==} \PY{l+s+s1}{\PYZsq{}}\PY{l+s+s1}{Chance}\PY{l+s+s1}{\PYZsq{}}         \PY{p}{]}\PY{p}{[}\PY{l+s+s1}{\PYZsq{}}\PY{l+s+s1}{Probability}\PY{l+s+s1}{\PYZsq{}}\PY{p}{]}\PY{p}{)}
\end{Verbatim}


    \begin{Verbatim}[commandchars=\\\{\}]
{\color{incolor}In [{\color{incolor}51}]:} \PY{n}{P\PYZus{}jail} \PY{o}{+} \PY{n}{P\PYZus{}go\PYZus{}to\PYZus{}jail} \PY{o}{+} \PY{l+m+mi}{1}\PY{o}{/}\PY{l+m+mi}{16} \PY{o}{*} \PY{p}{(}\PY{n}{P\PYZus{}community\PYZus{}card} \PY{o}{+} \PY{n}{P\PYZus{}chance\PYZus{}card}\PY{p}{)}
\end{Verbatim}


\begin{Verbatim}[commandchars=\\\{\}]
{\color{outcolor}Out[{\color{outcolor}51}]:} 0.058294249999999999
\end{Verbatim}
            
    \hypertarget{probability-to-draw-a-card}{%
\subsubsection{Probability to draw a
card}\label{probability-to-draw-a-card}}

    To find the probability to draw a card, we simply calculate:

\[ P(\bar{x}=\text{draw a card}) = P(\bar{x}=\text{community chest}) + P(\bar{x}=\text{chance})\]

    \begin{Verbatim}[commandchars=\\\{\}]
{\color{incolor}In [{\color{incolor}52}]:} \PY{n}{P\PYZus{}community\PYZus{}card} \PY{o}{+} \PY{n}{P\PYZus{}chance\PYZus{}card}
\end{Verbatim}


\begin{Verbatim}[commandchars=\\\{\}]
{\color{outcolor}Out[{\color{outcolor}52}]:} 0.141428
\end{Verbatim}
            
    Where the probabilities for the community chest square are:

    \begin{Verbatim}[commandchars=\\\{\}]
{\color{incolor}In [{\color{incolor}53}]:} \PY{n}{df}\PY{o}{.}\PY{n}{loc}\PY{p}{[}\PY{n}{df}\PY{p}{[}\PY{l+s+s1}{\PYZsq{}}\PY{l+s+s1}{Aggregate}\PY{l+s+s1}{\PYZsq{}}\PY{p}{]} \PY{o}{==} \PY{l+s+s1}{\PYZsq{}}\PY{l+s+s1}{Community Chest}\PY{l+s+s1}{\PYZsq{}}\PY{p}{,} \PY{l+s+s1}{\PYZsq{}}\PY{l+s+s1}{Square}\PY{l+s+s1}{\PYZsq{}}\PY{p}{:}\PY{l+s+s1}{\PYZsq{}}\PY{l+s+s1}{Probability}\PY{l+s+s1}{\PYZsq{}}\PY{p}{]}\PY{o}{.}\PY{n}{sort\PYZus{}values}\PY{p}{(}\PY{l+s+s1}{\PYZsq{}}\PY{l+s+s1}{Probability}\PY{l+s+s1}{\PYZsq{}}\PY{p}{,} \PY{n}{ascending}\PY{o}{=}\PY{k+kc}{False}\PY{p}{)}
\end{Verbatim}


\begin{Verbatim}[commandchars=\\\{\}]
{\color{outcolor}Out[{\color{outcolor}53}]:}    Square        Description  Purchasable  Visited  Probability
         17    cc2  Community Chest 2        False    29637     0.029637
         33    cc3  Community Chest 3        False    22656     0.022656
         2     cc1  Community Chest 1        False    20450     0.020450
\end{Verbatim}
            
    And the probabilities for the chance square are:

    \begin{Verbatim}[commandchars=\\\{\}]
{\color{incolor}In [{\color{incolor}54}]:} \PY{n}{df}\PY{o}{.}\PY{n}{loc}\PY{p}{[}\PY{n}{df}\PY{p}{[}\PY{l+s+s1}{\PYZsq{}}\PY{l+s+s1}{Aggregate}\PY{l+s+s1}{\PYZsq{}}\PY{p}{]} \PY{o}{==} \PY{l+s+s1}{\PYZsq{}}\PY{l+s+s1}{Chance}\PY{l+s+s1}{\PYZsq{}}\PY{p}{,} \PY{l+s+s1}{\PYZsq{}}\PY{l+s+s1}{Square}\PY{l+s+s1}{\PYZsq{}}\PY{p}{:}\PY{l+s+s1}{\PYZsq{}}\PY{l+s+s1}{Probability}\PY{l+s+s1}{\PYZsq{}}\PY{p}{]}\PY{o}{.}\PY{n}{sort\PYZus{}values}\PY{p}{(}\PY{l+s+s1}{\PYZsq{}}\PY{l+s+s1}{Probability}\PY{l+s+s1}{\PYZsq{}}\PY{p}{,} \PY{n}{ascending}\PY{o}{=}\PY{k+kc}{False}\PY{p}{)}
\end{Verbatim}


\begin{Verbatim}[commandchars=\\\{\}]
{\color{outcolor}Out[{\color{outcolor}54}]:}    Square Description  Purchasable  Visited  Probability
         22     c2    Chance 2        False    29980     0.029980
         7      c1    Chance 1        False    20565     0.020565
         36     c3    Chance 3        False    18140     0.018140
\end{Verbatim}
            

    % Add a bibliography block to the postdoc
    
    
    
    \end{document}
