
% Default to the notebook output style

    


% Inherit from the specified cell style.




    
\documentclass[11pt]{article}

    
    
    \usepackage[T1]{fontenc}
    % Nicer default font (+ math font) than Computer Modern for most use cases
    \usepackage{mathpazo}

    % Basic figure setup, for now with no caption control since it's done
    % automatically by Pandoc (which extracts ![](path) syntax from Markdown).
    \usepackage{graphicx}
    % We will generate all images so they have a width \maxwidth. This means
    % that they will get their normal width if they fit onto the page, but
    % are scaled down if they would overflow the margins.
    \makeatletter
    \def\maxwidth{\ifdim\Gin@nat@width>\linewidth\linewidth
    \else\Gin@nat@width\fi}
    \makeatother
    \let\Oldincludegraphics\includegraphics
    % Set max figure width to be 80% of text width, for now hardcoded.
    \renewcommand{\includegraphics}[1]{\Oldincludegraphics[width=.8\maxwidth]{#1}}
    % Ensure that by default, figures have no caption (until we provide a
    % proper Figure object with a Caption API and a way to capture that
    % in the conversion process - todo).
    \usepackage{caption}
    \DeclareCaptionLabelFormat{nolabel}{}
    \captionsetup{labelformat=nolabel}

    \usepackage{adjustbox} % Used to constrain images to a maximum size 
    \usepackage{xcolor} % Allow colors to be defined
    \usepackage{enumerate} % Needed for markdown enumerations to work
    \usepackage{geometry} % Used to adjust the document margins
    \usepackage{amsmath} % Equations
    \usepackage{amssymb} % Equations
    \usepackage{textcomp} % defines textquotesingle
    % Hack from http://tex.stackexchange.com/a/47451/13684:
    \AtBeginDocument{%
        \def\PYZsq{\textquotesingle}% Upright quotes in Pygmentized code
    }
    \usepackage{upquote} % Upright quotes for verbatim code
    \usepackage{eurosym} % defines \euro
    \usepackage[mathletters]{ucs} % Extended unicode (utf-8) support
    \usepackage[utf8x]{inputenc} % Allow utf-8 characters in the tex document
    \usepackage{fancyvrb} % verbatim replacement that allows latex
    \usepackage{grffile} % extends the file name processing of package graphics 
                         % to support a larger range 
    % The hyperref package gives us a pdf with properly built
    % internal navigation ('pdf bookmarks' for the table of contents,
    % internal cross-reference links, web links for URLs, etc.)
    \usepackage{hyperref}
    \usepackage{longtable} % longtable support required by pandoc >1.10
    \usepackage{booktabs}  % table support for pandoc > 1.12.2
    \usepackage[inline]{enumitem} % IRkernel/repr support (it uses the enumerate* environment)
    \usepackage[normalem]{ulem} % ulem is needed to support strikethroughs (\sout)
                                % normalem makes italics be italics, not underlines
    

    
    
    % Colors for the hyperref package
    \definecolor{urlcolor}{rgb}{0,.145,.698}
    \definecolor{linkcolor}{rgb}{.71,0.21,0.01}
    \definecolor{citecolor}{rgb}{.12,.54,.11}

    % ANSI colors
    \definecolor{ansi-black}{HTML}{3E424D}
    \definecolor{ansi-black-intense}{HTML}{282C36}
    \definecolor{ansi-red}{HTML}{E75C58}
    \definecolor{ansi-red-intense}{HTML}{B22B31}
    \definecolor{ansi-green}{HTML}{00A250}
    \definecolor{ansi-green-intense}{HTML}{007427}
    \definecolor{ansi-yellow}{HTML}{DDB62B}
    \definecolor{ansi-yellow-intense}{HTML}{B27D12}
    \definecolor{ansi-blue}{HTML}{208FFB}
    \definecolor{ansi-blue-intense}{HTML}{0065CA}
    \definecolor{ansi-magenta}{HTML}{D160C4}
    \definecolor{ansi-magenta-intense}{HTML}{A03196}
    \definecolor{ansi-cyan}{HTML}{60C6C8}
    \definecolor{ansi-cyan-intense}{HTML}{258F8F}
    \definecolor{ansi-white}{HTML}{C5C1B4}
    \definecolor{ansi-white-intense}{HTML}{A1A6B2}

    % commands and environments needed by pandoc snippets
    % extracted from the output of `pandoc -s`
    \providecommand{\tightlist}{%
      \setlength{\itemsep}{0pt}\setlength{\parskip}{0pt}}
    \DefineVerbatimEnvironment{Highlighting}{Verbatim}{commandchars=\\\{\}}
    % Add ',fontsize=\small' for more characters per line
    \newenvironment{Shaded}{}{}
    \newcommand{\KeywordTok}[1]{\textcolor[rgb]{0.00,0.44,0.13}{\textbf{{#1}}}}
    \newcommand{\DataTypeTok}[1]{\textcolor[rgb]{0.56,0.13,0.00}{{#1}}}
    \newcommand{\DecValTok}[1]{\textcolor[rgb]{0.25,0.63,0.44}{{#1}}}
    \newcommand{\BaseNTok}[1]{\textcolor[rgb]{0.25,0.63,0.44}{{#1}}}
    \newcommand{\FloatTok}[1]{\textcolor[rgb]{0.25,0.63,0.44}{{#1}}}
    \newcommand{\CharTok}[1]{\textcolor[rgb]{0.25,0.44,0.63}{{#1}}}
    \newcommand{\StringTok}[1]{\textcolor[rgb]{0.25,0.44,0.63}{{#1}}}
    \newcommand{\CommentTok}[1]{\textcolor[rgb]{0.38,0.63,0.69}{\textit{{#1}}}}
    \newcommand{\OtherTok}[1]{\textcolor[rgb]{0.00,0.44,0.13}{{#1}}}
    \newcommand{\AlertTok}[1]{\textcolor[rgb]{1.00,0.00,0.00}{\textbf{{#1}}}}
    \newcommand{\FunctionTok}[1]{\textcolor[rgb]{0.02,0.16,0.49}{{#1}}}
    \newcommand{\RegionMarkerTok}[1]{{#1}}
    \newcommand{\ErrorTok}[1]{\textcolor[rgb]{1.00,0.00,0.00}{\textbf{{#1}}}}
    \newcommand{\NormalTok}[1]{{#1}}
    
    % Additional commands for more recent versions of Pandoc
    \newcommand{\ConstantTok}[1]{\textcolor[rgb]{0.53,0.00,0.00}{{#1}}}
    \newcommand{\SpecialCharTok}[1]{\textcolor[rgb]{0.25,0.44,0.63}{{#1}}}
    \newcommand{\VerbatimStringTok}[1]{\textcolor[rgb]{0.25,0.44,0.63}{{#1}}}
    \newcommand{\SpecialStringTok}[1]{\textcolor[rgb]{0.73,0.40,0.53}{{#1}}}
    \newcommand{\ImportTok}[1]{{#1}}
    \newcommand{\DocumentationTok}[1]{\textcolor[rgb]{0.73,0.13,0.13}{\textit{{#1}}}}
    \newcommand{\AnnotationTok}[1]{\textcolor[rgb]{0.38,0.63,0.69}{\textbf{\textit{{#1}}}}}
    \newcommand{\CommentVarTok}[1]{\textcolor[rgb]{0.38,0.63,0.69}{\textbf{\textit{{#1}}}}}
    \newcommand{\VariableTok}[1]{\textcolor[rgb]{0.10,0.09,0.49}{{#1}}}
    \newcommand{\ControlFlowTok}[1]{\textcolor[rgb]{0.00,0.44,0.13}{\textbf{{#1}}}}
    \newcommand{\OperatorTok}[1]{\textcolor[rgb]{0.40,0.40,0.40}{{#1}}}
    \newcommand{\BuiltInTok}[1]{{#1}}
    \newcommand{\ExtensionTok}[1]{{#1}}
    \newcommand{\PreprocessorTok}[1]{\textcolor[rgb]{0.74,0.48,0.00}{{#1}}}
    \newcommand{\AttributeTok}[1]{\textcolor[rgb]{0.49,0.56,0.16}{{#1}}}
    \newcommand{\InformationTok}[1]{\textcolor[rgb]{0.38,0.63,0.69}{\textbf{\textit{{#1}}}}}
    \newcommand{\WarningTok}[1]{\textcolor[rgb]{0.38,0.63,0.69}{\textbf{\textit{{#1}}}}}
    
    
    % Define a nice break command that doesn't care if a line doesn't already
    % exist.
    \def\br{\hspace*{\fill} \\* }
    % Math Jax compatability definitions
    \def\gt{>}
    \def\lt{<}
    % Document parameters
    \title{Matrixrekening}
    
    
    

    % Pygments definitions
    
\makeatletter
\def\PY@reset{\let\PY@it=\relax \let\PY@bf=\relax%
    \let\PY@ul=\relax \let\PY@tc=\relax%
    \let\PY@bc=\relax \let\PY@ff=\relax}
\def\PY@tok#1{\csname PY@tok@#1\endcsname}
\def\PY@toks#1+{\ifx\relax#1\empty\else%
    \PY@tok{#1}\expandafter\PY@toks\fi}
\def\PY@do#1{\PY@bc{\PY@tc{\PY@ul{%
    \PY@it{\PY@bf{\PY@ff{#1}}}}}}}
\def\PY#1#2{\PY@reset\PY@toks#1+\relax+\PY@do{#2}}

\expandafter\def\csname PY@tok@w\endcsname{\def\PY@tc##1{\textcolor[rgb]{0.73,0.73,0.73}{##1}}}
\expandafter\def\csname PY@tok@c\endcsname{\let\PY@it=\textit\def\PY@tc##1{\textcolor[rgb]{0.25,0.50,0.50}{##1}}}
\expandafter\def\csname PY@tok@cp\endcsname{\def\PY@tc##1{\textcolor[rgb]{0.74,0.48,0.00}{##1}}}
\expandafter\def\csname PY@tok@k\endcsname{\let\PY@bf=\textbf\def\PY@tc##1{\textcolor[rgb]{0.00,0.50,0.00}{##1}}}
\expandafter\def\csname PY@tok@kp\endcsname{\def\PY@tc##1{\textcolor[rgb]{0.00,0.50,0.00}{##1}}}
\expandafter\def\csname PY@tok@kt\endcsname{\def\PY@tc##1{\textcolor[rgb]{0.69,0.00,0.25}{##1}}}
\expandafter\def\csname PY@tok@o\endcsname{\def\PY@tc##1{\textcolor[rgb]{0.40,0.40,0.40}{##1}}}
\expandafter\def\csname PY@tok@ow\endcsname{\let\PY@bf=\textbf\def\PY@tc##1{\textcolor[rgb]{0.67,0.13,1.00}{##1}}}
\expandafter\def\csname PY@tok@nb\endcsname{\def\PY@tc##1{\textcolor[rgb]{0.00,0.50,0.00}{##1}}}
\expandafter\def\csname PY@tok@nf\endcsname{\def\PY@tc##1{\textcolor[rgb]{0.00,0.00,1.00}{##1}}}
\expandafter\def\csname PY@tok@nc\endcsname{\let\PY@bf=\textbf\def\PY@tc##1{\textcolor[rgb]{0.00,0.00,1.00}{##1}}}
\expandafter\def\csname PY@tok@nn\endcsname{\let\PY@bf=\textbf\def\PY@tc##1{\textcolor[rgb]{0.00,0.00,1.00}{##1}}}
\expandafter\def\csname PY@tok@ne\endcsname{\let\PY@bf=\textbf\def\PY@tc##1{\textcolor[rgb]{0.82,0.25,0.23}{##1}}}
\expandafter\def\csname PY@tok@nv\endcsname{\def\PY@tc##1{\textcolor[rgb]{0.10,0.09,0.49}{##1}}}
\expandafter\def\csname PY@tok@no\endcsname{\def\PY@tc##1{\textcolor[rgb]{0.53,0.00,0.00}{##1}}}
\expandafter\def\csname PY@tok@nl\endcsname{\def\PY@tc##1{\textcolor[rgb]{0.63,0.63,0.00}{##1}}}
\expandafter\def\csname PY@tok@ni\endcsname{\let\PY@bf=\textbf\def\PY@tc##1{\textcolor[rgb]{0.60,0.60,0.60}{##1}}}
\expandafter\def\csname PY@tok@na\endcsname{\def\PY@tc##1{\textcolor[rgb]{0.49,0.56,0.16}{##1}}}
\expandafter\def\csname PY@tok@nt\endcsname{\let\PY@bf=\textbf\def\PY@tc##1{\textcolor[rgb]{0.00,0.50,0.00}{##1}}}
\expandafter\def\csname PY@tok@nd\endcsname{\def\PY@tc##1{\textcolor[rgb]{0.67,0.13,1.00}{##1}}}
\expandafter\def\csname PY@tok@s\endcsname{\def\PY@tc##1{\textcolor[rgb]{0.73,0.13,0.13}{##1}}}
\expandafter\def\csname PY@tok@sd\endcsname{\let\PY@it=\textit\def\PY@tc##1{\textcolor[rgb]{0.73,0.13,0.13}{##1}}}
\expandafter\def\csname PY@tok@si\endcsname{\let\PY@bf=\textbf\def\PY@tc##1{\textcolor[rgb]{0.73,0.40,0.53}{##1}}}
\expandafter\def\csname PY@tok@se\endcsname{\let\PY@bf=\textbf\def\PY@tc##1{\textcolor[rgb]{0.73,0.40,0.13}{##1}}}
\expandafter\def\csname PY@tok@sr\endcsname{\def\PY@tc##1{\textcolor[rgb]{0.73,0.40,0.53}{##1}}}
\expandafter\def\csname PY@tok@ss\endcsname{\def\PY@tc##1{\textcolor[rgb]{0.10,0.09,0.49}{##1}}}
\expandafter\def\csname PY@tok@sx\endcsname{\def\PY@tc##1{\textcolor[rgb]{0.00,0.50,0.00}{##1}}}
\expandafter\def\csname PY@tok@m\endcsname{\def\PY@tc##1{\textcolor[rgb]{0.40,0.40,0.40}{##1}}}
\expandafter\def\csname PY@tok@gh\endcsname{\let\PY@bf=\textbf\def\PY@tc##1{\textcolor[rgb]{0.00,0.00,0.50}{##1}}}
\expandafter\def\csname PY@tok@gu\endcsname{\let\PY@bf=\textbf\def\PY@tc##1{\textcolor[rgb]{0.50,0.00,0.50}{##1}}}
\expandafter\def\csname PY@tok@gd\endcsname{\def\PY@tc##1{\textcolor[rgb]{0.63,0.00,0.00}{##1}}}
\expandafter\def\csname PY@tok@gi\endcsname{\def\PY@tc##1{\textcolor[rgb]{0.00,0.63,0.00}{##1}}}
\expandafter\def\csname PY@tok@gr\endcsname{\def\PY@tc##1{\textcolor[rgb]{1.00,0.00,0.00}{##1}}}
\expandafter\def\csname PY@tok@ge\endcsname{\let\PY@it=\textit}
\expandafter\def\csname PY@tok@gs\endcsname{\let\PY@bf=\textbf}
\expandafter\def\csname PY@tok@gp\endcsname{\let\PY@bf=\textbf\def\PY@tc##1{\textcolor[rgb]{0.00,0.00,0.50}{##1}}}
\expandafter\def\csname PY@tok@go\endcsname{\def\PY@tc##1{\textcolor[rgb]{0.53,0.53,0.53}{##1}}}
\expandafter\def\csname PY@tok@gt\endcsname{\def\PY@tc##1{\textcolor[rgb]{0.00,0.27,0.87}{##1}}}
\expandafter\def\csname PY@tok@err\endcsname{\def\PY@bc##1{\setlength{\fboxsep}{0pt}\fcolorbox[rgb]{1.00,0.00,0.00}{1,1,1}{\strut ##1}}}
\expandafter\def\csname PY@tok@kc\endcsname{\let\PY@bf=\textbf\def\PY@tc##1{\textcolor[rgb]{0.00,0.50,0.00}{##1}}}
\expandafter\def\csname PY@tok@kd\endcsname{\let\PY@bf=\textbf\def\PY@tc##1{\textcolor[rgb]{0.00,0.50,0.00}{##1}}}
\expandafter\def\csname PY@tok@kn\endcsname{\let\PY@bf=\textbf\def\PY@tc##1{\textcolor[rgb]{0.00,0.50,0.00}{##1}}}
\expandafter\def\csname PY@tok@kr\endcsname{\let\PY@bf=\textbf\def\PY@tc##1{\textcolor[rgb]{0.00,0.50,0.00}{##1}}}
\expandafter\def\csname PY@tok@bp\endcsname{\def\PY@tc##1{\textcolor[rgb]{0.00,0.50,0.00}{##1}}}
\expandafter\def\csname PY@tok@fm\endcsname{\def\PY@tc##1{\textcolor[rgb]{0.00,0.00,1.00}{##1}}}
\expandafter\def\csname PY@tok@vc\endcsname{\def\PY@tc##1{\textcolor[rgb]{0.10,0.09,0.49}{##1}}}
\expandafter\def\csname PY@tok@vg\endcsname{\def\PY@tc##1{\textcolor[rgb]{0.10,0.09,0.49}{##1}}}
\expandafter\def\csname PY@tok@vi\endcsname{\def\PY@tc##1{\textcolor[rgb]{0.10,0.09,0.49}{##1}}}
\expandafter\def\csname PY@tok@vm\endcsname{\def\PY@tc##1{\textcolor[rgb]{0.10,0.09,0.49}{##1}}}
\expandafter\def\csname PY@tok@sa\endcsname{\def\PY@tc##1{\textcolor[rgb]{0.73,0.13,0.13}{##1}}}
\expandafter\def\csname PY@tok@sb\endcsname{\def\PY@tc##1{\textcolor[rgb]{0.73,0.13,0.13}{##1}}}
\expandafter\def\csname PY@tok@sc\endcsname{\def\PY@tc##1{\textcolor[rgb]{0.73,0.13,0.13}{##1}}}
\expandafter\def\csname PY@tok@dl\endcsname{\def\PY@tc##1{\textcolor[rgb]{0.73,0.13,0.13}{##1}}}
\expandafter\def\csname PY@tok@s2\endcsname{\def\PY@tc##1{\textcolor[rgb]{0.73,0.13,0.13}{##1}}}
\expandafter\def\csname PY@tok@sh\endcsname{\def\PY@tc##1{\textcolor[rgb]{0.73,0.13,0.13}{##1}}}
\expandafter\def\csname PY@tok@s1\endcsname{\def\PY@tc##1{\textcolor[rgb]{0.73,0.13,0.13}{##1}}}
\expandafter\def\csname PY@tok@mb\endcsname{\def\PY@tc##1{\textcolor[rgb]{0.40,0.40,0.40}{##1}}}
\expandafter\def\csname PY@tok@mf\endcsname{\def\PY@tc##1{\textcolor[rgb]{0.40,0.40,0.40}{##1}}}
\expandafter\def\csname PY@tok@mh\endcsname{\def\PY@tc##1{\textcolor[rgb]{0.40,0.40,0.40}{##1}}}
\expandafter\def\csname PY@tok@mi\endcsname{\def\PY@tc##1{\textcolor[rgb]{0.40,0.40,0.40}{##1}}}
\expandafter\def\csname PY@tok@il\endcsname{\def\PY@tc##1{\textcolor[rgb]{0.40,0.40,0.40}{##1}}}
\expandafter\def\csname PY@tok@mo\endcsname{\def\PY@tc##1{\textcolor[rgb]{0.40,0.40,0.40}{##1}}}
\expandafter\def\csname PY@tok@ch\endcsname{\let\PY@it=\textit\def\PY@tc##1{\textcolor[rgb]{0.25,0.50,0.50}{##1}}}
\expandafter\def\csname PY@tok@cm\endcsname{\let\PY@it=\textit\def\PY@tc##1{\textcolor[rgb]{0.25,0.50,0.50}{##1}}}
\expandafter\def\csname PY@tok@cpf\endcsname{\let\PY@it=\textit\def\PY@tc##1{\textcolor[rgb]{0.25,0.50,0.50}{##1}}}
\expandafter\def\csname PY@tok@c1\endcsname{\let\PY@it=\textit\def\PY@tc##1{\textcolor[rgb]{0.25,0.50,0.50}{##1}}}
\expandafter\def\csname PY@tok@cs\endcsname{\let\PY@it=\textit\def\PY@tc##1{\textcolor[rgb]{0.25,0.50,0.50}{##1}}}

\def\PYZbs{\char`\\}
\def\PYZus{\char`\_}
\def\PYZob{\char`\{}
\def\PYZcb{\char`\}}
\def\PYZca{\char`\^}
\def\PYZam{\char`\&}
\def\PYZlt{\char`\<}
\def\PYZgt{\char`\>}
\def\PYZsh{\char`\#}
\def\PYZpc{\char`\%}
\def\PYZdl{\char`\$}
\def\PYZhy{\char`\-}
\def\PYZsq{\char`\'}
\def\PYZdq{\char`\"}
\def\PYZti{\char`\~}
% for compatibility with earlier versions
\def\PYZat{@}
\def\PYZlb{[}
\def\PYZrb{]}
\makeatother


    % Exact colors from NB
    \definecolor{incolor}{rgb}{0.0, 0.0, 0.5}
    \definecolor{outcolor}{rgb}{0.545, 0.0, 0.0}



    
    % Prevent overflowing lines due to hard-to-break entities
    \sloppy 
    % Setup hyperref package
    \hypersetup{
      breaklinks=true,  % so long urls are correctly broken across lines
      colorlinks=true,
      urlcolor=urlcolor,
      linkcolor=linkcolor,
      citecolor=citecolor,
      }
    % Slightly bigger margins than the latex defaults
    
    \geometry{verbose,tmargin=1in,bmargin=1in,lmargin=1in,rmargin=1in}
    
    

    \begin{document}
    
    
    \maketitle
    
    

    
    \hypertarget{matrixrekening}{%
\section{Matrixrekening}\label{matrixrekening}}

    \hypertarget{week-1}{%
\subsection{Week 1}\label{week-1}}

    \hypertarget{matrix-definitie}{%
\paragraph{Matrix definitie}\label{matrix-definitie}}

    Een matrix wordt aangeduidt met:

    \(A = \begin{pmatrix} 1 & 2 \\ 3 & 4 \end{pmatrix} \qquad B = \begin{pmatrix} -3 & \frac{2}{5} & \frac{1}{2} \\ 0 & -10 & 2 \end{pmatrix} \qquad C = \begin{pmatrix} 2 \\ -1 \\ 8 \end{pmatrix} \qquad D = \begin{pmatrix}1 & 3 & 4 \end{pmatrix}\)

    De afmeting van de matrix zijn als volgt:

\begin{itemize}
\tightlist
\item
  \(A = 2 \times 2\)
\item
  \(B = 2 \times 3\)
\item
  \(C = 1 \times 3\)
\end{itemize}

Hier tellen we eerst de \emph{rijen} en daarna de \emph{kolommen}. Als
het aantal rijen en kolommen gelijk is dan spreken we van een
\emph{vierkante} matrix. Als een matrix \(1\) rij (of kolom) heeft zoals
matrix \(C\) dan spreken we van een \emph{vector}. \(C\) is een
\emph{kolomvector}. Indien er maar \(1\) rij zoals in \(D\) is dan
spreken we van een \emph{rijvector}.

De verschillende \emph{elementen} in de matrix worden op de volgende
manier benoemd:

    \(A = \begin{pmatrix} a_{11} & a_{12} & \ldots & a_{1n} \\ a_{21} & \ddots && \vdots \\ \vdots && \ddots & \vdots \\ a_{m1} & a_{m2} & \ldots & a_{mn} \end{pmatrix}\)

    Met deze methode kunnen we de elementen in de onderstaande
\(3 \times 2\) matrix op de volgende manier noteren:

\(L=\begin{pmatrix} 7 & 1 \\ 8 & 0 \\ 3 & -4 \end{pmatrix}\)

\begin{itemize}
\tightlist
\item
  \(l_{11} = 7\)
\item
  \(l_{12} = 1\)
\item
  \(l_{21} = 8\)
\item
  \(l_{22} = 0\)
\item
  \(l_{31} = 3\)
\item
  \(l_{32} = -4\)
\end{itemize}

    \hypertarget{vierkante-matrices}{%
\paragraph{Vierkante matrices}\label{vierkante-matrices}}

    Zoals al eerder is genoemd, is een \(n \times n\) matrix een
\emph{vierkante} matrix. De hoofddiagonaal in een vierkante matrix is
hieronder aangegeven:

\(\begin{pmatrix} \color{red}{1} & 2 & 3 \\ 4 & \color{red}{5} & 6 \\ 7 & 8 & \color{red}{9} \end{pmatrix}\)

De hoofddiagonaal bestaat uit de elementen
\(a_{11}, a_{22}, a_{33}, \ldots, a_{nn}\).

    \hypertarget{nul-matrix}{%
\paragraph{Nul-matrix}\label{nul-matrix}}

    Een matrix waarvan alle elementen de waarde \(0\) bevatten wordt de
\emph{nulmatrix} genoemd. De afmetingen maken hierbij niet uit.

\(\begin{pmatrix}0 & 0 & 0 & 0 \\ 0 & 0 & 0 & 0\end{pmatrix}\)

    \hypertarget{symmetrische-matrix}{%
\paragraph{Symmetrische matrix}\label{symmetrische-matrix}}

    Als een matrix symmetrisch is over de hoofddiagonaal dan wordt er
gesproken van een \emph{symmetrische matrix}.

\(\begin{pmatrix}1 & 3 & \frac{2}{3} \\ 3 & 5 & 9 \\ \frac{2}{3} & 9 & 9 \end{pmatrix}\)

Hiervoor geldt dat \(a_{ij} = a_{ji}\). En vinden we de eigenschap
\(a_{ij} = a_{ji} \forall i,j\).

    \hypertarget{transponeren}{%
\paragraph{Transponeren}\label{transponeren}}

    Als we de matrix \(A\) \emph{transponeren} krijgen we een
getransponeerde matrix, aangeduidt met \(A^T\):

\(A = \begin{pmatrix} 1 & 2 & 3 \\ 4 & 5 & 6 \end{pmatrix} \qquad A^T = \begin{pmatrix} 1 & 4 \\ 2 & 5 \\ 3 & 6 \end{pmatrix}\).

Hierbij geldt dat \(m \times n \rightarrow n \times m\) en
\(a_{ij} \rightarrow a_{ji}\).

    \hypertarget{section}{%
\paragraph{\texorpdfstring{(\(+\), \(-\))}{(+, -)}}\label{section}}

    Om matrices bij elkaar op te tellen \textbf{moeten} deze dezelfde
afmetingen hebben, indien dit niet zo is dan is deze \emph{niet
gedefinieerd}.

\(A=\begin{pmatrix} 1 & 3 & 7 \\ 2 & 8 & 6 \end{pmatrix} \qquad B=\begin{pmatrix} 4 & 3 & 1 \\ 9 & 2 & 5 \end{pmatrix} \qquad A+B=\begin{pmatrix} 5 & 6 & 8 \\ 11 & 10 & 11 \end{pmatrix}\)

    \hypertarget{times-div}{%
\paragraph{\texorpdfstring{(\(\times\),
\(\div\))}{(\textbackslash{}times, \textbackslash{}div)}}\label{times-div}}

    Een matrix kunnen we vermenigvuldigen met een getal. Dit wordt een
\emph{scalaire} matrix vermenigvuldigen genoemd. Het getal \(2\) is een
\emph{scalar}.

\$A =

\begin{pmatrix} 1 & 2 \\ 3 & 4 \end{pmatrix}

\qquad 2 \cdot A =

\begin{pmatrix} 2 & 4 \\ 6 & 8 \end{pmatrix}

\$

    Om matrices met elkaar te vermenigvuldigen kunnen we dit doen a.d.h.v.
de volgende definitie:

Als \(A\) een \(m \times n\) matrix en \(B\) een \(n \times q\) matrix
is, dan is het product \(C=A\cdot B\) de \(m\times q\) matrix waarvan de
elementen zijn

\(\begin{align}c_{ij} = a_{i1}b_{1j} + a_{i2}b_{2j} + \ldots + a_{in}b_{nj} = \sum\limits_{k=1}^n a_{ik}b_{kj}\end{align}\).

Een voorbeeld hiervan is:

\(\begin{pmatrix} 2 & 4 & 6 \\ 3 & 2 & 1 \end{pmatrix} \cdot \begin{pmatrix} 2 \\ 4 \\ 6 \end{pmatrix} = \begin{pmatrix} 2\cdot2+4\cdot4+6\cdot6 \\ 3\cdot2+2\cdot4+1\cdot6 \end{pmatrix} = \begin{pmatrix} 56 \\ 20\end{pmatrix}\)

Van een \(2\times3\) matrix kan alleen een product worden genomen met
een \(3\times n\) matrix en het resultaat heeft de afmeting
\(2\times n\).

Dit kan uiteraard ook met andere operaties worden gecombineerd zoals
bijvoorbeeld:

\(A^T \cdot A = \begin{pmatrix} 2 & 7 & 8 \end{pmatrix} \cdot \begin{pmatrix} 2 \\ 7 \\ 8 \end{pmatrix} = 127\)

In een meer gegeneraliseerde vorm kunnen we beter zien dat het benaderen
van elementen volgt dat \(a_{\text{rij},\text{kolom}}\), ofterwel
\(a_{y,x}\):

\(A = \begin{pmatrix} a_{11} & a_{12} & a_{13} \\ a_{21} & a_{22} & a_{23} \\ a_{31} & a_{32} & a_{33} \end{pmatrix} \qquad B = \begin{pmatrix} b_{11} & b_{12} & b_{13} \\ b_{21} & b_{22} & b_{23} \\ b_{31} & b_{32} & b_{33} \end{pmatrix}\)

En bij een matrixvermenigvuldiging wordt het volgende algoritme gevolgd:

\begin{enumerate}
\def\labelenumi{\arabic{enumi}.}
\tightlist
\item
  Voor elke kolom \(j\) in \(B\).
\item
  Voor elke rij \(i\) in \(A\).
\item
  Voor elk element \(k\) in rij \(i\).
\item
  \(C_{ij} = k_1 \cdot B_{ij} + k_2 \cdot B_{ij} + \ldots + k_n \cdot B_{ij}\).
\end{enumerate}

Hieruit volgt dat:

\(A \cdot B = \begin{pmatrix} a_{11}b_{11} + a_{12}b_{21} + a_{13}b_{31} & a_{11}b_{12} + a_{12}b_{22} + a_{13}b_{32} & a_{11}b_{13} + a_{12}b_{23} + a_{13}b_{33} \\ a_{21}b_{11} + a_{22}b_{21} + a_{23}b_{31} & a_{11}b_{12} + a_{12}b_{22} + a_{13}b_{32} & a_{11}b_{13} + a_{12}b_{23} + a_{13}b_{33}\\ a_{31}b_{11} + a_{32}b_{21} + a_{33}b_{31} & a_{11}b_{12} + a_{12}b_{22} + a_{13}b_{32} & a_{11}b_{13} + a_{12}b_{23} + a_{13}b_{33} \end{pmatrix}\)

    \hypertarget{eenheidsmatrix-identity}{%
\paragraph{Eenheidsmatrix (identity)}\label{eenheidsmatrix-identity}}

    Een eenheidsmatrix is altijd een vierkante matrix waarvan alle elementen
\(0\) zijn behalve de elementen van de hoofddiagonaal, deze zijn \(1\).

\(E = \begin{pmatrix} 1 & 0 & 0 \\ 0 & 1 & 0 \\ 0 & 0 & 1 \end{pmatrix}\)

De eeheidsmatrix wordt aangeduidt met \(A^E\) (eenheid) of \(A^I\)
(identity).

Wanneer er een matrixvermenigvuldiging wordt gedaan met de
eenheidsmatrix dan komt er altijd de oorspronkelijk matrix uit. De
volgorde waarin dit gebeurd maakt niet uit \(A\cdot E = E \cdot A = A\).

    \hypertarget{machtsverheffen}{%
\paragraph{Machtsverheffen}\label{machtsverheffen}}

    Machtsverheffen kan alleen met een vierkante matrix en hierbij gelden de
regels \(n \geq 0\), \(n \in \mathbb{N}\). Ga na waarom je niet kunt
machtsverheffen op een matrix die niet vierkant is.

\begin{itemize}
\tightlist
\item
  \(A^0 = E\) (identiteitsmatrix)
\item
  \(A^1 = A\)
\item
  \(A^2 = A\cdot A\)
\item
  \(A^n = A \cdot A \cdot \ldots\)
\end{itemize}

Een macht van een matrix kan geen gebroken exponenten hebben.

    \hypertarget{commutativiteit}{%
\paragraph{Commutativiteit}\label{commutativiteit}}

    Bij matrixvermenigvuldigen geldt de commutatieve eigenschap
\(a \cdot b = b \cdot a\) niet!

    \hypertarget{associativeit-en-distributiviteit}{%
\paragraph{Associativeit en
distributiviteit}\label{associativeit-en-distributiviteit}}

    Bij matrixvermenigvuldigen gelden de associatieve
\(a \cdot b \cdot c = (a \cdot b) \cdot c = a \cdot (b \cdot c)\) en
distributieve \(a \cdot (b + c) = a\cdot b + a \cdot c\) eigenschappen
wel.

    \hypertarget{nulvector}{%
\paragraph{Nulvector}\label{nulvector}}

    Een vector in bijvoorbeeld \(\mathbb{R}^3\) waarvan alle elementen \(0\)
zijn:

\(\begin{pmatrix} 0 \\ 0 \\ 0\end{pmatrix}\)

Deze vector geeft de oorsprong van het coordinatensysteem aan.

    \hypertarget{eenheidsvectoren}{%
\paragraph{Eenheidsvectoren}\label{eenheidsvectoren}}

    Dit zijn de vectoren die de basis vormen van het coordinatensysteem
waarin je werkt, voor \(\mathbb{R}^2\) zijn dit de vectoren:

\(\begin{pmatrix} 1 \\ 0\end{pmatrix} \qquad \begin{pmatrix} 0 \\ 1\end{pmatrix}\)

    \hypertarget{lineaire-combinatie-van-eenheidsvectoren}{%
\paragraph{Lineaire combinatie van
eenheidsvectoren}\label{lineaire-combinatie-van-eenheidsvectoren}}

    Elke vector in \(\mathbb{R}^2\) kan worden geschreven als een
\emph{lineaire combinatie} van eenheidsvectoren, bijvoorbeeld:

\(\begin{pmatrix} 3 \\ 4\end{pmatrix} = 3\cdot \begin{pmatrix} 1 \\ 0 \end{pmatrix} + 4 \cdot \begin{pmatrix} 0 \\ 1\end{pmatrix}\)

En voor \(\mathbb{R}^3\):

\(\begin{pmatrix} 2 \\ -1 \\ 6 \end{pmatrix} = 2 \cdot \begin{pmatrix} 1 \\ 0 \\ 0 \end{pmatrix} - \begin{pmatrix} 0 \\ 1 \\ 0\end{pmatrix} + 6 \cdot \begin{pmatrix} 0 \\ 0 \\ 1\end{pmatrix}\)

    \hypertarget{week-2}{%
\subsection{Week 2}\label{week-2}}

    \hypertarget{e-graads-vergelijking}{%
\paragraph{1e-graads vergelijking}\label{e-graads-vergelijking}}

    Dit wordt ook wel een \emph{lineaire vergelijking} genoemd en dit is in
de vorm van \(5x+7y=11\). Een voorbeeld een vergelijking die niet
lineair is zijn \(5x^2 + y = 8\).

    \hypertarget{equivalent-iff}{%
\paragraph{\texorpdfstring{Equivalent
\((\iff)\)}{Equivalent (\textbackslash{}iff)}}\label{equivalent-iff}}

    Als stelsels equivalent (gelijkwaardig) wil dat zeggen dat ze dezelfde
oplossing(en) hebben.

Twee stelsels zijn equivalent als:

\begin{itemize}
\tightlist
\item
  beide precies \(1\), en dezelfde, oplossing hebben
\item
  beide geen oplossing hebben
\item
  beide oneindig veel, en dezelfde, oplossingen hebben
\end{itemize}

    \hypertarget{standaard-vorm}{%
\paragraph{Standaard vorm}\label{standaard-vorm}}

    In alle vergelijkingen staan de variabelen aan de linker kant (op
volgorde \(x,y,z\)) en alle constanten staan aan de rechterkant van het
\(=\) teken.

    \hypertarget{elementaire-bewerkingen-elementairy-row-operations}{%
\paragraph{Elementaire bewerkingen (elementairy row
operations)}\label{elementaire-bewerkingen-elementairy-row-operations}}

    Op lineaire stelsels kunnen de volgende drie elementaire bewerkingen
worden uitgevoerd:

\begin{enumerate}
\def\labelenumi{\arabic{enumi}.}
\tightlist
\item
  Een vergelijking met een getal vermenigvuldigen (\(x\not=0\)). (Scale)
\item
  Een vergelijking bij een andere vergelijking optellen of aftrekken.
  (Pivot)
\item
  De volgorde van de vergelijking verwisselen. (Swap)
\end{enumerate}

Omdat dit al snel wat onoverzichtelijk wordt kan het handig zijn om met
de volgende notatie aan te geven welke stappen er zijn uitgevoerd:

\begin{itemize}
\tightlist
\item
  \(R_1 - 3R_2 \rightarrow R_1\), hiermee zeggen we dat we \(3\) maar
  rij \(2\) van rij \(1\) aftrekken en deze plaatsen op rij \(1\).
\item
  \(R_1 \iff R_2\), hiermee zeggen we dat wij de rijen \(1\) en \(2\)
  met elkaar verwisselen.
\end{itemize}

Met deze elementaire bewerkingen kun je stelsels van lineaire
vergelijkingen oplossen.

    \hypertarget{vegen}{%
\paragraph{Vegen}\label{vegen}}

    Vegen betekend achtereen volgend elimineren.

    \hypertarget{pivot}{%
\paragraph{Pivot}\label{pivot}}

    De pivot is de variabele waarmee geveegd wordt. Als notaties digitaal
zijn dan kan het helpen om de pivot dikgedrukt te maken.

    \hypertarget{stelsels-vergelijkingen-als-een-matrixvergelijking}{%
\paragraph{Stelsels vergelijkingen als een
matrixvergelijking}\label{stelsels-vergelijkingen-als-een-matrixvergelijking}}

    In plaats van \(x\) schrijven we \(x_1\), \(y\) wordt \(x_2\), en \(z\)
wordt \(x_3\).

Het stelsel:

\(\begin{cases}\begin{align}x_1 & -x_2 & +2x_3 &= 16 \\ -2x_1 & +4x_2 & +x_3 &= 5 \\ 3x_1 & -5x_2 & +3x_3 &= 25\end{align}\end{cases}\)

kunnen we schrijven als \(A \cdot X = B\), waarin:

\(A=\begin{pmatrix}1 & -2 & 2 \\ -2 & 4 & 1 \\ 3 & -5 & 3\end{pmatrix}, X = \begin{pmatrix}x_1\\x_2\\x_3\end{pmatrix}, B=\begin{pmatrix}16\\5\\25\end{pmatrix}\)

De matrix \(A\) bevat de coefficienten van \(x_1, x_2, x_3\) en heet
daarom de \emph{coefficient matrix}.

We plakken daarvoor \(B\) als een kolom aan \(A\), gescheiden door een
vertical lijn.

Dit is de \emph{aangevulde coefficient matrix} (augmented matrix).
Hiermee kunnen we de uitwerking van het stelsel verkorten. De pivot
elementen zijn dikgedrukt.

    \hypertarget{eenduidig-oplosbaar}{%
\paragraph{Eenduidig oplosbaar}\label{eenduidig-oplosbaar}}

    Stelsel heeft precies \(1\) oplossing.

    \hypertarget{strijdig-stelsel}{%
\paragraph{Strijdig stelsel}\label{strijdig-stelsel}}

    Als ergens bijvoorbeeld de vergelijking \(0=1\) of \(4=2\)
(ongelijkheid) optreedt.

    \hypertarget{afhankelijk-stelsel}{%
\paragraph{Afhankelijk stelsel}\label{afhankelijk-stelsel}}

    Heeft oneindig veel oplossingen.

    \hypertarget{algemene-oplossing}{%
\paragraph{Algemene oplossing}\label{algemene-oplossing}}

    Dit is de verzameling van alle oplossingen. Elke afzonderlijke oplossing
is een \emph{particuliere oplossing}.

    \hypertarget{oplossingsmogelijkheden}{%
\paragraph{Oplossingsmogelijkheden}\label{oplossingsmogelijkheden}}

    \begin{itemize}
\tightlist
\item
  Eenduidig oplosbaar stelsel heeft \(1\) oplossing.
\item
  Strijdig stelsel heeft geen oplossing.
\item
  Afhandkelijk stelsen heeft oneindig veel oplossingen.
\end{itemize}

Voor het aantal oplossingen geldt dat (\(\text{#vgl}=\) aantal
vergelijkingen, \(\text{#var}=\) aantal variabelen):

\begin{enumerate}
\def\labelenumi{\arabic{enumi}.}
\tightlist
\item
  \(\text{#vgl}=\text{#var} \rightarrow 0\) oplossingen in speciale
  gevallen oneindig veel.
\item
  \(\text{#vgl}>\text{#var} \rightarrow\) in het algemeen geen oplossing
  (in speciale gevallen \(1\) of oneindig).
\item
  \(\text{#vgl}<\text{#var} \rightarrow\) oneindig veel oplossingen (in
  speciale gevallen onoplosbaar).
\end{enumerate}

    \hypertarget{overige}{%
\paragraph{Overige}\label{overige}}

    In het dictaar wordt er gebruik gemaakt van de letters \(t,u,v\). Echter
in andere boeken wordt er gebruik gemaakt van
\(\alpha, \beta, \gamma, \lambda, \mu\) (alpha, beta, gamma, lambda,
mu).

    \hypertarget{week-3}{%
\subsection{Week 3}\label{week-3}}

    \hypertarget{aangevulde-coefficient-matrix}{%
\paragraph{(aangevulde) coefficient
matrix}\label{aangevulde-coefficient-matrix}}

    Een coefficient matrix is in de vorm van \(A \times \vec{x} = \vec{b}\).

\(A=\begin{pmatrix}1 & -1 & 2 \\ -2 & 4 & 1 \\ 3 & 5 & 3\end{pmatrix}\)

Hierbij is de aangevulde coefficient matrix in de vorm van
\(\begin{pmatrix}A|\vec{b}\end{pmatrix}\).

    \hypertarget{bovendriehoeksmatrix}{%
\paragraph{Bovendriehoeksmatrix}\label{bovendriehoeksmatrix}}

    Een bovendriehoeksmatrix (upper-triangular form) heeft alle elementen
linksonder de hoofddiagonaal op \(0\).

    \hypertarget{strijdigafhankelijk}{%
\paragraph{Strijdig/afhankelijk}\label{strijdigafhankelijk}}

    Een \emph{strijdig} stelsel heeft geen oplossing. Een \emph{afhankelijk}
stelsel heeft oneindig veel oplossingen.

    \hypertarget{eenduidig}{%
\paragraph{Eenduidig}\label{eenduidig}}

    Een \emph{eenduidig} stelsel heeft \(1\) oplossing.

    \hypertarget{equivalent}{%
\paragraph{Equivalent}\label{equivalent}}

    Een stelsel is \emph{equivalent} als ze beide dezelfde algemene
oplossing hebben.

\begin{itemize}
\tightlist
\item
  Geen oplossing
\item
  1 en dezelfde oplossing
\item
  oneindig veel, en dezelfde, oplossingen
\end{itemize}

    \hypertarget{algemene-oplossingen}{%
\paragraph{Algemene oplossing(en)}\label{algemene-oplossingen}}

    Dit zijn alle oplossing(en) van een stelsel. Hierbij wordt een enkele
oplossing een \emph{particuliere} oplossing genoemd.

    \hypertarget{algemene-oplossing-bij-afhankelijke-stelsels}{%
\paragraph{Algemene oplossing (bij afhankelijke
stelsels)}\label{algemene-oplossing-bij-afhankelijke-stelsels}}

    Om aan te geven wat alle oplossingen zijn van een afhankelijk stelsel
kunnen we de volgende notatie gebruiken. (Niet vergeten als je \(t\)
aangeeft in het stelsel dat je erbij vermeldt dat deze een willekeurige
waarde kan bevatten.)

\(\begin{pmatrix}x_1 \\ x_2 \\ x_3 \end{pmatrix} = \begin{pmatrix} \frac{1}{4}-\frac{9}{4}t \\ \frac{5}{8}+\frac{7}{8}t \\ t \end{pmatrix} = \begin{pmatrix}\frac{1}{4}\\\frac{5}{8}\\0\end{pmatrix}+t\begin{pmatrix}\frac{1}{4}\\\frac{7}{8}\\1\end{pmatrix}\)

Nu kunnen we op een gemakkelijke manier \(t\) vervangen met een getal en
uitrekenen welke oplossing er is.

    \hypertarget{week-4}{%
\subsection{Week 4}\label{week-4}}

    \hypertarget{determinanten}{%
\subsubsection{Determinanten}\label{determinanten}}

    \textbf{Definitie}

De \emph{determinant} is een hulpmiddel om vast te stellen of een
stelsel van lineaire vergelijking precies 1, geen, of oneindig veel
oplossingen heeft.

Voor elke \emph{vierkante matrix} (\(N \times N\)) kunnen we \(\det(A)\)
berekenen.

\textbf{\(2 \times 2\) matrix}

In het geval van een \(2 \times 2\) matrix,
\(A=\begin{pmatrix}a&b\\c&d\end{pmatrix}\), dan is \(\det(A) = ad-bc\).

\textbf{Voorbeeld:}

Om de determinant van \(A=\begin{pmatrix}1&5\\2&7\end{pmatrix}\) te
berekenen krijgen we \(1 \cdot 7 - 5 \cdot 2 = -3\).

\textbf{\(3 \times 3\) matrix}

Om de determinant van een \(3 \times 3\) matrix te berekenen kunnen we
de onderdeterminanten berekenen door af te wikkelen via een rij of
kolom.

Voor matrix
\(A = \begin{pmatrix}a_1&b_1&c_1\\a_2&b_2&c_2\\a_3&b_3&c_3 \end{pmatrix}\)
wordt dit op de volgende manier berekend:

\(\det(A) = a_1(b_2c_3-c_2b_3) - a_2(b_1c_3 - b_3c_1) + a_3(b_1c_2-b_2c_1)\)

\textbf{Determinant \(N\times N\) matrix}

Bij het bereken van de determinant wordt het onderstaande patroon
gevolgd:

\(\begin{pmatrix}+&-&+&-&\cdots\\-&+&-&+&\cdots\\+&-&+&-&\cdots\\-&+&-&+&\ldots\\\cdots& \vdots &\vdots&\vdots&\ddots\end{pmatrix}\)

\textbf{Eigenschappen van determinanten en veegprocedure}

\begin{enumerate}
\def\labelenumi{\arabic{enumi}.}
\tightlist
\item
  Bij het verwisselen van twee rijen of kolommen in een determinant
  veranderd deze van teken.
\item
  Als je een determinant spiegeld over de hoofddiagonaal blijft deze
  gelijk.
\item
  Als je alle getallen in een rij (of kolom) met een willekeurig getal
  \(k\) vermenigvuldigd, dan wordt de waarde van de determinant met
  \(k\) vermenigvuldigd.
\item
  Als je in een determinant \(k\) maal een rij (of kolom) optelt of
  aftrekt van een andere rij (of kolom), dan blijft de determinant
  gelijk.
\end{enumerate}

Het is mogelijk om de veegprocedure op determinant toe te passen om in
een rij/kolom meer nullen te krijgen. Hiervoor worden de bovenstaande
regels gebruikt. Dit is handig omdat het afwikkelen van de
rijen/kolommen hierdoor aanzienlijk wordt vereenvoudigd.

    \hypertarget{week-5}{%
\subsection{Week 5}\label{week-5}}

    \hypertarget{inverses}{%
\subsubsection{Inverses}\label{inverses}}

    Voor een matrix kan een inverse worden bepaald als:

\begin{enumerate}
\def\labelenumi{\arabic{enumi}.}
\tightlist
\item
  De matrix is vierkant (\(N \times N\))
\item
  \(\det(A) \not=0\)
\end{enumerate}

Een belangrijke eigenschap van de inverse is dat zodra deze met de
oorspronkelijke matrix wordt vermenigvuldigd we de identiteitsmatrix
krijgen. Ofterwel \(A \times A^{-1} = E = A^{-1} \times A = E\). Deze
operatie is \textbf{wel} commutatief, in tegenstelling tot
matrixvermenigvuldigen wat niet commutatief is.

Als een matrix een inverse heeft dan is de matrix \emph{inverteerbaar
(regulier)}. Als een matrix geen inverse heeft dan is de matrix
\emph{singulier}.

\textbf{Voorbeeld}

Stel we hebben \(A=\begin{pmatrix}1&2\\1&3\end{pmatrix}\). Om de inverse
te vinden zoeken we naar
\(A \times A^{-1} = \begin{pmatrix}1&0\\0&1\end{pmatrix}\).

Ofterwel,
\(\begin{pmatrix}1&2\\1&3\end{pmatrix} \cdot \begin{pmatrix}a&b\\c&d\end{pmatrix} = \begin{pmatrix}1&0\\0&1\end{pmatrix}\)

Dit is gemakkelijk op te lossen met vegen. Hiervoor gebruiken we een
aangevulde coefficientmatrix met links \(A\) en rechts de
identiteitsmatrix \(E\). Vervolgens vegen we zodat de coefficientmatrix
alleen maar \(1\) heeft op de hoofddiagonaal. De rechterkant is nu
\(A^{-1}\). Dit is ook weer te controleren door \(A \times A^{-1} = E\).

    \textbf{Eigenschappen}

    \begin{itemize}
\tightlist
\item
  Een matrix heeft een inverse als \(\det(A)\not=0\).
\item
  Elke inverse is uniek.
\item
  Als \(A\) inverteerbaar is dan is \(A^{-1}\) ook inverteerbaar.
\item
  Als een matrix een inverse heeft dan is stelsel \(Ax=b\) eenduidig
  oplosbaar.
\end{itemize}

    \textbf{Stelsels oplossen met \(A^{-1}\)}

Stel we hebben een vergelijking \(Ax=b\), dan geldt \(x = A^{-1}b\).

Bewijs:

\(\begin{align} Ax &= b \\ A^{-1}(Ax)&=A^{-1}b \quad &(\text{Add} \quad A^{-1}) \\ (A^{-1}A)x&=A^{-1}b \quad &(\text{By associativity}) \\ Ex &= A^{-1}b \quad &( A^{-1}A=E )\\ x &= A^{-1}b \quad &(Ex=x) \end{align}\)

    \hypertarget{week-6}{%
\subsection{Week 6}\label{week-6}}

    \hypertarget{definitie-eigenvector-eigenwaarde}{%
\paragraph{Definitie eigenvector \&
eigenwaarde}\label{definitie-eigenvector-eigenwaarde}}

Een eigenvector van een \(n \times n\) matrix is een vector \(x\),
waarvan tenminste 1 element niet gelijk is aan 0 zodaning dat
\(A\cdot x=k \cdot x\) voor een reëel getal \(k\). Dit getal \(k\) heet
een eingewaarde van \(A\) als er een niet-triviale oplossing \(x\)
bestaat van \(A \cdot x = k \cdot x\).

    \hypertarget{eigenvector}{%
\paragraph{Eigenvector}\label{eigenvector}}

Gegeven is matrix \(A=\begin{pmatrix}1&6\\5&2\end{pmatrix}\) en de
vector \(p=\begin{pmatrix}6\\-5\end{pmatrix}\). Is \(p\) een eigenvector
van \(A\)?

Om te zien of \(p\) een eigenvector is van \(A\) vermenigvuldigen we
\(A \cdot p\) en kijken we of we een getal \(k\) kunnen vinden voor
\(A \cdot p\).

\(A\cdot p = \begin{pmatrix}1&6\\5&2\end{pmatrix}\cdot\begin{pmatrix}6\\-5\end{pmatrix}=\begin{pmatrix}-24\\20\end{pmatrix}=-4\cdot\begin{pmatrix}6\\-5\end{pmatrix}=-4\cdot p\).

Ja, \(p\) is een eigenvector van \(A\) met eigenwaarde \(-4\).

    \hypertarget{eigenwaarde}{%
\paragraph{Eigenwaarde}\label{eigenwaarde}}

    Gegeven is matrix \(A=\begin{pmatrix}-2&2\\2&1\end{pmatrix}\) is het
getal \(2\) een eigenwaarde van \(A\)?

Om te zien of \(2\) een eigenwaarde is van \(A\) vullen we de formule
\(A \cdot x = 2 \cdot x\).

\(\begin{align} \begin{pmatrix}-2&2\\2&1\end{pmatrix}\begin{pmatrix}x_1\\x_2\end{pmatrix}&=2\cdot \begin{pmatrix}x_1\\x_2\end{pmatrix} \\ \begin{pmatrix}-2&2\\2&1\end{pmatrix}\begin{pmatrix}x_1\\x_2\end{pmatrix}-2\cdot \begin{pmatrix}x_1\\x_2\end{pmatrix}&=\begin{pmatrix}0\\0\end{pmatrix} \\ \begin{pmatrix}-2&2\\2&1\end{pmatrix}\begin{pmatrix}x_1\\x_2\end{pmatrix}-2\cdot \begin{pmatrix}1&0\\0&1\end{pmatrix}\begin{pmatrix}x_1\\x_2\end{pmatrix}&=\begin{pmatrix}0\\0\end{pmatrix} \\ \begin{pmatrix}-2&2\\2&1\end{pmatrix}\begin{pmatrix}x_1\\x_2\end{pmatrix}- \begin{pmatrix}2&0\\0&2\end{pmatrix}\begin{pmatrix}x_1\\x_2\end{pmatrix}&=\begin{pmatrix}0\\0\end{pmatrix} \\ \begin{pmatrix}-4&2\\2&-1\end{pmatrix}\begin{pmatrix}x_1\\x_2\end{pmatrix}&=\begin{pmatrix}0\\0\end{pmatrix} \end{align}\)

Beide vergelijkingen in het volgende stelsel
\(\begin{cases}\begin{align}-4x_1+2x_2&=0\\2x_1-x_2&=0\end{align}\end{cases}\)
zijn equivalent, dus \(2x_1-x_2=0 \iff 2x_1=x_2\). Nu is het makkelijk
te zien dat \(\begin{pmatrix}1\\2\end{pmatrix}\) een eigenvector is van
\(A\). Ofterwel, \(2\) is een eigenwaarde van de vorm
\(t\cdot\begin{pmatrix}1\\2\end{pmatrix}\) voor \(t\) ongelijk aan
\(0\).


    % Add a bibliography block to the postdoc
    
    
    
    \end{document}
