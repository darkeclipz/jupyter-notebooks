
% Default to the notebook output style

    


% Inherit from the specified cell style.




    
\documentclass[11pt]{article}

    
    
    \usepackage[T1]{fontenc}
    % Nicer default font (+ math font) than Computer Modern for most use cases
    \usepackage{mathpazo}

    % Basic figure setup, for now with no caption control since it's done
    % automatically by Pandoc (which extracts ![](path) syntax from Markdown).
    \usepackage{graphicx}
    % We will generate all images so they have a width \maxwidth. This means
    % that they will get their normal width if they fit onto the page, but
    % are scaled down if they would overflow the margins.
    \makeatletter
    \def\maxwidth{\ifdim\Gin@nat@width>\linewidth\linewidth
    \else\Gin@nat@width\fi}
    \makeatother
    \let\Oldincludegraphics\includegraphics
    % Set max figure width to be 80% of text width, for now hardcoded.
    \renewcommand{\includegraphics}[1]{\Oldincludegraphics[width=.8\maxwidth]{#1}}
    % Ensure that by default, figures have no caption (until we provide a
    % proper Figure object with a Caption API and a way to capture that
    % in the conversion process - todo).
    \usepackage{caption}
    \DeclareCaptionLabelFormat{nolabel}{}
    \captionsetup{labelformat=nolabel}

    \usepackage{adjustbox} % Used to constrain images to a maximum size 
    \usepackage{xcolor} % Allow colors to be defined
    \usepackage{enumerate} % Needed for markdown enumerations to work
    \usepackage{geometry} % Used to adjust the document margins
    \usepackage{amsmath} % Equations
    \usepackage{amssymb} % Equations
    \usepackage{textcomp} % defines textquotesingle
    % Hack from http://tex.stackexchange.com/a/47451/13684:
    \AtBeginDocument{%
        \def\PYZsq{\textquotesingle}% Upright quotes in Pygmentized code
    }
    \usepackage{upquote} % Upright quotes for verbatim code
    \usepackage{eurosym} % defines \euro
    \usepackage[mathletters]{ucs} % Extended unicode (utf-8) support
    \usepackage[utf8x]{inputenc} % Allow utf-8 characters in the tex document
    \usepackage{fancyvrb} % verbatim replacement that allows latex
    \usepackage{grffile} % extends the file name processing of package graphics 
                         % to support a larger range 
    % The hyperref package gives us a pdf with properly built
    % internal navigation ('pdf bookmarks' for the table of contents,
    % internal cross-reference links, web links for URLs, etc.)
    \usepackage{hyperref}
    \usepackage{longtable} % longtable support required by pandoc >1.10
    \usepackage{booktabs}  % table support for pandoc > 1.12.2
    \usepackage[inline]{enumitem} % IRkernel/repr support (it uses the enumerate* environment)
    \usepackage[normalem]{ulem} % ulem is needed to support strikethroughs (\sout)
                                % normalem makes italics be italics, not underlines
    

    
    
    % Colors for the hyperref package
    \definecolor{urlcolor}{rgb}{0,.145,.698}
    \definecolor{linkcolor}{rgb}{.71,0.21,0.01}
    \definecolor{citecolor}{rgb}{.12,.54,.11}

    % ANSI colors
    \definecolor{ansi-black}{HTML}{3E424D}
    \definecolor{ansi-black-intense}{HTML}{282C36}
    \definecolor{ansi-red}{HTML}{E75C58}
    \definecolor{ansi-red-intense}{HTML}{B22B31}
    \definecolor{ansi-green}{HTML}{00A250}
    \definecolor{ansi-green-intense}{HTML}{007427}
    \definecolor{ansi-yellow}{HTML}{DDB62B}
    \definecolor{ansi-yellow-intense}{HTML}{B27D12}
    \definecolor{ansi-blue}{HTML}{208FFB}
    \definecolor{ansi-blue-intense}{HTML}{0065CA}
    \definecolor{ansi-magenta}{HTML}{D160C4}
    \definecolor{ansi-magenta-intense}{HTML}{A03196}
    \definecolor{ansi-cyan}{HTML}{60C6C8}
    \definecolor{ansi-cyan-intense}{HTML}{258F8F}
    \definecolor{ansi-white}{HTML}{C5C1B4}
    \definecolor{ansi-white-intense}{HTML}{A1A6B2}

    % commands and environments needed by pandoc snippets
    % extracted from the output of `pandoc -s`
    \providecommand{\tightlist}{%
      \setlength{\itemsep}{0pt}\setlength{\parskip}{0pt}}
    \DefineVerbatimEnvironment{Highlighting}{Verbatim}{commandchars=\\\{\}}
    % Add ',fontsize=\small' for more characters per line
    \newenvironment{Shaded}{}{}
    \newcommand{\KeywordTok}[1]{\textcolor[rgb]{0.00,0.44,0.13}{\textbf{{#1}}}}
    \newcommand{\DataTypeTok}[1]{\textcolor[rgb]{0.56,0.13,0.00}{{#1}}}
    \newcommand{\DecValTok}[1]{\textcolor[rgb]{0.25,0.63,0.44}{{#1}}}
    \newcommand{\BaseNTok}[1]{\textcolor[rgb]{0.25,0.63,0.44}{{#1}}}
    \newcommand{\FloatTok}[1]{\textcolor[rgb]{0.25,0.63,0.44}{{#1}}}
    \newcommand{\CharTok}[1]{\textcolor[rgb]{0.25,0.44,0.63}{{#1}}}
    \newcommand{\StringTok}[1]{\textcolor[rgb]{0.25,0.44,0.63}{{#1}}}
    \newcommand{\CommentTok}[1]{\textcolor[rgb]{0.38,0.63,0.69}{\textit{{#1}}}}
    \newcommand{\OtherTok}[1]{\textcolor[rgb]{0.00,0.44,0.13}{{#1}}}
    \newcommand{\AlertTok}[1]{\textcolor[rgb]{1.00,0.00,0.00}{\textbf{{#1}}}}
    \newcommand{\FunctionTok}[1]{\textcolor[rgb]{0.02,0.16,0.49}{{#1}}}
    \newcommand{\RegionMarkerTok}[1]{{#1}}
    \newcommand{\ErrorTok}[1]{\textcolor[rgb]{1.00,0.00,0.00}{\textbf{{#1}}}}
    \newcommand{\NormalTok}[1]{{#1}}
    
    % Additional commands for more recent versions of Pandoc
    \newcommand{\ConstantTok}[1]{\textcolor[rgb]{0.53,0.00,0.00}{{#1}}}
    \newcommand{\SpecialCharTok}[1]{\textcolor[rgb]{0.25,0.44,0.63}{{#1}}}
    \newcommand{\VerbatimStringTok}[1]{\textcolor[rgb]{0.25,0.44,0.63}{{#1}}}
    \newcommand{\SpecialStringTok}[1]{\textcolor[rgb]{0.73,0.40,0.53}{{#1}}}
    \newcommand{\ImportTok}[1]{{#1}}
    \newcommand{\DocumentationTok}[1]{\textcolor[rgb]{0.73,0.13,0.13}{\textit{{#1}}}}
    \newcommand{\AnnotationTok}[1]{\textcolor[rgb]{0.38,0.63,0.69}{\textbf{\textit{{#1}}}}}
    \newcommand{\CommentVarTok}[1]{\textcolor[rgb]{0.38,0.63,0.69}{\textbf{\textit{{#1}}}}}
    \newcommand{\VariableTok}[1]{\textcolor[rgb]{0.10,0.09,0.49}{{#1}}}
    \newcommand{\ControlFlowTok}[1]{\textcolor[rgb]{0.00,0.44,0.13}{\textbf{{#1}}}}
    \newcommand{\OperatorTok}[1]{\textcolor[rgb]{0.40,0.40,0.40}{{#1}}}
    \newcommand{\BuiltInTok}[1]{{#1}}
    \newcommand{\ExtensionTok}[1]{{#1}}
    \newcommand{\PreprocessorTok}[1]{\textcolor[rgb]{0.74,0.48,0.00}{{#1}}}
    \newcommand{\AttributeTok}[1]{\textcolor[rgb]{0.49,0.56,0.16}{{#1}}}
    \newcommand{\InformationTok}[1]{\textcolor[rgb]{0.38,0.63,0.69}{\textbf{\textit{{#1}}}}}
    \newcommand{\WarningTok}[1]{\textcolor[rgb]{0.38,0.63,0.69}{\textbf{\textit{{#1}}}}}
    
    
    % Define a nice break command that doesn't care if a line doesn't already
    % exist.
    \def\br{\hspace*{\fill} \\* }
    % Math Jax compatability definitions
    \def\gt{>}
    \def\lt{<}
    % Document parameters
    \title{R - Week 2 (exercises)}
    
    
    

    % Pygments definitions
    
\makeatletter
\def\PY@reset{\let\PY@it=\relax \let\PY@bf=\relax%
    \let\PY@ul=\relax \let\PY@tc=\relax%
    \let\PY@bc=\relax \let\PY@ff=\relax}
\def\PY@tok#1{\csname PY@tok@#1\endcsname}
\def\PY@toks#1+{\ifx\relax#1\empty\else%
    \PY@tok{#1}\expandafter\PY@toks\fi}
\def\PY@do#1{\PY@bc{\PY@tc{\PY@ul{%
    \PY@it{\PY@bf{\PY@ff{#1}}}}}}}
\def\PY#1#2{\PY@reset\PY@toks#1+\relax+\PY@do{#2}}

\expandafter\def\csname PY@tok@w\endcsname{\def\PY@tc##1{\textcolor[rgb]{0.73,0.73,0.73}{##1}}}
\expandafter\def\csname PY@tok@c\endcsname{\let\PY@it=\textit\def\PY@tc##1{\textcolor[rgb]{0.25,0.50,0.50}{##1}}}
\expandafter\def\csname PY@tok@cp\endcsname{\def\PY@tc##1{\textcolor[rgb]{0.74,0.48,0.00}{##1}}}
\expandafter\def\csname PY@tok@k\endcsname{\let\PY@bf=\textbf\def\PY@tc##1{\textcolor[rgb]{0.00,0.50,0.00}{##1}}}
\expandafter\def\csname PY@tok@kp\endcsname{\def\PY@tc##1{\textcolor[rgb]{0.00,0.50,0.00}{##1}}}
\expandafter\def\csname PY@tok@kt\endcsname{\def\PY@tc##1{\textcolor[rgb]{0.69,0.00,0.25}{##1}}}
\expandafter\def\csname PY@tok@o\endcsname{\def\PY@tc##1{\textcolor[rgb]{0.40,0.40,0.40}{##1}}}
\expandafter\def\csname PY@tok@ow\endcsname{\let\PY@bf=\textbf\def\PY@tc##1{\textcolor[rgb]{0.67,0.13,1.00}{##1}}}
\expandafter\def\csname PY@tok@nb\endcsname{\def\PY@tc##1{\textcolor[rgb]{0.00,0.50,0.00}{##1}}}
\expandafter\def\csname PY@tok@nf\endcsname{\def\PY@tc##1{\textcolor[rgb]{0.00,0.00,1.00}{##1}}}
\expandafter\def\csname PY@tok@nc\endcsname{\let\PY@bf=\textbf\def\PY@tc##1{\textcolor[rgb]{0.00,0.00,1.00}{##1}}}
\expandafter\def\csname PY@tok@nn\endcsname{\let\PY@bf=\textbf\def\PY@tc##1{\textcolor[rgb]{0.00,0.00,1.00}{##1}}}
\expandafter\def\csname PY@tok@ne\endcsname{\let\PY@bf=\textbf\def\PY@tc##1{\textcolor[rgb]{0.82,0.25,0.23}{##1}}}
\expandafter\def\csname PY@tok@nv\endcsname{\def\PY@tc##1{\textcolor[rgb]{0.10,0.09,0.49}{##1}}}
\expandafter\def\csname PY@tok@no\endcsname{\def\PY@tc##1{\textcolor[rgb]{0.53,0.00,0.00}{##1}}}
\expandafter\def\csname PY@tok@nl\endcsname{\def\PY@tc##1{\textcolor[rgb]{0.63,0.63,0.00}{##1}}}
\expandafter\def\csname PY@tok@ni\endcsname{\let\PY@bf=\textbf\def\PY@tc##1{\textcolor[rgb]{0.60,0.60,0.60}{##1}}}
\expandafter\def\csname PY@tok@na\endcsname{\def\PY@tc##1{\textcolor[rgb]{0.49,0.56,0.16}{##1}}}
\expandafter\def\csname PY@tok@nt\endcsname{\let\PY@bf=\textbf\def\PY@tc##1{\textcolor[rgb]{0.00,0.50,0.00}{##1}}}
\expandafter\def\csname PY@tok@nd\endcsname{\def\PY@tc##1{\textcolor[rgb]{0.67,0.13,1.00}{##1}}}
\expandafter\def\csname PY@tok@s\endcsname{\def\PY@tc##1{\textcolor[rgb]{0.73,0.13,0.13}{##1}}}
\expandafter\def\csname PY@tok@sd\endcsname{\let\PY@it=\textit\def\PY@tc##1{\textcolor[rgb]{0.73,0.13,0.13}{##1}}}
\expandafter\def\csname PY@tok@si\endcsname{\let\PY@bf=\textbf\def\PY@tc##1{\textcolor[rgb]{0.73,0.40,0.53}{##1}}}
\expandafter\def\csname PY@tok@se\endcsname{\let\PY@bf=\textbf\def\PY@tc##1{\textcolor[rgb]{0.73,0.40,0.13}{##1}}}
\expandafter\def\csname PY@tok@sr\endcsname{\def\PY@tc##1{\textcolor[rgb]{0.73,0.40,0.53}{##1}}}
\expandafter\def\csname PY@tok@ss\endcsname{\def\PY@tc##1{\textcolor[rgb]{0.10,0.09,0.49}{##1}}}
\expandafter\def\csname PY@tok@sx\endcsname{\def\PY@tc##1{\textcolor[rgb]{0.00,0.50,0.00}{##1}}}
\expandafter\def\csname PY@tok@m\endcsname{\def\PY@tc##1{\textcolor[rgb]{0.40,0.40,0.40}{##1}}}
\expandafter\def\csname PY@tok@gh\endcsname{\let\PY@bf=\textbf\def\PY@tc##1{\textcolor[rgb]{0.00,0.00,0.50}{##1}}}
\expandafter\def\csname PY@tok@gu\endcsname{\let\PY@bf=\textbf\def\PY@tc##1{\textcolor[rgb]{0.50,0.00,0.50}{##1}}}
\expandafter\def\csname PY@tok@gd\endcsname{\def\PY@tc##1{\textcolor[rgb]{0.63,0.00,0.00}{##1}}}
\expandafter\def\csname PY@tok@gi\endcsname{\def\PY@tc##1{\textcolor[rgb]{0.00,0.63,0.00}{##1}}}
\expandafter\def\csname PY@tok@gr\endcsname{\def\PY@tc##1{\textcolor[rgb]{1.00,0.00,0.00}{##1}}}
\expandafter\def\csname PY@tok@ge\endcsname{\let\PY@it=\textit}
\expandafter\def\csname PY@tok@gs\endcsname{\let\PY@bf=\textbf}
\expandafter\def\csname PY@tok@gp\endcsname{\let\PY@bf=\textbf\def\PY@tc##1{\textcolor[rgb]{0.00,0.00,0.50}{##1}}}
\expandafter\def\csname PY@tok@go\endcsname{\def\PY@tc##1{\textcolor[rgb]{0.53,0.53,0.53}{##1}}}
\expandafter\def\csname PY@tok@gt\endcsname{\def\PY@tc##1{\textcolor[rgb]{0.00,0.27,0.87}{##1}}}
\expandafter\def\csname PY@tok@err\endcsname{\def\PY@bc##1{\setlength{\fboxsep}{0pt}\fcolorbox[rgb]{1.00,0.00,0.00}{1,1,1}{\strut ##1}}}
\expandafter\def\csname PY@tok@kc\endcsname{\let\PY@bf=\textbf\def\PY@tc##1{\textcolor[rgb]{0.00,0.50,0.00}{##1}}}
\expandafter\def\csname PY@tok@kd\endcsname{\let\PY@bf=\textbf\def\PY@tc##1{\textcolor[rgb]{0.00,0.50,0.00}{##1}}}
\expandafter\def\csname PY@tok@kn\endcsname{\let\PY@bf=\textbf\def\PY@tc##1{\textcolor[rgb]{0.00,0.50,0.00}{##1}}}
\expandafter\def\csname PY@tok@kr\endcsname{\let\PY@bf=\textbf\def\PY@tc##1{\textcolor[rgb]{0.00,0.50,0.00}{##1}}}
\expandafter\def\csname PY@tok@bp\endcsname{\def\PY@tc##1{\textcolor[rgb]{0.00,0.50,0.00}{##1}}}
\expandafter\def\csname PY@tok@fm\endcsname{\def\PY@tc##1{\textcolor[rgb]{0.00,0.00,1.00}{##1}}}
\expandafter\def\csname PY@tok@vc\endcsname{\def\PY@tc##1{\textcolor[rgb]{0.10,0.09,0.49}{##1}}}
\expandafter\def\csname PY@tok@vg\endcsname{\def\PY@tc##1{\textcolor[rgb]{0.10,0.09,0.49}{##1}}}
\expandafter\def\csname PY@tok@vi\endcsname{\def\PY@tc##1{\textcolor[rgb]{0.10,0.09,0.49}{##1}}}
\expandafter\def\csname PY@tok@vm\endcsname{\def\PY@tc##1{\textcolor[rgb]{0.10,0.09,0.49}{##1}}}
\expandafter\def\csname PY@tok@sa\endcsname{\def\PY@tc##1{\textcolor[rgb]{0.73,0.13,0.13}{##1}}}
\expandafter\def\csname PY@tok@sb\endcsname{\def\PY@tc##1{\textcolor[rgb]{0.73,0.13,0.13}{##1}}}
\expandafter\def\csname PY@tok@sc\endcsname{\def\PY@tc##1{\textcolor[rgb]{0.73,0.13,0.13}{##1}}}
\expandafter\def\csname PY@tok@dl\endcsname{\def\PY@tc##1{\textcolor[rgb]{0.73,0.13,0.13}{##1}}}
\expandafter\def\csname PY@tok@s2\endcsname{\def\PY@tc##1{\textcolor[rgb]{0.73,0.13,0.13}{##1}}}
\expandafter\def\csname PY@tok@sh\endcsname{\def\PY@tc##1{\textcolor[rgb]{0.73,0.13,0.13}{##1}}}
\expandafter\def\csname PY@tok@s1\endcsname{\def\PY@tc##1{\textcolor[rgb]{0.73,0.13,0.13}{##1}}}
\expandafter\def\csname PY@tok@mb\endcsname{\def\PY@tc##1{\textcolor[rgb]{0.40,0.40,0.40}{##1}}}
\expandafter\def\csname PY@tok@mf\endcsname{\def\PY@tc##1{\textcolor[rgb]{0.40,0.40,0.40}{##1}}}
\expandafter\def\csname PY@tok@mh\endcsname{\def\PY@tc##1{\textcolor[rgb]{0.40,0.40,0.40}{##1}}}
\expandafter\def\csname PY@tok@mi\endcsname{\def\PY@tc##1{\textcolor[rgb]{0.40,0.40,0.40}{##1}}}
\expandafter\def\csname PY@tok@il\endcsname{\def\PY@tc##1{\textcolor[rgb]{0.40,0.40,0.40}{##1}}}
\expandafter\def\csname PY@tok@mo\endcsname{\def\PY@tc##1{\textcolor[rgb]{0.40,0.40,0.40}{##1}}}
\expandafter\def\csname PY@tok@ch\endcsname{\let\PY@it=\textit\def\PY@tc##1{\textcolor[rgb]{0.25,0.50,0.50}{##1}}}
\expandafter\def\csname PY@tok@cm\endcsname{\let\PY@it=\textit\def\PY@tc##1{\textcolor[rgb]{0.25,0.50,0.50}{##1}}}
\expandafter\def\csname PY@tok@cpf\endcsname{\let\PY@it=\textit\def\PY@tc##1{\textcolor[rgb]{0.25,0.50,0.50}{##1}}}
\expandafter\def\csname PY@tok@c1\endcsname{\let\PY@it=\textit\def\PY@tc##1{\textcolor[rgb]{0.25,0.50,0.50}{##1}}}
\expandafter\def\csname PY@tok@cs\endcsname{\let\PY@it=\textit\def\PY@tc##1{\textcolor[rgb]{0.25,0.50,0.50}{##1}}}

\def\PYZbs{\char`\\}
\def\PYZus{\char`\_}
\def\PYZob{\char`\{}
\def\PYZcb{\char`\}}
\def\PYZca{\char`\^}
\def\PYZam{\char`\&}
\def\PYZlt{\char`\<}
\def\PYZgt{\char`\>}
\def\PYZsh{\char`\#}
\def\PYZpc{\char`\%}
\def\PYZdl{\char`\$}
\def\PYZhy{\char`\-}
\def\PYZsq{\char`\'}
\def\PYZdq{\char`\"}
\def\PYZti{\char`\~}
% for compatibility with earlier versions
\def\PYZat{@}
\def\PYZlb{[}
\def\PYZrb{]}
\makeatother


    % Exact colors from NB
    \definecolor{incolor}{rgb}{0.0, 0.0, 0.5}
    \definecolor{outcolor}{rgb}{0.545, 0.0, 0.0}



    
    % Prevent overflowing lines due to hard-to-break entities
    \sloppy 
    % Setup hyperref package
    \hypersetup{
      breaklinks=true,  % so long urls are correctly broken across lines
      colorlinks=true,
      urlcolor=urlcolor,
      linkcolor=linkcolor,
      citecolor=citecolor,
      }
    % Slightly bigger margins than the latex defaults
    
    \geometry{verbose,tmargin=1in,bmargin=1in,lmargin=1in,rmargin=1in}
    
    

    \begin{document}
    
    
    \maketitle
    
    

    
    \hypertarget{r---week-2-exercises}{%
\section{R - Week 2 (exercises)}\label{r---week-2-exercises}}

    \hypertarget{r-code-on-solving-equations-with-inverse-matrix}{%
\subsection{R-code on solving equations with inverse
matrix}\label{r-code-on-solving-equations-with-inverse-matrix}}

    Solve the following system of equations:

    \begin{enumerate}
\def\labelenumi{\arabic{enumi}.}
\tightlist
\item
  \(2x+y+2z=3\)
\item
  \(x-3z=-5\)
\item
  \(2y+5z=4\)
\end{enumerate}

    \[ \begin{bmatrix} 2 & 1 & 2 \\ 1 & 6 & -3 \\ 0 & 2 & 5 \end{bmatrix} \cdot \begin{bmatrix} x \\ y \\ z \end{bmatrix} = \begin{bmatrix} 3 & -5 & 4 \end{bmatrix} \]

    \$ A \vec{x} = \vec{b}\$

\(A^{-1}\cdot A \vec{x} = A^{-1} \vec{b}\)

\(I \vec{x} = A^{-1} \vec{b}\)

    Define matrix \(A\):

    \begin{Verbatim}[commandchars=\\\{\}]
{\color{incolor}In [{\color{incolor}2}]:} A \PY{o}{=} \PY{k+kt}{matrix}\PY{p}{(}\PY{k+kt}{c}\PY{p}{(}\PY{l+m}{2}\PY{p}{,}\PY{l+m}{1}\PY{p}{,}\PY{l+m}{2}\PY{p}{,}\PY{l+m}{1}\PY{p}{,}\PY{l+m}{6}\PY{p}{,}\PY{l+m}{\PYZhy{}3}\PY{p}{,}\PY{l+m}{0}\PY{p}{,}\PY{l+m}{2}\PY{p}{,}\PY{l+m}{5}\PY{p}{)}\PY{p}{,} nrow\PY{o}{=}\PY{l+m}{3}\PY{p}{)}
        A
\end{Verbatim}


    \begin{tabular}{lll}
	 2  &  1 & 0 \\
	 1  &  6 & 2 \\
	 2  & -3 & 5 \\
\end{tabular}


    
    The inverse \(A^{-1}\) is:

    \begin{Verbatim}[commandchars=\\\{\}]
{\color{incolor}In [{\color{incolor}3}]:} \PY{k+kp}{solve}\PY{p}{(}A\PY{p}{)}
\end{Verbatim}


    \begin{tabular}{lll}
	  0.50704225 & -0.07042254 &  0.02816901\\
	 -0.01408451 &  0.14084507 & -0.05633803\\
	 -0.21126761 &  0.11267606 &  0.15492958\\
\end{tabular}


    
    Define vector \(\vec{b}\):

    \begin{Verbatim}[commandchars=\\\{\}]
{\color{incolor}In [{\color{incolor}4}]:} b \PY{o}{=} \PY{k+kt}{c}\PY{p}{(}\PY{l+m}{3}\PY{p}{,} \PY{l+m}{\PYZhy{}5}\PY{p}{,} \PY{l+m}{4}\PY{p}{)}
\end{Verbatim}


    Solve the system with R functions \texttt{solve(A,b)}:

    \begin{Verbatim}[commandchars=\\\{\}]
{\color{incolor}In [{\color{incolor}5}]:} \PY{k+kp}{solve}\PY{p}{(}A\PY{p}{,}b\PY{p}{)}
\end{Verbatim}


    \begin{enumerate*}
\item 1.98591549295775
\item -0.971830985915493
\item -0.577464788732394
\end{enumerate*}


    
    Solve the system with \(\vec{x}=A^{-1}\vec{b}\):

    \begin{Verbatim}[commandchars=\\\{\}]
{\color{incolor}In [{\color{incolor}6}]:} \PY{k+kp}{solve}\PY{p}{(}A\PY{p}{)} \PY{o}{\PYZpc{}*\PYZpc{}} b
\end{Verbatim}


    \begin{tabular}{l}
	  1.9859155\\
	 -0.9718310\\
	 -0.5774648\\
\end{tabular}


    
    \hypertarget{r-code-on-least-square-method}{%
\subsection{R-code on least square
method}\label{r-code-on-least-square-method}}

    \(y=ax+b\)

    \(A\cdot \vec{x} = \vec{b}\)

\(A^T\cdot A \vec{x} = A^T \vec{b}\)

\((A^T A)^{-1}A^T A \vec{x} = A^T \vec{b}\)

\((A^T A)^{-1} ...\)

    Define
\(\vec{x}=\begin{bmatrix}12 & 2 & 3 & 5 & 10 & 9 & 8 \end{bmatrix}\):

    \begin{Verbatim}[commandchars=\\\{\}]
{\color{incolor}In [{\color{incolor}7}]:} x \PY{o}{=} \PY{k+kt}{c}\PY{p}{(}\PY{l+m}{12}\PY{p}{,} \PY{l+m}{2}\PY{p}{,} \PY{l+m}{3}\PY{p}{,} \PY{l+m}{5}\PY{p}{,} \PY{l+m}{10}\PY{p}{,} \PY{l+m}{9}\PY{p}{,} \PY{l+m}{8}\PY{p}{)}
\end{Verbatim}


    \begin{Verbatim}[commandchars=\\\{\}]
{\color{incolor}In [{\color{incolor}8}]:} x
\end{Verbatim}


    \begin{enumerate*}
\item 12
\item 2
\item 3
\item 5
\item 10
\item 9
\item 8
\end{enumerate*}


    
    Define
\(\vec{y} = \begin{bmatrix}125 & 30 & 43 & 62 & 108 & 102 & 90 \end{bmatrix}\):

    \begin{Verbatim}[commandchars=\\\{\}]
{\color{incolor}In [{\color{incolor}29}]:} y \PY{o}{=} \PY{k+kt}{c}\PY{p}{(}\PY{l+m}{125}\PY{p}{,} \PY{l+m}{30}\PY{p}{,} \PY{l+m}{43}\PY{p}{,} \PY{l+m}{62}\PY{p}{,} \PY{l+m}{108}\PY{p}{,} \PY{l+m}{102}\PY{p}{,} \PY{l+m}{90}\PY{p}{)}
\end{Verbatim}


    \begin{Verbatim}[commandchars=\\\{\}]
{\color{incolor}In [{\color{incolor}10}]:} y
\end{Verbatim}


    \begin{enumerate*}
\item 125
\item 30
\item 43
\item 62
\item 108
\item 102
\item 90
\end{enumerate*}


    
    \begin{Verbatim}[commandchars=\\\{\}]
{\color{incolor}In [{\color{incolor}11}]:} \PY{k+kp}{length}\PY{p}{(}x\PY{p}{)}\PY{o}{==}\PY{k+kp}{length}\PY{p}{(}y\PY{p}{)}
\end{Verbatim}


    TRUE

    
    \begin{Verbatim}[commandchars=\\\{\}]
{\color{incolor}In [{\color{incolor}12}]:} A \PY{o}{=} \PY{k+kt}{matrix}\PY{p}{(}\PY{k+kp}{union}\PY{p}{(}x\PY{p}{,}y\PY{p}{)}\PY{p}{,} nrow\PY{o}{=}\PY{k+kp}{length}\PY{p}{(}x\PY{p}{)}\PY{p}{)}
\end{Verbatim}


    \begin{Verbatim}[commandchars=\\\{\}]
{\color{incolor}In [{\color{incolor}13}]:} A
\end{Verbatim}


    \begin{tabular}{ll}
	 12  & 125\\
	  2  &  30\\
	  3  &  43\\
	  5  &  62\\
	 10  & 108\\
	  9  & 102\\
	  8  &  90\\
\end{tabular}


    
    \begin{Verbatim}[commandchars=\\\{\}]
{\color{incolor}In [{\color{incolor}14}]:} lm\PY{p}{(}y\PY{o}{\PYZti{}}x\PY{p}{)}
\end{Verbatim}


    
    \begin{verbatim}

Call:
lm(formula = y ~ x)

Coefficients:
(Intercept)            x  
     13.583        9.488  

    \end{verbatim}

    
    \begin{Verbatim}[commandchars=\\\{\}]
{\color{incolor}In [{\color{incolor}15}]:} fit \PY{o}{\PYZlt{}\PYZhy{}} \PY{k+kr}{function}\PY{p}{(}x\PY{p}{)} \PY{l+m}{9.488}\PY{o}{*}x\PY{l+m}{+13.583}
\end{Verbatim}


    \begin{Verbatim}[commandchars=\\\{\}]
{\color{incolor}In [{\color{incolor}16}]:} fit\PY{p}{(}\PY{l+m}{5}\PY{p}{)}
\end{Verbatim}


    61.023

    
    \begin{Verbatim}[commandchars=\\\{\}]
{\color{incolor}In [{\color{incolor}17}]:} plot\PY{p}{(}x\PY{p}{,}\PY{k+kp}{sapply}\PY{p}{(}x\PY{p}{,} fit\PY{p}{)}\PY{p}{,} \PY{l+s}{\PYZsq{}}\PY{l+s}{l\PYZsq{}}\PY{p}{,} col\PY{o}{=}\PY{l+s}{\PYZsq{}}\PY{l+s}{blue\PYZsq{}}\PY{p}{)}
         points\PY{p}{(}x\PY{p}{,}y\PY{p}{)}
\end{Verbatim}


    \begin{center}
    \adjustimage{max size={0.9\linewidth}{0.9\paperheight}}{output_31_0.png}
    \end{center}
    { \hspace*{\fill} \\}
    
    \begin{Verbatim}[commandchars=\\\{\}]
{\color{incolor}In [{\color{incolor}60}]:} x
\end{Verbatim}


    \begin{enumerate*}
\item 12
\item 2
\item 3
\item 5
\item 10
\item 9
\item 8
\end{enumerate*}


    
    \begin{Verbatim}[commandchars=\\\{\}]
{\color{incolor}In [{\color{incolor}61}]:} y
\end{Verbatim}


    \begin{enumerate*}
\item 125
\item 30
\item 43
\item 62
\item 108
\item 102
\item 90
\end{enumerate*}


    
    \(y=ax+b\)

    \begin{Verbatim}[commandchars=\\\{\}]
{\color{incolor}In [{\color{incolor}62}]:} X \PY{o}{=} \PY{k+kt}{matrix}\PY{p}{(}\PY{k+kt}{c}\PY{p}{(}y\PY{p}{,} \PY{k+kp}{rep}\PY{p}{(}\PY{l+m}{1}\PY{p}{,} \PY{k+kp}{length}\PY{p}{(}y\PY{p}{)}\PY{p}{)}\PY{p}{)}\PY{p}{,} ncol\PY{o}{=}\PY{l+m}{2}\PY{p}{)}
\end{Verbatim}


    \begin{Verbatim}[commandchars=\\\{\}]
{\color{incolor}In [{\color{incolor}63}]:} X
\end{Verbatim}


    \begin{tabular}{ll}
	 125 & 1  \\
	  30 & 1  \\
	  43 & 1  \\
	  62 & 1  \\
	 108 & 1  \\
	 102 & 1  \\
	  90 & 1  \\
\end{tabular}


    
    \begin{Verbatim}[commandchars=\\\{\}]
{\color{incolor}In [{\color{incolor}64}]:} b \PY{o}{=} \PY{k+kt}{matrix}\PY{p}{(}y\PY{p}{)}
\end{Verbatim}


    \begin{Verbatim}[commandchars=\\\{\}]
{\color{incolor}In [{\color{incolor}65}]:} b
\end{Verbatim}


    \begin{tabular}{l}
	 125\\
	  30\\
	  43\\
	  62\\
	 108\\
	 102\\
	  90\\
\end{tabular}


    
    Dit is de vorm \(A\cdot\vec{x} = \vec{b}\). Wat we op willen lossen met
\(A^{-1}A\vec{x}=A^{-1}\vec{b}\), maar dit werkt niet omdat \(A\) geen
vierkante matrix is en daardoor geen inverse kan bepalen voor \(A\).

Door gebruik te maken van de getransponeerde \(A^T\) kunnen we een
vierkante matrix krijgen. Dus matrix-vermenigvuldigen van
\(A\vec{x}=\vec{b}\) met \(A^T\) geeft
\(A^T\cdot A\vec{x} = A^T\cdot\vec{b}\).

    \begin{Verbatim}[commandchars=\\\{\}]
{\color{incolor}In [{\color{incolor}66}]:} \PY{k+kp}{t}\PY{p}{(}A\PY{p}{)} \PY{o}{\PYZpc{}*\PYZpc{}} A
\end{Verbatim}


    \begin{tabular}{ll}
	  427  &  4717\\
	 4717  & 52386\\
\end{tabular}


    
    Wat inderdaad een vierkant matrix geeft. Vervolgens is deze op te lossen
door de inverse te bepalen. De hele formule wordt dan:

\[ (A^T \cdot A)^{-1}\cdot(A^T \cdot A)\cdot\vec{x} = (A^T \cdot A)^{-1} \cdot A^T \cdot \vec{b} \]

Laat \(B = (A^T \cdot A)^{-1}\) zijn. Substitueren en vereenvoudigen
geeft:

\[ I\cdot\vec{x} = B \cdot A^T \cdot \vec{b} \]

    \begin{Verbatim}[commandchars=\\\{\}]
{\color{incolor}In [{\color{incolor}67}]:} B \PY{o}{=} \PY{k+kp}{solve}\PY{p}{(}\PY{k+kp}{t}\PY{p}{(}A\PY{p}{)} \PY{o}{\PYZpc{}*\PYZpc{}} A\PY{p}{)}
         B
\end{Verbatim}


    \begin{tabular}{ll}
	  0.44120843  & -0.039727793\\
	 -0.03972779  &  0.003596304\\
\end{tabular}


    
    \begin{Verbatim}[commandchars=\\\{\}]
{\color{incolor}In [{\color{incolor}74}]:} B \PY{o}{\PYZpc{}*\PYZpc{}} \PY{k+kp}{t}\PY{p}{(}A\PY{p}{)} \PY{o}{\PYZpc{}*\PYZpc{}} b
\end{Verbatim}


    \begin{tabular}{l}
	 -4.13003e-13\\
	  1.00000e+00\\
\end{tabular}


    
    \begin{Verbatim}[commandchars=\\\{\}]
{\color{incolor}In [{\color{incolor}69}]:} lm\PY{p}{(}y\PY{o}{\PYZti{}}x\PY{p}{)}
\end{Verbatim}


    
    \begin{verbatim}

Call:
lm(formula = y ~ x)

Coefficients:
(Intercept)            x  
     13.583        9.488  

    \end{verbatim}

    
    \textbf{Versimpeld voorbeeld lreg}

    \begin{Verbatim}[commandchars=\\\{\}]
{\color{incolor}In [{\color{incolor}118}]:} A \PY{o}{=} \PY{k+kt}{matrix}\PY{p}{(}\PY{k+kt}{c}\PY{p}{(}\PY{l+m}{\PYZhy{}1}\PY{p}{,}\PY{l+m}{0}\PY{p}{,}\PY{l+m}{2}\PY{p}{,}\PY{l+m}{3}\PY{p}{,}\PY{l+m}{1}\PY{p}{,}\PY{l+m}{1}\PY{p}{,}\PY{l+m}{1}\PY{p}{,}\PY{l+m}{1}\PY{p}{)}\PY{p}{,}ncol\PY{o}{=}\PY{l+m}{2}\PY{p}{)}
\end{Verbatim}


    \begin{Verbatim}[commandchars=\\\{\}]
{\color{incolor}In [{\color{incolor}119}]:} A
\end{Verbatim}


    \begin{tabular}{ll}
	 -1 & 1 \\
	  0 & 1 \\
	  2 & 1 \\
	  3 & 1 \\
\end{tabular}


    
    \begin{Verbatim}[commandchars=\\\{\}]
{\color{incolor}In [{\color{incolor}120}]:} b \PY{o}{=} \PY{k+kt}{c}\PY{p}{(}\PY{l+m}{\PYZhy{}1}\PY{p}{,}\PY{l+m}{2}\PY{p}{,}\PY{l+m}{1}\PY{p}{,}\PY{l+m}{2}\PY{p}{)}
\end{Verbatim}


    \begin{Verbatim}[commandchars=\\\{\}]
{\color{incolor}In [{\color{incolor}121}]:} b
\end{Verbatim}


    \begin{enumerate*}
\item -1
\item 2
\item 1
\item 2
\end{enumerate*}


    
    \begin{Verbatim}[commandchars=\\\{\}]
{\color{incolor}In [{\color{incolor}127}]:} x \PY{o}{=} A\PY{p}{[}\PY{l+m}{0}\PY{o}{:}\PY{l+m}{4}\PY{p}{]}
\end{Verbatim}


    \begin{Verbatim}[commandchars=\\\{\}]
{\color{incolor}In [{\color{incolor}128}]:} \PY{k+kp}{solve}\PY{p}{(}\PY{k+kp}{t}\PY{p}{(}A\PY{p}{)} \PY{o}{\PYZpc{}*\PYZpc{}} A\PY{p}{)} \PY{o}{\PYZpc{}*\PYZpc{}} \PY{k+kp}{t}\PY{p}{(}A\PY{p}{)} \PY{o}{\PYZpc{}*\PYZpc{}} b
\end{Verbatim}


    \begin{tabular}{l}
	 0.5\\
	 0.5\\
\end{tabular}


    
    \begin{Verbatim}[commandchars=\\\{\}]
{\color{incolor}In [{\color{incolor}129}]:} lm\PY{p}{(}b\PY{o}{\PYZti{}}x\PY{p}{)}
\end{Verbatim}


    
    \begin{verbatim}

Call:
lm(formula = b ~ x)

Coefficients:
(Intercept)            x  
        0.5          0.5  

    \end{verbatim}

    
    \begin{Verbatim}[commandchars=\\\{\}]
{\color{incolor}In [{\color{incolor}130}]:} fit \PY{o}{\PYZlt{}\PYZhy{}} \PY{k+kr}{function}\PY{p}{(}x\PY{p}{)} \PY{l+m}{0.5}\PY{o}{*}x\PY{l+m}{+0.5}
\end{Verbatim}


    \begin{Verbatim}[commandchars=\\\{\}]
{\color{incolor}In [{\color{incolor}131}]:} plot\PY{p}{(}\PY{l+m}{\PYZhy{}5}\PY{o}{:}\PY{l+m}{5}\PY{p}{,} \PY{k+kp}{sapply}\PY{p}{(}\PY{l+m}{\PYZhy{}5}\PY{o}{:}\PY{l+m}{5}\PY{p}{,} fit\PY{p}{)}\PY{p}{,} \PY{l+s}{\PYZsq{}}\PY{l+s}{l\PYZsq{}}\PY{p}{)}
          points\PY{p}{(}x\PY{p}{,}b\PY{p}{)}
\end{Verbatim}


    \begin{center}
    \adjustimage{max size={0.9\linewidth}{0.9\paperheight}}{output_54_0.png}
    \end{center}
    { \hspace*{\fill} \\}
    
    \begin{Verbatim}[commandchars=\\\{\}]
{\color{incolor}In [{\color{incolor}267}]:} lreg \PY{o}{\PYZlt{}\PYZhy{}} \PY{k+kr}{function}\PY{p}{(}x\PY{p}{,} y\PY{p}{)} \PY{p}{\PYZob{}}
              A \PY{o}{=} \PY{k+kp}{cbind}\PY{p}{(}x\PY{p}{,} \PY{k+kp}{rep}\PY{p}{(}\PY{l+m}{1}\PY{p}{,} \PY{k+kp}{length}\PY{p}{(}x\PY{p}{)}\PY{p}{)}\PY{p}{)}
              s \PY{o}{=} \PY{k+kp}{solve}\PY{p}{(}\PY{k+kp}{t}\PY{p}{(}A\PY{p}{)} \PY{o}{\PYZpc{}*\PYZpc{}} A\PY{p}{)} \PY{o}{\PYZpc{}*\PYZpc{}} \PY{k+kp}{t}\PY{p}{(}A\PY{p}{)} \PY{o}{\PYZpc{}*\PYZpc{}} y
              \PY{k+kr}{function}\PY{p}{(}x\PY{p}{)} s\PY{p}{[}\PY{l+m}{1}\PY{p}{]} \PY{o}{*} x \PY{o}{+} s\PY{p}{[}\PY{l+m}{2}\PY{p}{]} \PY{c+c1}{\PYZsh{} Return f(x)=ax+b}
          \PY{p}{\PYZcb{}}
\end{Verbatim}


    \begin{Verbatim}[commandchars=\\\{\}]
{\color{incolor}In [{\color{incolor}268}]:} x \PY{o}{=} \PY{k+kt}{c}\PY{p}{(}\PY{l+m}{12}\PY{p}{,}\PY{l+m}{2}\PY{p}{,}\PY{l+m}{3}\PY{p}{,}\PY{l+m}{5}\PY{p}{,}\PY{l+m}{10}\PY{p}{,}\PY{l+m}{9}\PY{p}{,}\PY{l+m}{8}\PY{p}{)}
          y \PY{o}{=} \PY{k+kt}{c}\PY{p}{(}\PY{l+m}{125}\PY{p}{,}\PY{l+m}{30}\PY{p}{,}\PY{l+m}{43}\PY{p}{,}\PY{l+m}{62}\PY{p}{,}\PY{l+m}{108}\PY{p}{,}\PY{l+m}{102}\PY{p}{,}\PY{l+m}{90}\PY{p}{)}
\end{Verbatim}


    \begin{Verbatim}[commandchars=\\\{\}]
{\color{incolor}In [{\color{incolor}269}]:} fit \PY{o}{\PYZlt{}\PYZhy{}} lreg\PY{p}{(}x\PY{p}{,} y\PY{p}{)}
\end{Verbatim}


    \begin{Verbatim}[commandchars=\\\{\}]
{\color{incolor}In [{\color{incolor}270}]:} plot\PY{p}{(}\PY{l+m}{0}\PY{o}{:}\PY{l+m}{15}\PY{p}{,} \PY{k+kp}{sapply}\PY{p}{(}\PY{l+m}{0}\PY{o}{:}\PY{l+m}{15}\PY{p}{,} fit\PY{p}{)}\PY{p}{,} \PY{l+s}{\PYZsq{}}\PY{l+s}{l\PYZsq{}}\PY{p}{)}
          points\PY{p}{(}x\PY{p}{,}y\PY{p}{)}
\end{Verbatim}


    \begin{center}
    \adjustimage{max size={0.9\linewidth}{0.9\paperheight}}{output_58_0.png}
    \end{center}
    { \hspace*{\fill} \\}
    
    With other functionality:

    \hypertarget{least-squares-regression-model}{%
\subsubsection{Least squares regression
model}\label{least-squares-regression-model}}

    Find the best fitting line \(y=ax+b\) for the following data points:

    \begin{Verbatim}[commandchars=\\\{\}]
{\color{incolor}In [{\color{incolor}311}]:} x \PY{o}{\PYZlt{}\PYZhy{}} \PY{k+kt}{c}\PY{p}{(}\PY{l+m}{12}\PY{p}{,}\PY{l+m}{2}\PY{p}{,}\PY{l+m}{3}\PY{p}{,}\PY{l+m}{5}\PY{p}{,}\PY{l+m}{10}\PY{p}{,}\PY{l+m}{9}\PY{p}{,}\PY{l+m}{8}\PY{p}{)}
          b \PY{o}{\PYZlt{}\PYZhy{}} \PY{k+kt}{c}\PY{p}{(}\PY{l+m}{125}\PY{p}{,}\PY{l+m}{30}\PY{p}{,}\PY{l+m}{43}\PY{p}{,}\PY{l+m}{62}\PY{p}{,}\PY{l+m}{108}\PY{p}{,}\PY{l+m}{102}\PY{p}{,}\PY{l+m}{90}\PY{p}{)}
\end{Verbatim}


    We can do this by solving the equation \(A\vec{x}=\vec{b}\).
Constructing the equation with the matrices for our data points yields:

\[ \begin{bmatrix} 12 & 1 \\ 2 & 1 \\ 3 & 1 \\ 5 & 1 \\ 10 & 1 \\ 9 & 1 \\ 8 & 1 \end{bmatrix} \cdot \begin{bmatrix}a \\ b \end{bmatrix} = \begin{bmatrix} 125 \\ 30 \\ 43 \\ 62 \\ 108 \\ 102 \\ 90 \end{bmatrix} \]

    First we will construct our matrix \(A\):

    \begin{Verbatim}[commandchars=\\\{\}]
{\color{incolor}In [{\color{incolor}312}]:} ones \PY{o}{\PYZlt{}\PYZhy{}} \PY{k+kp}{rep}\PY{p}{(}\PY{l+m}{1}\PY{p}{,} \PY{k+kp}{length}\PY{p}{(}x\PY{p}{)}\PY{p}{)}
\end{Verbatim}


    \begin{Verbatim}[commandchars=\\\{\}]
{\color{incolor}In [{\color{incolor}297}]:} A \PY{o}{\PYZlt{}\PYZhy{}} \PY{k+kp}{cbind}\PY{p}{(}x\PY{p}{,} ones\PY{p}{)}
\end{Verbatim}


    \begin{Verbatim}[commandchars=\\\{\}]
{\color{incolor}In [{\color{incolor}313}]:} A
\end{Verbatim}


    \begin{tabular}{ll}
 x & ones\\
\hline
	 12 & 1 \\
	  2 & 1 \\
	  3 & 1 \\
	  5 & 1 \\
	 10 & 1 \\
	  9 & 1 \\
	  8 & 1 \\
\end{tabular}


    
    If we want to solve the equation \(A\vec{x}=\vec{b}\) we can multiply
both sides \(A^{-1}\) to get:

\[ \begin{align} A\vec{x}&=\vec{b} \\ (A^{-1}\cdot A)\vec{x}&=A^{-1}\vec{b} \\ I\vec{x}&=A^{-1}\vec{b} \end{align} \]

    However, we need to calculate the inverse of \(A\), but \(A\) is not a
square matrix. To solve this problem we multiply \(A\) with \(A^T\) to
get a square matrix.

    \[ \begin{align} A\vec{x}&=\vec{b} \\ (A^T \cdot A) \cdot \vec{x} &= A^T \cdot \vec{b} \\ (A^T \cdot A)^{-1} \cdot (A^T \cdot A) \cdot \vec{x} &= (A^T \cdot A)^{-1} \cdot A^T \cdot \vec{b} \\ I \cdot \vec{x} &= (A^T \cdot A)^{-1} \cdot A^T \cdot \vec{b} \end{align} \]

    \begin{Verbatim}[commandchars=\\\{\}]
{\color{incolor}In [{\color{incolor}298}]:} S \PY{o}{=} \PY{k+kp}{solve}\PY{p}{(}\PY{k+kp}{t}\PY{p}{(}A\PY{p}{)} \PY{o}{\PYZpc{}*\PYZpc{}} A\PY{p}{)} \PY{o}{\PYZpc{}*\PYZpc{}} \PY{k+kp}{t}\PY{p}{(}A\PY{p}{)} \PY{o}{\PYZpc{}*\PYZpc{}} b
\end{Verbatim}


    The resulting matrix \(S\) will have our coefficients \(a\) and \(b\) to
construct the line:

    \begin{Verbatim}[commandchars=\\\{\}]
{\color{incolor}In [{\color{incolor}308}]:} lsm \PY{o}{=} \PY{k+kt}{c}\PY{p}{(}S\PY{p}{[}\PY{l+m}{2}\PY{p}{]}\PY{p}{,} S\PY{p}{[}\PY{l+m}{1}\PY{p}{]}\PY{p}{)}
\end{Verbatim}


    \begin{Verbatim}[commandchars=\\\{\}]
{\color{incolor}In [{\color{incolor}310}]:} lsm
\end{Verbatim}


    \begin{enumerate*}
\item 13.5833333333333
\item 9.48809523809525
\end{enumerate*}


    
    If we verify the coefficients with built-in R functionality for
least-squares regression, we can see that our solution is correct.

    \begin{Verbatim}[commandchars=\\\{\}]
{\color{incolor}In [{\color{incolor}301}]:} lm\PY{p}{(}b\PY{o}{\PYZti{}}x\PY{p}{)}
\end{Verbatim}


    
    \begin{verbatim}

Call:
lm(formula = b ~ x)

Coefficients:
(Intercept)            x  
     13.583        9.488  

    \end{verbatim}

    
    Plotting our values yields:

    \begin{Verbatim}[commandchars=\\\{\}]
{\color{incolor}In [{\color{incolor}309}]:} plot\PY{p}{(}x\PY{p}{,} b\PY{p}{)}
          abline\PY{p}{(}lsm\PY{p}{)}
\end{Verbatim}


    \begin{center}
    \adjustimage{max size={0.9\linewidth}{0.9\paperheight}}{output_78_0.png}
    \end{center}
    { \hspace*{\fill} \\}
    
    Tada.


    % Add a bibliography block to the postdoc
    
    
    
    \end{document}
