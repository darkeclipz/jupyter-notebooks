
% Default to the notebook output style

    


% Inherit from the specified cell style.




    
\documentclass[11pt]{article}

    
    
    \usepackage[T1]{fontenc}
    % Nicer default font (+ math font) than Computer Modern for most use cases
    \usepackage{mathpazo}

    % Basic figure setup, for now with no caption control since it's done
    % automatically by Pandoc (which extracts ![](path) syntax from Markdown).
    \usepackage{graphicx}
    % We will generate all images so they have a width \maxwidth. This means
    % that they will get their normal width if they fit onto the page, but
    % are scaled down if they would overflow the margins.
    \makeatletter
    \def\maxwidth{\ifdim\Gin@nat@width>\linewidth\linewidth
    \else\Gin@nat@width\fi}
    \makeatother
    \let\Oldincludegraphics\includegraphics
    % Set max figure width to be 80% of text width, for now hardcoded.
    \renewcommand{\includegraphics}[1]{\Oldincludegraphics[width=.8\maxwidth]{#1}}
    % Ensure that by default, figures have no caption (until we provide a
    % proper Figure object with a Caption API and a way to capture that
    % in the conversion process - todo).
    \usepackage{caption}
    \DeclareCaptionLabelFormat{nolabel}{}
    \captionsetup{labelformat=nolabel}

    \usepackage{adjustbox} % Used to constrain images to a maximum size 
    \usepackage{xcolor} % Allow colors to be defined
    \usepackage{enumerate} % Needed for markdown enumerations to work
    \usepackage{geometry} % Used to adjust the document margins
    \usepackage{amsmath} % Equations
    \usepackage{amssymb} % Equations
    \usepackage{textcomp} % defines textquotesingle
    % Hack from http://tex.stackexchange.com/a/47451/13684:
    \AtBeginDocument{%
        \def\PYZsq{\textquotesingle}% Upright quotes in Pygmentized code
    }
    \usepackage{upquote} % Upright quotes for verbatim code
    \usepackage{eurosym} % defines \euro
    \usepackage[mathletters]{ucs} % Extended unicode (utf-8) support
    \usepackage[utf8x]{inputenc} % Allow utf-8 characters in the tex document
    \usepackage{fancyvrb} % verbatim replacement that allows latex
    \usepackage{grffile} % extends the file name processing of package graphics 
                         % to support a larger range 
    % The hyperref package gives us a pdf with properly built
    % internal navigation ('pdf bookmarks' for the table of contents,
    % internal cross-reference links, web links for URLs, etc.)
    \usepackage{hyperref}
    \usepackage{longtable} % longtable support required by pandoc >1.10
    \usepackage{booktabs}  % table support for pandoc > 1.12.2
    \usepackage[inline]{enumitem} % IRkernel/repr support (it uses the enumerate* environment)
    \usepackage[normalem]{ulem} % ulem is needed to support strikethroughs (\sout)
                                % normalem makes italics be italics, not underlines
    

    
    
    % Colors for the hyperref package
    \definecolor{urlcolor}{rgb}{0,.145,.698}
    \definecolor{linkcolor}{rgb}{.71,0.21,0.01}
    \definecolor{citecolor}{rgb}{.12,.54,.11}

    % ANSI colors
    \definecolor{ansi-black}{HTML}{3E424D}
    \definecolor{ansi-black-intense}{HTML}{282C36}
    \definecolor{ansi-red}{HTML}{E75C58}
    \definecolor{ansi-red-intense}{HTML}{B22B31}
    \definecolor{ansi-green}{HTML}{00A250}
    \definecolor{ansi-green-intense}{HTML}{007427}
    \definecolor{ansi-yellow}{HTML}{DDB62B}
    \definecolor{ansi-yellow-intense}{HTML}{B27D12}
    \definecolor{ansi-blue}{HTML}{208FFB}
    \definecolor{ansi-blue-intense}{HTML}{0065CA}
    \definecolor{ansi-magenta}{HTML}{D160C4}
    \definecolor{ansi-magenta-intense}{HTML}{A03196}
    \definecolor{ansi-cyan}{HTML}{60C6C8}
    \definecolor{ansi-cyan-intense}{HTML}{258F8F}
    \definecolor{ansi-white}{HTML}{C5C1B4}
    \definecolor{ansi-white-intense}{HTML}{A1A6B2}

    % commands and environments needed by pandoc snippets
    % extracted from the output of `pandoc -s`
    \providecommand{\tightlist}{%
      \setlength{\itemsep}{0pt}\setlength{\parskip}{0pt}}
    \DefineVerbatimEnvironment{Highlighting}{Verbatim}{commandchars=\\\{\}}
    % Add ',fontsize=\small' for more characters per line
    \newenvironment{Shaded}{}{}
    \newcommand{\KeywordTok}[1]{\textcolor[rgb]{0.00,0.44,0.13}{\textbf{{#1}}}}
    \newcommand{\DataTypeTok}[1]{\textcolor[rgb]{0.56,0.13,0.00}{{#1}}}
    \newcommand{\DecValTok}[1]{\textcolor[rgb]{0.25,0.63,0.44}{{#1}}}
    \newcommand{\BaseNTok}[1]{\textcolor[rgb]{0.25,0.63,0.44}{{#1}}}
    \newcommand{\FloatTok}[1]{\textcolor[rgb]{0.25,0.63,0.44}{{#1}}}
    \newcommand{\CharTok}[1]{\textcolor[rgb]{0.25,0.44,0.63}{{#1}}}
    \newcommand{\StringTok}[1]{\textcolor[rgb]{0.25,0.44,0.63}{{#1}}}
    \newcommand{\CommentTok}[1]{\textcolor[rgb]{0.38,0.63,0.69}{\textit{{#1}}}}
    \newcommand{\OtherTok}[1]{\textcolor[rgb]{0.00,0.44,0.13}{{#1}}}
    \newcommand{\AlertTok}[1]{\textcolor[rgb]{1.00,0.00,0.00}{\textbf{{#1}}}}
    \newcommand{\FunctionTok}[1]{\textcolor[rgb]{0.02,0.16,0.49}{{#1}}}
    \newcommand{\RegionMarkerTok}[1]{{#1}}
    \newcommand{\ErrorTok}[1]{\textcolor[rgb]{1.00,0.00,0.00}{\textbf{{#1}}}}
    \newcommand{\NormalTok}[1]{{#1}}
    
    % Additional commands for more recent versions of Pandoc
    \newcommand{\ConstantTok}[1]{\textcolor[rgb]{0.53,0.00,0.00}{{#1}}}
    \newcommand{\SpecialCharTok}[1]{\textcolor[rgb]{0.25,0.44,0.63}{{#1}}}
    \newcommand{\VerbatimStringTok}[1]{\textcolor[rgb]{0.25,0.44,0.63}{{#1}}}
    \newcommand{\SpecialStringTok}[1]{\textcolor[rgb]{0.73,0.40,0.53}{{#1}}}
    \newcommand{\ImportTok}[1]{{#1}}
    \newcommand{\DocumentationTok}[1]{\textcolor[rgb]{0.73,0.13,0.13}{\textit{{#1}}}}
    \newcommand{\AnnotationTok}[1]{\textcolor[rgb]{0.38,0.63,0.69}{\textbf{\textit{{#1}}}}}
    \newcommand{\CommentVarTok}[1]{\textcolor[rgb]{0.38,0.63,0.69}{\textbf{\textit{{#1}}}}}
    \newcommand{\VariableTok}[1]{\textcolor[rgb]{0.10,0.09,0.49}{{#1}}}
    \newcommand{\ControlFlowTok}[1]{\textcolor[rgb]{0.00,0.44,0.13}{\textbf{{#1}}}}
    \newcommand{\OperatorTok}[1]{\textcolor[rgb]{0.40,0.40,0.40}{{#1}}}
    \newcommand{\BuiltInTok}[1]{{#1}}
    \newcommand{\ExtensionTok}[1]{{#1}}
    \newcommand{\PreprocessorTok}[1]{\textcolor[rgb]{0.74,0.48,0.00}{{#1}}}
    \newcommand{\AttributeTok}[1]{\textcolor[rgb]{0.49,0.56,0.16}{{#1}}}
    \newcommand{\InformationTok}[1]{\textcolor[rgb]{0.38,0.63,0.69}{\textbf{\textit{{#1}}}}}
    \newcommand{\WarningTok}[1]{\textcolor[rgb]{0.38,0.63,0.69}{\textbf{\textit{{#1}}}}}
    
    
    % Define a nice break command that doesn't care if a line doesn't already
    % exist.
    \def\br{\hspace*{\fill} \\* }
    % Math Jax compatability definitions
    \def\gt{>}
    \def\lt{<}
    % Document parameters
    \title{Oefententamen R}
    
    
    

    % Pygments definitions
    
\makeatletter
\def\PY@reset{\let\PY@it=\relax \let\PY@bf=\relax%
    \let\PY@ul=\relax \let\PY@tc=\relax%
    \let\PY@bc=\relax \let\PY@ff=\relax}
\def\PY@tok#1{\csname PY@tok@#1\endcsname}
\def\PY@toks#1+{\ifx\relax#1\empty\else%
    \PY@tok{#1}\expandafter\PY@toks\fi}
\def\PY@do#1{\PY@bc{\PY@tc{\PY@ul{%
    \PY@it{\PY@bf{\PY@ff{#1}}}}}}}
\def\PY#1#2{\PY@reset\PY@toks#1+\relax+\PY@do{#2}}

\expandafter\def\csname PY@tok@w\endcsname{\def\PY@tc##1{\textcolor[rgb]{0.73,0.73,0.73}{##1}}}
\expandafter\def\csname PY@tok@c\endcsname{\let\PY@it=\textit\def\PY@tc##1{\textcolor[rgb]{0.25,0.50,0.50}{##1}}}
\expandafter\def\csname PY@tok@cp\endcsname{\def\PY@tc##1{\textcolor[rgb]{0.74,0.48,0.00}{##1}}}
\expandafter\def\csname PY@tok@k\endcsname{\let\PY@bf=\textbf\def\PY@tc##1{\textcolor[rgb]{0.00,0.50,0.00}{##1}}}
\expandafter\def\csname PY@tok@kp\endcsname{\def\PY@tc##1{\textcolor[rgb]{0.00,0.50,0.00}{##1}}}
\expandafter\def\csname PY@tok@kt\endcsname{\def\PY@tc##1{\textcolor[rgb]{0.69,0.00,0.25}{##1}}}
\expandafter\def\csname PY@tok@o\endcsname{\def\PY@tc##1{\textcolor[rgb]{0.40,0.40,0.40}{##1}}}
\expandafter\def\csname PY@tok@ow\endcsname{\let\PY@bf=\textbf\def\PY@tc##1{\textcolor[rgb]{0.67,0.13,1.00}{##1}}}
\expandafter\def\csname PY@tok@nb\endcsname{\def\PY@tc##1{\textcolor[rgb]{0.00,0.50,0.00}{##1}}}
\expandafter\def\csname PY@tok@nf\endcsname{\def\PY@tc##1{\textcolor[rgb]{0.00,0.00,1.00}{##1}}}
\expandafter\def\csname PY@tok@nc\endcsname{\let\PY@bf=\textbf\def\PY@tc##1{\textcolor[rgb]{0.00,0.00,1.00}{##1}}}
\expandafter\def\csname PY@tok@nn\endcsname{\let\PY@bf=\textbf\def\PY@tc##1{\textcolor[rgb]{0.00,0.00,1.00}{##1}}}
\expandafter\def\csname PY@tok@ne\endcsname{\let\PY@bf=\textbf\def\PY@tc##1{\textcolor[rgb]{0.82,0.25,0.23}{##1}}}
\expandafter\def\csname PY@tok@nv\endcsname{\def\PY@tc##1{\textcolor[rgb]{0.10,0.09,0.49}{##1}}}
\expandafter\def\csname PY@tok@no\endcsname{\def\PY@tc##1{\textcolor[rgb]{0.53,0.00,0.00}{##1}}}
\expandafter\def\csname PY@tok@nl\endcsname{\def\PY@tc##1{\textcolor[rgb]{0.63,0.63,0.00}{##1}}}
\expandafter\def\csname PY@tok@ni\endcsname{\let\PY@bf=\textbf\def\PY@tc##1{\textcolor[rgb]{0.60,0.60,0.60}{##1}}}
\expandafter\def\csname PY@tok@na\endcsname{\def\PY@tc##1{\textcolor[rgb]{0.49,0.56,0.16}{##1}}}
\expandafter\def\csname PY@tok@nt\endcsname{\let\PY@bf=\textbf\def\PY@tc##1{\textcolor[rgb]{0.00,0.50,0.00}{##1}}}
\expandafter\def\csname PY@tok@nd\endcsname{\def\PY@tc##1{\textcolor[rgb]{0.67,0.13,1.00}{##1}}}
\expandafter\def\csname PY@tok@s\endcsname{\def\PY@tc##1{\textcolor[rgb]{0.73,0.13,0.13}{##1}}}
\expandafter\def\csname PY@tok@sd\endcsname{\let\PY@it=\textit\def\PY@tc##1{\textcolor[rgb]{0.73,0.13,0.13}{##1}}}
\expandafter\def\csname PY@tok@si\endcsname{\let\PY@bf=\textbf\def\PY@tc##1{\textcolor[rgb]{0.73,0.40,0.53}{##1}}}
\expandafter\def\csname PY@tok@se\endcsname{\let\PY@bf=\textbf\def\PY@tc##1{\textcolor[rgb]{0.73,0.40,0.13}{##1}}}
\expandafter\def\csname PY@tok@sr\endcsname{\def\PY@tc##1{\textcolor[rgb]{0.73,0.40,0.53}{##1}}}
\expandafter\def\csname PY@tok@ss\endcsname{\def\PY@tc##1{\textcolor[rgb]{0.10,0.09,0.49}{##1}}}
\expandafter\def\csname PY@tok@sx\endcsname{\def\PY@tc##1{\textcolor[rgb]{0.00,0.50,0.00}{##1}}}
\expandafter\def\csname PY@tok@m\endcsname{\def\PY@tc##1{\textcolor[rgb]{0.40,0.40,0.40}{##1}}}
\expandafter\def\csname PY@tok@gh\endcsname{\let\PY@bf=\textbf\def\PY@tc##1{\textcolor[rgb]{0.00,0.00,0.50}{##1}}}
\expandafter\def\csname PY@tok@gu\endcsname{\let\PY@bf=\textbf\def\PY@tc##1{\textcolor[rgb]{0.50,0.00,0.50}{##1}}}
\expandafter\def\csname PY@tok@gd\endcsname{\def\PY@tc##1{\textcolor[rgb]{0.63,0.00,0.00}{##1}}}
\expandafter\def\csname PY@tok@gi\endcsname{\def\PY@tc##1{\textcolor[rgb]{0.00,0.63,0.00}{##1}}}
\expandafter\def\csname PY@tok@gr\endcsname{\def\PY@tc##1{\textcolor[rgb]{1.00,0.00,0.00}{##1}}}
\expandafter\def\csname PY@tok@ge\endcsname{\let\PY@it=\textit}
\expandafter\def\csname PY@tok@gs\endcsname{\let\PY@bf=\textbf}
\expandafter\def\csname PY@tok@gp\endcsname{\let\PY@bf=\textbf\def\PY@tc##1{\textcolor[rgb]{0.00,0.00,0.50}{##1}}}
\expandafter\def\csname PY@tok@go\endcsname{\def\PY@tc##1{\textcolor[rgb]{0.53,0.53,0.53}{##1}}}
\expandafter\def\csname PY@tok@gt\endcsname{\def\PY@tc##1{\textcolor[rgb]{0.00,0.27,0.87}{##1}}}
\expandafter\def\csname PY@tok@err\endcsname{\def\PY@bc##1{\setlength{\fboxsep}{0pt}\fcolorbox[rgb]{1.00,0.00,0.00}{1,1,1}{\strut ##1}}}
\expandafter\def\csname PY@tok@kc\endcsname{\let\PY@bf=\textbf\def\PY@tc##1{\textcolor[rgb]{0.00,0.50,0.00}{##1}}}
\expandafter\def\csname PY@tok@kd\endcsname{\let\PY@bf=\textbf\def\PY@tc##1{\textcolor[rgb]{0.00,0.50,0.00}{##1}}}
\expandafter\def\csname PY@tok@kn\endcsname{\let\PY@bf=\textbf\def\PY@tc##1{\textcolor[rgb]{0.00,0.50,0.00}{##1}}}
\expandafter\def\csname PY@tok@kr\endcsname{\let\PY@bf=\textbf\def\PY@tc##1{\textcolor[rgb]{0.00,0.50,0.00}{##1}}}
\expandafter\def\csname PY@tok@bp\endcsname{\def\PY@tc##1{\textcolor[rgb]{0.00,0.50,0.00}{##1}}}
\expandafter\def\csname PY@tok@fm\endcsname{\def\PY@tc##1{\textcolor[rgb]{0.00,0.00,1.00}{##1}}}
\expandafter\def\csname PY@tok@vc\endcsname{\def\PY@tc##1{\textcolor[rgb]{0.10,0.09,0.49}{##1}}}
\expandafter\def\csname PY@tok@vg\endcsname{\def\PY@tc##1{\textcolor[rgb]{0.10,0.09,0.49}{##1}}}
\expandafter\def\csname PY@tok@vi\endcsname{\def\PY@tc##1{\textcolor[rgb]{0.10,0.09,0.49}{##1}}}
\expandafter\def\csname PY@tok@vm\endcsname{\def\PY@tc##1{\textcolor[rgb]{0.10,0.09,0.49}{##1}}}
\expandafter\def\csname PY@tok@sa\endcsname{\def\PY@tc##1{\textcolor[rgb]{0.73,0.13,0.13}{##1}}}
\expandafter\def\csname PY@tok@sb\endcsname{\def\PY@tc##1{\textcolor[rgb]{0.73,0.13,0.13}{##1}}}
\expandafter\def\csname PY@tok@sc\endcsname{\def\PY@tc##1{\textcolor[rgb]{0.73,0.13,0.13}{##1}}}
\expandafter\def\csname PY@tok@dl\endcsname{\def\PY@tc##1{\textcolor[rgb]{0.73,0.13,0.13}{##1}}}
\expandafter\def\csname PY@tok@s2\endcsname{\def\PY@tc##1{\textcolor[rgb]{0.73,0.13,0.13}{##1}}}
\expandafter\def\csname PY@tok@sh\endcsname{\def\PY@tc##1{\textcolor[rgb]{0.73,0.13,0.13}{##1}}}
\expandafter\def\csname PY@tok@s1\endcsname{\def\PY@tc##1{\textcolor[rgb]{0.73,0.13,0.13}{##1}}}
\expandafter\def\csname PY@tok@mb\endcsname{\def\PY@tc##1{\textcolor[rgb]{0.40,0.40,0.40}{##1}}}
\expandafter\def\csname PY@tok@mf\endcsname{\def\PY@tc##1{\textcolor[rgb]{0.40,0.40,0.40}{##1}}}
\expandafter\def\csname PY@tok@mh\endcsname{\def\PY@tc##1{\textcolor[rgb]{0.40,0.40,0.40}{##1}}}
\expandafter\def\csname PY@tok@mi\endcsname{\def\PY@tc##1{\textcolor[rgb]{0.40,0.40,0.40}{##1}}}
\expandafter\def\csname PY@tok@il\endcsname{\def\PY@tc##1{\textcolor[rgb]{0.40,0.40,0.40}{##1}}}
\expandafter\def\csname PY@tok@mo\endcsname{\def\PY@tc##1{\textcolor[rgb]{0.40,0.40,0.40}{##1}}}
\expandafter\def\csname PY@tok@ch\endcsname{\let\PY@it=\textit\def\PY@tc##1{\textcolor[rgb]{0.25,0.50,0.50}{##1}}}
\expandafter\def\csname PY@tok@cm\endcsname{\let\PY@it=\textit\def\PY@tc##1{\textcolor[rgb]{0.25,0.50,0.50}{##1}}}
\expandafter\def\csname PY@tok@cpf\endcsname{\let\PY@it=\textit\def\PY@tc##1{\textcolor[rgb]{0.25,0.50,0.50}{##1}}}
\expandafter\def\csname PY@tok@c1\endcsname{\let\PY@it=\textit\def\PY@tc##1{\textcolor[rgb]{0.25,0.50,0.50}{##1}}}
\expandafter\def\csname PY@tok@cs\endcsname{\let\PY@it=\textit\def\PY@tc##1{\textcolor[rgb]{0.25,0.50,0.50}{##1}}}

\def\PYZbs{\char`\\}
\def\PYZus{\char`\_}
\def\PYZob{\char`\{}
\def\PYZcb{\char`\}}
\def\PYZca{\char`\^}
\def\PYZam{\char`\&}
\def\PYZlt{\char`\<}
\def\PYZgt{\char`\>}
\def\PYZsh{\char`\#}
\def\PYZpc{\char`\%}
\def\PYZdl{\char`\$}
\def\PYZhy{\char`\-}
\def\PYZsq{\char`\'}
\def\PYZdq{\char`\"}
\def\PYZti{\char`\~}
% for compatibility with earlier versions
\def\PYZat{@}
\def\PYZlb{[}
\def\PYZrb{]}
\makeatother


    % Exact colors from NB
    \definecolor{incolor}{rgb}{0.0, 0.0, 0.5}
    \definecolor{outcolor}{rgb}{0.545, 0.0, 0.0}



    
    % Prevent overflowing lines due to hard-to-break entities
    \sloppy 
    % Setup hyperref package
    \hypersetup{
      breaklinks=true,  % so long urls are correctly broken across lines
      colorlinks=true,
      urlcolor=urlcolor,
      linkcolor=linkcolor,
      citecolor=citecolor,
      }
    % Slightly bigger margins than the latex defaults
    
    \geometry{verbose,tmargin=1in,bmargin=1in,lmargin=1in,rmargin=1in}
    
    

    \begin{document}
    
    
    \maketitle
    
    

    
    \hypertarget{opgave-1}{%
\section{Opgave 1}\label{opgave-1}}

    Importeer de dataset Exam Anxiety van ref{[}1{]} H4.

    \begin{Verbatim}[commandchars=\\\{\}]
{\color{incolor}In [{\color{incolor}82}]:} file \PY{o}{=} \PY{l+s}{\PYZdq{}}\PY{l+s}{C:\PYZbs{}\PYZbs{}Users\PYZbs{}\PYZbs{}Isomorphism\PYZbs{}\PYZbs{}Downloads\PYZbs{}\PYZbs{}Exam Anxiety.dat\PYZdq{}}
         exam \PY{o}{=} read.table\PY{p}{(}\PY{k+kp}{file}\PY{p}{,} header\PY{o}{=}\PY{k+kc}{TRUE}\PY{p}{)}
         \PY{k+kp}{head}\PY{p}{(}exam\PY{p}{)}
\end{Verbatim}


    \begin{tabular}{r|lllll}
 Code & Revise & Exam & Anxiety & Gender\\
\hline
	 1      &  4     & 40     & 86.298 & Male  \\
	 2      & 11     & 65     & 88.716 & Female\\
	 3      & 27     & 80     & 70.178 & Male  \\
	 4      & 53     & 80     & 61.312 & Male  \\
	 5      &  4     & 40     & 89.522 & Male  \\
	 6      & 22     & 70     & 60.506 & Female\\
\end{tabular}


    
    \begin{center}\rule{0.5\linewidth}{\linethickness}\end{center}

    \textbf{1A:} Maak een scatterplot van Exam als functie van Anxiety.

    \begin{Verbatim}[commandchars=\\\{\}]
{\color{incolor}In [{\color{incolor}152}]:} plot\PY{p}{(}exam\PY{o}{\PYZdl{}}Anxiety\PY{p}{,} exam\PY{o}{\PYZdl{}}Exam\PY{p}{,} main\PY{o}{=}\PY{l+s}{\PYZdq{}}\PY{l+s}{Anxiety vs. Exam\PYZdq{}}\PY{p}{,} xlab\PY{o}{=}\PY{l+s}{\PYZdq{}}\PY{l+s}{Anxiety\PYZdq{}}
               \PY{p}{,} ylab\PY{o}{=}\PY{l+s}{\PYZdq{}}\PY{l+s}{Exam\PYZdq{}}\PY{p}{)}
\end{Verbatim}


    \begin{center}
    \adjustimage{max size={0.9\linewidth}{0.9\paperheight}}{output_5_0.png}
    \end{center}
    { \hspace*{\fill} \\}
    
    \textbf{1B:} Bereken de gemiddelde waarde en standaardafwijking van de
variabele Exam.

    \begin{Verbatim}[commandchars=\\\{\}]
{\color{incolor}In [{\color{incolor}98}]:} x \PY{o}{=} \PY{k+kp}{mean}\PY{p}{(}exam\PY{o}{\PYZdl{}}Exam\PY{p}{,} na.rm\PY{o}{=}\PY{k+kc}{TRUE}\PY{p}{)}
         s \PY{o}{=} sd\PY{p}{(}exam\PY{o}{\PYZdl{}}Exam\PY{p}{,} na.rm\PY{o}{=}\PY{k+kc}{TRUE}\PY{p}{)}
\end{Verbatim}


    \begin{Verbatim}[commandchars=\\\{\}]
{\color{incolor}In [{\color{incolor}151}]:} \PY{k+kp}{paste}\PY{p}{(}\PY{l+s}{\PYZdq{}}\PY{l+s}{Het gemiddelde is\PYZdq{}}\PY{p}{,} \PY{k+kp}{round}\PY{p}{(}x\PY{p}{,}\PY{l+m}{3}\PY{p}{)}\PY{p}{,} \PY{l+s}{\PYZdq{}}\PY{l+s}{en de standaardafwijking is\PYZdq{}}
                \PY{p}{,} \PY{k+kp}{round}\PY{p}{(}s\PY{p}{,}\PY{l+m}{3}\PY{p}{)}\PY{p}{,} \PY{l+s}{\PYZdq{}}\PY{l+s}{.\PYZdq{}}\PY{p}{)}
\end{Verbatim}


    'Het gemiddelde is 56.573 en de standaardafwijking is 25.941 .'

    
    \begin{center}\rule{0.5\linewidth}{\linethickness}\end{center}

    \textbf{1C}: Controleer de standaardafwijking met een zelf
geprogrammeerde berekening.

    \begin{Verbatim}[commandchars=\\\{\}]
{\color{incolor}In [{\color{incolor}105}]:} stdev \PY{o}{\PYZlt{}\PYZhy{}} \PY{k+kr}{function}\PY{p}{(}L\PY{p}{)} \PY{k+kp}{sqrt}\PY{p}{(} \PY{k+kp}{sum}\PY{p}{(}\PY{p}{(}L\PY{o}{\PYZhy{}}\PY{k+kp}{mean}\PY{p}{(}L\PY{p}{)}\PY{p}{)}\PY{o}{\PYZca{}}\PY{l+m}{2}\PY{p}{)} \PY{o}{/} \PY{p}{(}\PY{k+kp}{length}\PY{p}{(}L\PY{p}{)} \PY{o}{\PYZhy{}} \PY{l+m}{1}\PY{p}{)} \PY{p}{)}
\end{Verbatim}


    \begin{Verbatim}[commandchars=\\\{\}]
{\color{incolor}In [{\color{incolor}106}]:} stdev\PY{p}{(}exam\PY{o}{\PYZdl{}}Exam\PY{p}{)}
\end{Verbatim}


    25.9405814037326

    
    \begin{center}\rule{0.5\linewidth}{\linethickness}\end{center}

    \textbf{1D}: Maak een boxplot van zowel Anxiety als Exam in een plot.

    \begin{Verbatim}[commandchars=\\\{\}]
{\color{incolor}In [{\color{incolor}115}]:} boxplot\PY{p}{(}exam\PY{o}{\PYZdl{}}Exam\PY{p}{,} exam\PY{o}{\PYZdl{}}Anxiety\PY{p}{)}
\end{Verbatim}


    \begin{center}
    \adjustimage{max size={0.9\linewidth}{0.9\paperheight}}{output_15_0.png}
    \end{center}
    { \hspace*{\fill} \\}
    
    \begin{center}\rule{0.5\linewidth}{\linethickness}\end{center}

    \textbf{1E}: Bereken de afstand van het eerste tot het derde kwartiel
van Anxiety.

    \begin{Verbatim}[commandchars=\\\{\}]
{\color{incolor}In [{\color{incolor}134}]:} q \PY{o}{=} quantile\PY{p}{(}exam\PY{o}{\PYZdl{}}Anxiety\PY{p}{,} \PY{k+kt}{c}\PY{p}{(}\PY{l+m}{.25}\PY{p}{,} \PY{l+m}{.75}\PY{p}{)}\PY{p}{,} na.rm\PY{o}{=}\PY{k+kc}{TRUE}\PY{p}{)}
          \PY{k+kp}{q}
\end{Verbatim}


    \begin{description*}
\item[25\textbackslash{}\%] 69.775
\item[75\textbackslash{}\%] 84.686
\end{description*}


    
    De afstand van het eerste tot het derde kwartiel is:

    \begin{Verbatim}[commandchars=\\\{\}]
{\color{incolor}In [{\color{incolor}133}]:} \PY{k+kp}{q}\PY{p}{[}\PY{l+m}{2}\PY{p}{]}\PY{o}{\PYZhy{}}\PY{k+kp}{q}\PY{p}{[}\PY{l+m}{1}\PY{p}{]}
\end{Verbatim}


    \textbf{75\textbackslash{}\%:} 14.911

    
    Ter controle:

    \begin{Verbatim}[commandchars=\\\{\}]
{\color{incolor}In [{\color{incolor}136}]:} \PY{l+m}{84.686} \PY{o}{\PYZhy{}} \PY{l+m}{69.775}
\end{Verbatim}


    14.911

    
    \begin{center}\rule{0.5\linewidth}{\linethickness}\end{center}

    \textbf{1F}: Zet de data op volgorde en bepaal het dertiende getal.

    \begin{Verbatim}[commandchars=\\\{\}]
{\color{incolor}In [{\color{incolor}143}]:} \PY{k+kp}{sort}\PY{p}{(}exam\PY{o}{\PYZdl{}}Exam\PY{p}{)}\PY{p}{[}\PY{l+m}{13}\PY{p}{]}
\end{Verbatim}


    20

    
    Ter controle:

    \begin{Verbatim}[commandchars=\\\{\}]
{\color{incolor}In [{\color{incolor}142}]:} \PY{k+kp}{head}\PY{p}{(}\PY{k+kp}{sort}\PY{p}{(}exam\PY{o}{\PYZdl{}}Exam\PY{p}{)}\PY{p}{,} \PY{l+m}{15}\PY{p}{)}
\end{Verbatim}


    \begin{enumerate*}
\item 2
\item 5
\item 5
\item 7
\item 10
\item 10
\item 10
\item 10
\item 15
\item 20
\item 20
\item 20
\item 20
\item 20
\item 20
\end{enumerate*}


    
    \begin{center}\rule{0.5\linewidth}{\linethickness}\end{center}

    \textbf{1G}: Maak een histogram van de variabele Exam met als titel
``TW1 histogram van Exam data''.

    \begin{Verbatim}[commandchars=\\\{\}]
{\color{incolor}In [{\color{incolor}144}]:} hist\PY{p}{(}exam\PY{o}{\PYZdl{}}Exam\PY{p}{,} main\PY{o}{=}\PY{l+s}{\PYZdq{}}\PY{l+s}{TW1 histogram van Exam data\PYZdq{}}\PY{p}{)}
\end{Verbatim}


    \begin{center}
    \adjustimage{max size={0.9\linewidth}{0.9\paperheight}}{output_30_0.png}
    \end{center}
    { \hspace*{\fill} \\}
    
    \hypertarget{opgave-2}{%
\section{Opgave 2}\label{opgave-2}}

    Importeer de dataset \texttt{cars} die behoort tot de standaard datasets
in R.

    \begin{Verbatim}[commandchars=\\\{\}]
{\color{incolor}In [{\color{incolor}150}]:} \PY{k+kp}{head}\PY{p}{(}cars\PY{p}{)} \PY{c+c1}{\PYZsh{} Dit importeert het niet, maar laat de eerste }
                     \PY{c+c1}{\PYZsh{} records in de dataset zien.}
\end{Verbatim}


    \begin{tabular}{r|ll}
 speed & dist\\
\hline
	 4  &  2\\
	 4  & 10\\
	 7  &  4\\
	 7  & 22\\
	 8  & 16\\
	 9  & 10\\
\end{tabular}


    
    \begin{center}\rule{0.5\linewidth}{\linethickness}\end{center}

    \textbf{2A:} Plot de stopweg als functie van de snelheid.

Met behulp van \texttt{help(cars)} kan je de eenheden vinden voor de
snelheid en remweg.

    \begin{Verbatim}[commandchars=\\\{\}]
{\color{incolor}In [{\color{incolor}149}]:} plot\PY{p}{(}cars\PY{o}{\PYZdl{}}speed\PY{p}{,} cars\PY{o}{\PYZdl{}}dist\PY{p}{,} main\PY{o}{=}\PY{l+s}{\PYZdq{}}\PY{l+s}{Snelheid vs. remweg\PYZdq{}}\PY{p}{,} xlab\PY{o}{=}\PY{l+s}{\PYZdq{}}\PY{l+s}{Snelheid (mph)\PYZdq{}}
               \PY{p}{,} ylab\PY{o}{=}\PY{l+s}{\PYZdq{}}\PY{l+s}{Remweg (feet)\PYZdq{}}\PY{p}{)}
\end{Verbatim}


    \begin{center}
    \adjustimage{max size={0.9\linewidth}{0.9\paperheight}}{output_36_0.png}
    \end{center}
    { \hspace*{\fill} \\}
    
    \begin{center}\rule{0.5\linewidth}{\linethickness}\end{center}

    \textbf{2B:} Bepaal of er een verband bestaat tussen de snelheid en de
remweg.

    \begin{Verbatim}[commandchars=\\\{\}]
{\color{incolor}In [{\color{incolor}56}]:} r \PY{o}{=} cor\PY{p}{(}cars\PY{o}{\PYZdl{}}speed\PY{p}{,} cars\PY{o}{\PYZdl{}}dist\PY{p}{)}
         r
\end{Verbatim}


    0.80689490068921

    
    Er is een zeer sterk positief verband tussen de variabelen
\texttt{speed} en \texttt{dist}.

    \begin{center}\rule{0.5\linewidth}{\linethickness}\end{center}

    \textbf{2C}: Plot de regressielijn in het rood in het scatterplot.

    \begin{Verbatim}[commandchars=\\\{\}]
{\color{incolor}In [{\color{incolor}57}]:} lreg \PY{o}{=} lm\PY{p}{(}cars\PY{o}{\PYZdl{}}dist\PY{o}{\PYZti{}}cars\PY{o}{\PYZdl{}}speed\PY{p}{)}
         lreg
\end{Verbatim}


    
    \begin{verbatim}

Call:
lm(formula = cars$dist ~ cars$speed)

Coefficients:
(Intercept)   cars$speed  
    -17.579        3.932  

    \end{verbatim}

    
    \begin{Verbatim}[commandchars=\\\{\}]
{\color{incolor}In [{\color{incolor}148}]:} plot\PY{p}{(}cars\PY{o}{\PYZdl{}}speed\PY{p}{,} cars\PY{o}{\PYZdl{}}dist\PY{p}{,} main\PY{o}{=}\PY{l+s}{\PYZdq{}}\PY{l+s}{Snelheid vs. remweg\PYZdq{}}\PY{p}{,} xlab\PY{o}{=}\PY{l+s}{\PYZdq{}}\PY{l+s}{Snelheid (mph)\PYZdq{}}
               \PY{p}{,} ylab\PY{o}{=}\PY{l+s}{\PYZdq{}}\PY{l+s}{Remweg (feet)\PYZdq{}}\PY{p}{)}
          
          abline\PY{p}{(}lreg\PY{p}{,} col\PY{o}{=}\PY{l+s}{\PYZdq{}}\PY{l+s}{red\PYZdq{}}\PY{p}{)}
\end{Verbatim}


    \begin{center}
    \adjustimage{max size={0.9\linewidth}{0.9\paperheight}}{output_44_0.png}
    \end{center}
    { \hspace*{\fill} \\}
    
    \begin{center}\rule{0.5\linewidth}{\linethickness}\end{center}

    \textbf{2D:} Hoe groot is de correlatie tussen de snelheid en de remweg?

    \begin{Verbatim}[commandchars=\\\{\}]
{\color{incolor}In [{\color{incolor}72}]:} r \PY{c+c1}{\PYZsh{} Berekend in 2B met: cor(cars\PYZdl{}speed, cars\PYZdl{}dist)}
\end{Verbatim}


    0.80689490068921

    
    Er is een zeer sterk positief verband tussen de variabelen
\texttt{speed} en \texttt{dist}.

    \hypertarget{opgave-3}{%
\section{Opgave 3}\label{opgave-3}}

    \textbf{3A}: Los het onderstaande stelsel van vergelijkingen op m.b.v.
matrixrekening.

    \[ \begin{aligned}\begin{cases}  x-y-z &= 2 \\ 3x-3y+2z &= 19 \\ 2x-y+z &= 9 \end{cases}\end{aligned} \]

    Eerst zetten we dit om naar een matrix:

    \[ A \cdot \vec{x} = \vec{b} \implies \begin{bmatrix} 1 & -1 & -1 \\ 3 & -3 & 2 \\ 2 & -1 & 1 \end{bmatrix} \cdot \vec{x} = \begin{bmatrix}2\\19\\9\end{bmatrix} \]

    \begin{Verbatim}[commandchars=\\\{\}]
{\color{incolor}In [{\color{incolor}66}]:} x \PY{o}{=} \PY{k+kt}{c}\PY{p}{(}\PY{l+m}{1}\PY{p}{,}\PY{l+m}{3}\PY{p}{,}\PY{l+m}{2}\PY{p}{)}
         y \PY{o}{=} \PY{k+kt}{c}\PY{p}{(}\PY{l+m}{\PYZhy{}1}\PY{p}{,}\PY{l+m}{\PYZhy{}3}\PY{p}{,}\PY{l+m}{\PYZhy{}1}\PY{p}{)}
         z \PY{o}{=} \PY{k+kt}{c}\PY{p}{(}\PY{l+m}{\PYZhy{}1}\PY{p}{,} \PY{l+m}{2}\PY{p}{,} \PY{l+m}{1}\PY{p}{)}
         A \PY{o}{=} \PY{k+kp}{cbind}\PY{p}{(}x\PY{p}{,}y\PY{p}{,}z\PY{p}{)}
         A
\end{Verbatim}


    \begin{tabular}{lll}
 x & y & z\\
\hline
	 1  & -1 & -1\\
	 3  & -3 &  2\\
	 2  & -1 &  1\\
\end{tabular}


    
    \begin{Verbatim}[commandchars=\\\{\}]
{\color{incolor}In [{\color{incolor}67}]:} b \PY{o}{=} \PY{k+kt}{c}\PY{p}{(}\PY{l+m}{2}\PY{p}{,}\PY{l+m}{19}\PY{p}{,}\PY{l+m}{9}\PY{p}{)}
         b
\end{Verbatim}


    \begin{enumerate*}
\item 2
\item 19
\item 9
\end{enumerate*}


    
    Een stelsel van vergelijkingen kunnen we oplossing met de inverse van
\(A\).

\[ \begin{aligned} A \cdot \vec{x} &= \vec{b} \\ A^{-1}\cdot A \cdot \vec{x} &= A^{-1} \cdot \vec{b} \\ I \cdot \vec{x} &= A^{-1}\cdot\vec{b} \end{aligned} \]

De oplossing is dus te vinden met \(\vec{x} = A^{-1} \cdot \vec{b}\).
Eerst bepalen we de inverse van \(A\):

    \begin{Verbatim}[commandchars=\\\{\}]
{\color{incolor}In [{\color{incolor}68}]:} Ainv \PY{o}{=} \PY{k+kp}{solve}\PY{p}{(}A\PY{p}{)}
         Ainv
\end{Verbatim}


    \begin{tabular}{r|lll}
	x &  0.2 & -0.4 & 1   \\
	y & -0.2 & -0.6 & 1   \\
	z & -0.6 &  0.2 & 0   \\
\end{tabular}


    
    En vervolgens het toepassen van matrixvermenigvuldiging (\texttt{\%*\%})
om de oplossing te vinden:

    \begin{Verbatim}[commandchars=\\\{\}]
{\color{incolor}In [{\color{incolor}70}]:} Ainv \PY{o}{\PYZpc{}*\PYZpc{}} b
\end{Verbatim}


    \begin{tabular}{r|l}
	x &  1.8\\
	y & -2.8\\
	z &  2.6\\
\end{tabular}


    
    \begin{center}\rule{0.5\linewidth}{\linethickness}\end{center}

    \textbf{3B:} Controleer je oplossing.

Dit kan ook in een keer met \texttt{solve(A,b)} en dient tevens als
controle:

    \begin{Verbatim}[commandchars=\\\{\}]
{\color{incolor}In [{\color{incolor}43}]:} \PY{k+kp}{solve}\PY{p}{(}A\PY{p}{,}b\PY{p}{)}
\end{Verbatim}


    \begin{description*}
\item[x] 1.8
\item[y] -2.8
\item[z] 2.6
\end{description*}


    

    % Add a bibliography block to the postdoc
    
    
    
    \end{document}
