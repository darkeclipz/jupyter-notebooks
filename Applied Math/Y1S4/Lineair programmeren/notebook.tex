
% Default to the notebook output style

    


% Inherit from the specified cell style.




    
\documentclass[11pt]{article}

    
    
    \usepackage[T1]{fontenc}
    % Nicer default font (+ math font) than Computer Modern for most use cases
    \usepackage{mathpazo}

    % Basic figure setup, for now with no caption control since it's done
    % automatically by Pandoc (which extracts ![](path) syntax from Markdown).
    \usepackage{graphicx}
    % We will generate all images so they have a width \maxwidth. This means
    % that they will get their normal width if they fit onto the page, but
    % are scaled down if they would overflow the margins.
    \makeatletter
    \def\maxwidth{\ifdim\Gin@nat@width>\linewidth\linewidth
    \else\Gin@nat@width\fi}
    \makeatother
    \let\Oldincludegraphics\includegraphics
    % Set max figure width to be 80% of text width, for now hardcoded.
    \renewcommand{\includegraphics}[1]{\Oldincludegraphics[width=.8\maxwidth]{#1}}
    % Ensure that by default, figures have no caption (until we provide a
    % proper Figure object with a Caption API and a way to capture that
    % in the conversion process - todo).
    \usepackage{caption}
    \DeclareCaptionLabelFormat{nolabel}{}
    \captionsetup{labelformat=nolabel}

    \usepackage{adjustbox} % Used to constrain images to a maximum size 
    \usepackage{xcolor} % Allow colors to be defined
    \usepackage{enumerate} % Needed for markdown enumerations to work
    \usepackage{geometry} % Used to adjust the document margins
    \usepackage{amsmath} % Equations
    \usepackage{amssymb} % Equations
    \usepackage{textcomp} % defines textquotesingle
    % Hack from http://tex.stackexchange.com/a/47451/13684:
    \AtBeginDocument{%
        \def\PYZsq{\textquotesingle}% Upright quotes in Pygmentized code
    }
    \usepackage{upquote} % Upright quotes for verbatim code
    \usepackage{eurosym} % defines \euro
    \usepackage[mathletters]{ucs} % Extended unicode (utf-8) support
    \usepackage[utf8x]{inputenc} % Allow utf-8 characters in the tex document
    \usepackage{fancyvrb} % verbatim replacement that allows latex
    \usepackage{grffile} % extends the file name processing of package graphics 
                         % to support a larger range 
    % The hyperref package gives us a pdf with properly built
    % internal navigation ('pdf bookmarks' for the table of contents,
    % internal cross-reference links, web links for URLs, etc.)
    \usepackage{hyperref}
    \usepackage{longtable} % longtable support required by pandoc >1.10
    \usepackage{booktabs}  % table support for pandoc > 1.12.2
    \usepackage[inline]{enumitem} % IRkernel/repr support (it uses the enumerate* environment)
    \usepackage[normalem]{ulem} % ulem is needed to support strikethroughs (\sout)
                                % normalem makes italics be italics, not underlines
    

    
    
    % Colors for the hyperref package
    \definecolor{urlcolor}{rgb}{0,.145,.698}
    \definecolor{linkcolor}{rgb}{.71,0.21,0.01}
    \definecolor{citecolor}{rgb}{.12,.54,.11}

    % ANSI colors
    \definecolor{ansi-black}{HTML}{3E424D}
    \definecolor{ansi-black-intense}{HTML}{282C36}
    \definecolor{ansi-red}{HTML}{E75C58}
    \definecolor{ansi-red-intense}{HTML}{B22B31}
    \definecolor{ansi-green}{HTML}{00A250}
    \definecolor{ansi-green-intense}{HTML}{007427}
    \definecolor{ansi-yellow}{HTML}{DDB62B}
    \definecolor{ansi-yellow-intense}{HTML}{B27D12}
    \definecolor{ansi-blue}{HTML}{208FFB}
    \definecolor{ansi-blue-intense}{HTML}{0065CA}
    \definecolor{ansi-magenta}{HTML}{D160C4}
    \definecolor{ansi-magenta-intense}{HTML}{A03196}
    \definecolor{ansi-cyan}{HTML}{60C6C8}
    \definecolor{ansi-cyan-intense}{HTML}{258F8F}
    \definecolor{ansi-white}{HTML}{C5C1B4}
    \definecolor{ansi-white-intense}{HTML}{A1A6B2}

    % commands and environments needed by pandoc snippets
    % extracted from the output of `pandoc -s`
    \providecommand{\tightlist}{%
      \setlength{\itemsep}{0pt}\setlength{\parskip}{0pt}}
    \DefineVerbatimEnvironment{Highlighting}{Verbatim}{commandchars=\\\{\}}
    % Add ',fontsize=\small' for more characters per line
    \newenvironment{Shaded}{}{}
    \newcommand{\KeywordTok}[1]{\textcolor[rgb]{0.00,0.44,0.13}{\textbf{{#1}}}}
    \newcommand{\DataTypeTok}[1]{\textcolor[rgb]{0.56,0.13,0.00}{{#1}}}
    \newcommand{\DecValTok}[1]{\textcolor[rgb]{0.25,0.63,0.44}{{#1}}}
    \newcommand{\BaseNTok}[1]{\textcolor[rgb]{0.25,0.63,0.44}{{#1}}}
    \newcommand{\FloatTok}[1]{\textcolor[rgb]{0.25,0.63,0.44}{{#1}}}
    \newcommand{\CharTok}[1]{\textcolor[rgb]{0.25,0.44,0.63}{{#1}}}
    \newcommand{\StringTok}[1]{\textcolor[rgb]{0.25,0.44,0.63}{{#1}}}
    \newcommand{\CommentTok}[1]{\textcolor[rgb]{0.38,0.63,0.69}{\textit{{#1}}}}
    \newcommand{\OtherTok}[1]{\textcolor[rgb]{0.00,0.44,0.13}{{#1}}}
    \newcommand{\AlertTok}[1]{\textcolor[rgb]{1.00,0.00,0.00}{\textbf{{#1}}}}
    \newcommand{\FunctionTok}[1]{\textcolor[rgb]{0.02,0.16,0.49}{{#1}}}
    \newcommand{\RegionMarkerTok}[1]{{#1}}
    \newcommand{\ErrorTok}[1]{\textcolor[rgb]{1.00,0.00,0.00}{\textbf{{#1}}}}
    \newcommand{\NormalTok}[1]{{#1}}
    
    % Additional commands for more recent versions of Pandoc
    \newcommand{\ConstantTok}[1]{\textcolor[rgb]{0.53,0.00,0.00}{{#1}}}
    \newcommand{\SpecialCharTok}[1]{\textcolor[rgb]{0.25,0.44,0.63}{{#1}}}
    \newcommand{\VerbatimStringTok}[1]{\textcolor[rgb]{0.25,0.44,0.63}{{#1}}}
    \newcommand{\SpecialStringTok}[1]{\textcolor[rgb]{0.73,0.40,0.53}{{#1}}}
    \newcommand{\ImportTok}[1]{{#1}}
    \newcommand{\DocumentationTok}[1]{\textcolor[rgb]{0.73,0.13,0.13}{\textit{{#1}}}}
    \newcommand{\AnnotationTok}[1]{\textcolor[rgb]{0.38,0.63,0.69}{\textbf{\textit{{#1}}}}}
    \newcommand{\CommentVarTok}[1]{\textcolor[rgb]{0.38,0.63,0.69}{\textbf{\textit{{#1}}}}}
    \newcommand{\VariableTok}[1]{\textcolor[rgb]{0.10,0.09,0.49}{{#1}}}
    \newcommand{\ControlFlowTok}[1]{\textcolor[rgb]{0.00,0.44,0.13}{\textbf{{#1}}}}
    \newcommand{\OperatorTok}[1]{\textcolor[rgb]{0.40,0.40,0.40}{{#1}}}
    \newcommand{\BuiltInTok}[1]{{#1}}
    \newcommand{\ExtensionTok}[1]{{#1}}
    \newcommand{\PreprocessorTok}[1]{\textcolor[rgb]{0.74,0.48,0.00}{{#1}}}
    \newcommand{\AttributeTok}[1]{\textcolor[rgb]{0.49,0.56,0.16}{{#1}}}
    \newcommand{\InformationTok}[1]{\textcolor[rgb]{0.38,0.63,0.69}{\textbf{\textit{{#1}}}}}
    \newcommand{\WarningTok}[1]{\textcolor[rgb]{0.38,0.63,0.69}{\textbf{\textit{{#1}}}}}
    
    
    % Define a nice break command that doesn't care if a line doesn't already
    % exist.
    \def\br{\hspace*{\fill} \\* }
    % Math Jax compatability definitions
    \def\gt{>}
    \def\lt{<}
    % Document parameters
    \title{Opgave 34}
    
    
    

    % Pygments definitions
    
\makeatletter
\def\PY@reset{\let\PY@it=\relax \let\PY@bf=\relax%
    \let\PY@ul=\relax \let\PY@tc=\relax%
    \let\PY@bc=\relax \let\PY@ff=\relax}
\def\PY@tok#1{\csname PY@tok@#1\endcsname}
\def\PY@toks#1+{\ifx\relax#1\empty\else%
    \PY@tok{#1}\expandafter\PY@toks\fi}
\def\PY@do#1{\PY@bc{\PY@tc{\PY@ul{%
    \PY@it{\PY@bf{\PY@ff{#1}}}}}}}
\def\PY#1#2{\PY@reset\PY@toks#1+\relax+\PY@do{#2}}

\expandafter\def\csname PY@tok@w\endcsname{\def\PY@tc##1{\textcolor[rgb]{0.73,0.73,0.73}{##1}}}
\expandafter\def\csname PY@tok@c\endcsname{\let\PY@it=\textit\def\PY@tc##1{\textcolor[rgb]{0.25,0.50,0.50}{##1}}}
\expandafter\def\csname PY@tok@cp\endcsname{\def\PY@tc##1{\textcolor[rgb]{0.74,0.48,0.00}{##1}}}
\expandafter\def\csname PY@tok@k\endcsname{\let\PY@bf=\textbf\def\PY@tc##1{\textcolor[rgb]{0.00,0.50,0.00}{##1}}}
\expandafter\def\csname PY@tok@kp\endcsname{\def\PY@tc##1{\textcolor[rgb]{0.00,0.50,0.00}{##1}}}
\expandafter\def\csname PY@tok@kt\endcsname{\def\PY@tc##1{\textcolor[rgb]{0.69,0.00,0.25}{##1}}}
\expandafter\def\csname PY@tok@o\endcsname{\def\PY@tc##1{\textcolor[rgb]{0.40,0.40,0.40}{##1}}}
\expandafter\def\csname PY@tok@ow\endcsname{\let\PY@bf=\textbf\def\PY@tc##1{\textcolor[rgb]{0.67,0.13,1.00}{##1}}}
\expandafter\def\csname PY@tok@nb\endcsname{\def\PY@tc##1{\textcolor[rgb]{0.00,0.50,0.00}{##1}}}
\expandafter\def\csname PY@tok@nf\endcsname{\def\PY@tc##1{\textcolor[rgb]{0.00,0.00,1.00}{##1}}}
\expandafter\def\csname PY@tok@nc\endcsname{\let\PY@bf=\textbf\def\PY@tc##1{\textcolor[rgb]{0.00,0.00,1.00}{##1}}}
\expandafter\def\csname PY@tok@nn\endcsname{\let\PY@bf=\textbf\def\PY@tc##1{\textcolor[rgb]{0.00,0.00,1.00}{##1}}}
\expandafter\def\csname PY@tok@ne\endcsname{\let\PY@bf=\textbf\def\PY@tc##1{\textcolor[rgb]{0.82,0.25,0.23}{##1}}}
\expandafter\def\csname PY@tok@nv\endcsname{\def\PY@tc##1{\textcolor[rgb]{0.10,0.09,0.49}{##1}}}
\expandafter\def\csname PY@tok@no\endcsname{\def\PY@tc##1{\textcolor[rgb]{0.53,0.00,0.00}{##1}}}
\expandafter\def\csname PY@tok@nl\endcsname{\def\PY@tc##1{\textcolor[rgb]{0.63,0.63,0.00}{##1}}}
\expandafter\def\csname PY@tok@ni\endcsname{\let\PY@bf=\textbf\def\PY@tc##1{\textcolor[rgb]{0.60,0.60,0.60}{##1}}}
\expandafter\def\csname PY@tok@na\endcsname{\def\PY@tc##1{\textcolor[rgb]{0.49,0.56,0.16}{##1}}}
\expandafter\def\csname PY@tok@nt\endcsname{\let\PY@bf=\textbf\def\PY@tc##1{\textcolor[rgb]{0.00,0.50,0.00}{##1}}}
\expandafter\def\csname PY@tok@nd\endcsname{\def\PY@tc##1{\textcolor[rgb]{0.67,0.13,1.00}{##1}}}
\expandafter\def\csname PY@tok@s\endcsname{\def\PY@tc##1{\textcolor[rgb]{0.73,0.13,0.13}{##1}}}
\expandafter\def\csname PY@tok@sd\endcsname{\let\PY@it=\textit\def\PY@tc##1{\textcolor[rgb]{0.73,0.13,0.13}{##1}}}
\expandafter\def\csname PY@tok@si\endcsname{\let\PY@bf=\textbf\def\PY@tc##1{\textcolor[rgb]{0.73,0.40,0.53}{##1}}}
\expandafter\def\csname PY@tok@se\endcsname{\let\PY@bf=\textbf\def\PY@tc##1{\textcolor[rgb]{0.73,0.40,0.13}{##1}}}
\expandafter\def\csname PY@tok@sr\endcsname{\def\PY@tc##1{\textcolor[rgb]{0.73,0.40,0.53}{##1}}}
\expandafter\def\csname PY@tok@ss\endcsname{\def\PY@tc##1{\textcolor[rgb]{0.10,0.09,0.49}{##1}}}
\expandafter\def\csname PY@tok@sx\endcsname{\def\PY@tc##1{\textcolor[rgb]{0.00,0.50,0.00}{##1}}}
\expandafter\def\csname PY@tok@m\endcsname{\def\PY@tc##1{\textcolor[rgb]{0.40,0.40,0.40}{##1}}}
\expandafter\def\csname PY@tok@gh\endcsname{\let\PY@bf=\textbf\def\PY@tc##1{\textcolor[rgb]{0.00,0.00,0.50}{##1}}}
\expandafter\def\csname PY@tok@gu\endcsname{\let\PY@bf=\textbf\def\PY@tc##1{\textcolor[rgb]{0.50,0.00,0.50}{##1}}}
\expandafter\def\csname PY@tok@gd\endcsname{\def\PY@tc##1{\textcolor[rgb]{0.63,0.00,0.00}{##1}}}
\expandafter\def\csname PY@tok@gi\endcsname{\def\PY@tc##1{\textcolor[rgb]{0.00,0.63,0.00}{##1}}}
\expandafter\def\csname PY@tok@gr\endcsname{\def\PY@tc##1{\textcolor[rgb]{1.00,0.00,0.00}{##1}}}
\expandafter\def\csname PY@tok@ge\endcsname{\let\PY@it=\textit}
\expandafter\def\csname PY@tok@gs\endcsname{\let\PY@bf=\textbf}
\expandafter\def\csname PY@tok@gp\endcsname{\let\PY@bf=\textbf\def\PY@tc##1{\textcolor[rgb]{0.00,0.00,0.50}{##1}}}
\expandafter\def\csname PY@tok@go\endcsname{\def\PY@tc##1{\textcolor[rgb]{0.53,0.53,0.53}{##1}}}
\expandafter\def\csname PY@tok@gt\endcsname{\def\PY@tc##1{\textcolor[rgb]{0.00,0.27,0.87}{##1}}}
\expandafter\def\csname PY@tok@err\endcsname{\def\PY@bc##1{\setlength{\fboxsep}{0pt}\fcolorbox[rgb]{1.00,0.00,0.00}{1,1,1}{\strut ##1}}}
\expandafter\def\csname PY@tok@kc\endcsname{\let\PY@bf=\textbf\def\PY@tc##1{\textcolor[rgb]{0.00,0.50,0.00}{##1}}}
\expandafter\def\csname PY@tok@kd\endcsname{\let\PY@bf=\textbf\def\PY@tc##1{\textcolor[rgb]{0.00,0.50,0.00}{##1}}}
\expandafter\def\csname PY@tok@kn\endcsname{\let\PY@bf=\textbf\def\PY@tc##1{\textcolor[rgb]{0.00,0.50,0.00}{##1}}}
\expandafter\def\csname PY@tok@kr\endcsname{\let\PY@bf=\textbf\def\PY@tc##1{\textcolor[rgb]{0.00,0.50,0.00}{##1}}}
\expandafter\def\csname PY@tok@bp\endcsname{\def\PY@tc##1{\textcolor[rgb]{0.00,0.50,0.00}{##1}}}
\expandafter\def\csname PY@tok@fm\endcsname{\def\PY@tc##1{\textcolor[rgb]{0.00,0.00,1.00}{##1}}}
\expandafter\def\csname PY@tok@vc\endcsname{\def\PY@tc##1{\textcolor[rgb]{0.10,0.09,0.49}{##1}}}
\expandafter\def\csname PY@tok@vg\endcsname{\def\PY@tc##1{\textcolor[rgb]{0.10,0.09,0.49}{##1}}}
\expandafter\def\csname PY@tok@vi\endcsname{\def\PY@tc##1{\textcolor[rgb]{0.10,0.09,0.49}{##1}}}
\expandafter\def\csname PY@tok@vm\endcsname{\def\PY@tc##1{\textcolor[rgb]{0.10,0.09,0.49}{##1}}}
\expandafter\def\csname PY@tok@sa\endcsname{\def\PY@tc##1{\textcolor[rgb]{0.73,0.13,0.13}{##1}}}
\expandafter\def\csname PY@tok@sb\endcsname{\def\PY@tc##1{\textcolor[rgb]{0.73,0.13,0.13}{##1}}}
\expandafter\def\csname PY@tok@sc\endcsname{\def\PY@tc##1{\textcolor[rgb]{0.73,0.13,0.13}{##1}}}
\expandafter\def\csname PY@tok@dl\endcsname{\def\PY@tc##1{\textcolor[rgb]{0.73,0.13,0.13}{##1}}}
\expandafter\def\csname PY@tok@s2\endcsname{\def\PY@tc##1{\textcolor[rgb]{0.73,0.13,0.13}{##1}}}
\expandafter\def\csname PY@tok@sh\endcsname{\def\PY@tc##1{\textcolor[rgb]{0.73,0.13,0.13}{##1}}}
\expandafter\def\csname PY@tok@s1\endcsname{\def\PY@tc##1{\textcolor[rgb]{0.73,0.13,0.13}{##1}}}
\expandafter\def\csname PY@tok@mb\endcsname{\def\PY@tc##1{\textcolor[rgb]{0.40,0.40,0.40}{##1}}}
\expandafter\def\csname PY@tok@mf\endcsname{\def\PY@tc##1{\textcolor[rgb]{0.40,0.40,0.40}{##1}}}
\expandafter\def\csname PY@tok@mh\endcsname{\def\PY@tc##1{\textcolor[rgb]{0.40,0.40,0.40}{##1}}}
\expandafter\def\csname PY@tok@mi\endcsname{\def\PY@tc##1{\textcolor[rgb]{0.40,0.40,0.40}{##1}}}
\expandafter\def\csname PY@tok@il\endcsname{\def\PY@tc##1{\textcolor[rgb]{0.40,0.40,0.40}{##1}}}
\expandafter\def\csname PY@tok@mo\endcsname{\def\PY@tc##1{\textcolor[rgb]{0.40,0.40,0.40}{##1}}}
\expandafter\def\csname PY@tok@ch\endcsname{\let\PY@it=\textit\def\PY@tc##1{\textcolor[rgb]{0.25,0.50,0.50}{##1}}}
\expandafter\def\csname PY@tok@cm\endcsname{\let\PY@it=\textit\def\PY@tc##1{\textcolor[rgb]{0.25,0.50,0.50}{##1}}}
\expandafter\def\csname PY@tok@cpf\endcsname{\let\PY@it=\textit\def\PY@tc##1{\textcolor[rgb]{0.25,0.50,0.50}{##1}}}
\expandafter\def\csname PY@tok@c1\endcsname{\let\PY@it=\textit\def\PY@tc##1{\textcolor[rgb]{0.25,0.50,0.50}{##1}}}
\expandafter\def\csname PY@tok@cs\endcsname{\let\PY@it=\textit\def\PY@tc##1{\textcolor[rgb]{0.25,0.50,0.50}{##1}}}

\def\PYZbs{\char`\\}
\def\PYZus{\char`\_}
\def\PYZob{\char`\{}
\def\PYZcb{\char`\}}
\def\PYZca{\char`\^}
\def\PYZam{\char`\&}
\def\PYZlt{\char`\<}
\def\PYZgt{\char`\>}
\def\PYZsh{\char`\#}
\def\PYZpc{\char`\%}
\def\PYZdl{\char`\$}
\def\PYZhy{\char`\-}
\def\PYZsq{\char`\'}
\def\PYZdq{\char`\"}
\def\PYZti{\char`\~}
% for compatibility with earlier versions
\def\PYZat{@}
\def\PYZlb{[}
\def\PYZrb{]}
\makeatother


    % Exact colors from NB
    \definecolor{incolor}{rgb}{0.0, 0.0, 0.5}
    \definecolor{outcolor}{rgb}{0.545, 0.0, 0.0}



    
    % Prevent overflowing lines due to hard-to-break entities
    \sloppy 
    % Setup hyperref package
    \hypersetup{
      breaklinks=true,  % so long urls are correctly broken across lines
      colorlinks=true,
      urlcolor=urlcolor,
      linkcolor=linkcolor,
      citecolor=citecolor,
      }
    % Slightly bigger margins than the latex defaults
    
    \geometry{verbose,tmargin=1in,bmargin=1in,lmargin=1in,rmargin=1in}
    
    

    \begin{document}
    
    
    \maketitle
    
    

    
    In dit document worden de onderstaande instellingen/functies gebruikt:

    \begin{Verbatim}[commandchars=\\\{\}]
{\color{incolor}In [{\color{incolor}77}]:} \PY{c+c1}{\PYZsh{} Remove the scientific notation for \PYZsq{}big\PYZsq{} numbers.}
         \PY{k+kp}{options}\PY{p}{(}scipen\PY{o}{=}\PY{l+m}{999}\PY{p}{)}
\end{Verbatim}


    \begin{Verbatim}[commandchars=\\\{\}]
{\color{incolor}In [{\color{incolor}78}]:} \PY{c+c1}{\PYZsh{} Add the column and row numbers to the matrix for easier indexing.}
         \PY{c+c1}{\PYZsh{} It preserves the original naming.}
         colrownames \PY{o}{\PYZlt{}\PYZhy{}} \PY{k+kr}{function}\PY{p}{(}M\PY{p}{)} \PY{p}{\PYZob{}}
             cn\PY{o}{=}\PY{k+kp}{colnames}\PY{p}{(}M\PY{p}{)}
             cn\PY{o}{=}\PY{k+kp}{paste}\PY{p}{(}cn\PY{p}{,} \PY{l+s}{\PYZsq{}}\PY{l+s}{ (\PYZsq{}}\PY{p}{,} \PY{k+kt}{c}\PY{p}{(}\PY{l+m}{1}\PY{o}{:}\PY{k+kp}{length}\PY{p}{(}M\PY{p}{[}\PY{l+m}{1}\PY{p}{,}\PY{p}{]}\PY{p}{)}\PY{p}{)}\PY{p}{,} \PY{l+s}{\PYZsq{}}\PY{l+s}{)\PYZsq{}}\PY{p}{,} sep\PY{o}{=}\PY{l+s}{\PYZsq{}}\PY{l+s}{\PYZsq{}}\PY{p}{)}
             \PY{k+kp}{colnames}\PY{p}{(}M\PY{p}{)}\PY{o}{=}cn
             rn\PY{o}{=}\PY{k+kp}{rownames}\PY{p}{(}M\PY{p}{)}
             rn\PY{o}{=}\PY{k+kp}{paste}\PY{p}{(}rn\PY{p}{,} \PY{l+s}{\PYZsq{}}\PY{l+s}{ (\PYZsq{}}\PY{p}{,} \PY{k+kt}{c}\PY{p}{(}\PY{l+m}{1}\PY{o}{:}\PY{k+kp}{length}\PY{p}{(}M\PY{p}{[}\PY{p}{,}\PY{l+m}{1}\PY{p}{]}\PY{p}{)}\PY{p}{)}\PY{p}{,} \PY{l+s}{\PYZsq{}}\PY{l+s}{)\PYZsq{}}\PY{p}{,} sep\PY{o}{=}\PY{l+s}{\PYZsq{}}\PY{l+s}{\PYZsq{}}\PY{p}{)}
             \PY{k+kp}{rownames}\PY{p}{(}M\PY{p}{)}\PY{o}{=}rn
             M
         \PY{p}{\PYZcb{}}
\end{Verbatim}


    \hypertarget{lp-model}{%
\section{LP model}\label{lp-model}}

    Het LP-model voor opgave 34 is als volgt:

\[ \begin{aligned}
\min 3000x+4500y&+4000z\\
x+y+z &= 100 \qquad&\text{(1)} \\
x &\geq 30 \qquad&\text{(2)} \\
-x+y &\leq 0 \qquad&\text{(3)} \\
-y+2z &\leq 0 \qquad&\text{(4)}\\
z &\leq 10 \qquad&\text{(5)}
\end{aligned}\]

    Alle mogelijke combinaties van \(\leq, \geq, =\) komen hierin voor. Ook
moet het probleem worden omgezet in een maximalisatieprobleem. Doordat
er kunstmatige variabelen nodig zijn, wordt er een twee-fasen model
gebruikt.

    \hypertarget{canonieke-vorm}{%
\section{Canonieke vorm}\label{canonieke-vorm}}

    Eerst stellen we de doelfunctie op: \(d=3000x+4500y+4000z\). Omdat dit
een minimalisatie probleem is, en we er een maximalisatieprobleem van
willen maken, wordt \(d^*=-d\). De doelfunctie wordt dan:
\(-d=-3000x-4500y-4000z \iff d^*=-3000x-4500y-4000z\).

Vervolgens herschrijven we de andere restricties in de canonieke vorm,
dit geeft de volgende vergelijkingen:

\[ \tag{0} d^*+3000x+4500y+4000z = 0 \] \[ \tag{1} x+y+z+A_1=100 \]
\[ \tag{2} x-S_2+A_2=30 \] \[ \tag{3} -x+y+s_3=0 \]
\[ \tag{4} -y+2z+s_4=0 \] \[ \tag{5} z+s_5=10 \]

    Omdat er kunstmatige variabelen worden gebruikt hebben we een extra
doelfunctie nodig, namelijk: \(A=-A_1-A_2\). Deze doelfunctie wordt
gemaximaliseerd tot \(0\). In de canonieke vorm schrijven we dit als
\(A+A_1+A_2=0 \quad(0^*)\). Echter mogen \(A_1\) en \(A_2\) maar één
keer in de vergelijkingen voorkomen omdat dit basisvariabelen zijn. Dit
probleem is als volgt op te lossen: \((0^*)-(1)-(2)\). Dit geeft:
\(A+A_1+A_2-(x+y+z+A_1)-(x-S_2+A_2)=0-100-30\) ofwel,
\(A-2x-y-z+S_2=-130\).

Hieronder staan alle vergelijkingen in de canonieke vorm:

\[ \tag{0*} A-2x-y-z+S_2=-130 \] \[ \tag{0} d^*+3000x+4500y+4000z = 0 \]
\[ \tag{1} x+y+z+A_1=100 \] \[ \tag{2} x-S_2+A_2=30 \]
\[ \tag{3} -x+y+s_3=0 \] \[ \tag{4} -y+2z+s_4=0 \]
\[ \tag{5} z+s_5=10 \]

    Nu kunnen we dit omzetten naar een simplex tableau.

    \hypertarget{simplex}{%
\section{Simplex}\label{simplex}}

    Eerst stellen we het begintableau op:

    \begin{Verbatim}[commandchars=\\\{\}]
{\color{incolor}In [{\color{incolor}68}]:} A\PY{o}{=}\PY{k+kt}{c}\PY{p}{(}\PY{l+m}{1}\PY{p}{,}\PY{l+m}{0}\PY{p}{,}\PY{l+m}{0}\PY{p}{,}\PY{l+m}{0}\PY{p}{,}\PY{l+m}{0}\PY{p}{,}\PY{l+m}{0}\PY{p}{,}\PY{l+m}{0}\PY{p}{)}
         d\PY{o}{=}\PY{k+kt}{c}\PY{p}{(}\PY{l+m}{0}\PY{p}{,}\PY{l+m}{1}\PY{p}{,}\PY{l+m}{0}\PY{p}{,}\PY{l+m}{0}\PY{p}{,}\PY{l+m}{0}\PY{p}{,}\PY{l+m}{0}\PY{p}{,}\PY{l+m}{0}\PY{p}{)}
         x\PY{o}{=}\PY{k+kt}{c}\PY{p}{(}\PY{l+m}{\PYZhy{}2}\PY{p}{,}\PY{l+m}{3000}\PY{p}{,}\PY{l+m}{1}\PY{p}{,}\PY{l+m}{1}\PY{p}{,}\PY{l+m}{\PYZhy{}1}\PY{p}{,}\PY{l+m}{0}\PY{p}{,}\PY{l+m}{0}\PY{p}{)}
         y\PY{o}{=}\PY{k+kt}{c}\PY{p}{(}\PY{l+m}{\PYZhy{}1}\PY{p}{,}\PY{l+m}{4500}\PY{p}{,}\PY{l+m}{1}\PY{p}{,}\PY{l+m}{0}\PY{p}{,}\PY{l+m}{1}\PY{p}{,}\PY{l+m}{\PYZhy{}1}\PY{p}{,}\PY{l+m}{0}\PY{p}{)}
         z\PY{o}{=}\PY{k+kt}{c}\PY{p}{(}\PY{l+m}{\PYZhy{}1}\PY{p}{,}\PY{l+m}{4000}\PY{p}{,}\PY{l+m}{1}\PY{p}{,}\PY{l+m}{0}\PY{p}{,}\PY{l+m}{0}\PY{p}{,}\PY{l+m}{2}\PY{p}{,}\PY{l+m}{1}\PY{p}{)}
         A1\PY{o}{=}\PY{k+kt}{c}\PY{p}{(}\PY{l+m}{0}\PY{p}{,}\PY{l+m}{0}\PY{p}{,}\PY{l+m}{1}\PY{p}{,}\PY{l+m}{0}\PY{p}{,}\PY{l+m}{0}\PY{p}{,}\PY{l+m}{0}\PY{p}{,}\PY{l+m}{0}\PY{p}{)}
         S2\PY{o}{=}\PY{k+kt}{c}\PY{p}{(}\PY{l+m}{1}\PY{p}{,}\PY{l+m}{0}\PY{p}{,}\PY{l+m}{0}\PY{p}{,}\PY{l+m}{\PYZhy{}1}\PY{p}{,}\PY{l+m}{0}\PY{p}{,}\PY{l+m}{0}\PY{p}{,}\PY{l+m}{0}\PY{p}{)}
         A2\PY{o}{=}\PY{k+kt}{c}\PY{p}{(}\PY{l+m}{0}\PY{p}{,}\PY{l+m}{0}\PY{p}{,}\PY{l+m}{0}\PY{p}{,}\PY{l+m}{1}\PY{p}{,}\PY{l+m}{0}\PY{p}{,}\PY{l+m}{0}\PY{p}{,}\PY{l+m}{0}\PY{p}{)}
         s3\PY{o}{=}\PY{k+kt}{c}\PY{p}{(}\PY{l+m}{0}\PY{p}{,}\PY{l+m}{0}\PY{p}{,}\PY{l+m}{0}\PY{p}{,}\PY{l+m}{0}\PY{p}{,}\PY{l+m}{1}\PY{p}{,}\PY{l+m}{0}\PY{p}{,}\PY{l+m}{0}\PY{p}{)}
         s4\PY{o}{=}\PY{k+kt}{c}\PY{p}{(}\PY{l+m}{0}\PY{p}{,}\PY{l+m}{0}\PY{p}{,}\PY{l+m}{0}\PY{p}{,}\PY{l+m}{0}\PY{p}{,}\PY{l+m}{0}\PY{p}{,}\PY{l+m}{1}\PY{p}{,}\PY{l+m}{0}\PY{p}{)}
         s5\PY{o}{=}\PY{k+kt}{c}\PY{p}{(}\PY{l+m}{0}\PY{p}{,}\PY{l+m}{0}\PY{p}{,}\PY{l+m}{0}\PY{p}{,}\PY{l+m}{0}\PY{p}{,}\PY{l+m}{0}\PY{p}{,}\PY{l+m}{0}\PY{p}{,}\PY{l+m}{1}\PY{p}{)}
         RHS\PY{o}{=}\PY{k+kt}{c}\PY{p}{(}\PY{l+m}{\PYZhy{}130}\PY{p}{,}\PY{l+m}{0}\PY{p}{,}\PY{l+m}{100}\PY{p}{,}\PY{l+m}{30}\PY{p}{,}\PY{l+m}{0}\PY{p}{,}\PY{l+m}{0}\PY{p}{,}\PY{l+m}{10}\PY{p}{)}
         M\PY{o}{=}\PY{k+kp}{cbind}\PY{p}{(}A\PY{p}{,}d\PY{p}{,}x\PY{p}{,}y\PY{p}{,}z\PY{p}{,}A1\PY{p}{,}S2\PY{p}{,}A2\PY{p}{,}s3\PY{p}{,}s4\PY{p}{,}s5\PY{p}{,}RHS\PY{p}{)}
         \PY{k+kp}{rownames}\PY{p}{(}M\PY{p}{)}\PY{o}{=}\PY{k+kt}{c}\PY{p}{(}\PY{l+s}{\PYZsq{}}\PY{l+s}{A\PYZsq{}}\PY{p}{,}\PY{l+s}{\PYZsq{}}\PY{l+s}{d*\PYZsq{}}\PY{p}{,}\PY{l+s}{\PYZsq{}}\PY{l+s}{A1\PYZsq{}}\PY{p}{,}\PY{l+s}{\PYZsq{}}\PY{l+s}{A2\PYZsq{}}\PY{p}{,}\PY{l+s}{\PYZsq{}}\PY{l+s}{s3\PYZsq{}}\PY{p}{,}\PY{l+s}{\PYZsq{}}\PY{l+s}{s4\PYZsq{}}\PY{p}{,} \PY{l+s}{\PYZsq{}}\PY{l+s}{s5\PYZsq{}}\PY{p}{)}
         M\PY{o}{=}colrownames\PY{p}{(}M\PY{p}{)}
         
         \PY{c+c1}{\PYZsh{} Iteratie 1}
         M
\end{Verbatim}


    \begin{tabular}{r|llllllllllll}
  & A (1) & d (2) & x (3) & y (4) & z (5) & A1 (6) & S2 (7) & A2 (8) & s3 (9) & s4 (10) & s5 (11) & RHS (12)\\
\hline
	A (1) & 1    & 0    &   -2 &   -1 &   -1 & 0    &  1   & 0    & 0    & 0    & 0    & -130\\
	d* (2) & 0    & 1    & 3000 & 4500 & 4000 & 0    &  0   & 0    & 0    & 0    & 0    &    0\\
	A1 (3) & 0    & 0    &    1 &    1 &    1 & 1    &  0   & 0    & 0    & 0    & 0    &  100\\
	A2 (4) & 0    & 0    &    1 &    0 &    0 & 0    & -1   & 1    & 0    & 0    & 0    &   30\\
	s3 (5) & 0    & 0    &   -1 &    1 &    0 & 0    &  0   & 0    & 1    & 0    & 0    &    0\\
	s4 (6) & 0    & 0    &    0 &   -1 &    2 & 0    &  0   & 0    & 0    & 1    & 0    &    0\\
	s5 (7) & 0    & 0    &    0 &    0 &    1 & 0    &  0   & 0    & 0    & 0    & 1    &   10\\
\end{tabular}


    
    Wat opvalt in het begintableau, met uitzondering van vergelijking
\((0^*)\), voor de toegevoegde variabelen is het volgende:

\begin{enumerate}
\def\labelenumi{\arabic{enumi}.}
\tightlist
\item
  Een spelingsvariabel (SlackN) \(s_i\) is altijd \(\geq\) 0 in het
  begintableau.
\item
  Een surplusvariabel (Surp.N) \(S_i\) is altijd \(\leq\) 0 in het
  begintableau.
\item
  Een kunstmatige variabel (Art.N) \(A_i\) is altijd \(\geq\) 0 in het
  begintableau.
\end{enumerate}

Deze observatie kan worden gebruikt om het begintableau op correctheid
te controleren, voordat er wordt begonnen met vegen.

De meeste negatieve waarde is in dit geval \(x=-2\), vegen geeft:

    \begin{Verbatim}[commandchars=\\\{\}]
{\color{incolor}In [{\color{incolor}69}]:} I2\PY{o}{=}M
         I2\PY{p}{[}\PY{l+m}{1}\PY{p}{,}\PY{p}{]}\PY{o}{=}I2\PY{p}{[}\PY{l+m}{1}\PY{p}{,}\PY{p}{]}\PY{l+m}{+2}\PY{o}{*}I2\PY{p}{[}\PY{l+m}{4}\PY{p}{,}\PY{p}{]}
         I2\PY{p}{[}\PY{l+m}{2}\PY{p}{,}\PY{p}{]}\PY{o}{=}I2\PY{p}{[}\PY{l+m}{2}\PY{p}{,}\PY{p}{]}\PY{l+m}{\PYZhy{}3000}\PY{o}{*}I2\PY{p}{[}\PY{l+m}{4}\PY{p}{,}\PY{p}{]}
         I2\PY{p}{[}\PY{l+m}{3}\PY{p}{,}\PY{p}{]}\PY{o}{=}I2\PY{p}{[}\PY{l+m}{3}\PY{p}{,}\PY{p}{]}\PY{o}{\PYZhy{}}I2\PY{p}{[}\PY{l+m}{4}\PY{p}{,}\PY{p}{]}
         I2\PY{p}{[}\PY{l+m}{5}\PY{p}{,}\PY{p}{]}\PY{o}{=}I2\PY{p}{[}\PY{l+m}{5}\PY{p}{,}\PY{p}{]}\PY{o}{+}I2\PY{p}{[}\PY{l+m}{4}\PY{p}{,}\PY{p}{]}
         I2
\end{Verbatim}


    \begin{tabular}{r|llllllllllll}
  & A (1) & d (2) & x (3) & y (4) & z (5) & A1 (6) & S2 (7) & A2 (8) & s3 (9) & s4 (10) & s5 (11) & RHS (12)\\
\hline
	A (1) & 1      & 0      & 0      &   -1   &   -1   & 0      &   -1   &     2  & 0      & 0      & 0      &    -70\\
	d* (2) & 0      & 1      & 0      & 4500   & 4000   & 0      & 3000   & -3000  & 0      & 0      & 0      & -90000\\
	A1 (3) & 0      & 0      & 0      &    1   &    1   & 1      &    1   &    -1  & 0      & 0      & 0      &     70\\
	A2 (4) & 0      & 0      & 1      &    0   &    0   & 0      &   -1   &     1  & 0      & 0      & 0      &     30\\
	s3 (5) & 0      & 0      & 0      &    1   &    0   & 0      &   -1   &     1  & 1      & 0      & 0      &     30\\
	s4 (6) & 0      & 0      & 0      &   -1   &    2   & 0      &    0   &     0  & 0      & 1      & 0      &      0\\
	s5 (7) & 0      & 0      & 0      &    0   &    1   & 0      &    0   &     0  & 0      & 0      & 1      &     10\\
\end{tabular}


    
    De meeste negatieve waarde is in dit geval \(y=-1\) en \(z=-1\), beiden
zijn gelijk, dus het wordt \(y\) omdat dit de eerste is.

    \begin{Verbatim}[commandchars=\\\{\}]
{\color{incolor}In [{\color{incolor}70}]:} I3\PY{o}{=}I2
         I3\PY{p}{[}\PY{l+m}{1}\PY{p}{,}\PY{p}{]}\PY{o}{=}I3\PY{p}{[}\PY{l+m}{1}\PY{p}{,}\PY{p}{]}\PY{o}{+}I3\PY{p}{[}\PY{l+m}{5}\PY{p}{,}\PY{p}{]}
         I3\PY{p}{[}\PY{l+m}{2}\PY{p}{,}\PY{p}{]}\PY{o}{=}I3\PY{p}{[}\PY{l+m}{2}\PY{p}{,}\PY{p}{]}\PY{l+m}{\PYZhy{}4500}\PY{o}{*}I3\PY{p}{[}\PY{l+m}{5}\PY{p}{,}\PY{p}{]}
         I3\PY{p}{[}\PY{l+m}{3}\PY{p}{,}\PY{p}{]}\PY{o}{=}I3\PY{p}{[}\PY{l+m}{3}\PY{p}{,}\PY{p}{]}\PY{o}{\PYZhy{}}I3\PY{p}{[}\PY{l+m}{5}\PY{p}{,}\PY{p}{]}
         I3\PY{p}{[}\PY{l+m}{6}\PY{p}{,}\PY{p}{]}\PY{o}{=}I3\PY{p}{[}\PY{l+m}{6}\PY{p}{,}\PY{p}{]}\PY{o}{+}I3\PY{p}{[}\PY{l+m}{5}\PY{p}{,}\PY{p}{]}
         I3
\end{Verbatim}


    \begin{tabular}{r|llllllllllll}
  & A (1) & d (2) & x (3) & y (4) & z (5) & A1 (6) & S2 (7) & A2 (8) & s3 (9) & s4 (10) & s5 (11) & RHS (12)\\
\hline
	A (1) & 1       & 0       & 0       & 0       &   -1    & 0       &   -2    &     3   &     1   & 0       & 0       &     -40\\
	d* (2) & 0       & 1       & 0       & 0       & 4000    & 0       & 7500    & -7500   & -4500   & 0       & 0       & -225000\\
	A1 (3) & 0       & 0       & 0       & 0       &    1    & 1       &    2    &    -2   &    -1   & 0       & 0       &      40\\
	A2 (4) & 0       & 0       & 1       & 0       &    0    & 0       &   -1    &     1   &     0   & 0       & 0       &      30\\
	s3 (5) & 0       & 0       & 0       & 1       &    0    & 0       &   -1    &     1   &     1   & 0       & 0       &      30\\
	s4 (6) & 0       & 0       & 0       & 0       &    2    & 0       &   -1    &     1   &     1   & 1       & 0       &      30\\
	s5 (7) & 0       & 0       & 0       & 0       &    1    & 0       &    0    &     0   &     0   & 0       & 1       &      10\\
\end{tabular}


    
    De meeste negatieve waarde is nu \(S_2\), vegen geeft:

    \begin{Verbatim}[commandchars=\\\{\}]
{\color{incolor}In [{\color{incolor}71}]:} I4\PY{o}{=}I3
         I4\PY{p}{[}\PY{l+m}{1}\PY{p}{,}\PY{p}{]}\PY{o}{=}I4\PY{p}{[}\PY{l+m}{1}\PY{p}{,}\PY{p}{]}\PY{o}{+}I4\PY{p}{[}\PY{l+m}{3}\PY{p}{,}\PY{p}{]}
         I4\PY{p}{[}\PY{l+m}{3}\PY{p}{,}\PY{p}{]}\PY{o}{=}I4\PY{p}{[}\PY{l+m}{3}\PY{p}{,}\PY{p}{]}
         I4\PY{p}{[}\PY{l+m}{2}\PY{p}{,}\PY{p}{]}\PY{o}{=}I4\PY{p}{[}\PY{l+m}{2}\PY{p}{,}\PY{p}{]}\PY{l+m}{\PYZhy{}7500}\PY{o}{/}\PY{l+m}{2}\PY{o}{*}I4\PY{p}{[}\PY{l+m}{3}\PY{p}{,}\PY{p}{]}
         I4\PY{p}{[}\PY{l+m}{4}\PY{p}{,}\PY{p}{]}\PY{o}{=}I4\PY{p}{[}\PY{l+m}{4}\PY{p}{,}\PY{p}{]}\PY{l+m}{+1}\PY{o}{/}\PY{l+m}{2}\PY{o}{*}I4\PY{p}{[}\PY{l+m}{3}\PY{p}{,}\PY{p}{]}
         I4\PY{p}{[}\PY{l+m}{5}\PY{p}{,}\PY{p}{]}\PY{o}{=}I4\PY{p}{[}\PY{l+m}{5}\PY{p}{,}\PY{p}{]}\PY{l+m}{+1}\PY{o}{/}\PY{l+m}{2}\PY{o}{*}I4\PY{p}{[}\PY{l+m}{3}\PY{p}{,}\PY{p}{]}
         I4\PY{p}{[}\PY{l+m}{6}\PY{p}{,}\PY{p}{]}\PY{o}{=}I4\PY{p}{[}\PY{l+m}{6}\PY{p}{,}\PY{p}{]}\PY{l+m}{+1}\PY{o}{/}\PY{l+m}{2}\PY{o}{*}I4\PY{p}{[}\PY{l+m}{3}\PY{p}{,}\PY{p}{]}
         I4\PY{p}{[}\PY{l+m}{3}\PY{p}{,}\PY{p}{]}\PY{o}{=}I4\PY{p}{[}\PY{l+m}{3}\PY{p}{,}\PY{p}{]}\PY{o}{/}\PY{l+m}{2}
         I4
\end{Verbatim}


    \begin{tabular}{r|llllllllllll}
  & A (1) & d (2) & x (3) & y (4) & z (5) & A1 (6) & S2 (7) & A2 (8) & s3 (9) & s4 (10) & s5 (11) & RHS (12)\\
\hline
	A (1) & 1       & 0       & 0       & 0       &   0.0   &     1.0 & 0       &  1      &    0.0  & 0       & 0       &       0\\
	d* (2) & 0       & 1       & 0       & 0       & 250.0   & -3750.0 & 0       &  0      & -750.0  & 0       & 0       & -375000\\
	A1 (3) & 0       & 0       & 0       & 0       &   0.5   &     0.5 & 1       & -1      &   -0.5  & 0       & 0       &      20\\
	A2 (4) & 0       & 0       & 1       & 0       &   0.5   &     0.5 & 0       &  0      &   -0.5  & 0       & 0       &      50\\
	s3 (5) & 0       & 0       & 0       & 1       &   0.5   &     0.5 & 0       &  0      &    0.5  & 0       & 0       &      50\\
	s4 (6) & 0       & 0       & 0       & 0       &   2.5   &     0.5 & 0       &  0      &    0.5  & 1       & 0       &      50\\
	s5 (7) & 0       & 0       & 0       & 0       &   1.0   &     0.0 & 0       &  0      &    0.0  & 0       & 1       &      10\\
\end{tabular}


    
    Nu zijn we klaar met de eerste fase en is \(A=0\). Vervolgens gaan we
verder met het optimaliseren van de doelfunctie \(d^*\). Let er op dat
we nu alle kolommen en rijen met \(A\) negeren.

De meeste negatieve waarde is \(A_1\), maar deze kolom telt niet meer
mee. De volgende waarde is \(s_3\), vegen geeft:

    \begin{Verbatim}[commandchars=\\\{\}]
{\color{incolor}In [{\color{incolor}76}]:} I5\PY{o}{=}I4
         I5\PY{p}{[}\PY{l+m}{5}\PY{p}{,}\PY{p}{]}\PY{o}{=}I5\PY{p}{[}\PY{l+m}{5}\PY{p}{,}\PY{p}{]}\PY{o}{*}\PY{l+m}{2}
         I5\PY{p}{[}\PY{l+m}{2}\PY{p}{,}\PY{p}{]}\PY{o}{=}I5\PY{p}{[}\PY{l+m}{2}\PY{p}{,}\PY{p}{]}\PY{o}{+}I5\PY{p}{[}\PY{l+m}{5}\PY{p}{,}\PY{p}{]}\PY{o}{*}\PY{l+m}{750}
         I5
\end{Verbatim}


    \begin{tabular}{r|llllllllllll}
  & A (1) & d (2) & x (3) & y (4) & z (5) & A1 (6) & S2 (7) & A2 (8) & s3 (9) & s4 (10) & s5 (11) & RHS (12)\\
\hline
	A (1) & 1       & 0       & 0       &    0    &    0.0  &     1.0 & 0       &  1      &  0.0    & 0       & 0       &       0\\
	d* (2) & 0       & 1       & 0       & 1500    & 1000.0  & -3000.0 & 0       &  0      &  0.0    & 0       & 0       & -300000\\
	A1 (3) & 0       & 0       & 0       &    0    &    0.5  &     0.5 & 1       & -1      & -0.5    & 0       & 0       &      20\\
	A2 (4) & 0       & 0       & 1       &    0    &    0.5  &     0.5 & 0       &  0      & -0.5    & 0       & 0       &      50\\
	s3 (5) & 0       & 0       & 0       &    2    &    1.0  &     1.0 & 0       &  0      &  1.0    & 0       & 0       &     100\\
	s4 (6) & 0       & 0       & 0       &    0    &    2.5  &     0.5 & 0       &  0      &  0.5    & 1       & 0       &      50\\
	s5 (7) & 0       & 0       & 0       &    0    &    1.0  &     0.0 & 0       &  0      &  0.0    & 0       & 1       &      10\\
\end{tabular}


    
    Alle getallen in de rij \(d^*\) zijn nu positief, we zijn dus klaar. De
rechterkant is nu \(-300.000\), en als we \(d^*=-d\) toepassen (het was
een minimalisatieprobleem), dan is de minimale waarde \(300.000\).

De oplossing is \((x,y,z)=(0,1500,1000)\) met een minimale waarde van
\(€300.000\) .


    % Add a bibliography block to the postdoc
    
    
    
    \end{document}
