
% Default to the notebook output style

    


% Inherit from the specified cell style.




    
\documentclass[11pt]{article}

    
    
    \usepackage[T1]{fontenc}
    % Nicer default font (+ math font) than Computer Modern for most use cases
    \usepackage{mathpazo}

    % Basic figure setup, for now with no caption control since it's done
    % automatically by Pandoc (which extracts ![](path) syntax from Markdown).
    \usepackage{graphicx}
    % We will generate all images so they have a width \maxwidth. This means
    % that they will get their normal width if they fit onto the page, but
    % are scaled down if they would overflow the margins.
    \makeatletter
    \def\maxwidth{\ifdim\Gin@nat@width>\linewidth\linewidth
    \else\Gin@nat@width\fi}
    \makeatother
    \let\Oldincludegraphics\includegraphics
    % Set max figure width to be 80% of text width, for now hardcoded.
    \renewcommand{\includegraphics}[1]{\Oldincludegraphics[width=.8\maxwidth]{#1}}
    % Ensure that by default, figures have no caption (until we provide a
    % proper Figure object with a Caption API and a way to capture that
    % in the conversion process - todo).
    \usepackage{caption}
    \DeclareCaptionLabelFormat{nolabel}{}
    \captionsetup{labelformat=nolabel}

    \usepackage{adjustbox} % Used to constrain images to a maximum size 
    \usepackage{xcolor} % Allow colors to be defined
    \usepackage{enumerate} % Needed for markdown enumerations to work
    \usepackage{geometry} % Used to adjust the document margins
    \usepackage{amsmath} % Equations
    \usepackage{amssymb} % Equations
    \usepackage{textcomp} % defines textquotesingle
    % Hack from http://tex.stackexchange.com/a/47451/13684:
    \AtBeginDocument{%
        \def\PYZsq{\textquotesingle}% Upright quotes in Pygmentized code
    }
    \usepackage{upquote} % Upright quotes for verbatim code
    \usepackage{eurosym} % defines \euro
    \usepackage[mathletters]{ucs} % Extended unicode (utf-8) support
    \usepackage[utf8x]{inputenc} % Allow utf-8 characters in the tex document
    \usepackage{fancyvrb} % verbatim replacement that allows latex
    \usepackage{grffile} % extends the file name processing of package graphics 
                         % to support a larger range 
    % The hyperref package gives us a pdf with properly built
    % internal navigation ('pdf bookmarks' for the table of contents,
    % internal cross-reference links, web links for URLs, etc.)
    \usepackage{hyperref}
    \usepackage{longtable} % longtable support required by pandoc >1.10
    \usepackage{booktabs}  % table support for pandoc > 1.12.2
    \usepackage[inline]{enumitem} % IRkernel/repr support (it uses the enumerate* environment)
    \usepackage[normalem]{ulem} % ulem is needed to support strikethroughs (\sout)
                                % normalem makes italics be italics, not underlines
    

    
    
    % Colors for the hyperref package
    \definecolor{urlcolor}{rgb}{0,.145,.698}
    \definecolor{linkcolor}{rgb}{.71,0.21,0.01}
    \definecolor{citecolor}{rgb}{.12,.54,.11}

    % ANSI colors
    \definecolor{ansi-black}{HTML}{3E424D}
    \definecolor{ansi-black-intense}{HTML}{282C36}
    \definecolor{ansi-red}{HTML}{E75C58}
    \definecolor{ansi-red-intense}{HTML}{B22B31}
    \definecolor{ansi-green}{HTML}{00A250}
    \definecolor{ansi-green-intense}{HTML}{007427}
    \definecolor{ansi-yellow}{HTML}{DDB62B}
    \definecolor{ansi-yellow-intense}{HTML}{B27D12}
    \definecolor{ansi-blue}{HTML}{208FFB}
    \definecolor{ansi-blue-intense}{HTML}{0065CA}
    \definecolor{ansi-magenta}{HTML}{D160C4}
    \definecolor{ansi-magenta-intense}{HTML}{A03196}
    \definecolor{ansi-cyan}{HTML}{60C6C8}
    \definecolor{ansi-cyan-intense}{HTML}{258F8F}
    \definecolor{ansi-white}{HTML}{C5C1B4}
    \definecolor{ansi-white-intense}{HTML}{A1A6B2}

    % commands and environments needed by pandoc snippets
    % extracted from the output of `pandoc -s`
    \providecommand{\tightlist}{%
      \setlength{\itemsep}{0pt}\setlength{\parskip}{0pt}}
    \DefineVerbatimEnvironment{Highlighting}{Verbatim}{commandchars=\\\{\}}
    % Add ',fontsize=\small' for more characters per line
    \newenvironment{Shaded}{}{}
    \newcommand{\KeywordTok}[1]{\textcolor[rgb]{0.00,0.44,0.13}{\textbf{{#1}}}}
    \newcommand{\DataTypeTok}[1]{\textcolor[rgb]{0.56,0.13,0.00}{{#1}}}
    \newcommand{\DecValTok}[1]{\textcolor[rgb]{0.25,0.63,0.44}{{#1}}}
    \newcommand{\BaseNTok}[1]{\textcolor[rgb]{0.25,0.63,0.44}{{#1}}}
    \newcommand{\FloatTok}[1]{\textcolor[rgb]{0.25,0.63,0.44}{{#1}}}
    \newcommand{\CharTok}[1]{\textcolor[rgb]{0.25,0.44,0.63}{{#1}}}
    \newcommand{\StringTok}[1]{\textcolor[rgb]{0.25,0.44,0.63}{{#1}}}
    \newcommand{\CommentTok}[1]{\textcolor[rgb]{0.38,0.63,0.69}{\textit{{#1}}}}
    \newcommand{\OtherTok}[1]{\textcolor[rgb]{0.00,0.44,0.13}{{#1}}}
    \newcommand{\AlertTok}[1]{\textcolor[rgb]{1.00,0.00,0.00}{\textbf{{#1}}}}
    \newcommand{\FunctionTok}[1]{\textcolor[rgb]{0.02,0.16,0.49}{{#1}}}
    \newcommand{\RegionMarkerTok}[1]{{#1}}
    \newcommand{\ErrorTok}[1]{\textcolor[rgb]{1.00,0.00,0.00}{\textbf{{#1}}}}
    \newcommand{\NormalTok}[1]{{#1}}
    
    % Additional commands for more recent versions of Pandoc
    \newcommand{\ConstantTok}[1]{\textcolor[rgb]{0.53,0.00,0.00}{{#1}}}
    \newcommand{\SpecialCharTok}[1]{\textcolor[rgb]{0.25,0.44,0.63}{{#1}}}
    \newcommand{\VerbatimStringTok}[1]{\textcolor[rgb]{0.25,0.44,0.63}{{#1}}}
    \newcommand{\SpecialStringTok}[1]{\textcolor[rgb]{0.73,0.40,0.53}{{#1}}}
    \newcommand{\ImportTok}[1]{{#1}}
    \newcommand{\DocumentationTok}[1]{\textcolor[rgb]{0.73,0.13,0.13}{\textit{{#1}}}}
    \newcommand{\AnnotationTok}[1]{\textcolor[rgb]{0.38,0.63,0.69}{\textbf{\textit{{#1}}}}}
    \newcommand{\CommentVarTok}[1]{\textcolor[rgb]{0.38,0.63,0.69}{\textbf{\textit{{#1}}}}}
    \newcommand{\VariableTok}[1]{\textcolor[rgb]{0.10,0.09,0.49}{{#1}}}
    \newcommand{\ControlFlowTok}[1]{\textcolor[rgb]{0.00,0.44,0.13}{\textbf{{#1}}}}
    \newcommand{\OperatorTok}[1]{\textcolor[rgb]{0.40,0.40,0.40}{{#1}}}
    \newcommand{\BuiltInTok}[1]{{#1}}
    \newcommand{\ExtensionTok}[1]{{#1}}
    \newcommand{\PreprocessorTok}[1]{\textcolor[rgb]{0.74,0.48,0.00}{{#1}}}
    \newcommand{\AttributeTok}[1]{\textcolor[rgb]{0.49,0.56,0.16}{{#1}}}
    \newcommand{\InformationTok}[1]{\textcolor[rgb]{0.38,0.63,0.69}{\textbf{\textit{{#1}}}}}
    \newcommand{\WarningTok}[1]{\textcolor[rgb]{0.38,0.63,0.69}{\textbf{\textit{{#1}}}}}
    
    
    % Define a nice break command that doesn't care if a line doesn't already
    % exist.
    \def\br{\hspace*{\fill} \\* }
    % Math Jax compatability definitions
    \def\gt{>}
    \def\lt{<}
    % Document parameters
    \title{Simplex method (Linear programming)}
    
    
    

    % Pygments definitions
    
\makeatletter
\def\PY@reset{\let\PY@it=\relax \let\PY@bf=\relax%
    \let\PY@ul=\relax \let\PY@tc=\relax%
    \let\PY@bc=\relax \let\PY@ff=\relax}
\def\PY@tok#1{\csname PY@tok@#1\endcsname}
\def\PY@toks#1+{\ifx\relax#1\empty\else%
    \PY@tok{#1}\expandafter\PY@toks\fi}
\def\PY@do#1{\PY@bc{\PY@tc{\PY@ul{%
    \PY@it{\PY@bf{\PY@ff{#1}}}}}}}
\def\PY#1#2{\PY@reset\PY@toks#1+\relax+\PY@do{#2}}

\expandafter\def\csname PY@tok@w\endcsname{\def\PY@tc##1{\textcolor[rgb]{0.73,0.73,0.73}{##1}}}
\expandafter\def\csname PY@tok@c\endcsname{\let\PY@it=\textit\def\PY@tc##1{\textcolor[rgb]{0.25,0.50,0.50}{##1}}}
\expandafter\def\csname PY@tok@cp\endcsname{\def\PY@tc##1{\textcolor[rgb]{0.74,0.48,0.00}{##1}}}
\expandafter\def\csname PY@tok@k\endcsname{\let\PY@bf=\textbf\def\PY@tc##1{\textcolor[rgb]{0.00,0.50,0.00}{##1}}}
\expandafter\def\csname PY@tok@kp\endcsname{\def\PY@tc##1{\textcolor[rgb]{0.00,0.50,0.00}{##1}}}
\expandafter\def\csname PY@tok@kt\endcsname{\def\PY@tc##1{\textcolor[rgb]{0.69,0.00,0.25}{##1}}}
\expandafter\def\csname PY@tok@o\endcsname{\def\PY@tc##1{\textcolor[rgb]{0.40,0.40,0.40}{##1}}}
\expandafter\def\csname PY@tok@ow\endcsname{\let\PY@bf=\textbf\def\PY@tc##1{\textcolor[rgb]{0.67,0.13,1.00}{##1}}}
\expandafter\def\csname PY@tok@nb\endcsname{\def\PY@tc##1{\textcolor[rgb]{0.00,0.50,0.00}{##1}}}
\expandafter\def\csname PY@tok@nf\endcsname{\def\PY@tc##1{\textcolor[rgb]{0.00,0.00,1.00}{##1}}}
\expandafter\def\csname PY@tok@nc\endcsname{\let\PY@bf=\textbf\def\PY@tc##1{\textcolor[rgb]{0.00,0.00,1.00}{##1}}}
\expandafter\def\csname PY@tok@nn\endcsname{\let\PY@bf=\textbf\def\PY@tc##1{\textcolor[rgb]{0.00,0.00,1.00}{##1}}}
\expandafter\def\csname PY@tok@ne\endcsname{\let\PY@bf=\textbf\def\PY@tc##1{\textcolor[rgb]{0.82,0.25,0.23}{##1}}}
\expandafter\def\csname PY@tok@nv\endcsname{\def\PY@tc##1{\textcolor[rgb]{0.10,0.09,0.49}{##1}}}
\expandafter\def\csname PY@tok@no\endcsname{\def\PY@tc##1{\textcolor[rgb]{0.53,0.00,0.00}{##1}}}
\expandafter\def\csname PY@tok@nl\endcsname{\def\PY@tc##1{\textcolor[rgb]{0.63,0.63,0.00}{##1}}}
\expandafter\def\csname PY@tok@ni\endcsname{\let\PY@bf=\textbf\def\PY@tc##1{\textcolor[rgb]{0.60,0.60,0.60}{##1}}}
\expandafter\def\csname PY@tok@na\endcsname{\def\PY@tc##1{\textcolor[rgb]{0.49,0.56,0.16}{##1}}}
\expandafter\def\csname PY@tok@nt\endcsname{\let\PY@bf=\textbf\def\PY@tc##1{\textcolor[rgb]{0.00,0.50,0.00}{##1}}}
\expandafter\def\csname PY@tok@nd\endcsname{\def\PY@tc##1{\textcolor[rgb]{0.67,0.13,1.00}{##1}}}
\expandafter\def\csname PY@tok@s\endcsname{\def\PY@tc##1{\textcolor[rgb]{0.73,0.13,0.13}{##1}}}
\expandafter\def\csname PY@tok@sd\endcsname{\let\PY@it=\textit\def\PY@tc##1{\textcolor[rgb]{0.73,0.13,0.13}{##1}}}
\expandafter\def\csname PY@tok@si\endcsname{\let\PY@bf=\textbf\def\PY@tc##1{\textcolor[rgb]{0.73,0.40,0.53}{##1}}}
\expandafter\def\csname PY@tok@se\endcsname{\let\PY@bf=\textbf\def\PY@tc##1{\textcolor[rgb]{0.73,0.40,0.13}{##1}}}
\expandafter\def\csname PY@tok@sr\endcsname{\def\PY@tc##1{\textcolor[rgb]{0.73,0.40,0.53}{##1}}}
\expandafter\def\csname PY@tok@ss\endcsname{\def\PY@tc##1{\textcolor[rgb]{0.10,0.09,0.49}{##1}}}
\expandafter\def\csname PY@tok@sx\endcsname{\def\PY@tc##1{\textcolor[rgb]{0.00,0.50,0.00}{##1}}}
\expandafter\def\csname PY@tok@m\endcsname{\def\PY@tc##1{\textcolor[rgb]{0.40,0.40,0.40}{##1}}}
\expandafter\def\csname PY@tok@gh\endcsname{\let\PY@bf=\textbf\def\PY@tc##1{\textcolor[rgb]{0.00,0.00,0.50}{##1}}}
\expandafter\def\csname PY@tok@gu\endcsname{\let\PY@bf=\textbf\def\PY@tc##1{\textcolor[rgb]{0.50,0.00,0.50}{##1}}}
\expandafter\def\csname PY@tok@gd\endcsname{\def\PY@tc##1{\textcolor[rgb]{0.63,0.00,0.00}{##1}}}
\expandafter\def\csname PY@tok@gi\endcsname{\def\PY@tc##1{\textcolor[rgb]{0.00,0.63,0.00}{##1}}}
\expandafter\def\csname PY@tok@gr\endcsname{\def\PY@tc##1{\textcolor[rgb]{1.00,0.00,0.00}{##1}}}
\expandafter\def\csname PY@tok@ge\endcsname{\let\PY@it=\textit}
\expandafter\def\csname PY@tok@gs\endcsname{\let\PY@bf=\textbf}
\expandafter\def\csname PY@tok@gp\endcsname{\let\PY@bf=\textbf\def\PY@tc##1{\textcolor[rgb]{0.00,0.00,0.50}{##1}}}
\expandafter\def\csname PY@tok@go\endcsname{\def\PY@tc##1{\textcolor[rgb]{0.53,0.53,0.53}{##1}}}
\expandafter\def\csname PY@tok@gt\endcsname{\def\PY@tc##1{\textcolor[rgb]{0.00,0.27,0.87}{##1}}}
\expandafter\def\csname PY@tok@err\endcsname{\def\PY@bc##1{\setlength{\fboxsep}{0pt}\fcolorbox[rgb]{1.00,0.00,0.00}{1,1,1}{\strut ##1}}}
\expandafter\def\csname PY@tok@kc\endcsname{\let\PY@bf=\textbf\def\PY@tc##1{\textcolor[rgb]{0.00,0.50,0.00}{##1}}}
\expandafter\def\csname PY@tok@kd\endcsname{\let\PY@bf=\textbf\def\PY@tc##1{\textcolor[rgb]{0.00,0.50,0.00}{##1}}}
\expandafter\def\csname PY@tok@kn\endcsname{\let\PY@bf=\textbf\def\PY@tc##1{\textcolor[rgb]{0.00,0.50,0.00}{##1}}}
\expandafter\def\csname PY@tok@kr\endcsname{\let\PY@bf=\textbf\def\PY@tc##1{\textcolor[rgb]{0.00,0.50,0.00}{##1}}}
\expandafter\def\csname PY@tok@bp\endcsname{\def\PY@tc##1{\textcolor[rgb]{0.00,0.50,0.00}{##1}}}
\expandafter\def\csname PY@tok@fm\endcsname{\def\PY@tc##1{\textcolor[rgb]{0.00,0.00,1.00}{##1}}}
\expandafter\def\csname PY@tok@vc\endcsname{\def\PY@tc##1{\textcolor[rgb]{0.10,0.09,0.49}{##1}}}
\expandafter\def\csname PY@tok@vg\endcsname{\def\PY@tc##1{\textcolor[rgb]{0.10,0.09,0.49}{##1}}}
\expandafter\def\csname PY@tok@vi\endcsname{\def\PY@tc##1{\textcolor[rgb]{0.10,0.09,0.49}{##1}}}
\expandafter\def\csname PY@tok@vm\endcsname{\def\PY@tc##1{\textcolor[rgb]{0.10,0.09,0.49}{##1}}}
\expandafter\def\csname PY@tok@sa\endcsname{\def\PY@tc##1{\textcolor[rgb]{0.73,0.13,0.13}{##1}}}
\expandafter\def\csname PY@tok@sb\endcsname{\def\PY@tc##1{\textcolor[rgb]{0.73,0.13,0.13}{##1}}}
\expandafter\def\csname PY@tok@sc\endcsname{\def\PY@tc##1{\textcolor[rgb]{0.73,0.13,0.13}{##1}}}
\expandafter\def\csname PY@tok@dl\endcsname{\def\PY@tc##1{\textcolor[rgb]{0.73,0.13,0.13}{##1}}}
\expandafter\def\csname PY@tok@s2\endcsname{\def\PY@tc##1{\textcolor[rgb]{0.73,0.13,0.13}{##1}}}
\expandafter\def\csname PY@tok@sh\endcsname{\def\PY@tc##1{\textcolor[rgb]{0.73,0.13,0.13}{##1}}}
\expandafter\def\csname PY@tok@s1\endcsname{\def\PY@tc##1{\textcolor[rgb]{0.73,0.13,0.13}{##1}}}
\expandafter\def\csname PY@tok@mb\endcsname{\def\PY@tc##1{\textcolor[rgb]{0.40,0.40,0.40}{##1}}}
\expandafter\def\csname PY@tok@mf\endcsname{\def\PY@tc##1{\textcolor[rgb]{0.40,0.40,0.40}{##1}}}
\expandafter\def\csname PY@tok@mh\endcsname{\def\PY@tc##1{\textcolor[rgb]{0.40,0.40,0.40}{##1}}}
\expandafter\def\csname PY@tok@mi\endcsname{\def\PY@tc##1{\textcolor[rgb]{0.40,0.40,0.40}{##1}}}
\expandafter\def\csname PY@tok@il\endcsname{\def\PY@tc##1{\textcolor[rgb]{0.40,0.40,0.40}{##1}}}
\expandafter\def\csname PY@tok@mo\endcsname{\def\PY@tc##1{\textcolor[rgb]{0.40,0.40,0.40}{##1}}}
\expandafter\def\csname PY@tok@ch\endcsname{\let\PY@it=\textit\def\PY@tc##1{\textcolor[rgb]{0.25,0.50,0.50}{##1}}}
\expandafter\def\csname PY@tok@cm\endcsname{\let\PY@it=\textit\def\PY@tc##1{\textcolor[rgb]{0.25,0.50,0.50}{##1}}}
\expandafter\def\csname PY@tok@cpf\endcsname{\let\PY@it=\textit\def\PY@tc##1{\textcolor[rgb]{0.25,0.50,0.50}{##1}}}
\expandafter\def\csname PY@tok@c1\endcsname{\let\PY@it=\textit\def\PY@tc##1{\textcolor[rgb]{0.25,0.50,0.50}{##1}}}
\expandafter\def\csname PY@tok@cs\endcsname{\let\PY@it=\textit\def\PY@tc##1{\textcolor[rgb]{0.25,0.50,0.50}{##1}}}

\def\PYZbs{\char`\\}
\def\PYZus{\char`\_}
\def\PYZob{\char`\{}
\def\PYZcb{\char`\}}
\def\PYZca{\char`\^}
\def\PYZam{\char`\&}
\def\PYZlt{\char`\<}
\def\PYZgt{\char`\>}
\def\PYZsh{\char`\#}
\def\PYZpc{\char`\%}
\def\PYZdl{\char`\$}
\def\PYZhy{\char`\-}
\def\PYZsq{\char`\'}
\def\PYZdq{\char`\"}
\def\PYZti{\char`\~}
% for compatibility with earlier versions
\def\PYZat{@}
\def\PYZlb{[}
\def\PYZrb{]}
\makeatother


    % Exact colors from NB
    \definecolor{incolor}{rgb}{0.0, 0.0, 0.5}
    \definecolor{outcolor}{rgb}{0.545, 0.0, 0.0}



    
    % Prevent overflowing lines due to hard-to-break entities
    \sloppy 
    % Setup hyperref package
    \hypersetup{
      breaklinks=true,  % so long urls are correctly broken across lines
      colorlinks=true,
      urlcolor=urlcolor,
      linkcolor=linkcolor,
      citecolor=citecolor,
      }
    % Slightly bigger margins than the latex defaults
    
    \geometry{verbose,tmargin=1in,bmargin=1in,lmargin=1in,rmargin=1in}
    
    

    \begin{document}
    
    
    \maketitle
    
    

    
    \hypertarget{problem}{%
\section{Problem}\label{problem}}

    A business produces two products \(X\) and \(Y\). To make these two
products he needs \(3\) machines and a quantity of labor.

To produce \(1\)kg of \(X\) we need:

\begin{itemize}
\tightlist
\item
  \(2\) hours on machine \(1\)
\item
  \(1\) hour on machine \(2\)
\item
  \(2\) hours on machine \(3\)
\item
  \(1\) hour of labor
\end{itemize}

To produce \(1\)kg of \(Y\) we need:

\begin{itemize}
\tightlist
\item
  \(1\) hour on machine \(1\)
\item
  \(2\) hours on machine \(2\)
\item
  \(1\) hour of labor
\end{itemize}

On the first and second machine there are a maximum of \(140\) hours
available. The last machine has \(130\) hours available. There are a
maximum of \(90\) hours of labor.

The profit for \(1\)kg of \(X\) is \(\$30\), and for \(1\)kg of \(Y\) it
is \(\$20\). The business wants to maximize their profits.

How many kg \(X\) and \(Y\) does he needs to make? Assuming that all
produced products will be sold.

    \hypertarget{linear-programming-model}{%
\section{Linear programming model}\label{linear-programming-model}}

    First we convert the linear programming problem into a linear
programming model.

    \begin{itemize}
\tightlist
\item
  Let \(x\): quantity required to produce in kg for \(X\).
\item
  Let \(y\): quantity required to product in kg for \(Y\).
\item
  max \(30x + 20y\)
\item
  \(2x+y \leq 140\)
\item
  \(x+2y\leq 140\)
\item
  \(2x \leq 130\)
\item
  \(x + y \leq 90\)
\end{itemize}

    \hypertarget{solving-it-graphically}{%
\subsection{Solving it graphically}\label{solving-it-graphically}}

    To find the intersections we solve:

\[ S_1 = \begin{cases} x+y=90 \\ x+2y=140 \end{cases} \]

\[ S_2 = \begin{cases} 2x+y=140 \\ x+2y=140 \end{cases} \]

\[ S_3 = \begin{cases} x+2y=140 \\ 2x=130 \end{cases} \]

\[ S_4 = \begin{cases} x+y=90 \\ 2x+y=140 \end{cases} \]

    \begin{Verbatim}[commandchars=\\\{\}]
{\color{incolor}In [{\color{incolor}299}]:} S1 \PY{o}{=} \PY{k+kp}{solve}\PY{p}{(}\PY{k+kp}{cbind}\PY{p}{(}\PY{k+kt}{c}\PY{p}{(}\PY{l+m}{1}\PY{p}{,}\PY{l+m}{1}\PY{p}{)}\PY{p}{,}\PY{k+kt}{c}\PY{p}{(}\PY{l+m}{1}\PY{p}{,}\PY{l+m}{2}\PY{p}{)}\PY{p}{)}\PY{p}{,}\PY{k+kt}{c}\PY{p}{(}\PY{l+m}{90}\PY{p}{,}\PY{l+m}{140}\PY{p}{)}\PY{p}{)} 
          S2 \PY{o}{=} \PY{k+kp}{solve}\PY{p}{(}\PY{k+kp}{cbind}\PY{p}{(}\PY{k+kt}{c}\PY{p}{(}\PY{l+m}{2}\PY{p}{,}\PY{l+m}{1}\PY{p}{)}\PY{p}{,}\PY{k+kt}{c}\PY{p}{(}\PY{l+m}{1}\PY{p}{,}\PY{l+m}{2}\PY{p}{)}\PY{p}{)}\PY{p}{,}\PY{k+kt}{c}\PY{p}{(}\PY{l+m}{140}\PY{p}{,}\PY{l+m}{140}\PY{p}{)}\PY{p}{)}
          S3 \PY{o}{=} \PY{k+kp}{solve}\PY{p}{(}\PY{k+kp}{cbind}\PY{p}{(}\PY{k+kt}{c}\PY{p}{(}\PY{l+m}{1}\PY{p}{,}\PY{l+m}{2}\PY{p}{)}\PY{p}{,}\PY{k+kt}{c}\PY{p}{(}\PY{l+m}{2}\PY{p}{,}\PY{l+m}{0}\PY{p}{)}\PY{p}{)}\PY{p}{,}\PY{k+kt}{c}\PY{p}{(}\PY{l+m}{140}\PY{p}{,}\PY{l+m}{130}\PY{p}{)}\PY{p}{)}
          S4 \PY{o}{=} \PY{k+kp}{solve}\PY{p}{(}\PY{k+kp}{cbind}\PY{p}{(}\PY{k+kt}{c}\PY{p}{(}\PY{l+m}{1}\PY{p}{,}\PY{l+m}{2}\PY{p}{)}\PY{p}{,}\PY{k+kt}{c}\PY{p}{(}\PY{l+m}{1}\PY{p}{,}\PY{l+m}{1}\PY{p}{)}\PY{p}{)}\PY{p}{,}\PY{k+kt}{c}\PY{p}{(}\PY{l+m}{90}\PY{p}{,}\PY{l+m}{140}\PY{p}{)}\PY{p}{)}
          s \PY{o}{=} \PY{k+kp}{cbind}\PY{p}{(}S1\PY{p}{,}S2\PY{p}{,}S3\PY{p}{,}S4\PY{p}{)}
          \PY{k+kp}{rownames}\PY{p}{(}s\PY{p}{)} \PY{o}{\PYZlt{}\PYZhy{}} \PY{k+kt}{c}\PY{p}{(}\PY{l+s}{\PYZsq{}}\PY{l+s}{x\PYZsq{}}\PY{p}{,} \PY{l+s}{\PYZsq{}}\PY{l+s}{y\PYZsq{}}\PY{p}{)}
          s
\end{Verbatim}


    \begin{tabular}{r|llll}
  & S1 & S2 & S3 & S4\\
\hline
	x & 40       & 46.66667 & 65.0     & 50      \\
	y & 50       & 46.66667 & 37.5     & 40      \\
\end{tabular}


    
    We calculate the profit for each of the points:

    \begin{Verbatim}[commandchars=\\\{\}]
{\color{incolor}In [{\color{incolor}300}]:} \PY{k+kr}{for} \PY{p}{(}i \PY{k+kr}{in} \PY{l+m}{1}\PY{o}{:}\PY{l+m}{4}\PY{p}{)} \PY{p}{\PYZob{}}
              \PY{k+kp}{print}\PY{p}{(}\PY{k+kp}{paste}\PY{p}{(}\PY{l+s}{\PYZsq{}}\PY{l+s}{X=\PYZsq{}}\PY{p}{,} s\PY{p}{[}\PY{l+m}{1}\PY{p}{,}i\PY{p}{]}\PY{p}{,} \PY{l+s}{\PYZsq{}}\PY{l+s}{Y=\PYZsq{}}\PY{p}{,} s\PY{p}{[}\PY{l+m}{2}\PY{p}{,}i\PY{p}{]}\PY{p}{,} \PY{l+s}{\PYZsq{}}\PY{l+s}{Profit=\PYZsq{}}\PY{p}{,} \PY{l+m}{30}\PY{o}{*}s\PY{p}{[}\PY{l+m}{1}\PY{p}{,}i\PY{p}{]}\PY{l+m}{+20}\PY{o}{*}s\PY{p}{[}\PY{l+m}{2}\PY{p}{,}i\PY{p}{]}\PY{p}{)}\PY{p}{)}
          \PY{p}{\PYZcb{}}
\end{Verbatim}


    \begin{Verbatim}[commandchars=\\\{\}]
[1] "X= 40 Y= 50 Profit= 2200"
[1] "X= 46.6666666666667 Y= 46.6666666666667 Profit= 2333.33333333333"
[1] "X= 65 Y= 37.5 Profit= 2700"
[1] "X= 50 Y= 40 Profit= 2300"

    \end{Verbatim}

    Only \(S_1\) and \(S_4\) are viable solutions. We create the goal
function \(d(x)\) through \(S_4\) which is \((50, 40)\):

\[ 30x + 20y = 0 \iff y = -\frac{3}{2}x\]

Finally we plug-in the point \((50,40)\):

\[ y = -\frac{3}{2}(x-50)+40 \]

    \begin{Verbatim}[commandchars=\\\{\}]
{\color{incolor}In [{\color{incolor}301}]:} d \PY{o}{\PYZlt{}\PYZhy{}} \PY{k+kr}{function}\PY{p}{(}x\PY{p}{)} \PY{l+m}{\PYZhy{}3}\PY{o}{/}\PY{l+m}{2}\PY{o}{*}\PY{p}{(}x\PY{l+m}{\PYZhy{}50}\PY{p}{)}\PY{l+m}{+40}
\end{Verbatim}


    \begin{Verbatim}[commandchars=\\\{\}]
{\color{incolor}In [{\color{incolor}302}]:} a \PY{o}{=} \PY{l+m}{0}
          b \PY{o}{=} \PY{l+m}{140}
          X \PY{o}{=} Y \PY{o}{=} a\PY{o}{:}b
          plot\PY{p}{(}X\PY{p}{,}Y\PY{p}{,}col\PY{o}{=}\PY{l+s}{\PYZsq{}}\PY{l+s}{white\PYZsq{}}\PY{p}{)}
          points\PY{p}{(}s\PY{p}{[}\PY{l+m}{1}\PY{p}{,}\PY{p}{]}\PY{p}{,} s\PY{p}{[}\PY{l+m}{2}\PY{p}{,}\PY{p}{]}\PY{p}{)}
          c1 \PY{o}{=} line\PY{p}{(}a\PY{o}{:}b\PY{p}{,} \PY{k+kp}{sapply}\PY{p}{(}a\PY{o}{:}b\PY{p}{,} d\PY{p}{)}\PY{p}{)}
          c2 \PY{o}{=} line\PY{p}{(}a\PY{o}{:}b\PY{p}{,} \PY{k+kp}{sapply}\PY{p}{(}a\PY{o}{:}b\PY{p}{,} \PY{k+kr}{function}\PY{p}{(}x\PY{p}{)} \PY{p}{(}\PY{l+m}{140}\PY{l+m}{\PYZhy{}2}\PY{o}{*}x\PY{p}{)}\PY{p}{)}\PY{p}{)}
          c3 \PY{o}{=} line\PY{p}{(}a\PY{o}{:}b\PY{p}{,} \PY{k+kp}{sapply}\PY{p}{(}a\PY{o}{:}b\PY{p}{,} \PY{k+kr}{function}\PY{p}{(}x\PY{p}{)} \PY{p}{(}\PY{l+m}{1}\PY{o}{/}\PY{l+m}{2}\PY{o}{*}\PY{p}{(}\PY{l+m}{140}\PY{o}{\PYZhy{}}x\PY{p}{)}\PY{p}{)}\PY{p}{)}\PY{p}{)}
          c4 \PY{o}{=} line\PY{p}{(}a\PY{o}{:}b\PY{p}{,} \PY{k+kp}{sapply}\PY{p}{(}a\PY{o}{:}b\PY{p}{,} \PY{k+kr}{function}\PY{p}{(}x\PY{p}{)} \PY{p}{(}\PY{l+m}{90}\PY{o}{\PYZhy{}}x\PY{p}{)}\PY{p}{)}\PY{p}{)}
          abline\PY{p}{(}c1\PY{p}{,} col\PY{o}{=}\PY{l+s}{\PYZsq{}}\PY{l+s}{red\PYZsq{}}\PY{p}{,} lwd\PY{o}{=}\PY{l+m}{3}\PY{p}{,} lty\PY{o}{=}\PY{l+m}{2}\PY{p}{)}
          abline\PY{p}{(}c2\PY{p}{,} lwd\PY{o}{=}\PY{l+m}{2}\PY{p}{)}
          abline\PY{p}{(}c3\PY{p}{,} lwd\PY{o}{=}\PY{l+m}{2}\PY{p}{)}
          abline\PY{p}{(}c4\PY{p}{,} lwd\PY{o}{=}\PY{l+m}{2}\PY{p}{)}
          abline\PY{p}{(}v\PY{o}{=}\PY{l+m}{130}\PY{o}{/}\PY{l+m}{2}\PY{p}{,} lwd\PY{o}{=}\PY{l+m}{2}\PY{p}{)}
\end{Verbatim}


    \begin{center}
    \adjustimage{max size={0.9\linewidth}{0.9\paperheight}}{output_12_0.png}
    \end{center}
    { \hspace*{\fill} \\}
    
    \hypertarget{solution}{%
\subsection{Solution}\label{solution}}

    The solution is \(50\) kg of \(X\) and \(40\) kg of \(Y\) with a total
profit of \(\$2300\).

    \hypertarget{simplex-method}{%
\section{Simplex method}\label{simplex-method}}

    Now we are going to find the solution with the simplex method.

    \begin{Verbatim}[commandchars=\\\{\}]
{\color{incolor}In [{\color{incolor}303}]:} d \PY{o}{=} \PY{k+kt}{c}\PY{p}{(}\PY{l+m}{1}\PY{p}{,}\PY{l+m}{0}\PY{p}{,}\PY{l+m}{0}\PY{p}{,}\PY{l+m}{0}\PY{p}{,}\PY{l+m}{0}\PY{p}{)}
          x \PY{o}{=} \PY{k+kt}{c}\PY{p}{(}\PY{l+m}{\PYZhy{}30}\PY{p}{,} \PY{l+m}{2}\PY{p}{,}\PY{l+m}{1}\PY{p}{,}\PY{l+m}{2}\PY{p}{,}\PY{l+m}{1}\PY{p}{)}
          y \PY{o}{=} \PY{k+kt}{c}\PY{p}{(}\PY{l+m}{\PYZhy{}20}\PY{p}{,}\PY{l+m}{1}\PY{p}{,}\PY{l+m}{2}\PY{p}{,}\PY{l+m}{0}\PY{p}{,}\PY{l+m}{1}\PY{p}{)}
          s1 \PY{o}{=} \PY{k+kt}{c}\PY{p}{(}\PY{l+m}{0}\PY{p}{,}\PY{l+m}{1}\PY{p}{,}\PY{l+m}{0}\PY{p}{,}\PY{l+m}{0}\PY{p}{,}\PY{l+m}{0}\PY{p}{)}
          s2 \PY{o}{=} \PY{k+kt}{c}\PY{p}{(}\PY{l+m}{0}\PY{p}{,}\PY{l+m}{0}\PY{p}{,}\PY{l+m}{1}\PY{p}{,}\PY{l+m}{0}\PY{p}{,}\PY{l+m}{0}\PY{p}{)}
          s3 \PY{o}{=} \PY{k+kt}{c}\PY{p}{(}\PY{l+m}{0}\PY{p}{,}\PY{l+m}{0}\PY{p}{,}\PY{l+m}{0}\PY{p}{,}\PY{l+m}{1}\PY{p}{,}\PY{l+m}{0}\PY{p}{)}
          s4 \PY{o}{=} \PY{k+kt}{c}\PY{p}{(}\PY{l+m}{0}\PY{p}{,}\PY{l+m}{0}\PY{p}{,}\PY{l+m}{0}\PY{p}{,}\PY{l+m}{0}\PY{p}{,}\PY{l+m}{1}\PY{p}{)}
          RHS \PY{o}{=} \PY{k+kt}{c}\PY{p}{(}\PY{l+m}{0}\PY{p}{,}\PY{l+m}{140}\PY{p}{,}\PY{l+m}{140}\PY{p}{,}\PY{l+m}{130}\PY{p}{,}\PY{l+m}{90}\PY{p}{)}
          M \PY{o}{=} \PY{k+kp}{cbind}\PY{p}{(}d\PY{p}{,}x\PY{p}{,}y\PY{p}{,}s1\PY{p}{,}s2\PY{p}{,}s3\PY{p}{,}s4\PY{p}{,}RHS\PY{p}{)}
          \PY{k+kp}{rownames}\PY{p}{(}M\PY{p}{)} \PY{o}{=} \PY{k+kt}{c}\PY{p}{(}\PY{l+s}{\PYZsq{}}\PY{l+s}{d\PYZsq{}}\PY{p}{,}\PY{l+s}{\PYZsq{}}\PY{l+s}{s1\PYZsq{}}\PY{p}{,}\PY{l+s}{\PYZsq{}}\PY{l+s}{s2\PYZsq{}}\PY{p}{,}\PY{l+s}{\PYZsq{}}\PY{l+s}{s3\PYZsq{}}\PY{p}{,}\PY{l+s}{\PYZsq{}}\PY{l+s}{s4\PYZsq{}}\PY{p}{)}
          M
\end{Verbatim}


    \begin{tabular}{r|llllllll}
  & d & x & y & s1 & s2 & s3 & s4 & RHS\\
\hline
	d & 1   & -30 & -20 & 0   & 0   & 0   & 0   &   0\\
	s1 & 0   &   2 &   1 & 1   & 0   & 0   & 0   & 140\\
	s2 & 0   &   1 &   2 & 0   & 1   & 0   & 0   & 140\\
	s3 & 0   &   2 &   0 & 0   & 0   & 1   & 0   & 130\\
	s4 & 0   &   1 &   1 & 0   & 0   & 0   & 1   &  90\\
\end{tabular}


    
    Solving for \(X\):

    \begin{Verbatim}[commandchars=\\\{\}]
{\color{incolor}In [{\color{incolor}304}]:} M\PY{p}{[}\PY{l+m}{1}\PY{p}{,}\PY{p}{]} \PY{o}{=} M\PY{p}{[}\PY{l+m}{1}\PY{p}{,}\PY{p}{]} \PY{o}{+} \PY{l+m}{15}\PY{o}{*}M\PY{p}{[}\PY{l+m}{4}\PY{p}{,}\PY{p}{]}
          M\PY{p}{[}\PY{l+m}{2}\PY{p}{,}\PY{p}{]} \PY{o}{=} M\PY{p}{[}\PY{l+m}{2}\PY{p}{,}\PY{p}{]} \PY{o}{\PYZhy{}} M\PY{p}{[}\PY{l+m}{4}\PY{p}{,}\PY{p}{]}
          M\PY{p}{[}\PY{l+m}{3}\PY{p}{,}\PY{p}{]} \PY{o}{=} M\PY{p}{[}\PY{l+m}{3}\PY{p}{,}\PY{p}{]} \PY{o}{\PYZhy{}} \PY{l+m}{1}\PY{o}{/}\PY{l+m}{2}\PY{o}{*}M\PY{p}{[}\PY{l+m}{4}\PY{p}{,}\PY{p}{]}
          M\PY{p}{[}\PY{l+m}{5}\PY{p}{,}\PY{p}{]} \PY{o}{=} M\PY{p}{[}\PY{l+m}{5}\PY{p}{,}\PY{p}{]} \PY{o}{\PYZhy{}} \PY{l+m}{1}\PY{o}{/}\PY{l+m}{2}\PY{o}{*}M\PY{p}{[}\PY{l+m}{4}\PY{p}{,}\PY{p}{]}
          M\PY{p}{[}\PY{l+m}{4}\PY{p}{,}\PY{p}{]} \PY{o}{=} \PY{l+m}{1}\PY{o}{/}\PY{l+m}{2}\PY{o}{*}M\PY{p}{[}\PY{l+m}{4}\PY{p}{,}\PY{p}{]}
          \PY{k+kp}{rownames}\PY{p}{(}M\PY{p}{)} \PY{o}{=} \PY{k+kt}{c}\PY{p}{(}\PY{l+s}{\PYZsq{}}\PY{l+s}{d\PYZsq{}}\PY{p}{,}\PY{l+s}{\PYZsq{}}\PY{l+s}{s1\PYZsq{}}\PY{p}{,}\PY{l+s}{\PYZsq{}}\PY{l+s}{s2\PYZsq{}}\PY{p}{,}\PY{l+s}{\PYZsq{}}\PY{l+s}{x\PYZsq{}}\PY{p}{,}\PY{l+s}{\PYZsq{}}\PY{l+s}{s4\PYZsq{}}\PY{p}{)}
\end{Verbatim}


    \begin{Verbatim}[commandchars=\\\{\}]
{\color{incolor}In [{\color{incolor}305}]:} M
\end{Verbatim}


    \begin{tabular}{r|llllllll}
  & d & x & y & s1 & s2 & s3 & s4 & RHS\\
\hline
	d & 1    & 0    & -20  & 0    & 0    & 15.0 & 0    & 1950\\
	s1 & 0    & 0    &   1  & 1    & 0    & -1.0 & 0    &   10\\
	s2 & 0    & 0    &   2  & 0    & 1    & -0.5 & 0    &   75\\
	x & 0    & 1    &   0  & 0    & 0    &  0.5 & 0    &   65\\
	s4 & 0    & 0    &   1  & 0    & 0    & -0.5 & 1    &   25\\
\end{tabular}


    
    Solving for \(Y\):

    \begin{Verbatim}[commandchars=\\\{\}]
{\color{incolor}In [{\color{incolor}306}]:} M\PY{p}{[}\PY{l+m}{1}\PY{p}{,}\PY{p}{]} \PY{o}{=} M\PY{p}{[}\PY{l+m}{1}\PY{p}{,}\PY{p}{]} \PY{o}{+} \PY{l+m}{20}\PY{o}{*}M\PY{p}{[}\PY{l+m}{2}\PY{p}{,}\PY{p}{]}
          M\PY{p}{[}\PY{l+m}{3}\PY{p}{,}\PY{p}{]} \PY{o}{=} M\PY{p}{[}\PY{l+m}{3}\PY{p}{,}\PY{p}{]} \PY{o}{\PYZhy{}} \PY{l+m}{2}\PY{o}{*}M\PY{p}{[}\PY{l+m}{2}\PY{p}{,}\PY{p}{]}
          M\PY{p}{[}\PY{l+m}{5}\PY{p}{,}\PY{p}{]} \PY{o}{=} M\PY{p}{[}\PY{l+m}{5}\PY{p}{,}\PY{p}{]} \PY{o}{\PYZhy{}} M\PY{p}{[}\PY{l+m}{2}\PY{p}{,}\PY{p}{]}
          \PY{k+kp}{rownames}\PY{p}{(}M\PY{p}{)} \PY{o}{=} \PY{k+kt}{c}\PY{p}{(}\PY{l+s}{\PYZsq{}}\PY{l+s}{d\PYZsq{}}\PY{p}{,}\PY{l+s}{\PYZsq{}}\PY{l+s}{y\PYZsq{}}\PY{p}{,}\PY{l+s}{\PYZsq{}}\PY{l+s}{s2\PYZsq{}}\PY{p}{,}\PY{l+s}{\PYZsq{}}\PY{l+s}{x\PYZsq{}}\PY{p}{,}\PY{l+s}{\PYZsq{}}\PY{l+s}{s4\PYZsq{}}\PY{p}{)}
          M
\end{Verbatim}


    \begin{tabular}{r|llllllll}
  & d & x & y & s1 & s2 & s3 & s4 & RHS\\
\hline
	d & 1    & 0    & 0    & 20   & 0    & -5.0 & 0    & 2150\\
	y & 0    & 0    & 1    &  1   & 0    & -1.0 & 0    &   10\\
	s2 & 0    & 0    & 0    & -2   & 1    &  1.5 & 0    &   55\\
	x & 0    & 1    & 0    &  0   & 0    &  0.5 & 0    &   65\\
	s4 & 0    & 0    & 0    & -1   & 0    &  0.5 & 1    &   15\\
\end{tabular}


    
    Solving for \(s_3\):

    \begin{Verbatim}[commandchars=\\\{\}]
{\color{incolor}In [{\color{incolor}307}]:} M\PY{p}{[}\PY{l+m}{1}\PY{p}{,}\PY{p}{]} \PY{o}{=} M\PY{p}{[}\PY{l+m}{1}\PY{p}{,}\PY{p}{]} \PY{o}{+} \PY{l+m}{10}\PY{o}{*}M\PY{p}{[}\PY{l+m}{5}\PY{p}{,}\PY{p}{]}
          M\PY{p}{[}\PY{l+m}{2}\PY{p}{,}\PY{p}{]} \PY{o}{=} M\PY{p}{[}\PY{l+m}{2}\PY{p}{,}\PY{p}{]} \PY{o}{+} \PY{l+m}{2}\PY{o}{*}M\PY{p}{[}\PY{l+m}{5}\PY{p}{,}\PY{p}{]}
          M\PY{p}{[}\PY{l+m}{3}\PY{p}{,}\PY{p}{]} \PY{o}{=} M\PY{p}{[}\PY{l+m}{3}\PY{p}{,}\PY{p}{]} \PY{o}{\PYZhy{}} \PY{l+m}{3}\PY{o}{*}M\PY{p}{[}\PY{l+m}{5}\PY{p}{,}\PY{p}{]}
          M\PY{p}{[}\PY{l+m}{4}\PY{p}{,}\PY{p}{]} \PY{o}{=} M\PY{p}{[}\PY{l+m}{4}\PY{p}{,}\PY{p}{]} \PY{o}{\PYZhy{}} M\PY{p}{[}\PY{l+m}{5}\PY{p}{,}\PY{p}{]}
          M\PY{p}{[}\PY{l+m}{5}\PY{p}{,}\PY{p}{]} \PY{o}{=} \PY{l+m}{2}\PY{o}{*}M\PY{p}{[}\PY{l+m}{5}\PY{p}{,}\PY{p}{]}
          \PY{k+kp}{rownames}\PY{p}{(}M\PY{p}{)} \PY{o}{=} \PY{k+kt}{c}\PY{p}{(}\PY{l+s}{\PYZsq{}}\PY{l+s}{d\PYZsq{}}\PY{p}{,}\PY{l+s}{\PYZsq{}}\PY{l+s}{y\PYZsq{}}\PY{p}{,}\PY{l+s}{\PYZsq{}}\PY{l+s}{s2\PYZsq{}}\PY{p}{,}\PY{l+s}{\PYZsq{}}\PY{l+s}{x\PYZsq{}}\PY{p}{,}\PY{l+s}{\PYZsq{}}\PY{l+s}{s3\PYZsq{}}\PY{p}{)}
          M
\end{Verbatim}


    \begin{tabular}{r|llllllll}
  & d & x & y & s1 & s2 & s3 & s4 & RHS\\
\hline
	d & 1    & 0    & 0    & 10   & 0    & 0    & 10   & 2300\\
	y & 0    & 0    & 1    & -1   & 0    & 0    &  2   &   40\\
	s2 & 0    & 0    & 0    &  1   & 1    & 0    & -3   &   10\\
	x & 0    & 1    & 0    &  1   & 0    & 0    & -1   &   50\\
	s3 & 0    & 0    & 0    & -2   & 0    & 1    &  2   &   30\\
\end{tabular}


    
    \hypertarget{solution}{%
\subsection{Solution}\label{solution}}

The total profit is \(\$2300\). For maximum profit we need to produce
\(50\) kg of \(X\), and \(40\) kg of \(Y\). With a remainder for
\(s_2=10\) and \(s_3=30\).


    % Add a bibliography block to the postdoc
    
    
    
    \end{document}
