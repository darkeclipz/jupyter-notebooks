
% Default to the notebook output style

    


% Inherit from the specified cell style.




    
\documentclass[11pt]{article}

    
    
    \usepackage[T1]{fontenc}
    % Nicer default font (+ math font) than Computer Modern for most use cases
    \usepackage{mathpazo}

    % Basic figure setup, for now with no caption control since it's done
    % automatically by Pandoc (which extracts ![](path) syntax from Markdown).
    \usepackage{graphicx}
    % We will generate all images so they have a width \maxwidth. This means
    % that they will get their normal width if they fit onto the page, but
    % are scaled down if they would overflow the margins.
    \makeatletter
    \def\maxwidth{\ifdim\Gin@nat@width>\linewidth\linewidth
    \else\Gin@nat@width\fi}
    \makeatother
    \let\Oldincludegraphics\includegraphics
    % Set max figure width to be 80% of text width, for now hardcoded.
    \renewcommand{\includegraphics}[1]{\Oldincludegraphics[width=.8\maxwidth]{#1}}
    % Ensure that by default, figures have no caption (until we provide a
    % proper Figure object with a Caption API and a way to capture that
    % in the conversion process - todo).
    \usepackage{caption}
    \DeclareCaptionLabelFormat{nolabel}{}
    \captionsetup{labelformat=nolabel}

    \usepackage{adjustbox} % Used to constrain images to a maximum size 
    \usepackage{xcolor} % Allow colors to be defined
    \usepackage{enumerate} % Needed for markdown enumerations to work
    \usepackage{geometry} % Used to adjust the document margins
    \usepackage{amsmath} % Equations
    \usepackage{amssymb} % Equations
    \usepackage{textcomp} % defines textquotesingle
    % Hack from http://tex.stackexchange.com/a/47451/13684:
    \AtBeginDocument{%
        \def\PYZsq{\textquotesingle}% Upright quotes in Pygmentized code
    }
    \usepackage{upquote} % Upright quotes for verbatim code
    \usepackage{eurosym} % defines \euro
    \usepackage[mathletters]{ucs} % Extended unicode (utf-8) support
    \usepackage[utf8x]{inputenc} % Allow utf-8 characters in the tex document
    \usepackage{fancyvrb} % verbatim replacement that allows latex
    \usepackage{grffile} % extends the file name processing of package graphics 
                         % to support a larger range 
    % The hyperref package gives us a pdf with properly built
    % internal navigation ('pdf bookmarks' for the table of contents,
    % internal cross-reference links, web links for URLs, etc.)
    \usepackage{hyperref}
    \usepackage{longtable} % longtable support required by pandoc >1.10
    \usepackage{booktabs}  % table support for pandoc > 1.12.2
    \usepackage[inline]{enumitem} % IRkernel/repr support (it uses the enumerate* environment)
    \usepackage[normalem]{ulem} % ulem is needed to support strikethroughs (\sout)
                                % normalem makes italics be italics, not underlines
    \usepackage{mathrsfs}
    

    
    
    % Colors for the hyperref package
    \definecolor{urlcolor}{rgb}{0,.145,.698}
    \definecolor{linkcolor}{rgb}{.71,0.21,0.01}
    \definecolor{citecolor}{rgb}{.12,.54,.11}

    % ANSI colors
    \definecolor{ansi-black}{HTML}{3E424D}
    \definecolor{ansi-black-intense}{HTML}{282C36}
    \definecolor{ansi-red}{HTML}{E75C58}
    \definecolor{ansi-red-intense}{HTML}{B22B31}
    \definecolor{ansi-green}{HTML}{00A250}
    \definecolor{ansi-green-intense}{HTML}{007427}
    \definecolor{ansi-yellow}{HTML}{DDB62B}
    \definecolor{ansi-yellow-intense}{HTML}{B27D12}
    \definecolor{ansi-blue}{HTML}{208FFB}
    \definecolor{ansi-blue-intense}{HTML}{0065CA}
    \definecolor{ansi-magenta}{HTML}{D160C4}
    \definecolor{ansi-magenta-intense}{HTML}{A03196}
    \definecolor{ansi-cyan}{HTML}{60C6C8}
    \definecolor{ansi-cyan-intense}{HTML}{258F8F}
    \definecolor{ansi-white}{HTML}{C5C1B4}
    \definecolor{ansi-white-intense}{HTML}{A1A6B2}
    \definecolor{ansi-default-inverse-fg}{HTML}{FFFFFF}
    \definecolor{ansi-default-inverse-bg}{HTML}{000000}

    % commands and environments needed by pandoc snippets
    % extracted from the output of `pandoc -s`
    \providecommand{\tightlist}{%
      \setlength{\itemsep}{0pt}\setlength{\parskip}{0pt}}
    \DefineVerbatimEnvironment{Highlighting}{Verbatim}{commandchars=\\\{\}}
    % Add ',fontsize=\small' for more characters per line
    \newenvironment{Shaded}{}{}
    \newcommand{\KeywordTok}[1]{\textcolor[rgb]{0.00,0.44,0.13}{\textbf{{#1}}}}
    \newcommand{\DataTypeTok}[1]{\textcolor[rgb]{0.56,0.13,0.00}{{#1}}}
    \newcommand{\DecValTok}[1]{\textcolor[rgb]{0.25,0.63,0.44}{{#1}}}
    \newcommand{\BaseNTok}[1]{\textcolor[rgb]{0.25,0.63,0.44}{{#1}}}
    \newcommand{\FloatTok}[1]{\textcolor[rgb]{0.25,0.63,0.44}{{#1}}}
    \newcommand{\CharTok}[1]{\textcolor[rgb]{0.25,0.44,0.63}{{#1}}}
    \newcommand{\StringTok}[1]{\textcolor[rgb]{0.25,0.44,0.63}{{#1}}}
    \newcommand{\CommentTok}[1]{\textcolor[rgb]{0.38,0.63,0.69}{\textit{{#1}}}}
    \newcommand{\OtherTok}[1]{\textcolor[rgb]{0.00,0.44,0.13}{{#1}}}
    \newcommand{\AlertTok}[1]{\textcolor[rgb]{1.00,0.00,0.00}{\textbf{{#1}}}}
    \newcommand{\FunctionTok}[1]{\textcolor[rgb]{0.02,0.16,0.49}{{#1}}}
    \newcommand{\RegionMarkerTok}[1]{{#1}}
    \newcommand{\ErrorTok}[1]{\textcolor[rgb]{1.00,0.00,0.00}{\textbf{{#1}}}}
    \newcommand{\NormalTok}[1]{{#1}}
    
    % Additional commands for more recent versions of Pandoc
    \newcommand{\ConstantTok}[1]{\textcolor[rgb]{0.53,0.00,0.00}{{#1}}}
    \newcommand{\SpecialCharTok}[1]{\textcolor[rgb]{0.25,0.44,0.63}{{#1}}}
    \newcommand{\VerbatimStringTok}[1]{\textcolor[rgb]{0.25,0.44,0.63}{{#1}}}
    \newcommand{\SpecialStringTok}[1]{\textcolor[rgb]{0.73,0.40,0.53}{{#1}}}
    \newcommand{\ImportTok}[1]{{#1}}
    \newcommand{\DocumentationTok}[1]{\textcolor[rgb]{0.73,0.13,0.13}{\textit{{#1}}}}
    \newcommand{\AnnotationTok}[1]{\textcolor[rgb]{0.38,0.63,0.69}{\textbf{\textit{{#1}}}}}
    \newcommand{\CommentVarTok}[1]{\textcolor[rgb]{0.38,0.63,0.69}{\textbf{\textit{{#1}}}}}
    \newcommand{\VariableTok}[1]{\textcolor[rgb]{0.10,0.09,0.49}{{#1}}}
    \newcommand{\ControlFlowTok}[1]{\textcolor[rgb]{0.00,0.44,0.13}{\textbf{{#1}}}}
    \newcommand{\OperatorTok}[1]{\textcolor[rgb]{0.40,0.40,0.40}{{#1}}}
    \newcommand{\BuiltInTok}[1]{{#1}}
    \newcommand{\ExtensionTok}[1]{{#1}}
    \newcommand{\PreprocessorTok}[1]{\textcolor[rgb]{0.74,0.48,0.00}{{#1}}}
    \newcommand{\AttributeTok}[1]{\textcolor[rgb]{0.49,0.56,0.16}{{#1}}}
    \newcommand{\InformationTok}[1]{\textcolor[rgb]{0.38,0.63,0.69}{\textbf{\textit{{#1}}}}}
    \newcommand{\WarningTok}[1]{\textcolor[rgb]{0.38,0.63,0.69}{\textbf{\textit{{#1}}}}}
    
    
    % Define a nice break command that doesn't care if a line doesn't already
    % exist.
    \def\br{\hspace*{\fill} \\* }
    % Math Jax compatibility definitions
    \def\gt{>}
    \def\lt{<}
    \let\Oldtex\TeX
    \let\Oldlatex\LaTeX
    \renewcommand{\TeX}{\textrm{\Oldtex}}
    \renewcommand{\LaTeX}{\textrm{\Oldlatex}}
    % Document parameters
    % Document title
    \title{Surface graphs of functions with two variables}
    
    
    
    
    

    % Pygments definitions
    
\makeatletter
\def\PY@reset{\let\PY@it=\relax \let\PY@bf=\relax%
    \let\PY@ul=\relax \let\PY@tc=\relax%
    \let\PY@bc=\relax \let\PY@ff=\relax}
\def\PY@tok#1{\csname PY@tok@#1\endcsname}
\def\PY@toks#1+{\ifx\relax#1\empty\else%
    \PY@tok{#1}\expandafter\PY@toks\fi}
\def\PY@do#1{\PY@bc{\PY@tc{\PY@ul{%
    \PY@it{\PY@bf{\PY@ff{#1}}}}}}}
\def\PY#1#2{\PY@reset\PY@toks#1+\relax+\PY@do{#2}}

\expandafter\def\csname PY@tok@w\endcsname{\def\PY@tc##1{\textcolor[rgb]{0.73,0.73,0.73}{##1}}}
\expandafter\def\csname PY@tok@c\endcsname{\let\PY@it=\textit\def\PY@tc##1{\textcolor[rgb]{0.25,0.50,0.50}{##1}}}
\expandafter\def\csname PY@tok@cp\endcsname{\def\PY@tc##1{\textcolor[rgb]{0.74,0.48,0.00}{##1}}}
\expandafter\def\csname PY@tok@k\endcsname{\let\PY@bf=\textbf\def\PY@tc##1{\textcolor[rgb]{0.00,0.50,0.00}{##1}}}
\expandafter\def\csname PY@tok@kp\endcsname{\def\PY@tc##1{\textcolor[rgb]{0.00,0.50,0.00}{##1}}}
\expandafter\def\csname PY@tok@kt\endcsname{\def\PY@tc##1{\textcolor[rgb]{0.69,0.00,0.25}{##1}}}
\expandafter\def\csname PY@tok@o\endcsname{\def\PY@tc##1{\textcolor[rgb]{0.40,0.40,0.40}{##1}}}
\expandafter\def\csname PY@tok@ow\endcsname{\let\PY@bf=\textbf\def\PY@tc##1{\textcolor[rgb]{0.67,0.13,1.00}{##1}}}
\expandafter\def\csname PY@tok@nb\endcsname{\def\PY@tc##1{\textcolor[rgb]{0.00,0.50,0.00}{##1}}}
\expandafter\def\csname PY@tok@nf\endcsname{\def\PY@tc##1{\textcolor[rgb]{0.00,0.00,1.00}{##1}}}
\expandafter\def\csname PY@tok@nc\endcsname{\let\PY@bf=\textbf\def\PY@tc##1{\textcolor[rgb]{0.00,0.00,1.00}{##1}}}
\expandafter\def\csname PY@tok@nn\endcsname{\let\PY@bf=\textbf\def\PY@tc##1{\textcolor[rgb]{0.00,0.00,1.00}{##1}}}
\expandafter\def\csname PY@tok@ne\endcsname{\let\PY@bf=\textbf\def\PY@tc##1{\textcolor[rgb]{0.82,0.25,0.23}{##1}}}
\expandafter\def\csname PY@tok@nv\endcsname{\def\PY@tc##1{\textcolor[rgb]{0.10,0.09,0.49}{##1}}}
\expandafter\def\csname PY@tok@no\endcsname{\def\PY@tc##1{\textcolor[rgb]{0.53,0.00,0.00}{##1}}}
\expandafter\def\csname PY@tok@nl\endcsname{\def\PY@tc##1{\textcolor[rgb]{0.63,0.63,0.00}{##1}}}
\expandafter\def\csname PY@tok@ni\endcsname{\let\PY@bf=\textbf\def\PY@tc##1{\textcolor[rgb]{0.60,0.60,0.60}{##1}}}
\expandafter\def\csname PY@tok@na\endcsname{\def\PY@tc##1{\textcolor[rgb]{0.49,0.56,0.16}{##1}}}
\expandafter\def\csname PY@tok@nt\endcsname{\let\PY@bf=\textbf\def\PY@tc##1{\textcolor[rgb]{0.00,0.50,0.00}{##1}}}
\expandafter\def\csname PY@tok@nd\endcsname{\def\PY@tc##1{\textcolor[rgb]{0.67,0.13,1.00}{##1}}}
\expandafter\def\csname PY@tok@s\endcsname{\def\PY@tc##1{\textcolor[rgb]{0.73,0.13,0.13}{##1}}}
\expandafter\def\csname PY@tok@sd\endcsname{\let\PY@it=\textit\def\PY@tc##1{\textcolor[rgb]{0.73,0.13,0.13}{##1}}}
\expandafter\def\csname PY@tok@si\endcsname{\let\PY@bf=\textbf\def\PY@tc##1{\textcolor[rgb]{0.73,0.40,0.53}{##1}}}
\expandafter\def\csname PY@tok@se\endcsname{\let\PY@bf=\textbf\def\PY@tc##1{\textcolor[rgb]{0.73,0.40,0.13}{##1}}}
\expandafter\def\csname PY@tok@sr\endcsname{\def\PY@tc##1{\textcolor[rgb]{0.73,0.40,0.53}{##1}}}
\expandafter\def\csname PY@tok@ss\endcsname{\def\PY@tc##1{\textcolor[rgb]{0.10,0.09,0.49}{##1}}}
\expandafter\def\csname PY@tok@sx\endcsname{\def\PY@tc##1{\textcolor[rgb]{0.00,0.50,0.00}{##1}}}
\expandafter\def\csname PY@tok@m\endcsname{\def\PY@tc##1{\textcolor[rgb]{0.40,0.40,0.40}{##1}}}
\expandafter\def\csname PY@tok@gh\endcsname{\let\PY@bf=\textbf\def\PY@tc##1{\textcolor[rgb]{0.00,0.00,0.50}{##1}}}
\expandafter\def\csname PY@tok@gu\endcsname{\let\PY@bf=\textbf\def\PY@tc##1{\textcolor[rgb]{0.50,0.00,0.50}{##1}}}
\expandafter\def\csname PY@tok@gd\endcsname{\def\PY@tc##1{\textcolor[rgb]{0.63,0.00,0.00}{##1}}}
\expandafter\def\csname PY@tok@gi\endcsname{\def\PY@tc##1{\textcolor[rgb]{0.00,0.63,0.00}{##1}}}
\expandafter\def\csname PY@tok@gr\endcsname{\def\PY@tc##1{\textcolor[rgb]{1.00,0.00,0.00}{##1}}}
\expandafter\def\csname PY@tok@ge\endcsname{\let\PY@it=\textit}
\expandafter\def\csname PY@tok@gs\endcsname{\let\PY@bf=\textbf}
\expandafter\def\csname PY@tok@gp\endcsname{\let\PY@bf=\textbf\def\PY@tc##1{\textcolor[rgb]{0.00,0.00,0.50}{##1}}}
\expandafter\def\csname PY@tok@go\endcsname{\def\PY@tc##1{\textcolor[rgb]{0.53,0.53,0.53}{##1}}}
\expandafter\def\csname PY@tok@gt\endcsname{\def\PY@tc##1{\textcolor[rgb]{0.00,0.27,0.87}{##1}}}
\expandafter\def\csname PY@tok@err\endcsname{\def\PY@bc##1{\setlength{\fboxsep}{0pt}\fcolorbox[rgb]{1.00,0.00,0.00}{1,1,1}{\strut ##1}}}
\expandafter\def\csname PY@tok@kc\endcsname{\let\PY@bf=\textbf\def\PY@tc##1{\textcolor[rgb]{0.00,0.50,0.00}{##1}}}
\expandafter\def\csname PY@tok@kd\endcsname{\let\PY@bf=\textbf\def\PY@tc##1{\textcolor[rgb]{0.00,0.50,0.00}{##1}}}
\expandafter\def\csname PY@tok@kn\endcsname{\let\PY@bf=\textbf\def\PY@tc##1{\textcolor[rgb]{0.00,0.50,0.00}{##1}}}
\expandafter\def\csname PY@tok@kr\endcsname{\let\PY@bf=\textbf\def\PY@tc##1{\textcolor[rgb]{0.00,0.50,0.00}{##1}}}
\expandafter\def\csname PY@tok@bp\endcsname{\def\PY@tc##1{\textcolor[rgb]{0.00,0.50,0.00}{##1}}}
\expandafter\def\csname PY@tok@fm\endcsname{\def\PY@tc##1{\textcolor[rgb]{0.00,0.00,1.00}{##1}}}
\expandafter\def\csname PY@tok@vc\endcsname{\def\PY@tc##1{\textcolor[rgb]{0.10,0.09,0.49}{##1}}}
\expandafter\def\csname PY@tok@vg\endcsname{\def\PY@tc##1{\textcolor[rgb]{0.10,0.09,0.49}{##1}}}
\expandafter\def\csname PY@tok@vi\endcsname{\def\PY@tc##1{\textcolor[rgb]{0.10,0.09,0.49}{##1}}}
\expandafter\def\csname PY@tok@vm\endcsname{\def\PY@tc##1{\textcolor[rgb]{0.10,0.09,0.49}{##1}}}
\expandafter\def\csname PY@tok@sa\endcsname{\def\PY@tc##1{\textcolor[rgb]{0.73,0.13,0.13}{##1}}}
\expandafter\def\csname PY@tok@sb\endcsname{\def\PY@tc##1{\textcolor[rgb]{0.73,0.13,0.13}{##1}}}
\expandafter\def\csname PY@tok@sc\endcsname{\def\PY@tc##1{\textcolor[rgb]{0.73,0.13,0.13}{##1}}}
\expandafter\def\csname PY@tok@dl\endcsname{\def\PY@tc##1{\textcolor[rgb]{0.73,0.13,0.13}{##1}}}
\expandafter\def\csname PY@tok@s2\endcsname{\def\PY@tc##1{\textcolor[rgb]{0.73,0.13,0.13}{##1}}}
\expandafter\def\csname PY@tok@sh\endcsname{\def\PY@tc##1{\textcolor[rgb]{0.73,0.13,0.13}{##1}}}
\expandafter\def\csname PY@tok@s1\endcsname{\def\PY@tc##1{\textcolor[rgb]{0.73,0.13,0.13}{##1}}}
\expandafter\def\csname PY@tok@mb\endcsname{\def\PY@tc##1{\textcolor[rgb]{0.40,0.40,0.40}{##1}}}
\expandafter\def\csname PY@tok@mf\endcsname{\def\PY@tc##1{\textcolor[rgb]{0.40,0.40,0.40}{##1}}}
\expandafter\def\csname PY@tok@mh\endcsname{\def\PY@tc##1{\textcolor[rgb]{0.40,0.40,0.40}{##1}}}
\expandafter\def\csname PY@tok@mi\endcsname{\def\PY@tc##1{\textcolor[rgb]{0.40,0.40,0.40}{##1}}}
\expandafter\def\csname PY@tok@il\endcsname{\def\PY@tc##1{\textcolor[rgb]{0.40,0.40,0.40}{##1}}}
\expandafter\def\csname PY@tok@mo\endcsname{\def\PY@tc##1{\textcolor[rgb]{0.40,0.40,0.40}{##1}}}
\expandafter\def\csname PY@tok@ch\endcsname{\let\PY@it=\textit\def\PY@tc##1{\textcolor[rgb]{0.25,0.50,0.50}{##1}}}
\expandafter\def\csname PY@tok@cm\endcsname{\let\PY@it=\textit\def\PY@tc##1{\textcolor[rgb]{0.25,0.50,0.50}{##1}}}
\expandafter\def\csname PY@tok@cpf\endcsname{\let\PY@it=\textit\def\PY@tc##1{\textcolor[rgb]{0.25,0.50,0.50}{##1}}}
\expandafter\def\csname PY@tok@c1\endcsname{\let\PY@it=\textit\def\PY@tc##1{\textcolor[rgb]{0.25,0.50,0.50}{##1}}}
\expandafter\def\csname PY@tok@cs\endcsname{\let\PY@it=\textit\def\PY@tc##1{\textcolor[rgb]{0.25,0.50,0.50}{##1}}}

\def\PYZbs{\char`\\}
\def\PYZus{\char`\_}
\def\PYZob{\char`\{}
\def\PYZcb{\char`\}}
\def\PYZca{\char`\^}
\def\PYZam{\char`\&}
\def\PYZlt{\char`\<}
\def\PYZgt{\char`\>}
\def\PYZsh{\char`\#}
\def\PYZpc{\char`\%}
\def\PYZdl{\char`\$}
\def\PYZhy{\char`\-}
\def\PYZsq{\char`\'}
\def\PYZdq{\char`\"}
\def\PYZti{\char`\~}
% for compatibility with earlier versions
\def\PYZat{@}
\def\PYZlb{[}
\def\PYZrb{]}
\makeatother


    % Exact colors from NB
    \definecolor{incolor}{rgb}{0.0, 0.0, 0.5}
    \definecolor{outcolor}{rgb}{0.545, 0.0, 0.0}



    
    % Prevent overflowing lines due to hard-to-break entities
    \sloppy 
    % Setup hyperref package
    \hypersetup{
      breaklinks=true,  % so long urls are correctly broken across lines
      colorlinks=true,
      urlcolor=urlcolor,
      linkcolor=linkcolor,
      citecolor=citecolor,
      }
    % Slightly bigger margins than the latex defaults
    
    \geometry{verbose,tmargin=1in,bmargin=1in,lmargin=1in,rmargin=1in}
    
    

    \begin{document}
    
    
    \maketitle
    
    
    \newpage
    \tableofcontents
    \newpage
    
    

    
    \hypertarget{definitions-of-functions-with-two-variables}{%
\section{Definitions of functions with two
variables}\label{definitions-of-functions-with-two-variables}}

    \begin{itemize}
\tightlist
\item
  \textbf{Definition (function of two variables):} A function \(f\) of
  two variables is a rule that assigns to each ordered pair of real
  numbers \((x,y)\) in a set \(D\) a unique real number denoted by
  \(f(x,y)\). The set \(D\) is the domain of \(f\) and its range is the
  set of values that \(f\) takes on, that is,
  \(\{f(x,y)\ |\ (x,y)\in D\}\). The function \(f\) is a mapping from
  \(\mathbb{R}^2\rightarrow\mathbb{R}\).
\item
  \textbf{Definition (graph)}: If \(f\) is a function of two variables
  with domain \(D\), then the \textbf{graph} of \(f\) is the set of all
  points \((x,y,z)\) in \(\mathbb{R}^3\) such that \(z=f(x,y)\) and
  \((x,y) \in D\).
\end{itemize}

    \hypertarget{matplotlib-surface-plot}{%
\section{\texorpdfstring{\texttt{matplotlib} surface
plot}{matplotlib Surface Plot}}\label{matplotlib-surface-plot}}

    \begin{Verbatim}[commandchars=\\\{\}]
{\color{incolor}In [{\color{incolor}1}]:} \PY{o}{\PYZpc{}}\PY{k}{pylab} inline
\end{Verbatim}

    \begin{Verbatim}[commandchars=\\\{\}]
Populating the interactive namespace from numpy and matplotlib

    \end{Verbatim}

    First we will create a function that will create a service plot of a
function \(f(x,y)\). To create a surface plot with \texttt{matplotlib},
we need the following imports:

    \begin{Verbatim}[commandchars=\\\{\}]
{\color{incolor}In [{\color{incolor}2}]:} \PY{k+kn}{import} \PY{n+nn}{matplotlib}\PY{n+nn}{.}\PY{n+nn}{pyplot} \PY{k}{as} \PY{n+nn}{plt}
        \PY{k+kn}{from} \PY{n+nn}{mpl\PYZus{}toolkits}\PY{n+nn}{.}\PY{n+nn}{mplot3d} \PY{k}{import} \PY{n}{Axes3D}
        \PY{n}{font} \PY{o}{=} \PY{p}{\PYZob{}}\PY{l+s+s1}{\PYZsq{}}\PY{l+s+s1}{size}\PY{l+s+s1}{\PYZsq{}}\PY{p}{:} \PY{l+m+mi}{16}\PY{p}{\PYZcb{}}
        \PY{n}{matplotlib}\PY{o}{.}\PY{n}{rc}\PY{p}{(}\PY{l+s+s1}{\PYZsq{}}\PY{l+s+s1}{font}\PY{l+s+s1}{\PYZsq{}}\PY{p}{,} \PY{o}{*}\PY{o}{*}\PY{n}{font}\PY{p}{)}
\end{Verbatim}

    To plot a surface, we need to create a grid of \(xy\)-pairs, and
calculate the height (\(z\)), which is \(f(x,y)\). Then we feed this to
\texttt{matplotlib} to plot the surface.

    \begin{Verbatim}[commandchars=\\\{\}]
{\color{incolor}In [{\color{incolor}3}]:} \PY{k}{def} \PY{n+nf}{plot3d}\PY{p}{(}\PY{n}{f}\PY{p}{,}\PY{n}{lim}\PY{o}{=}\PY{p}{(}\PY{o}{\PYZhy{}}\PY{l+m+mi}{5}\PY{p}{,}\PY{l+m+mi}{5}\PY{p}{)}\PY{p}{,}\PY{n}{title}\PY{o}{=}\PY{l+s+s1}{\PYZsq{}}\PY{l+s+s1}{Surface plot}\PY{l+s+s1}{\PYZsq{}}\PY{p}{,}\PY{n}{detail}\PY{o}{=}\PY{l+m+mf}{0.05}\PY{p}{)}\PY{p}{:}
            \PY{n}{fig} \PY{o}{=} \PY{n}{plt}\PY{o}{.}\PY{n}{figure}\PY{p}{(}\PY{n}{figsize}\PY{o}{=}\PY{p}{(}\PY{l+m+mi}{16}\PY{p}{,}\PY{l+m+mi}{12}\PY{p}{)}\PY{p}{)}
            \PY{n}{ax} \PY{o}{=} \PY{n}{fig}\PY{o}{.}\PY{n}{add\PYZus{}subplot}\PY{p}{(}\PY{l+m+mi}{111}\PY{p}{,} \PY{n}{projection}\PY{o}{=}\PY{l+s+s1}{\PYZsq{}}\PY{l+s+s1}{3d}\PY{l+s+s1}{\PYZsq{}}\PY{p}{)}
            \PY{n}{xs} \PY{o}{=} \PY{n}{ys} \PY{o}{=} \PY{n}{np}\PY{o}{.}\PY{n}{arange}\PY{p}{(}\PY{n}{lim}\PY{p}{[}\PY{l+m+mi}{0}\PY{p}{]}\PY{p}{,}\PY{n}{lim}\PY{p}{[}\PY{l+m+mi}{1}\PY{p}{]}\PY{p}{,} \PY{n}{detail}\PY{p}{)}
            \PY{n}{X}\PY{p}{,} \PY{n}{Y} \PY{o}{=} \PY{n}{np}\PY{o}{.}\PY{n}{meshgrid}\PY{p}{(}\PY{n}{xs}\PY{p}{,} \PY{n}{ys}\PY{p}{)}
            \PY{n}{zs} \PY{o}{=} \PY{n}{np}\PY{o}{.}\PY{n}{array}\PY{p}{(}\PY{p}{[}\PY{n}{f}\PY{p}{(}\PY{n}{x}\PY{p}{,} \PY{n}{y}\PY{p}{)} \PY{k}{for} \PY{n}{x}\PY{p}{,} \PY{n}{y} \PY{o+ow}{in} \PY{n+nb}{zip}\PY{p}{(}\PY{n}{np}\PY{o}{.}\PY{n}{ravel}\PY{p}{(}\PY{n}{X}\PY{p}{)}\PY{p}{,} \PY{n}{np}\PY{o}{.}\PY{n}{ravel}\PY{p}{(}\PY{n}{Y}\PY{p}{)}\PY{p}{)}\PY{p}{]}\PY{p}{)}
            \PY{n}{Z} \PY{o}{=} \PY{n}{zs}\PY{o}{.}\PY{n}{reshape}\PY{p}{(}\PY{n}{X}\PY{o}{.}\PY{n}{shape}\PY{p}{)}
            \PY{n}{surf} \PY{o}{=} \PY{n}{ax}\PY{o}{.}\PY{n}{plot\PYZus{}surface}\PY{p}{(}\PY{n}{X}\PY{p}{,} \PY{n}{Y}\PY{p}{,} \PY{n}{Z}\PY{p}{,} \PY{n}{cmap}\PY{o}{=}\PY{n}{cm}\PY{o}{.}\PY{n}{jet}\PY{p}{)}
            \PY{n}{fig}\PY{o}{.}\PY{n}{colorbar}\PY{p}{(}\PY{n}{surf}\PY{p}{,} \PY{n}{shrink}\PY{o}{=}\PY{l+m+mf}{0.5}\PY{p}{,} \PY{n}{aspect}\PY{o}{=}\PY{l+m+mi}{5}\PY{p}{)}   
            \PY{n}{xlabel}\PY{p}{(}\PY{l+s+s1}{\PYZsq{}}\PY{l+s+s1}{X}\PY{l+s+s1}{\PYZsq{}}\PY{p}{)}\PY{p}{;}\PY{n}{ylabel}\PY{p}{(}\PY{l+s+s1}{\PYZsq{}}\PY{l+s+s1}{Y}\PY{l+s+s1}{\PYZsq{}}\PY{p}{)}\PY{p}{;}\PY{n}{ax}\PY{o}{.}\PY{n}{set\PYZus{}zlabel}\PY{p}{(}\PY{l+s+s1}{\PYZsq{}}\PY{l+s+s1}{\PYZdl{}f(x,y)\PYZdl{}}\PY{l+s+s1}{\PYZsq{}}\PY{p}{)}
            \PY{n}{plt}\PY{o}{.}\PY{n}{title}\PY{p}{(}\PY{n}{title}\PY{p}{)}\PY{p}{;}
\end{Verbatim}

    \begin{itemize}
\tightlist
\item
  \texttt{f} is a lambda function, which is the function \(f(x,y)\)
\item
  \texttt{lim=(a,b)} is the region \(x\in[a,b]\) and \(y\in[a,b]\) for
  which \(f\) is plotted.
\item
  \texttt{title=\textquotesingle{}\textquotesingle{}} is an optional
  title for the plot.
\item
  \texttt{detail=0.05} is the detail of the surface plot. High detail
  takes more time.
\end{itemize}

\textbf{Example:}
\texttt{plot3d(lambda:\ x+y,\ lim=(-1,1),\ title="title",\ detail=0.05)}

    \hypertarget{surface-graphs-of-basic-functions}{%
\newpage\section{Surface graphs of basic
functions}\label{surface-graphs-of-basic-functions}}

    \hypertarget{linear-xy}{%
\subsection{\texorpdfstring{Linear:
\(x+y\)}{Linear: x+y}}\label{linear-xy}}

    \begin{Verbatim}[commandchars=\\\{\}]
{\color{incolor}In [{\color{incolor}4}]:} \PY{n}{plot3d}\PY{p}{(}\PY{k}{lambda} \PY{n}{x}\PY{p}{,}\PY{n}{y}\PY{p}{:} \PY{n}{x}\PY{o}{+}\PY{n}{y}\PY{p}{)}
\end{Verbatim}

    \begin{center}
    \adjustimage{max size={0.9\linewidth}{0.9\paperheight}}{output_11_0.png}
    \end{center}
    { \hspace*{\fill} \\}
    
    \hypertarget{linear-x-y}{%
\newpage\subsection{\texorpdfstring{Linear:
\(x-y\)}{Linear: x-y}}\label{linear-x-y}}

    \begin{Verbatim}[commandchars=\\\{\}]
{\color{incolor}In [{\color{incolor}5}]:} \PY{n}{plot3d}\PY{p}{(}\PY{k}{lambda} \PY{n}{x}\PY{p}{,}\PY{n}{y}\PY{p}{:} \PY{n}{x}\PY{o}{\PYZhy{}}\PY{n}{y}\PY{p}{)}
\end{Verbatim}

    \begin{center}
    \adjustimage{max size={0.9\linewidth}{0.9\paperheight}}{output_13_0.png}
    \end{center}
    { \hspace*{\fill} \\}
    
    \hypertarget{power-xy}{%
\newpage\subsection{\texorpdfstring{Power: \(xy\)}{Power: xy}}\label{power-xy}}

    \begin{Verbatim}[commandchars=\\\{\}]
{\color{incolor}In [{\color{incolor}6}]:} \PY{n}{plot3d}\PY{p}{(}\PY{k}{lambda} \PY{n}{x}\PY{p}{,}\PY{n}{y}\PY{p}{:} \PY{n}{x}\PY{o}{*}\PY{n}{y}\PY{p}{)}
\end{Verbatim}

    \begin{center}
    \adjustimage{max size={0.9\linewidth}{0.9\paperheight}}{output_15_0.png}
    \end{center}
    { \hspace*{\fill} \\}
    
    \hypertarget{power-x2y2}{%
\newpage\subsection{\texorpdfstring{Power:
\(x^2+y^2\)}{Power: x\^{}2+y\^{}2}}\label{power-x2y2}}

    \begin{Verbatim}[commandchars=\\\{\}]
{\color{incolor}In [{\color{incolor}7}]:} \PY{n}{plot3d}\PY{p}{(}\PY{k}{lambda} \PY{n}{x}\PY{p}{,}\PY{n}{y}\PY{p}{:} \PY{n}{x}\PY{o}{*}\PY{o}{*}\PY{l+m+mi}{2}\PY{o}{+}\PY{n}{y}\PY{o}{*}\PY{o}{*}\PY{l+m+mi}{2}\PY{p}{)}
\end{Verbatim}

    \begin{center}
    \adjustimage{max size={0.9\linewidth}{0.9\paperheight}}{output_17_0.png}
    \end{center}
    { \hspace*{\fill} \\}
    
    \hypertarget{power-x2-y2}{%
\newpage\subsection{\texorpdfstring{Power:
\(x^2-y^2\)}{Power: x\^{}2-y\^{}2}}\label{power-x2-y2}}

    \begin{Verbatim}[commandchars=\\\{\}]
{\color{incolor}In [{\color{incolor}8}]:} \PY{n}{plot3d}\PY{p}{(}\PY{k}{lambda} \PY{n}{x}\PY{p}{,}\PY{n}{y}\PY{p}{:} \PY{n}{x}\PY{o}{*}\PY{o}{*}\PY{l+m+mi}{2}\PY{o}{\PYZhy{}}\PY{n}{y}\PY{o}{*}\PY{o}{*}\PY{l+m+mi}{2}\PY{p}{)}
\end{Verbatim}

    \begin{center}
    \adjustimage{max size={0.9\linewidth}{0.9\paperheight}}{output_19_0.png}
    \end{center}
    { \hspace*{\fill} \\}
    
    \hypertarget{power-x2y2}{%
\newpage\subsection{\texorpdfstring{Power:
\(x^2y^2\)}{Power: x\^{}2y\^{}2}}\label{power-x2y2}}

    \begin{Verbatim}[commandchars=\\\{\}]
{\color{incolor}In [{\color{incolor}9}]:} \PY{n}{plot3d}\PY{p}{(}\PY{k}{lambda} \PY{n}{x}\PY{p}{,}\PY{n}{y}\PY{p}{:} \PY{n}{x}\PY{o}{*}\PY{o}{*}\PY{l+m+mi}{2}\PY{o}{*}\PY{n}{y}\PY{o}{*}\PY{o}{*}\PY{l+m+mi}{2}\PY{p}{)}
\end{Verbatim}

    \begin{center}
    \adjustimage{max size={0.9\linewidth}{0.9\paperheight}}{output_21_0.png}
    \end{center}
    { \hspace*{\fill} \\}
    
    \hypertarget{reciprocal-dfracxy}{%
\newpage\subsection{\texorpdfstring{Reciprocal:
\(\dfrac{x}{y}\)}{Reciprocal: \textbackslash{}dfrac\{x\}\{y\}}}\label{reciprocal-dfracxy}}

    \begin{Verbatim}[commandchars=\\\{\}]
{\color{incolor}In [{\color{incolor}55}]:} \PY{n}{plot3d}\PY{p}{(}\PY{k}{lambda} \PY{n}{x}\PY{p}{,}\PY{n}{y}\PY{p}{:} \PY{n}{x}\PY{o}{/}\PY{n}{y}\PY{p}{,} \PY{n}{lim}\PY{o}{=}\PY{p}{(}\PY{o}{\PYZhy{}}\PY{o}{.}\PY{l+m+mi}{01}\PY{p}{,}\PY{o}{.}\PY{l+m+mi}{01}\PY{p}{)}\PY{p}{,} \PY{n}{detail}\PY{o}{=}\PY{l+m+mf}{0.001}\PY{p}{)}
\end{Verbatim}

    \begin{center}
    \adjustimage{max size={0.9\linewidth}{0.9\paperheight}}{output_23_0.png}
    \end{center}
    { \hspace*{\fill} \\}
    
    \hypertarget{root-sqrtxy}{%
\newpage\subsection{\texorpdfstring{Root:
\(\sqrt{x+y}\)}{Root: \textbackslash{}sqrt\{x+y\}}}\label{root-sqrtxy}}

    \begin{Verbatim}[commandchars=\\\{\}]
{\color{incolor}In [{\color{incolor}31}]:} \PY{n}{plot3d}\PY{p}{(}\PY{k}{lambda} \PY{n}{x}\PY{p}{,}\PY{n}{y}\PY{p}{:} \PY{n}{sqrt}\PY{p}{(}\PY{n}{x}\PY{o}{+}\PY{n}{y}\PY{p}{)}\PY{p}{,} \PY{n}{lim}\PY{o}{=}\PY{p}{(}\PY{l+m+mf}{0.01}\PY{p}{,}\PY{l+m+mi}{2}\PY{p}{)}\PY{p}{)}
\end{Verbatim}

    \begin{center}
    \adjustimage{max size={0.9\linewidth}{0.9\paperheight}}{output_25_0.png}
    \end{center}
    { \hspace*{\fill} \\}
    
    \hypertarget{root-sqrtx-y}{%
\newpage\subsection{\texorpdfstring{Root:
\(\sqrt{x-y}\)}{Root: \textbackslash{}sqrt\{x-y\}}}\label{root-sqrtx-y}}

    \begin{Verbatim}[commandchars=\\\{\}]
{\color{incolor}In [{\color{incolor}36}]:} \PY{n}{plot3d}\PY{p}{(}\PY{k}{lambda} \PY{n}{x}\PY{p}{,}\PY{n}{y}\PY{p}{:} \PY{n}{sqrt}\PY{p}{(}\PY{n}{x}\PY{o}{\PYZhy{}}\PY{n}{y}\PY{p}{)}\PY{p}{,} \PY{n}{lim}\PY{o}{=}\PY{p}{(}\PY{o}{\PYZhy{}}\PY{l+m+mi}{1}\PY{p}{,}\PY{l+m+mi}{1}\PY{p}{)}\PY{p}{)}
\end{Verbatim}

    \begin{Verbatim}[commandchars=\\\{\}]
c:\textbackslash{}users\textbackslash{}isomorphism\textbackslash{}miniconda3\textbackslash{}lib\textbackslash{}site-packages\textbackslash{}ipykernel\_launcher.py:1: RuntimeWarning: invalid value encountered in sqrt
  """Entry point for launching an IPython kernel.

    \end{Verbatim}

    \begin{center}
    \adjustimage{max size={0.9\linewidth}{0.9\paperheight}}{output_27_1.png}
    \end{center}
    { \hspace*{\fill} \\}
    
    \hypertarget{root-sqrtxy}{%
\newpage\subsection{\texorpdfstring{Root:
\(\sqrt{xy}\)}{Root: \textbackslash{}sqrt\{xy\}}}\label{root-sqrtxy}}

    \begin{Verbatim}[commandchars=\\\{\}]
{\color{incolor}In [{\color{incolor}35}]:} \PY{n}{plot3d}\PY{p}{(}\PY{k}{lambda} \PY{n}{x}\PY{p}{,}\PY{n}{y}\PY{p}{:} \PY{n}{sqrt}\PY{p}{(}\PY{n}{x}\PY{o}{*}\PY{n}{y}\PY{p}{)}\PY{p}{,} \PY{n}{lim}\PY{o}{=}\PY{p}{(}\PY{o}{\PYZhy{}}\PY{l+m+mi}{1}\PY{p}{,}\PY{l+m+mi}{1}\PY{p}{)}\PY{p}{)}
\end{Verbatim}

    \begin{Verbatim}[commandchars=\\\{\}]
c:\textbackslash{}users\textbackslash{}isomorphism\textbackslash{}miniconda3\textbackslash{}lib\textbackslash{}site-packages\textbackslash{}ipykernel\_launcher.py:1: RuntimeWarning: invalid value encountered in sqrt
  """Entry point for launching an IPython kernel.

    \end{Verbatim}

    \begin{center}
    \adjustimage{max size={0.9\linewidth}{0.9\paperheight}}{output_29_1.png}
    \end{center}
    { \hspace*{\fill} \\}
    
    \hypertarget{root-sqrtx2y2}{%
\newpage\subsection{\texorpdfstring{Root:
\(\sqrt{x^2+y^2}\)}{Root: \textbackslash{}sqrt\{x\^{}2+y\^{}2\}}}\label{root-sqrtx2y2}}

    \begin{Verbatim}[commandchars=\\\{\}]
{\color{incolor}In [{\color{incolor}38}]:} \PY{n}{plot3d}\PY{p}{(}\PY{k}{lambda} \PY{n}{x}\PY{p}{,}\PY{n}{y}\PY{p}{:} \PY{n}{sqrt}\PY{p}{(}\PY{n}{x}\PY{o}{*}\PY{n}{x}\PY{o}{+}\PY{n}{y}\PY{o}{*}\PY{n}{y}\PY{p}{)}\PY{p}{,} \PY{n}{lim}\PY{o}{=}\PY{p}{(}\PY{o}{\PYZhy{}}\PY{l+m+mi}{1}\PY{p}{,}\PY{l+m+mi}{1}\PY{p}{)}\PY{p}{)}
\end{Verbatim}

    \begin{center}
    \adjustimage{max size={0.9\linewidth}{0.9\paperheight}}{output_31_0.png}
    \end{center}
    { \hspace*{\fill} \\}
    
    \hypertarget{absolute-xy}{%
\newpage\subsection{\texorpdfstring{Absolute:
\(|\ x+y\ |\)}{Absolute: \textbar{}\textbackslash{} x+y\textbackslash{} \textbar{}}}\label{absolute-xy}}

    \begin{Verbatim}[commandchars=\\\{\}]
{\color{incolor}In [{\color{incolor}43}]:} \PY{n}{plot3d}\PY{p}{(}\PY{k}{lambda} \PY{n}{x}\PY{p}{,}\PY{n}{y}\PY{p}{:} \PY{n+nb}{abs}\PY{p}{(}\PY{n}{x}\PY{o}{+}\PY{n}{y}\PY{p}{)}\PY{p}{,} \PY{n}{lim}\PY{o}{=}\PY{p}{(}\PY{o}{\PYZhy{}}\PY{l+m+mi}{1}\PY{p}{,}\PY{l+m+mi}{1}\PY{p}{)}\PY{p}{)}
\end{Verbatim}

    \begin{center}
    \adjustimage{max size={0.9\linewidth}{0.9\paperheight}}{output_33_0.png}
    \end{center}
    { \hspace*{\fill} \\}
    
    \hypertarget{absolute-x-y}{%
\newpage\subsection{\texorpdfstring{Absolute:
\(|\ x-y\ |\)}{Absolute: \textbar{}\textbackslash{} x-y\textbackslash{} \textbar{}}}\label{absolute-x-y}}

    \begin{Verbatim}[commandchars=\\\{\}]
{\color{incolor}In [{\color{incolor}44}]:} \PY{n}{plot3d}\PY{p}{(}\PY{k}{lambda} \PY{n}{x}\PY{p}{,}\PY{n}{y}\PY{p}{:} \PY{n+nb}{abs}\PY{p}{(}\PY{n}{x}\PY{o}{\PYZhy{}}\PY{n}{y}\PY{p}{)}\PY{p}{,} \PY{n}{lim}\PY{o}{=}\PY{p}{(}\PY{o}{\PYZhy{}}\PY{l+m+mi}{1}\PY{p}{,}\PY{l+m+mi}{1}\PY{p}{)}\PY{p}{)}
\end{Verbatim}

    \begin{center}
    \adjustimage{max size={0.9\linewidth}{0.9\paperheight}}{output_35_0.png}
    \end{center}
    { \hspace*{\fill} \\}
    
    \hypertarget{absolute-xy}{%
\newpage\subsection{\texorpdfstring{Absolute:
\(|\ xy\ |\)}{Absolute: \textbar{}\textbackslash{} xy\textbackslash{} \textbar{}}}\label{absolute-xy}}

    \begin{Verbatim}[commandchars=\\\{\}]
{\color{incolor}In [{\color{incolor}45}]:} \PY{n}{plot3d}\PY{p}{(}\PY{k}{lambda} \PY{n}{x}\PY{p}{,}\PY{n}{y}\PY{p}{:} \PY{n+nb}{abs}\PY{p}{(}\PY{n}{x}\PY{o}{*}\PY{n}{y}\PY{p}{)}\PY{p}{,} \PY{n}{lim}\PY{o}{=}\PY{p}{(}\PY{o}{\PYZhy{}}\PY{l+m+mi}{1}\PY{p}{,}\PY{l+m+mi}{1}\PY{p}{)}\PY{p}{)}
\end{Verbatim}

    \begin{center}
    \adjustimage{max size={0.9\linewidth}{0.9\paperheight}}{output_37_0.png}
    \end{center}
    { \hspace*{\fill} \\}
    
    \hypertarget{trigonometric-cosxsiny}{%
\newpage\subsection{\texorpdfstring{Trigonometric:
\(\cos(x)+\sin(y)\)}{Trigonometric: \textbackslash{}cos(x)+\textbackslash{}sin(y)}}\label{trigonometric-cosxsiny}}

    \begin{Verbatim}[commandchars=\\\{\}]
{\color{incolor}In [{\color{incolor}57}]:} \PY{n}{plot3d}\PY{p}{(}\PY{k}{lambda} \PY{n}{x}\PY{p}{,}\PY{n}{y}\PY{p}{:} \PY{n}{cos}\PY{p}{(}\PY{n}{x}\PY{p}{)}\PY{o}{+}\PY{n}{sin}\PY{p}{(}\PY{n}{y}\PY{p}{)}\PY{p}{)}
\end{Verbatim}

    \begin{center}
    \adjustimage{max size={0.9\linewidth}{0.9\paperheight}}{output_39_0.png}
    \end{center}
    { \hspace*{\fill} \\}
    
    \hypertarget{trigonometric-sinxy}{%
\newpage\subsection{\texorpdfstring{Trigonometric:
\(\sin(x+y)\)}{Trigonometric: \textbackslash{}sin(x+y)}}\label{trigonometric-sinxy}}

    \begin{Verbatim}[commandchars=\\\{\}]
{\color{incolor}In [{\color{incolor}62}]:} \PY{n}{plot3d}\PY{p}{(}\PY{k}{lambda} \PY{n}{x}\PY{p}{,}\PY{n}{y}\PY{p}{:} \PY{n}{sin}\PY{p}{(}\PY{n}{x}\PY{o}{+}\PY{n}{y}\PY{p}{)}\PY{p}{)}
\end{Verbatim}

    \begin{center}
    \adjustimage{max size={0.9\linewidth}{0.9\paperheight}}{output_41_0.png}
    \end{center}
    { \hspace*{\fill} \\}
    
    \hypertarget{trigonometric-sinxy}{%
\newpage\subsection{\texorpdfstring{Trigonometric:
\(\sin(xy)\)}{Trigonometric: \textbackslash{}sin(xy)}}\label{trigonometric-sinxy}}

    \begin{Verbatim}[commandchars=\\\{\}]
{\color{incolor}In [{\color{incolor}74}]:} \PY{n}{plot3d}\PY{p}{(}\PY{k}{lambda} \PY{n}{x}\PY{p}{,}\PY{n}{y}\PY{p}{:} \PY{n}{sin}\PY{p}{(}\PY{n}{x}\PY{o}{*}\PY{n}{y}\PY{p}{)}\PY{p}{,}\PY{n}{lim}\PY{o}{=}\PY{p}{(}\PY{o}{\PYZhy{}}\PY{n}{pi}\PY{p}{,}\PY{n}{pi}\PY{p}{)}\PY{p}{,}\PY{n}{detail}\PY{o}{=}\PY{l+m+mf}{0.01}\PY{p}{)}
\end{Verbatim}

    \begin{center}
    \adjustimage{max size={0.9\linewidth}{0.9\paperheight}}{output_43_0.png}
    \end{center}
    { \hspace*{\fill} \\}
    
    \hypertarget{exponential-xy}{%
\newpage\subsection{\texorpdfstring{Exponential:
\(x^y\)}{Exponential: x\^{}y}}\label{exponential-xy}}

    \begin{Verbatim}[commandchars=\\\{\}]
{\color{incolor}In [{\color{incolor}77}]:} \PY{n}{plot3d}\PY{p}{(}\PY{k}{lambda} \PY{n}{x}\PY{p}{,}\PY{n}{y}\PY{p}{:} \PY{n}{x}\PY{o}{*}\PY{o}{*}\PY{n}{y}\PY{p}{,} \PY{n}{lim}\PY{o}{=}\PY{p}{(}\PY{l+m+mf}{0.001}\PY{p}{,} \PY{l+m+mi}{3}\PY{p}{)}\PY{p}{)}
\end{Verbatim}

    \begin{center}
    \adjustimage{max size={0.9\linewidth}{0.9\paperheight}}{output_45_0.png}
    \end{center}
    { \hspace*{\fill} \\}
    
    \hypertarget{exponential-exy}{%
\newpage\subsection{\texorpdfstring{Exponential:
\(e^{x+y}\)}{Exponential: e\^{}\{x+y\}}}\label{exponential-exy}}

    \begin{Verbatim}[commandchars=\\\{\}]
{\color{incolor}In [{\color{incolor}84}]:} \PY{n}{plot3d}\PY{p}{(}\PY{k}{lambda} \PY{n}{x}\PY{p}{,}\PY{n}{y}\PY{p}{:} \PY{n}{exp}\PY{p}{(}\PY{n}{x}\PY{o}{+}\PY{n}{y}\PY{p}{)}\PY{p}{,} \PY{n}{lim}\PY{o}{=}\PY{p}{(}\PY{o}{\PYZhy{}}\PY{l+m+mi}{2}\PY{p}{,}\PY{l+m+mi}{2}\PY{p}{)}\PY{p}{)}
\end{Verbatim}

    \begin{center}
    \adjustimage{max size={0.9\linewidth}{0.9\paperheight}}{output_47_0.png}
    \end{center}
    { \hspace*{\fill} \\}
    
    \hypertarget{exponential-exy}{%
\newpage\subsection{\texorpdfstring{Exponential:
\(e^{xy}\)}{Exponential: e\^{}\{xy\}}}\label{exponential-exy}}

    \begin{Verbatim}[commandchars=\\\{\}]
{\color{incolor}In [{\color{incolor}85}]:} \PY{n}{plot3d}\PY{p}{(}\PY{k}{lambda} \PY{n}{x}\PY{p}{,}\PY{n}{y}\PY{p}{:} \PY{n}{exp}\PY{p}{(}\PY{n}{x}\PY{o}{*}\PY{n}{y}\PY{p}{)}\PY{p}{,} \PY{n}{lim}\PY{o}{=}\PY{p}{(}\PY{o}{\PYZhy{}}\PY{l+m+mi}{2}\PY{p}{,}\PY{l+m+mi}{2}\PY{p}{)}\PY{p}{)}
\end{Verbatim}

    \begin{center}
    \adjustimage{max size={0.9\linewidth}{0.9\paperheight}}{output_49_0.png}
    \end{center}
    { \hspace*{\fill} \\}
    
    \hypertarget{logarithmic-lnxy}{%
\newpage\subsection{\texorpdfstring{Logarithmic:
\(\ln(x+y)\)}{Logarithmic: \textbackslash{}ln(x+y)}}\label{logarithmic-lnxy}}

    \begin{Verbatim}[commandchars=\\\{\}]
{\color{incolor}In [{\color{incolor}87}]:} \PY{n}{plot3d}\PY{p}{(}\PY{k}{lambda} \PY{n}{x}\PY{p}{,}\PY{n}{y}\PY{p}{:} \PY{n}{log}\PY{p}{(}\PY{n}{x}\PY{o}{+}\PY{n}{y}\PY{p}{)}\PY{p}{,} \PY{n}{lim}\PY{o}{=}\PY{p}{(}\PY{l+m+mf}{0.01}\PY{p}{,}\PY{l+m+mi}{2}\PY{p}{)}\PY{p}{)}
\end{Verbatim}

    \begin{center}
    \adjustimage{max size={0.9\linewidth}{0.9\paperheight}}{output_51_0.png}
    \end{center}
    { \hspace*{\fill} \\}
    
    \hypertarget{algebraic-exsin-y}{%
\newpage\subsection{\texorpdfstring{Algebraic:
\(e^x\sin y\)}{Algebraic: e\^{}x\textbackslash{}sin y}}\label{algebraic-exsin-y}}

    \begin{Verbatim}[commandchars=\\\{\}]
{\color{incolor}In [{\color{incolor}96}]:} \PY{n}{plot3d}\PY{p}{(}\PY{k}{lambda} \PY{n}{x}\PY{p}{,}\PY{n}{y}\PY{p}{:} \PY{n}{exp}\PY{p}{(}\PY{n}{x}\PY{p}{)}\PY{o}{*}\PY{n}{sin}\PY{p}{(}\PY{n}{y}\PY{p}{)}\PY{p}{)}
\end{Verbatim}

    \begin{center}
    \adjustimage{max size={0.9\linewidth}{0.9\paperheight}}{output_53_0.png}
    \end{center}
    { \hspace*{\fill} \\}
    
    \hypertarget{surfaces-of-algebraic-equations}{%
\newpage\section{Surfaces of algebraic
equations}\label{surfaces-of-algebraic-equations}}

    \hypertarget{example-fxy-2x3---3xy-2y}{%
\subsection{\texorpdfstring{Example:
\(f(x,y) = 2x^3 - 3xy + 2y\)}{Example: f(x,y) = 2x\^{}3 - 3xy + 2y}}\label{example-fxy-2x3---3xy-2y}}

    \begin{Verbatim}[commandchars=\\\{\}]
{\color{incolor}In [{\color{incolor}10}]:} \PY{n}{plot3d}\PY{p}{(}\PY{k}{lambda} \PY{n}{x}\PY{p}{,}\PY{n}{y}\PY{p}{:} \PY{l+m+mi}{2}\PY{o}{*}\PY{n}{x}\PY{o}{*}\PY{o}{*}\PY{l+m+mi}{3} \PY{o}{\PYZhy{}} \PY{l+m+mi}{3}\PY{o}{*}\PY{n}{x}\PY{o}{*}\PY{n}{y} \PY{o}{+} \PY{l+m+mi}{2}\PY{o}{*}\PY{n}{y}\PY{p}{)}
\end{Verbatim}

    \begin{center}
    \adjustimage{max size={0.9\linewidth}{0.9\paperheight}}{output_56_0.png}
    \end{center}
    { \hspace*{\fill} \\}
    
    \hypertarget{example-fxysinxcosy}{%
\newpage\subsection{\texorpdfstring{Example:
\(f(x,y)=\sin(x)+\cos(y)\)}{Example: f(x,y)=\textbackslash{}sin(x)+\textbackslash{}cos(y)}}\label{example-fxysinxcosy}}

    \begin{Verbatim}[commandchars=\\\{\}]
{\color{incolor}In [{\color{incolor}11}]:} \PY{n}{plot3d}\PY{p}{(}\PY{k}{lambda} \PY{n}{x}\PY{p}{,} \PY{n}{y}\PY{p}{:} \PY{n}{sin}\PY{p}{(}\PY{n}{x}\PY{p}{)} \PY{o}{+} \PY{n}{cos}\PY{p}{(}\PY{n}{y}\PY{p}{)}\PY{p}{,} \PY{n}{lim}\PY{o}{=}\PY{p}{(}\PY{o}{\PYZhy{}}\PY{l+m+mi}{8}\PY{p}{,}\PY{l+m+mi}{8}\PY{p}{)}\PY{p}{)}
\end{Verbatim}

    \begin{center}
    \adjustimage{max size={0.9\linewidth}{0.9\paperheight}}{output_58_0.png}
    \end{center}
    { \hspace*{\fill} \\}
    
    \hypertarget{example-fxyx2y2}{%
\newpage\subsection{\texorpdfstring{Example:
\(f(x,y)=x^2+y^2\)}{Example: f(x,y)=x\^{}2+y\^{}2}}\label{example-fxyx2y2}}

    \begin{Verbatim}[commandchars=\\\{\}]
{\color{incolor}In [{\color{incolor}12}]:} \PY{n}{plot3d}\PY{p}{(}\PY{k}{lambda} \PY{n}{x}\PY{p}{,} \PY{n}{y}\PY{p}{:} \PY{n}{x}\PY{o}{*}\PY{o}{*}\PY{l+m+mi}{2}\PY{o}{+}\PY{n}{y}\PY{o}{*}\PY{o}{*}\PY{l+m+mi}{2} \PY{p}{)}
\end{Verbatim}

    \begin{center}
    \adjustimage{max size={0.9\linewidth}{0.9\paperheight}}{output_60_0.png}
    \end{center}
    { \hspace*{\fill} \\}
    
    \hypertarget{example-fxyx2y3-y3}{%
\newpage\subsection{\texorpdfstring{Example:
\(f(x,y)=x^2y^3-y^3\)}{Example: f(x,y)=x\^{}2y\^{}3-y\^{}3}}\label{example-fxyx2y3-y3}}

    \begin{Verbatim}[commandchars=\\\{\}]
{\color{incolor}In [{\color{incolor}13}]:} \PY{n}{plot3d}\PY{p}{(}\PY{k}{lambda} \PY{n}{x}\PY{p}{,} \PY{n}{y}\PY{p}{:} \PY{n}{x}\PY{o}{*}\PY{o}{*}\PY{l+m+mi}{2}\PY{o}{*}\PY{n}{y}\PY{o}{*}\PY{o}{*}\PY{l+m+mi}{3}\PY{o}{\PYZhy{}}\PY{n}{y}\PY{o}{*}\PY{o}{*}\PY{l+m+mi}{3}\PY{p}{)}
\end{Verbatim}

    \begin{center}
    \adjustimage{max size={0.9\linewidth}{0.9\paperheight}}{output_62_0.png}
    \end{center}
    { \hspace*{\fill} \\}
    
    \hypertarget{example-fxylnsqrtx2y2}{%
\newpage\subsection{\texorpdfstring{Example
\(f(x,y)=\ln\sqrt{x^2+y^2}\)}{Example f(x,y)=\textbackslash{}ln\textbackslash{}sqrt\{x\^{}2+y\^{}2\}}}\label{example-fxylnsqrtx2y2}}

    \begin{Verbatim}[commandchars=\\\{\}]
{\color{incolor}In [{\color{incolor}27}]:} \PY{n}{plot3d}\PY{p}{(}\PY{k}{lambda} \PY{n}{x}\PY{p}{,}\PY{n}{y}\PY{p}{:} \PY{n}{log}\PY{p}{(}\PY{n}{sqrt}\PY{p}{(}\PY{n}{x}\PY{o}{*}\PY{o}{*}\PY{l+m+mi}{2}\PY{o}{+}\PY{n}{y}\PY{o}{*}\PY{o}{*}\PY{l+m+mi}{2}\PY{p}{)}\PY{p}{)}\PY{p}{,} \PY{n}{lim}\PY{o}{=}\PY{p}{(}\PY{l+m+mf}{0.1}\PY{p}{,}\PY{l+m+mi}{5}\PY{p}{)}\PY{p}{,} \PY{n}{detail}\PY{o}{=}\PY{l+m+mf}{0.025}\PY{p}{)}
\end{Verbatim}

    \begin{center}
    \adjustimage{max size={0.9\linewidth}{0.9\paperheight}}{output_64_0.png}
    \end{center}
    { \hspace*{\fill} \\}
    
    \hypertarget{example-fxysin-x-cosh-y-cos-x-sinh-y}{%
\newpage\subsection{\texorpdfstring{Example
\(f(x,y)=\sin x \cosh y + \cos x \sinh y\)}{Example f(x,y)=\textbackslash{}sin x \textbackslash{}cosh y + \textbackslash{}cos x \textbackslash{}sinh y}}\label{example-fxysin-x-cosh-y-cos-x-sinh-y}}

    \begin{Verbatim}[commandchars=\\\{\}]
{\color{incolor}In [{\color{incolor}28}]:} \PY{n}{plot3d}\PY{p}{(}\PY{k}{lambda} \PY{n}{x}\PY{p}{,}\PY{n}{y}\PY{p}{:} \PY{n}{sin}\PY{p}{(}\PY{n}{x}\PY{p}{)}\PY{o}{*}\PY{n}{cosh}\PY{p}{(}\PY{n}{y}\PY{p}{)}\PY{o}{+}\PY{n}{cos}\PY{p}{(}\PY{n}{x}\PY{p}{)}\PY{o}{*}\PY{n}{sin}\PY{p}{(}\PY{n}{y}\PY{p}{)}\PY{p}{,}\PY{n}{lim}\PY{o}{=}\PY{p}{(}\PY{o}{\PYZhy{}}\PY{l+m+mi}{8}\PY{p}{,}\PY{l+m+mi}{8}\PY{p}{)}\PY{p}{)}
\end{Verbatim}

    \begin{center}
    \adjustimage{max size={0.9\linewidth}{0.9\paperheight}}{output_66_0.png}
    \end{center}
    { \hspace*{\fill} \\}
    
    \hypertarget{example-fxye-xcos-y---e-y-cos-x}{%
\newpage\subsection{\texorpdfstring{Example:
\(f(x,y)=e^{-x}\cos y - e^{-y} \cos x\)}{Example: f(x,y)=e\^{}\{-x\}\textbackslash{}cos y - e\^{}\{-y\} \textbackslash{}cos x}}\label{example-fxye-xcos-y---e-y-cos-x}}

    \begin{Verbatim}[commandchars=\\\{\}]
{\color{incolor}In [{\color{incolor}29}]:} \PY{n}{plot3d}\PY{p}{(}\PY{k}{lambda} \PY{n}{x}\PY{p}{,}\PY{n}{y}\PY{p}{:} \PY{n}{exp}\PY{p}{(}\PY{o}{\PYZhy{}}\PY{n}{x}\PY{p}{)}\PY{o}{*}\PY{n}{cos}\PY{p}{(}\PY{n}{y}\PY{p}{)} \PY{o}{\PYZhy{}} \PY{n}{exp}\PY{p}{(}\PY{o}{\PYZhy{}}\PY{n}{y}\PY{p}{)}\PY{o}{*}\PY{n}{cos}\PY{p}{(}\PY{n}{x}\PY{p}{)}\PY{p}{,} \PY{n}{lim}\PY{o}{=}\PY{p}{(}\PY{o}{\PYZhy{}}\PY{l+m+mi}{10}\PY{p}{,}\PY{l+m+mi}{10}\PY{p}{)}\PY{p}{)}
\end{Verbatim}

    \begin{center}
    \adjustimage{max size={0.9\linewidth}{0.9\paperheight}}{output_68_0.png}
    \end{center}
    { \hspace*{\fill} \\}
    
    \hypertarget{surfaces-of-partial-derivatives}{%
\newpage\section{Surfaces of partial
derivatives}\label{surfaces-of-partial-derivatives}}

    \hypertarget{example-surface-area-of-a-human}{%
\subsection{Example: surface area of a
human}\label{example-surface-area-of-a-human}}

    The surface area of a human in \(\dfrac{m^2}{kg}\) is approximated by:

\[ S = 0.007184w^{0.425}h^{0.725} \]

What grows surface area faster for a person weighing 70kg with height
180cm, weight or height?

    First we will take a look at the surface plot of \(S(w,h)\) where
\(w=[10,125]\) and \(h=[10,125]\) (asymptote at \(0\)):

    \begin{Verbatim}[commandchars=\\\{\}]
{\color{incolor}In [{\color{incolor}14}]:} \PY{n}{plot3d}\PY{p}{(}\PY{k}{lambda} \PY{n}{w}\PY{p}{,}\PY{n}{h}\PY{p}{:} \PY{l+m+mf}{0.007184}\PY{o}{*}\PY{n}{w}\PY{o}{*}\PY{o}{*}\PY{l+m+mf}{0.425}\PY{o}{*}\PY{n}{h}\PY{o}{*}\PY{o}{*}\PY{l+m+mf}{0.725}\PY{p}{,} \PY{n}{lim}\PY{o}{=}\PY{p}{(}\PY{l+m+mi}{10}\PY{p}{,} \PY{l+m+mi}{125}\PY{p}{)}\PY{p}{,} 
                \PY{n}{title}\PY{o}{=}\PY{l+s+s1}{\PYZsq{}}\PY{l+s+s1}{Surface plot of \PYZdl{}S(w,h)\PYZdl{}}\PY{l+s+s1}{\PYZsq{}}\PY{p}{)}
\end{Verbatim}

    \begin{center}
    \adjustimage{max size={0.9\linewidth}{0.9\paperheight}}{output_73_0.png}
    \end{center}
    { \hspace*{\fill} \\}
    
    Taking the partial derivative with respect to \(w\) gives
\(\dfrac{\partial S}{\partial w} = 0.007184(0.425)w^{-0.575}h^{0.725}\),
and then the partial derivative with respect to \(h\) gives
\(\dfrac{\partial S}{\partial h} = 0.007184(0.725)w^{0.425}h^{-0.275}\).
Plotting the surfaces of both partial derivatives gives:

    \begin{Verbatim}[commandchars=\\\{\}]
{\color{incolor}In [{\color{incolor}15}]:} \PY{n}{plot3d}\PY{p}{(}\PY{k}{lambda} \PY{n}{w}\PY{p}{,}\PY{n}{h}\PY{p}{:} \PY{l+m+mf}{0.007184}\PY{o}{*}\PY{p}{(}\PY{l+m+mf}{0.425}\PY{p}{)}\PY{o}{*}\PY{n}{w}\PY{o}{*}\PY{o}{*}\PY{o}{\PYZhy{}}\PY{l+m+mf}{0.572}\PY{o}{*}\PY{n}{h}\PY{o}{*}\PY{o}{*}\PY{l+m+mf}{0.725}\PY{p}{,} \PY{n}{lim}\PY{o}{=}\PY{p}{(}\PY{l+m+mi}{10}\PY{p}{,} \PY{l+m+mi}{125}\PY{p}{)}\PY{p}{,} 
                \PY{n}{title}\PY{o}{=}\PY{l+s+s1}{\PYZsq{}}\PY{l+s+s1}{Surface plot of \PYZdl{}}\PY{l+s+s1}{\PYZbs{}}\PY{l+s+s1}{dfrac}\PY{l+s+s1}{\PYZob{}}\PY{l+s+s1}{\PYZbs{}}\PY{l+s+s1}{partial S\PYZcb{}}\PY{l+s+s1}{\PYZob{}}\PY{l+s+s1}{\PYZbs{}}\PY{l+s+s1}{partial w\PYZcb{}\PYZdl{}}\PY{l+s+s1}{\PYZsq{}}\PY{p}{)}
\end{Verbatim}

    \begin{center}
    \adjustimage{max size={0.9\linewidth}{0.9\paperheight}}{output_75_0.png}
    \end{center}
    { \hspace*{\fill} \\}
    
    \begin{Verbatim}[commandchars=\\\{\}]
{\color{incolor}In [{\color{incolor}16}]:} \PY{n+nb}{print}\PY{p}{(}\PY{l+s+s1}{\PYZsq{}}\PY{l+s+si}{\PYZob{}\PYZcb{}}\PY{l+s+s1}{ m²/kg}\PY{l+s+s1}{\PYZsq{}}\PY{o}{.}\PY{n}{format}\PY{p}{(}\PY{p}{(}\PY{k}{lambda} \PY{n}{w}\PY{p}{,}\PY{n}{h}\PY{p}{:} \PY{l+m+mf}{0.007184}\PY{o}{*}\PY{p}{(}\PY{l+m+mf}{0.425}\PY{p}{)}\PY{o}{*}\PY{n}{w}\PY{o}{*}\PY{o}{*}\PY{o}{\PYZhy{}}\PY{l+m+mf}{0.572}\PY{o}{*}\PY{n}{h}\PY{o}{*}\PY{o}{*}\PY{l+m+mf}{0.725}\PY{p}{)}
                                 \PY{p}{(}\PY{l+m+mi}{70}\PY{p}{,} \PY{l+m+mi}{180}\PY{p}{)}\PY{p}{)}\PY{p}{)}
\end{Verbatim}

    \begin{Verbatim}[commandchars=\\\{\}]
0.011599298931953175 m²/kg

    \end{Verbatim}

    \begin{Verbatim}[commandchars=\\\{\}]
{\color{incolor}In [{\color{incolor}17}]:} \PY{n}{plot3d}\PY{p}{(}\PY{k}{lambda} \PY{n}{w}\PY{p}{,}\PY{n}{h}\PY{p}{:} \PY{l+m+mf}{0.007184}\PY{o}{*}\PY{p}{(}\PY{l+m+mf}{0.725}\PY{p}{)}\PY{o}{*}\PY{n}{w}\PY{o}{*}\PY{o}{*}\PY{l+m+mf}{0.425}\PY{o}{*}\PY{n}{h}\PY{o}{*}\PY{o}{*}\PY{o}{\PYZhy{}}\PY{l+m+mf}{0.275}\PY{p}{,} \PY{n}{lim}\PY{o}{=}\PY{p}{(}\PY{l+m+mi}{10}\PY{p}{,} \PY{l+m+mi}{125}\PY{p}{)}\PY{p}{,} 
                \PY{n}{title}\PY{o}{=}\PY{l+s+s1}{\PYZsq{}}\PY{l+s+s1}{Surface plot of \PYZdl{}}\PY{l+s+s1}{\PYZbs{}}\PY{l+s+s1}{dfrac}\PY{l+s+s1}{\PYZob{}}\PY{l+s+s1}{\PYZbs{}}\PY{l+s+s1}{partial S\PYZcb{}}\PY{l+s+s1}{\PYZob{}}\PY{l+s+s1}{\PYZbs{}}\PY{l+s+s1}{partial h\PYZcb{}\PYZdl{}}\PY{l+s+s1}{\PYZsq{}}\PY{p}{)}
\end{Verbatim}

    \begin{center}
    \adjustimage{max size={0.9\linewidth}{0.9\paperheight}}{output_77_0.png}
    \end{center}
    { \hspace*{\fill} \\}
    
    \begin{Verbatim}[commandchars=\\\{\}]
{\color{incolor}In [{\color{incolor}18}]:} \PY{n+nb}{print}\PY{p}{(}\PY{l+s+s1}{\PYZsq{}}\PY{l+s+si}{\PYZob{}\PYZcb{}}\PY{l+s+s1}{ m²/kg}\PY{l+s+s1}{\PYZsq{}}\PY{o}{.}\PY{n}{format}\PY{p}{(}\PY{p}{(}\PY{k}{lambda} \PY{n}{w}\PY{p}{,}\PY{n}{h}\PY{p}{:} \PY{l+m+mf}{0.007184}\PY{o}{*}\PY{p}{(}\PY{l+m+mf}{0.725}\PY{p}{)}\PY{o}{*}\PY{n}{w}\PY{o}{*}\PY{o}{*}\PY{l+m+mf}{0.425}\PY{o}{*}\PY{n}{h}\PY{o}{*}\PY{o}{*}\PY{o}{\PYZhy{}}\PY{l+m+mf}{0.275}\PY{p}{)}
                                 \PY{p}{(}\PY{l+m+mi}{70}\PY{p}{,} \PY{l+m+mi}{180}\PY{p}{)}\PY{p}{)}\PY{p}{)}
\end{Verbatim}

    \begin{Verbatim}[commandchars=\\\{\}]
0.007597506115532552 m²/kg

    \end{Verbatim}

    We can clearly see that surface area in \(\dfrac{m^2}{kg}\) increases
that fastest when \(w\) increases for a person weighing 70 kg with
height 180 cm.

    As a bonus:

    \begin{Verbatim}[commandchars=\\\{\}]
{\color{incolor}In [{\color{incolor}19}]:} \PY{n}{plot3d}\PY{p}{(}\PY{k}{lambda} \PY{n}{w}\PY{p}{,}\PY{n}{h}\PY{p}{:} \PY{l+m+mf}{0.007184}\PY{o}{*}\PY{p}{(}\PY{l+m+mf}{0.425}\PY{p}{)}\PY{o}{*}\PY{n}{w}\PY{o}{*}\PY{o}{*}\PY{o}{\PYZhy{}}\PY{l+m+mf}{0.572}\PY{o}{*}\PY{n}{h}\PY{o}{*}\PY{o}{*}\PY{l+m+mf}{0.725} \PY{o}{+} 
                \PY{l+m+mf}{0.007184}\PY{o}{*}\PY{p}{(}\PY{l+m+mf}{0.725}\PY{p}{)}\PY{o}{*}\PY{n}{w}\PY{o}{*}\PY{o}{*}\PY{l+m+mf}{0.425}\PY{o}{*}\PY{n}{h}\PY{o}{*}\PY{o}{*}\PY{o}{\PYZhy{}}\PY{l+m+mf}{0.275}\PY{p}{,} \PY{n}{lim}\PY{o}{=}\PY{p}{(}\PY{l+m+mi}{10}\PY{p}{,} \PY{l+m+mi}{125}\PY{p}{)}\PY{p}{,} 
                \PY{n}{title}\PY{o}{=}\PY{l+s+s1}{\PYZsq{}}\PY{l+s+s1}{Surface plot of \PYZdl{}}\PY{l+s+s1}{\PYZbs{}}\PY{l+s+s1}{dfrac}\PY{l+s+s1}{\PYZob{}}\PY{l+s+s1}{\PYZbs{}}\PY{l+s+s1}{partial S\PYZcb{}}\PY{l+s+s1}{\PYZob{}}\PY{l+s+s1}{\PYZbs{}}\PY{l+s+s1}{partial w\PYZcb{} + }\PY{l+s+s1}{\PYZbs{}}\PY{l+s+s1}{dfrac}\PY{l+s+s1}{\PYZob{}}\PY{l+s+s1}{\PYZbs{}}\PY{l+s+s1}{partial S\PYZcb{}}\PY{l+s+s1}{\PYZob{}}\PY{l+s+s1}{\PYZbs{}}\PY{l+s+s1}{partial h\PYZcb{}\PYZdl{}}\PY{l+s+s1}{\PYZsq{}}\PY{p}{)}
\end{Verbatim}

    \begin{center}
    \adjustimage{max size={0.9\linewidth}{0.9\paperheight}}{output_81_0.png}
    \end{center}
    { \hspace*{\fill} \\}
    
    \hypertarget{example-wind-chill-index}{%
\newpage\subsection{Example: wind-chill index}\label{example-wind-chill-index}}

    The wind-chill index is modeled by the function:

\[ W = 13.12 + 0.6215T - 11.37v^{0.16} + 0.3965Tv^{0.16}\]

    \begin{Verbatim}[commandchars=\\\{\}]
{\color{incolor}In [{\color{incolor}20}]:} \PY{n}{plot3d}\PY{p}{(}\PY{k}{lambda} \PY{n}{T}\PY{p}{,}\PY{n}{v}\PY{p}{:} \PY{l+m+mf}{13.12} \PY{o}{+} \PY{l+m+mf}{0.6215}\PY{o}{*}\PY{n}{T} \PY{o}{\PYZhy{}} \PY{l+m+mf}{11.37}\PY{o}{*}\PY{n}{v}\PY{o}{*}\PY{o}{*}\PY{l+m+mf}{0.16} \PY{o}{+} \PY{l+m+mf}{0.3965}\PY{o}{*}\PY{n}{T}\PY{o}{*}\PY{n}{v}\PY{o}{*}\PY{o}{*}\PY{l+m+mf}{0.16}\PY{p}{,} 
                \PY{n}{lim}\PY{o}{=}\PY{p}{(}\PY{l+m+mf}{0.1}\PY{p}{,} \PY{l+m+mi}{50}\PY{p}{)}\PY{p}{)}
\end{Verbatim}

    \begin{center}
    \adjustimage{max size={0.9\linewidth}{0.9\paperheight}}{output_84_0.png}
    \end{center}
    { \hspace*{\fill} \\}
    
    \hypertarget{example-idk}{%
\newpage\subsection{Example: idk}\label{example-idk}}

    Let \(f(x,y)=x(x^2+y^2)^{-\frac{3}{2}}\cdot e^{\sin(x^2y)}\).

    \begin{Verbatim}[commandchars=\\\{\}]
{\color{incolor}In [{\color{incolor}21}]:} \PY{n}{plot3d}\PY{p}{(}\PY{k}{lambda} \PY{n}{x}\PY{p}{,} \PY{n}{y}\PY{p}{:} \PY{n}{x}\PY{o}{*}\PY{p}{(}\PY{n}{x}\PY{o}{*}\PY{o}{*}\PY{l+m+mi}{2} \PY{o}{+} \PY{n}{y}\PY{o}{*}\PY{o}{*}\PY{l+m+mi}{2}\PY{p}{)}\PY{o}{*}\PY{o}{*}\PY{p}{(}\PY{o}{\PYZhy{}}\PY{l+m+mi}{3}\PY{o}{/}\PY{l+m+mi}{2}\PY{p}{)}\PY{o}{*}\PY{n}{exp}\PY{p}{(}\PY{n}{sin}\PY{p}{(}\PY{n}{x}\PY{o}{*}\PY{o}{*}\PY{l+m+mi}{2}\PY{o}{*}\PY{n}{y}\PY{p}{)}\PY{p}{)}\PY{p}{,} \PY{n}{lim}\PY{o}{=}\PY{p}{(}\PY{l+m+mi}{1}\PY{p}{,}\PY{l+m+mi}{3}\PY{p}{)}\PY{p}{,}
               \PY{n}{title}\PY{o}{=}\PY{l+s+s2}{\PYZdq{}}\PY{l+s+s2}{Surface plot of \PYZdl{}f(x,y)=x(x\PYZca{}2+y\PYZca{}2)\PYZca{}}\PY{l+s+s2}{\PYZob{}}\PY{l+s+s2}{\PYZhy{}}\PY{l+s+se}{\PYZbs{}\PYZbs{}}\PY{l+s+s2}{frac}\PY{l+s+si}{\PYZob{}3\PYZcb{}}\PY{l+s+si}{\PYZob{}2\PYZcb{}}\PY{l+s+s2}{\PYZcb{}}\PY{l+s+s2}{\PYZbs{}}\PY{l+s+s2}{cdot e\PYZca{}}\PY{l+s+s2}{\PYZob{}}\PY{l+s+se}{\PYZbs{}\PYZbs{}}\PY{l+s+s2}{sin(x\PYZca{}2y)\PYZcb{}\PYZdl{}}\PY{l+s+s2}{\PYZdq{}}\PY{p}{)}
\end{Verbatim}

    \begin{center}
    \adjustimage{max size={0.9\linewidth}{0.9\paperheight}}{output_87_0.png}
    \end{center}
    { \hspace*{\fill} \\}
    
    If we find \(\dfrac{\partial f}{\partial x}\) for \(f(x,y)\), we get:

\[ \dfrac{\partial f}{\partial x} = (x^2+y^2)e^{\sin(x^2y)} - 3xe^{\sin(x^2y)}(x^2 + y^2)^{-\frac{5}{2}} + 2x^2ye^{\sin(x^2y)}\cdot (x^2+y^2)^{-\frac{3}{2}}\]

    \begin{Verbatim}[commandchars=\\\{\}]
{\color{incolor}In [{\color{incolor}22}]:} \PY{n}{plot3d}\PY{p}{(}\PY{k}{lambda} \PY{n}{x}\PY{p}{,} \PY{n}{y}\PY{p}{:} \PY{p}{(}\PY{n}{x}\PY{o}{*}\PY{o}{*}\PY{l+m+mi}{2}\PY{o}{+}\PY{n}{y}\PY{o}{*}\PY{o}{*}\PY{l+m+mi}{2}\PY{p}{)}\PY{o}{*}\PY{n}{exp}\PY{p}{(}\PY{n}{sin}\PY{p}{(}\PY{n}{x}\PY{o}{*}\PY{o}{*}\PY{l+m+mi}{2}\PY{o}{*}\PY{n}{y}\PY{p}{)}\PY{p}{)}\PY{o}{\PYZhy{}}
                \PY{l+m+mi}{3}\PY{o}{*}\PY{n}{x}\PY{o}{*}\PY{n}{exp}\PY{p}{(}\PY{n}{sin}\PY{p}{(}\PY{n}{x}\PY{o}{*}\PY{o}{*}\PY{l+m+mi}{2}\PY{o}{*}\PY{n}{y}\PY{p}{)}\PY{p}{)}\PY{o}{*}\PY{p}{(}\PY{n}{x}\PY{o}{*}\PY{o}{*}\PY{l+m+mi}{2}\PY{o}{+}\PY{n}{y}\PY{o}{*}\PY{o}{*}\PY{l+m+mi}{2}\PY{p}{)}\PY{o}{*}\PY{o}{*}\PY{p}{(}\PY{o}{\PYZhy{}}\PY{l+m+mi}{5}\PY{o}{/}\PY{l+m+mi}{2}\PY{p}{)}\PY{o}{+}
                \PY{l+m+mi}{2}\PY{o}{*}\PY{n}{x}\PY{o}{*}\PY{n}{y}\PY{o}{*}\PY{n}{exp}\PY{p}{(}\PY{n}{sin}\PY{p}{(}\PY{n}{x}\PY{o}{*}\PY{o}{*}\PY{l+m+mi}{2}\PY{o}{*}\PY{n}{y}\PY{p}{)}\PY{o}{*}\PY{n}{x}\PY{o}{*}\PY{p}{(}\PY{n}{x}\PY{o}{*}\PY{o}{*}\PY{l+m+mi}{2}\PY{o}{+}\PY{n}{y}\PY{o}{*}\PY{o}{*}\PY{l+m+mi}{2}\PY{p}{)}\PY{o}{*}\PY{o}{*}\PY{p}{(}\PY{o}{\PYZhy{}}\PY{l+m+mi}{3}\PY{o}{/}\PY{l+m+mi}{2}\PY{p}{)}\PY{p}{)}\PY{p}{,} 
             \PY{n}{lim}\PY{o}{=}\PY{p}{(}\PY{l+m+mi}{1}\PY{p}{,}\PY{l+m+mi}{3}\PY{p}{)}\PY{p}{,} \PY{n}{title}\PY{o}{=}\PY{l+s+s2}{\PYZdq{}}\PY{l+s+s2}{Surface plot of \PYZdl{}}\PY{l+s+se}{\PYZbs{}\PYZbs{}}\PY{l+s+s2}{dfrac}\PY{l+s+s2}{\PYZob{}}\PY{l+s+se}{\PYZbs{}\PYZbs{}}\PY{l+s+s2}{partial f\PYZcb{}}\PY{l+s+s2}{\PYZob{}}\PY{l+s+se}{\PYZbs{}\PYZbs{}}\PY{l+s+s2}{partial x\PYZcb{}\PYZdl{}}\PY{l+s+s2}{\PYZdq{}}\PY{p}{)}
\end{Verbatim}

    \begin{center}
    \adjustimage{max size={0.9\linewidth}{0.9\paperheight}}{output_89_0.png}
    \end{center}
    { \hspace*{\fill} \\}
    
    \hypertarget{example-fxyx5y5}{%
\newpage\subsection{\texorpdfstring{Example
\(f(x,y)=x^5y^5\)}{Example f(x,y)=x\^{}5y\^{}5}}\label{example-fxyx5y5}}

    \begin{Verbatim}[commandchars=\\\{\}]
{\color{incolor}In [{\color{incolor}23}]:} \PY{n}{plot3d}\PY{p}{(}\PY{k}{lambda} \PY{n}{x}\PY{p}{,}\PY{n}{y}\PY{p}{:} \PY{n}{x}\PY{o}{*}\PY{o}{*}\PY{l+m+mi}{5}\PY{o}{*}\PY{n}{y}\PY{o}{*}\PY{o}{*}\PY{l+m+mi}{5}\PY{p}{,}\PY{n}{lim}\PY{o}{=}\PY{p}{(}\PY{o}{\PYZhy{}}\PY{l+m+mi}{1}\PY{p}{,}\PY{l+m+mi}{1}\PY{p}{)}\PY{p}{)}
\end{Verbatim}

    \begin{center}
    \adjustimage{max size={0.9\linewidth}{0.9\paperheight}}{output_91_0.png}
    \end{center}
    { \hspace*{\fill} \\}
    
    Now let's create a surface plot of \(\dfrac{\partial f}{\partial x}\):

    \begin{Verbatim}[commandchars=\\\{\}]
{\color{incolor}In [{\color{incolor}24}]:} \PY{n}{plot3d}\PY{p}{(}\PY{k}{lambda} \PY{n}{x}\PY{p}{,}\PY{n}{y}\PY{p}{:} \PY{l+m+mi}{5}\PY{o}{*}\PY{n}{x}\PY{o}{*}\PY{o}{*}\PY{l+m+mi}{4}\PY{o}{*}\PY{n}{y}\PY{o}{*}\PY{o}{*}\PY{l+m+mi}{5}\PY{p}{,}\PY{n}{lim}\PY{o}{=}\PY{p}{(}\PY{o}{\PYZhy{}}\PY{l+m+mi}{1}\PY{p}{,}\PY{l+m+mi}{1}\PY{p}{)}\PY{p}{,}
                \PY{n}{title}\PY{o}{=}\PY{l+s+s1}{\PYZsq{}}\PY{l+s+s1}{Surface plot of \PYZdl{}}\PY{l+s+se}{\PYZbs{}\PYZbs{}}\PY{l+s+s1}{dfrac}\PY{l+s+s1}{\PYZob{}}\PY{l+s+se}{\PYZbs{}\PYZbs{}}\PY{l+s+s1}{partial f\PYZcb{}}\PY{l+s+s1}{\PYZob{}}\PY{l+s+se}{\PYZbs{}\PYZbs{}}\PY{l+s+s1}{partial x\PYZcb{}\PYZdl{}}\PY{l+s+s1}{\PYZsq{}}\PY{p}{)}
\end{Verbatim}

    \begin{center}
    \adjustimage{max size={0.9\linewidth}{0.9\paperheight}}{output_93_0.png}
    \end{center}
    { \hspace*{\fill} \\}
    
    And again, but now for \(\dfrac{\partial f}{\partial y}\):

    \begin{Verbatim}[commandchars=\\\{\}]
{\color{incolor}In [{\color{incolor}25}]:} \PY{n}{plot3d}\PY{p}{(}\PY{k}{lambda} \PY{n}{x}\PY{p}{,}\PY{n}{y}\PY{p}{:} \PY{l+m+mi}{5}\PY{o}{*}\PY{n}{x}\PY{o}{*}\PY{o}{*}\PY{l+m+mi}{5}\PY{o}{*}\PY{n}{y}\PY{o}{*}\PY{o}{*}\PY{l+m+mi}{4}\PY{p}{,}\PY{n}{lim}\PY{o}{=}\PY{p}{(}\PY{o}{\PYZhy{}}\PY{l+m+mi}{1}\PY{p}{,}\PY{l+m+mi}{1}\PY{p}{)}\PY{p}{,}
                \PY{n}{title}\PY{o}{=}\PY{l+s+s1}{\PYZsq{}}\PY{l+s+s1}{Surface plot of \PYZdl{}}\PY{l+s+se}{\PYZbs{}\PYZbs{}}\PY{l+s+s1}{dfrac}\PY{l+s+s1}{\PYZob{}}\PY{l+s+se}{\PYZbs{}\PYZbs{}}\PY{l+s+s1}{partial f\PYZcb{}}\PY{l+s+s1}{\PYZob{}}\PY{l+s+se}{\PYZbs{}\PYZbs{}}\PY{l+s+s1}{partial y\PYZcb{}\PYZdl{}}\PY{l+s+s1}{\PYZsq{}}\PY{p}{)}
\end{Verbatim}

    \begin{center}
    \adjustimage{max size={0.9\linewidth}{0.9\paperheight}}{output_95_0.png}
    \end{center}
    { \hspace*{\fill} \\}
    
    If we combine both partial derivatives, that is
\(\dfrac{\partial f}{\partial x}\)+\(\dfrac{\partial f}{\partial y}\),
we get:

    \begin{Verbatim}[commandchars=\\\{\}]
{\color{incolor}In [{\color{incolor}26}]:} \PY{n}{plot3d}\PY{p}{(}\PY{k}{lambda} \PY{n}{x}\PY{p}{,}\PY{n}{y}\PY{p}{:} \PY{l+m+mi}{5}\PY{o}{*}\PY{n}{x}\PY{o}{*}\PY{o}{*}\PY{l+m+mi}{4}\PY{o}{*}\PY{n}{y}\PY{o}{*}\PY{o}{*}\PY{l+m+mi}{5}\PY{o}{+}\PY{l+m+mi}{5}\PY{o}{*}\PY{n}{x}\PY{o}{*}\PY{o}{*}\PY{l+m+mi}{5}\PY{o}{*}\PY{n}{y}\PY{o}{*}\PY{o}{*}\PY{l+m+mi}{4}\PY{p}{,}\PY{n}{lim}\PY{o}{=}\PY{p}{(}\PY{o}{\PYZhy{}}\PY{l+m+mi}{1}\PY{p}{,}\PY{l+m+mi}{1}\PY{p}{)}\PY{p}{,}
                \PY{n}{title}\PY{o}{=}\PY{l+s+s1}{\PYZsq{}}\PY{l+s+s1}{Surface plot of \PYZdl{}}\PY{l+s+se}{\PYZbs{}\PYZbs{}}\PY{l+s+s1}{dfrac}\PY{l+s+s1}{\PYZob{}}\PY{l+s+se}{\PYZbs{}\PYZbs{}}\PY{l+s+s1}{partial f\PYZcb{}}\PY{l+s+s1}{\PYZob{}}\PY{l+s+se}{\PYZbs{}\PYZbs{}}\PY{l+s+s1}{partial x\PYZcb{}\PYZdl{}+\PYZdl{}}\PY{l+s+se}{\PYZbs{}\PYZbs{}}\PY{l+s+s1}{dfrac}\PY{l+s+s1}{\PYZob{}}\PY{l+s+se}{\PYZbs{}\PYZbs{}}\PY{l+s+s1}{partial f\PYZcb{}}\PY{l+s+s1}{\PYZob{}}\PY{l+s+se}{\PYZbs{}\PYZbs{}}\PY{l+s+s1}{partial y\PYZcb{}\PYZdl{}}\PY{l+s+s1}{\PYZsq{}}\PY{p}{)}
\end{Verbatim}

    \begin{center}
    \adjustimage{max size={0.9\linewidth}{0.9\paperheight}}{output_97_0.png}
    \end{center}
    { \hspace*{\fill} \\}
    
    A nice observation is that at the point \((1,-1)\) we can see that
\(\dfrac{\partial f}{\partial x}\) is negative, and
\(\dfrac{\partial f}{\partial y}\) is positive. Now we have lost
information for both derivatives, which makes adding derivatives a bad
idea. Both are completely different quantities, namely the slope of a
point on the surface with the tangent line parallel to the \(xz\) plane
for the x-axis (and the \(yz\) plane for the y-axis).


    % Add a bibliography block to the postdoc
    
    
    
    \end{document}
