
% Default to the notebook output style

    


% Inherit from the specified cell style.




    
\documentclass[11pt]{article}

    
    
    \usepackage[T1]{fontenc}
    % Nicer default font (+ math font) than Computer Modern for most use cases
    \usepackage{mathpazo}

    % Basic figure setup, for now with no caption control since it's done
    % automatically by Pandoc (which extracts ![](path) syntax from Markdown).
    \usepackage{graphicx}
    % We will generate all images so they have a width \maxwidth. This means
    % that they will get their normal width if they fit onto the page, but
    % are scaled down if they would overflow the margins.
    \makeatletter
    \def\maxwidth{\ifdim\Gin@nat@width>\linewidth\linewidth
    \else\Gin@nat@width\fi}
    \makeatother
    \let\Oldincludegraphics\includegraphics
    % Set max figure width to be 80% of text width, for now hardcoded.
    \renewcommand{\includegraphics}[1]{\Oldincludegraphics[width=.8\maxwidth]{#1}}
    % Ensure that by default, figures have no caption (until we provide a
    % proper Figure object with a Caption API and a way to capture that
    % in the conversion process - todo).
    \usepackage{caption}
    \DeclareCaptionLabelFormat{nolabel}{}
    \captionsetup{labelformat=nolabel}

    \usepackage{adjustbox} % Used to constrain images to a maximum size 
    \usepackage{xcolor} % Allow colors to be defined
    \usepackage{enumerate} % Needed for markdown enumerations to work
    \usepackage{geometry} % Used to adjust the document margins
    \usepackage{amsmath} % Equations
    \usepackage{amssymb} % Equations
    \usepackage{textcomp} % defines textquotesingle
    % Hack from http://tex.stackexchange.com/a/47451/13684:
    \AtBeginDocument{%
        \def\PYZsq{\textquotesingle}% Upright quotes in Pygmentized code
    }
    \usepackage{upquote} % Upright quotes for verbatim code
    \usepackage{eurosym} % defines \euro
    \usepackage[mathletters]{ucs} % Extended unicode (utf-8) support
    \usepackage[utf8x]{inputenc} % Allow utf-8 characters in the tex document
    \usepackage{fancyvrb} % verbatim replacement that allows latex
    \usepackage{grffile} % extends the file name processing of package graphics 
                         % to support a larger range 
    % The hyperref package gives us a pdf with properly built
    % internal navigation ('pdf bookmarks' for the table of contents,
    % internal cross-reference links, web links for URLs, etc.)
    \usepackage{hyperref}
    \usepackage{longtable} % longtable support required by pandoc >1.10
    \usepackage{booktabs}  % table support for pandoc > 1.12.2
    \usepackage[inline]{enumitem} % IRkernel/repr support (it uses the enumerate* environment)
    \usepackage[normalem]{ulem} % ulem is needed to support strikethroughs (\sout)
                                % normalem makes italics be italics, not underlines
    

    
    
    % Colors for the hyperref package
    \definecolor{urlcolor}{rgb}{0,.145,.698}
    \definecolor{linkcolor}{rgb}{.71,0.21,0.01}
    \definecolor{citecolor}{rgb}{.12,.54,.11}

    % ANSI colors
    \definecolor{ansi-black}{HTML}{3E424D}
    \definecolor{ansi-black-intense}{HTML}{282C36}
    \definecolor{ansi-red}{HTML}{E75C58}
    \definecolor{ansi-red-intense}{HTML}{B22B31}
    \definecolor{ansi-green}{HTML}{00A250}
    \definecolor{ansi-green-intense}{HTML}{007427}
    \definecolor{ansi-yellow}{HTML}{DDB62B}
    \definecolor{ansi-yellow-intense}{HTML}{B27D12}
    \definecolor{ansi-blue}{HTML}{208FFB}
    \definecolor{ansi-blue-intense}{HTML}{0065CA}
    \definecolor{ansi-magenta}{HTML}{D160C4}
    \definecolor{ansi-magenta-intense}{HTML}{A03196}
    \definecolor{ansi-cyan}{HTML}{60C6C8}
    \definecolor{ansi-cyan-intense}{HTML}{258F8F}
    \definecolor{ansi-white}{HTML}{C5C1B4}
    \definecolor{ansi-white-intense}{HTML}{A1A6B2}

    % commands and environments needed by pandoc snippets
    % extracted from the output of `pandoc -s`
    \providecommand{\tightlist}{%
      \setlength{\itemsep}{0pt}\setlength{\parskip}{0pt}}
    \DefineVerbatimEnvironment{Highlighting}{Verbatim}{commandchars=\\\{\}}
    % Add ',fontsize=\small' for more characters per line
    \newenvironment{Shaded}{}{}
    \newcommand{\KeywordTok}[1]{\textcolor[rgb]{0.00,0.44,0.13}{\textbf{{#1}}}}
    \newcommand{\DataTypeTok}[1]{\textcolor[rgb]{0.56,0.13,0.00}{{#1}}}
    \newcommand{\DecValTok}[1]{\textcolor[rgb]{0.25,0.63,0.44}{{#1}}}
    \newcommand{\BaseNTok}[1]{\textcolor[rgb]{0.25,0.63,0.44}{{#1}}}
    \newcommand{\FloatTok}[1]{\textcolor[rgb]{0.25,0.63,0.44}{{#1}}}
    \newcommand{\CharTok}[1]{\textcolor[rgb]{0.25,0.44,0.63}{{#1}}}
    \newcommand{\StringTok}[1]{\textcolor[rgb]{0.25,0.44,0.63}{{#1}}}
    \newcommand{\CommentTok}[1]{\textcolor[rgb]{0.38,0.63,0.69}{\textit{{#1}}}}
    \newcommand{\OtherTok}[1]{\textcolor[rgb]{0.00,0.44,0.13}{{#1}}}
    \newcommand{\AlertTok}[1]{\textcolor[rgb]{1.00,0.00,0.00}{\textbf{{#1}}}}
    \newcommand{\FunctionTok}[1]{\textcolor[rgb]{0.02,0.16,0.49}{{#1}}}
    \newcommand{\RegionMarkerTok}[1]{{#1}}
    \newcommand{\ErrorTok}[1]{\textcolor[rgb]{1.00,0.00,0.00}{\textbf{{#1}}}}
    \newcommand{\NormalTok}[1]{{#1}}
    
    % Additional commands for more recent versions of Pandoc
    \newcommand{\ConstantTok}[1]{\textcolor[rgb]{0.53,0.00,0.00}{{#1}}}
    \newcommand{\SpecialCharTok}[1]{\textcolor[rgb]{0.25,0.44,0.63}{{#1}}}
    \newcommand{\VerbatimStringTok}[1]{\textcolor[rgb]{0.25,0.44,0.63}{{#1}}}
    \newcommand{\SpecialStringTok}[1]{\textcolor[rgb]{0.73,0.40,0.53}{{#1}}}
    \newcommand{\ImportTok}[1]{{#1}}
    \newcommand{\DocumentationTok}[1]{\textcolor[rgb]{0.73,0.13,0.13}{\textit{{#1}}}}
    \newcommand{\AnnotationTok}[1]{\textcolor[rgb]{0.38,0.63,0.69}{\textbf{\textit{{#1}}}}}
    \newcommand{\CommentVarTok}[1]{\textcolor[rgb]{0.38,0.63,0.69}{\textbf{\textit{{#1}}}}}
    \newcommand{\VariableTok}[1]{\textcolor[rgb]{0.10,0.09,0.49}{{#1}}}
    \newcommand{\ControlFlowTok}[1]{\textcolor[rgb]{0.00,0.44,0.13}{\textbf{{#1}}}}
    \newcommand{\OperatorTok}[1]{\textcolor[rgb]{0.40,0.40,0.40}{{#1}}}
    \newcommand{\BuiltInTok}[1]{{#1}}
    \newcommand{\ExtensionTok}[1]{{#1}}
    \newcommand{\PreprocessorTok}[1]{\textcolor[rgb]{0.74,0.48,0.00}{{#1}}}
    \newcommand{\AttributeTok}[1]{\textcolor[rgb]{0.49,0.56,0.16}{{#1}}}
    \newcommand{\InformationTok}[1]{\textcolor[rgb]{0.38,0.63,0.69}{\textbf{\textit{{#1}}}}}
    \newcommand{\WarningTok}[1]{\textcolor[rgb]{0.38,0.63,0.69}{\textbf{\textit{{#1}}}}}
    
    
    % Define a nice break command that doesn't care if a line doesn't already
    % exist.
    \def\br{\hspace*{\fill} \\* }
    % Math Jax compatability definitions
    \def\gt{>}
    \def\lt{<}
    % Document parameters
    \title{MCL dataset analysis}
    
    
    

    % Pygments definitions
    
\makeatletter
\def\PY@reset{\let\PY@it=\relax \let\PY@bf=\relax%
    \let\PY@ul=\relax \let\PY@tc=\relax%
    \let\PY@bc=\relax \let\PY@ff=\relax}
\def\PY@tok#1{\csname PY@tok@#1\endcsname}
\def\PY@toks#1+{\ifx\relax#1\empty\else%
    \PY@tok{#1}\expandafter\PY@toks\fi}
\def\PY@do#1{\PY@bc{\PY@tc{\PY@ul{%
    \PY@it{\PY@bf{\PY@ff{#1}}}}}}}
\def\PY#1#2{\PY@reset\PY@toks#1+\relax+\PY@do{#2}}

\expandafter\def\csname PY@tok@w\endcsname{\def\PY@tc##1{\textcolor[rgb]{0.73,0.73,0.73}{##1}}}
\expandafter\def\csname PY@tok@c\endcsname{\let\PY@it=\textit\def\PY@tc##1{\textcolor[rgb]{0.25,0.50,0.50}{##1}}}
\expandafter\def\csname PY@tok@cp\endcsname{\def\PY@tc##1{\textcolor[rgb]{0.74,0.48,0.00}{##1}}}
\expandafter\def\csname PY@tok@k\endcsname{\let\PY@bf=\textbf\def\PY@tc##1{\textcolor[rgb]{0.00,0.50,0.00}{##1}}}
\expandafter\def\csname PY@tok@kp\endcsname{\def\PY@tc##1{\textcolor[rgb]{0.00,0.50,0.00}{##1}}}
\expandafter\def\csname PY@tok@kt\endcsname{\def\PY@tc##1{\textcolor[rgb]{0.69,0.00,0.25}{##1}}}
\expandafter\def\csname PY@tok@o\endcsname{\def\PY@tc##1{\textcolor[rgb]{0.40,0.40,0.40}{##1}}}
\expandafter\def\csname PY@tok@ow\endcsname{\let\PY@bf=\textbf\def\PY@tc##1{\textcolor[rgb]{0.67,0.13,1.00}{##1}}}
\expandafter\def\csname PY@tok@nb\endcsname{\def\PY@tc##1{\textcolor[rgb]{0.00,0.50,0.00}{##1}}}
\expandafter\def\csname PY@tok@nf\endcsname{\def\PY@tc##1{\textcolor[rgb]{0.00,0.00,1.00}{##1}}}
\expandafter\def\csname PY@tok@nc\endcsname{\let\PY@bf=\textbf\def\PY@tc##1{\textcolor[rgb]{0.00,0.00,1.00}{##1}}}
\expandafter\def\csname PY@tok@nn\endcsname{\let\PY@bf=\textbf\def\PY@tc##1{\textcolor[rgb]{0.00,0.00,1.00}{##1}}}
\expandafter\def\csname PY@tok@ne\endcsname{\let\PY@bf=\textbf\def\PY@tc##1{\textcolor[rgb]{0.82,0.25,0.23}{##1}}}
\expandafter\def\csname PY@tok@nv\endcsname{\def\PY@tc##1{\textcolor[rgb]{0.10,0.09,0.49}{##1}}}
\expandafter\def\csname PY@tok@no\endcsname{\def\PY@tc##1{\textcolor[rgb]{0.53,0.00,0.00}{##1}}}
\expandafter\def\csname PY@tok@nl\endcsname{\def\PY@tc##1{\textcolor[rgb]{0.63,0.63,0.00}{##1}}}
\expandafter\def\csname PY@tok@ni\endcsname{\let\PY@bf=\textbf\def\PY@tc##1{\textcolor[rgb]{0.60,0.60,0.60}{##1}}}
\expandafter\def\csname PY@tok@na\endcsname{\def\PY@tc##1{\textcolor[rgb]{0.49,0.56,0.16}{##1}}}
\expandafter\def\csname PY@tok@nt\endcsname{\let\PY@bf=\textbf\def\PY@tc##1{\textcolor[rgb]{0.00,0.50,0.00}{##1}}}
\expandafter\def\csname PY@tok@nd\endcsname{\def\PY@tc##1{\textcolor[rgb]{0.67,0.13,1.00}{##1}}}
\expandafter\def\csname PY@tok@s\endcsname{\def\PY@tc##1{\textcolor[rgb]{0.73,0.13,0.13}{##1}}}
\expandafter\def\csname PY@tok@sd\endcsname{\let\PY@it=\textit\def\PY@tc##1{\textcolor[rgb]{0.73,0.13,0.13}{##1}}}
\expandafter\def\csname PY@tok@si\endcsname{\let\PY@bf=\textbf\def\PY@tc##1{\textcolor[rgb]{0.73,0.40,0.53}{##1}}}
\expandafter\def\csname PY@tok@se\endcsname{\let\PY@bf=\textbf\def\PY@tc##1{\textcolor[rgb]{0.73,0.40,0.13}{##1}}}
\expandafter\def\csname PY@tok@sr\endcsname{\def\PY@tc##1{\textcolor[rgb]{0.73,0.40,0.53}{##1}}}
\expandafter\def\csname PY@tok@ss\endcsname{\def\PY@tc##1{\textcolor[rgb]{0.10,0.09,0.49}{##1}}}
\expandafter\def\csname PY@tok@sx\endcsname{\def\PY@tc##1{\textcolor[rgb]{0.00,0.50,0.00}{##1}}}
\expandafter\def\csname PY@tok@m\endcsname{\def\PY@tc##1{\textcolor[rgb]{0.40,0.40,0.40}{##1}}}
\expandafter\def\csname PY@tok@gh\endcsname{\let\PY@bf=\textbf\def\PY@tc##1{\textcolor[rgb]{0.00,0.00,0.50}{##1}}}
\expandafter\def\csname PY@tok@gu\endcsname{\let\PY@bf=\textbf\def\PY@tc##1{\textcolor[rgb]{0.50,0.00,0.50}{##1}}}
\expandafter\def\csname PY@tok@gd\endcsname{\def\PY@tc##1{\textcolor[rgb]{0.63,0.00,0.00}{##1}}}
\expandafter\def\csname PY@tok@gi\endcsname{\def\PY@tc##1{\textcolor[rgb]{0.00,0.63,0.00}{##1}}}
\expandafter\def\csname PY@tok@gr\endcsname{\def\PY@tc##1{\textcolor[rgb]{1.00,0.00,0.00}{##1}}}
\expandafter\def\csname PY@tok@ge\endcsname{\let\PY@it=\textit}
\expandafter\def\csname PY@tok@gs\endcsname{\let\PY@bf=\textbf}
\expandafter\def\csname PY@tok@gp\endcsname{\let\PY@bf=\textbf\def\PY@tc##1{\textcolor[rgb]{0.00,0.00,0.50}{##1}}}
\expandafter\def\csname PY@tok@go\endcsname{\def\PY@tc##1{\textcolor[rgb]{0.53,0.53,0.53}{##1}}}
\expandafter\def\csname PY@tok@gt\endcsname{\def\PY@tc##1{\textcolor[rgb]{0.00,0.27,0.87}{##1}}}
\expandafter\def\csname PY@tok@err\endcsname{\def\PY@bc##1{\setlength{\fboxsep}{0pt}\fcolorbox[rgb]{1.00,0.00,0.00}{1,1,1}{\strut ##1}}}
\expandafter\def\csname PY@tok@kc\endcsname{\let\PY@bf=\textbf\def\PY@tc##1{\textcolor[rgb]{0.00,0.50,0.00}{##1}}}
\expandafter\def\csname PY@tok@kd\endcsname{\let\PY@bf=\textbf\def\PY@tc##1{\textcolor[rgb]{0.00,0.50,0.00}{##1}}}
\expandafter\def\csname PY@tok@kn\endcsname{\let\PY@bf=\textbf\def\PY@tc##1{\textcolor[rgb]{0.00,0.50,0.00}{##1}}}
\expandafter\def\csname PY@tok@kr\endcsname{\let\PY@bf=\textbf\def\PY@tc##1{\textcolor[rgb]{0.00,0.50,0.00}{##1}}}
\expandafter\def\csname PY@tok@bp\endcsname{\def\PY@tc##1{\textcolor[rgb]{0.00,0.50,0.00}{##1}}}
\expandafter\def\csname PY@tok@fm\endcsname{\def\PY@tc##1{\textcolor[rgb]{0.00,0.00,1.00}{##1}}}
\expandafter\def\csname PY@tok@vc\endcsname{\def\PY@tc##1{\textcolor[rgb]{0.10,0.09,0.49}{##1}}}
\expandafter\def\csname PY@tok@vg\endcsname{\def\PY@tc##1{\textcolor[rgb]{0.10,0.09,0.49}{##1}}}
\expandafter\def\csname PY@tok@vi\endcsname{\def\PY@tc##1{\textcolor[rgb]{0.10,0.09,0.49}{##1}}}
\expandafter\def\csname PY@tok@vm\endcsname{\def\PY@tc##1{\textcolor[rgb]{0.10,0.09,0.49}{##1}}}
\expandafter\def\csname PY@tok@sa\endcsname{\def\PY@tc##1{\textcolor[rgb]{0.73,0.13,0.13}{##1}}}
\expandafter\def\csname PY@tok@sb\endcsname{\def\PY@tc##1{\textcolor[rgb]{0.73,0.13,0.13}{##1}}}
\expandafter\def\csname PY@tok@sc\endcsname{\def\PY@tc##1{\textcolor[rgb]{0.73,0.13,0.13}{##1}}}
\expandafter\def\csname PY@tok@dl\endcsname{\def\PY@tc##1{\textcolor[rgb]{0.73,0.13,0.13}{##1}}}
\expandafter\def\csname PY@tok@s2\endcsname{\def\PY@tc##1{\textcolor[rgb]{0.73,0.13,0.13}{##1}}}
\expandafter\def\csname PY@tok@sh\endcsname{\def\PY@tc##1{\textcolor[rgb]{0.73,0.13,0.13}{##1}}}
\expandafter\def\csname PY@tok@s1\endcsname{\def\PY@tc##1{\textcolor[rgb]{0.73,0.13,0.13}{##1}}}
\expandafter\def\csname PY@tok@mb\endcsname{\def\PY@tc##1{\textcolor[rgb]{0.40,0.40,0.40}{##1}}}
\expandafter\def\csname PY@tok@mf\endcsname{\def\PY@tc##1{\textcolor[rgb]{0.40,0.40,0.40}{##1}}}
\expandafter\def\csname PY@tok@mh\endcsname{\def\PY@tc##1{\textcolor[rgb]{0.40,0.40,0.40}{##1}}}
\expandafter\def\csname PY@tok@mi\endcsname{\def\PY@tc##1{\textcolor[rgb]{0.40,0.40,0.40}{##1}}}
\expandafter\def\csname PY@tok@il\endcsname{\def\PY@tc##1{\textcolor[rgb]{0.40,0.40,0.40}{##1}}}
\expandafter\def\csname PY@tok@mo\endcsname{\def\PY@tc##1{\textcolor[rgb]{0.40,0.40,0.40}{##1}}}
\expandafter\def\csname PY@tok@ch\endcsname{\let\PY@it=\textit\def\PY@tc##1{\textcolor[rgb]{0.25,0.50,0.50}{##1}}}
\expandafter\def\csname PY@tok@cm\endcsname{\let\PY@it=\textit\def\PY@tc##1{\textcolor[rgb]{0.25,0.50,0.50}{##1}}}
\expandafter\def\csname PY@tok@cpf\endcsname{\let\PY@it=\textit\def\PY@tc##1{\textcolor[rgb]{0.25,0.50,0.50}{##1}}}
\expandafter\def\csname PY@tok@c1\endcsname{\let\PY@it=\textit\def\PY@tc##1{\textcolor[rgb]{0.25,0.50,0.50}{##1}}}
\expandafter\def\csname PY@tok@cs\endcsname{\let\PY@it=\textit\def\PY@tc##1{\textcolor[rgb]{0.25,0.50,0.50}{##1}}}

\def\PYZbs{\char`\\}
\def\PYZus{\char`\_}
\def\PYZob{\char`\{}
\def\PYZcb{\char`\}}
\def\PYZca{\char`\^}
\def\PYZam{\char`\&}
\def\PYZlt{\char`\<}
\def\PYZgt{\char`\>}
\def\PYZsh{\char`\#}
\def\PYZpc{\char`\%}
\def\PYZdl{\char`\$}
\def\PYZhy{\char`\-}
\def\PYZsq{\char`\'}
\def\PYZdq{\char`\"}
\def\PYZti{\char`\~}
% for compatibility with earlier versions
\def\PYZat{@}
\def\PYZlb{[}
\def\PYZrb{]}
\makeatother


    % Exact colors from NB
    \definecolor{incolor}{rgb}{0.0, 0.0, 0.5}
    \definecolor{outcolor}{rgb}{0.545, 0.0, 0.0}



    
    % Prevent overflowing lines due to hard-to-break entities
    \sloppy 
    % Setup hyperref package
    \hypersetup{
      breaklinks=true,  % so long urls are correctly broken across lines
      colorlinks=true,
      urlcolor=urlcolor,
      linkcolor=linkcolor,
      citecolor=citecolor,
      }
    % Slightly bigger margins than the latex defaults
    
    \geometry{verbose,tmargin=1in,bmargin=1in,lmargin=1in,rmargin=1in}
    
    

    \begin{document}
    
    
    \maketitle
    
    

    
    \hypertarget{onderzoekvragen}{%
\subsection{Onderzoekvragen}\label{onderzoekvragen}}

Vanuit de data analyse is het de bedoeling dat er een antwoord komt op
de onderstaande vragen:

\begin{itemize}
\tightlist
\item
  Hoeveel verschillende operaties zijn er?
\item
  Hoe zijn de verschillende operaties verdeeld (frequentietabel)
\item
  Voor elke operatie het volgende bepalen (bedieningstijd):

  \begin{itemize}
  \tightlist
  \item
    Hoeveel minuten duurt de operatie?
  \item
    Hoeveel minuten duurt een operatie gemiddeld?
  \item
    Wat is de afwijking in minuten voor een operatie?
  \item
    Wat voor verdeling de behandeltijd van de operatie?
  \end{itemize}
\item
  Op welke dagen vinden de operaties plaats? (frequentietabel)
\item
  Hoeveel operaties zijn er gemiddeld per dag?
\item
  Hoeveel operaties zijn er gemiddeld per week?
\item
  Wat voor verdeling volgen de operaties per dag in de week (is het
  bijv. op maandag drukker dan op een andere dag)
\item
  Hoeveel verschillende operatiekamers zijn er en welke wordt het vaakst
  gebruikt? (Frequentietabel, operaties per week per operatiekamer)
\end{itemize}

    \hypertarget{dataset}{%
\subsection{Dataset}\label{dataset}}

    \begin{Verbatim}[commandchars=\\\{\}]
{\color{incolor}In [{\color{incolor}62}]:} df \PY{o}{=} read.csv\PY{p}{(}\PY{l+s}{\PYZdq{}}\PY{l+s}{dataset.csv\PYZdq{}}\PY{p}{,} header \PY{o}{=} \PY{k+kc}{TRUE}\PY{p}{)}
         procedure.codes \PY{o}{=} \PY{k+kt}{list}\PY{p}{(}\PY{l+s}{\PYZsq{}}\PY{l+s}{331800B\PYZsq{}}\PY{p}{,} \PY{l+s}{\PYZsq{}}\PY{l+s}{331801C\PYZsq{}}\PY{p}{,} \PY{l+s}{\PYZsq{}}\PY{l+s}{332090H\PYZsq{}}\PY{p}{,} \PY{l+s}{\PYZsq{}}\PY{l+s}{332320B\PYZsq{}}\PY{p}{,} \PY{l+s}{\PYZsq{}}\PY{l+s}{332320S\PYZsq{}}\PY{p}{,} \PY{l+s}{\PYZsq{}}\PY{l+s}{332323\PYZsq{}}\PY{p}{,} \PY{l+s}{\PYZsq{}}\PY{l+s}{332330\PYZsq{}}\PY{p}{)}
         dff \PY{o}{=} df\PY{p}{[}df\PY{o}{\PYZdl{}}Registration.Procedure.Code \PY{o}{\PYZpc{}in\PYZpc{}} procedure.codes\PY{p}{,} \PY{p}{]}
         dff\PY{o}{\PYZdl{}}Registration.Procedure.Code \PY{o}{=} \PY{k+kp}{droplevels}\PY{p}{(}dff\PY{o}{\PYZdl{}}Registration.Procedure.Code\PY{p}{)}
         \PY{k+kp}{colnames}\PY{p}{(}dff\PY{p}{)}
\end{Verbatim}


    \begin{enumerate*}
\item 'Patient.Primary.Mrn'
\item 'Patient.Birth.Date'
\item 'Performing.Department.Department.Name'
\item 'Registration.Procedure.Code'
\item 'Registration.Procedure.Name'
\item 'Service.Date.Day.Of.Week'
\item 'Service.Date.Year'
\item 'Service.Provider.Id'
\item 'Log.ID'
\item 'Log.Status'
\item 'Patient.Class'
\item 'Date'
\item 'Case.Request'
\item 'Scheduled.in.Room'
\item 'Patient.in.Room'
\item 'Scheduled.out.of.Room'
\item 'Patient.out.of.Room'
\item 'Procedure.Complete'
\item 'Procedure.Start'
\item 'Room'
\item 'Admission.Date'
\item 'Discharge.Date'
\item 'Department'
\end{enumerate*}


    
    De code \texttt{332311H} heeft geen records.

    \begin{Verbatim}[commandchars=\\\{\}]
{\color{incolor}In [{\color{incolor}63}]:} \PY{k+kp}{nrow}\PY{p}{(}dff\PY{p}{[}dff\PY{o}{\PYZdl{}}Registration.Procedure.Codes \PY{o}{==} \PY{l+s}{\PYZsq{}}\PY{l+s}{332311H\PYZsq{}}\PY{p}{,}\PY{p}{]}\PY{p}{)}
\end{Verbatim}


    0

    
    \hypertarget{aantal-rijen}{%
\subsubsection{Aantal rijen}\label{aantal-rijen}}

    De dataset bevat een totaal aantal rijen van:

    \begin{Verbatim}[commandchars=\\\{\}]
{\color{incolor}In [{\color{incolor}64}]:} \PY{k+kp}{nrow}\PY{p}{(}df\PY{p}{)}
\end{Verbatim}


    6624

    
    \hypertarget{aantal-relevante-rijen}{%
\subsubsection{Aantal relevante rijen}\label{aantal-relevante-rijen}}

    Hierop is een filtering gemaakt op basis van procedure codes. De
procedure codes die relevant zijn voor het onderzoek zijn:

\begin{itemize}
\tightlist
\item
  331800B, 331801C, 332090H, 332311H, 332320B, 332320S, 332323, 332330
\end{itemize}

Hiervan is het volgende aantal rijen relevant voor het onderzoek:

    \begin{Verbatim}[commandchars=\\\{\}]
{\color{incolor}In [{\color{incolor}65}]:} \PY{k+kp}{nrow}\PY{p}{(}dff\PY{p}{)}
\end{Verbatim}


    3048

    
    Extra toelichtingen voor de dataset (tips van het MCL):

\begin{itemize}
\tightlist
\item
  \texttt{Service.Provider.Id} is de arts (arts code)
\item
  \texttt{Datum} is de uitvoerdatum van de operatie
\item
  Bedieningstijd = kamer uit - kamer in (in minuten)
\item
  Geannuleerde behandeling: hiervan de \texttt{Patient.in.Room} =
  \texttt{00:00:00}. Deze moeten worden weggefilterd.
\end{itemize}

    \hypertarget{aantal-geannuleerde-behandelingen}{%
\subsubsection{Aantal geannuleerde
behandelingen}\label{aantal-geannuleerde-behandelingen}}

    Het aantal geannuleerde behandelingen is:

    \begin{Verbatim}[commandchars=\\\{\}]
{\color{incolor}In [{\color{incolor}66}]:} \PY{k+kp}{nrow}\PY{p}{(}dff\PY{p}{[}\PY{p}{(}dff\PY{o}{\PYZdl{}}Patient.in.Room \PY{o}{==} \PY{l+s}{\PYZdq{}}\PY{l+s}{00:00:00\PYZdq{}}\PY{p}{)}\PY{p}{,} \PY{p}{]}\PY{p}{)}
\end{Verbatim}


    11

    
    Deze moeten worden weggefilterd uit de dataset.

    \begin{Verbatim}[commandchars=\\\{\}]
{\color{incolor}In [{\color{incolor}67}]:} dff \PY{o}{=} dff\PY{p}{[}\PY{p}{(}dff\PY{o}{\PYZdl{}}Patient.in.Room \PY{o}{!=} \PY{l+s}{\PYZdq{}}\PY{l+s}{00:00:00\PYZdq{}}\PY{p}{)}\PY{p}{,} \PY{p}{]}
\end{Verbatim}


    \begin{Verbatim}[commandchars=\\\{\}]
{\color{incolor}In [{\color{incolor}68}]:} \PY{k+kp}{nrow}\PY{p}{(}dff\PY{p}{)}
\end{Verbatim}


    3037

    
    \hypertarget{berekende-variabelen}{%
\subsection{Berekende variabelen}\label{berekende-variabelen}}

Om de bedieningstijden te kunnen bepalen, moeten we eerst een extra
kolom toevoegen met de behandeltijd in minuten. Hiervoor gebruiken we
\texttt{bedieningstijd\ =\ kamer\ uit\ -\ kamer\ in}.

    \begin{Verbatim}[commandchars=\\\{\}]
{\color{incolor}In [{\color{incolor}69}]:} time.format \PY{o}{=} \PY{l+s}{\PYZdq{}}\PY{l+s}{\PYZpc{}H:\PYZpc{}M:\PYZpc{}S\PYZdq{}}
         time.patient.in.room \PY{o}{=} \PY{k+kp}{as.POSIXlt}\PY{p}{(}dff\PY{o}{\PYZdl{}}Patient.in.Room\PY{p}{,} format\PY{o}{=}time.format\PY{p}{)}
         time.patient.out.room \PY{o}{=} \PY{k+kp}{as.POSIXlt}\PY{p}{(}dff\PY{o}{\PYZdl{}}Patient.out.of.Room\PY{p}{,} format\PY{o}{=}time.format\PY{p}{)}
         dff\PY{o}{\PYZdl{}}Service.Time \PY{o}{=} \PY{k+kp}{as.numeric}\PY{p}{(}\PY{k+kp}{difftime}\PY{p}{(}time.patient.out.room\PY{p}{,} time.patient.in.room\PY{p}{,} units\PY{o}{=}\PY{l+s}{\PYZdq{}}\PY{l+s}{mins\PYZdq{}}\PY{p}{)}\PY{p}{)}
         hist\PY{p}{(}dff\PY{p}{[}dff\PY{o}{\PYZdl{}}Service.Time \PY{o}{\PYZlt{}} \PY{l+m}{60}\PY{p}{,} \PY{p}{]}\PY{o}{\PYZdl{}}Service.Time\PY{p}{,} breaks\PY{o}{=}\PY{l+m}{50}\PY{p}{,} main\PY{o}{=}\PY{l+s}{\PYZdq{}}\PY{l+s}{Behandeltijd (alle operatiecodes)\PYZdq{}}\PY{p}{,} xlab\PY{o}{=}\PY{l+s}{\PYZdq{}}\PY{l+s}{Behandeltijd in minuten\PYZdq{}}\PY{p}{,} ylab\PY{o}{=}\PY{l+s}{\PYZdq{}}\PY{l+s}{Frequentie\PYZdq{}}\PY{p}{)}
\end{Verbatim}


    \begin{center}
    \adjustimage{max size={0.9\linewidth}{0.9\paperheight}}{output_19_0.png}
    \end{center}
    { \hspace*{\fill} \\}
    
    \begin{Verbatim}[commandchars=\\\{\}]
{\color{incolor}In [{\color{incolor}70}]:} boxplot\PY{p}{(}dff\PY{o}{\PYZdl{}}Service.Time\PY{p}{)}
\end{Verbatim}


    \begin{center}
    \adjustimage{max size={0.9\linewidth}{0.9\paperheight}}{output_20_0.png}
    \end{center}
    { \hspace*{\fill} \\}
    
    \begin{Verbatim}[commandchars=\\\{\}]
{\color{incolor}In [{\color{incolor}71}]:} boxplot\PY{p}{(}dff\PY{p}{[}dff\PY{o}{\PYZdl{}}Service.Time \PY{o}{\PYZlt{}} \PY{l+m}{30}\PY{p}{,}\PY{p}{]}\PY{o}{\PYZdl{}}Service.Time\PY{p}{)}
\end{Verbatim}


    \begin{center}
    \adjustimage{max size={0.9\linewidth}{0.9\paperheight}}{output_21_0.png}
    \end{center}
    { \hspace*{\fill} \\}
    
    \hypertarget{antwoorden-op-de-onderzoekvragen}{%
\subsection{Antwoorden op de
onderzoekvragen}\label{antwoorden-op-de-onderzoekvragen}}

    \hypertarget{hoeveel-verschillende-operaties-zijn-er}{%
\subsubsection{Hoeveel verschillende operaties zijn
er?}\label{hoeveel-verschillende-operaties-zijn-er}}

    Het aantal verschillende operaties is hetzelfde als het aantal
procedurecodes:

    \begin{Verbatim}[commandchars=\\\{\}]
{\color{incolor}In [{\color{incolor}75}]:} \PY{k+kp}{length}\PY{p}{(}procedure.codes\PY{p}{)}
\end{Verbatim}


    7

    
    \hypertarget{hoe-zijn-de-verschillende-operaties-verdeeld}{%
\subsubsection{Hoe zijn de verschillende operaties
verdeeld?}\label{hoe-zijn-de-verschillende-operaties-verdeeld}}

    Dit is gemakkelijk te zien met een frequentietabel. Hiervoor worden het
aantal records geteld per procedure code:

    \begin{Verbatim}[commandchars=\\\{\}]
{\color{incolor}In [{\color{incolor}76}]:} \PY{k+kp}{summary}\PY{p}{(}dff\PY{o}{\PYZdl{}}Registration.Procedure.Code\PY{p}{)}
\end{Verbatim}


    \begin{description*}
\item[331800B] 18
\item[331801C] 1739
\item[332090H] 28
\item[332320B] 71
\item[332320S] 60
\item[332323] 646
\item[332330] 475
\end{description*}


    
    \begin{Verbatim}[commandchars=\\\{\}]
{\color{incolor}In [{\color{incolor}77}]:} plot\PY{p}{(}dff\PY{o}{\PYZdl{}}Registration.Procedure.Code\PY{p}{)}
\end{Verbatim}


    \begin{center}
    \adjustimage{max size={0.9\linewidth}{0.9\paperheight}}{output_29_0.png}
    \end{center}
    { \hspace*{\fill} \\}
    
    \hypertarget{wat-is-de-bedieningstijd-voor-elke-operatie}{%
\subsubsection{Wat is de bedieningstijd voor elke
operatie?}\label{wat-is-de-bedieningstijd-voor-elke-operatie}}

    In dit gedeelte willen we per operatie code het volgende bepalen:

\begin{itemize}
\tightlist
\item
  Gemiddelde
\item
  Standaardafwijking
\item
  Verdeling
\item
  Betrouwbaarheidsinterval
\end{itemize}

    \begin{Verbatim}[commandchars=\\\{\}]
{\color{incolor}In [{\color{incolor}78}]:} n\PY{o}{=}\PY{l+m}{50}
         mu \PY{o}{=} \PY{k+kt}{c}\PY{p}{(}\PY{k+kp}{length}\PY{p}{(}procedure.codes\PY{p}{)}\PY{p}{)}
         sigma \PY{o}{=} \PY{k+kt}{c}\PY{p}{(}\PY{k+kp}{length}\PY{p}{(}procedure.codes\PY{p}{)}\PY{p}{)}
         count \PY{o}{=} \PY{k+kt}{c}\PY{p}{(}\PY{k+kp}{length}\PY{p}{(}procedure.codes\PY{p}{)}\PY{p}{)}
         \PY{k+kr}{for}\PY{p}{(}i \PY{k+kr}{in} \PY{l+m}{1}\PY{o}{:}\PY{p}{(}\PY{k+kp}{length}\PY{p}{(}procedure.codes\PY{p}{)}\PY{p}{)}\PY{p}{)} \PY{p}{\PYZob{}}
             mu\PY{p}{[}i\PY{p}{]} \PY{o}{=} \PY{k+kp}{mean}\PY{p}{(}dff\PY{p}{[}dff\PY{o}{\PYZdl{}}Registration.Procedure.Code \PY{o}{==} procedure.codes\PY{p}{[}i\PY{p}{]}\PY{p}{,}\PY{p}{]}\PY{o}{\PYZdl{}}Service.Time\PY{p}{)}
             sigma\PY{p}{[}i\PY{p}{]} \PY{o}{=} sd\PY{p}{(}dff\PY{p}{[}dff\PY{o}{\PYZdl{}}Registration.Procedure.Code \PY{o}{==} procedure.codes\PY{p}{[}i\PY{p}{]}\PY{p}{,}\PY{p}{]}\PY{o}{\PYZdl{}}Service.Time\PY{p}{)}
             count\PY{p}{[}i\PY{p}{]} \PY{o}{=} \PY{k+kp}{nrow}\PY{p}{(}dff\PY{p}{[}dff\PY{o}{\PYZdl{}}Registration.Procedure.Code \PY{o}{==} procedure.codes\PY{p}{[}i\PY{p}{]}\PY{p}{,}\PY{p}{]}\PY{p}{)}
         \PY{p}{\PYZcb{}}
         \PY{k+kp}{cbind}\PY{p}{(}procedure.codes\PY{p}{,} mu\PY{p}{,} sigma\PY{p}{,} count\PY{p}{)}
\end{Verbatim}


    \begin{tabular}{llll}
 procedure.codes & mu & sigma & count\\
\hline
	 331800B  & 25.66667 & 74.87951 & 18      \\
	 331801C  & 11.39333 & 20.97718 & 1739    \\
	 332090H  & 19.42857 & 20.29857 & 28      \\
	 332320B & 20.8169 & 20.2395 & 71     \\
	 332320S  & 14.31667 & 14.96945 & 60      \\
	 332323   & 8.170279 & 15.63029 & 646     \\
	 332330   & 14.31158 & 19.05111 & 475     \\
\end{tabular}


    
    Om een aselecte steekproef te nemen met \(n=50\), kunnen we niet van
alle procedures een steekproef nemen. In eerste instantie is er een
betrouwbaarheidsinterval opgesteld voor alle procedures waarvan er 50 of
meer records beschikbaar zijn. Om een interval voor \(\mu\) te bepalen
gebruiken we:

\[ \bar{x} \pm z_{1-\alpha/2}\cdot\frac{\sigma}{\sqrt{n}} \]

    \begin{Verbatim}[commandchars=\\\{\}]
{\color{incolor}In [{\color{incolor}79}]:} mean.interval \PY{o}{=} \PY{k+kr}{function}\PY{p}{(}X\PY{p}{,} alpha\PY{p}{)} \PY{p}{\PYZob{}}
             n \PY{o}{=} \PY{k+kp}{length}\PY{p}{(}X\PY{p}{)}
             sigma \PY{o}{=} sd\PY{p}{(}X\PY{p}{)}
             sem \PY{o}{=} sigma \PY{o}{/} \PY{k+kp}{sqrt}\PY{p}{(}n\PY{p}{)}
             E \PY{o}{=} qnorm\PY{p}{(}\PY{l+m}{1}\PY{o}{\PYZhy{}}alpha\PY{o}{/}\PY{l+m}{2}\PY{p}{)}\PY{o}{*}sem
             xbar \PY{o}{=} \PY{k+kp}{mean}\PY{p}{(}X\PY{p}{)}
             xbar \PY{o}{+} \PY{k+kt}{c}\PY{p}{(}\PY{o}{\PYZhy{}}E\PY{p}{,}E\PY{p}{)}
         \PY{p}{\PYZcb{}}
\end{Verbatim}


    Om even te testen of het werkt, proberen we het uit met de volgende
gegevens:

    \begin{Verbatim}[commandchars=\\\{\}]
{\color{incolor}In [{\color{incolor}80}]:} mean.interval\PY{p}{(}\PY{k+kt}{c}\PY{p}{(}\PY{l+m}{1}\PY{p}{,}\PY{l+m}{2}\PY{p}{,}\PY{l+m}{3}\PY{p}{)}\PY{p}{,} \PY{l+m}{.1}\PY{p}{)}
\end{Verbatim}


    \begin{enumerate*}
\item 1.05034331570204
\item 2.94965668429796
\end{enumerate*}


    
    Voor alle procedure codes waarvan \(n>=50\) berekenen we nu het
gemiddelde/stdev en het betrouwbaarheidsinterval voor het
populatiegemiddelde:

    \begin{Verbatim}[commandchars=\\\{\}]
{\color{incolor}In [{\color{incolor}111}]:} plot.samples \PY{o}{=} \PY{k+kr}{function}\PY{p}{(}X\PY{p}{,} title\PY{p}{)} \PY{p}{\PYZob{}}
              hist\PY{p}{(}X\PY{p}{,} main\PY{o}{=}title\PY{p}{)}
              boxplot\PY{p}{(}X\PY{p}{,} main\PY{o}{=}title\PY{p}{)}
          \PY{p}{\PYZcb{}}
\end{Verbatim}


    \begin{Verbatim}[commandchars=\\\{\}]
{\color{incolor}In [{\color{incolor}112}]:} selected.procedure.codes \PY{o}{=} \PY{k+kt}{c}\PY{p}{(}\PY{l+s}{\PYZsq{}}\PY{l+s}{331801C\PYZsq{}}\PY{p}{,} \PY{l+s}{\PYZsq{}}\PY{l+s}{332320B\PYZsq{}}\PY{p}{,} \PY{l+s}{\PYZsq{}}\PY{l+s}{332320S\PYZsq{}}\PY{p}{,} \PY{l+s}{\PYZsq{}}\PY{l+s}{332323\PYZsq{}}\PY{p}{,} \PY{l+s}{\PYZsq{}}\PY{l+s}{332330\PYZsq{}}\PY{p}{)}
          n\PY{o}{=}\PY{l+m}{50}
          alpha\PY{o}{=}\PY{l+m}{0.05}
          
          \PY{k+kr}{for}\PY{p}{(}spc \PY{k+kr}{in} selected.procedure.codes\PY{p}{)} \PY{p}{\PYZob{}}
              records \PY{o}{=} dff\PY{p}{[}dff\PY{o}{\PYZdl{}}Registration.Procedure.Code \PY{o}{==} spc\PY{p}{,}\PY{p}{]}
              samples \PY{o}{=} \PY{k+kp}{sample}\PY{p}{(}records\PY{o}{\PYZdl{}}Service.Time\PY{p}{,} n\PY{p}{)}
              \PY{k+kp}{print}\PY{p}{(}\PY{k+kp}{paste}\PY{p}{(}\PY{l+s}{\PYZsq{}}\PY{l+s}{Interval Estimation for\PYZsq{}}\PY{p}{,} spc\PY{p}{)}\PY{p}{)}
              \PY{k+kp}{print}\PY{p}{(}\PY{k+kp}{paste}\PY{p}{(}\PY{l+s}{\PYZsq{}}\PY{l+s}{Population size:\PYZsq{}}\PY{p}{,} \PY{k+kp}{nrow}\PY{p}{(}records\PY{p}{)}\PY{p}{)}\PY{p}{)}
              \PY{k+kp}{print}\PY{p}{(}\PY{k+kp}{paste}\PY{p}{(}\PY{l+s}{\PYZsq{}}\PY{l+s}{n =\PYZsq{}}\PY{p}{,}n\PY{p}{)}\PY{p}{)}
              \PY{k+kp}{print}\PY{p}{(}\PY{k+kp}{paste}\PY{p}{(}\PY{l+s}{\PYZsq{}}\PY{l+s}{mean =\PYZsq{}}\PY{p}{,} \PY{k+kp}{mean}\PY{p}{(}samples\PY{p}{)}\PY{p}{)}\PY{p}{)}
              \PY{k+kp}{print}\PY{p}{(}\PY{k+kp}{paste}\PY{p}{(}\PY{l+s}{\PYZsq{}}\PY{l+s}{stdev = \PYZsq{}}\PY{p}{,} sd\PY{p}{(}samples\PY{p}{)}\PY{p}{)}\PY{p}{)}
              \PY{k+kp}{print}\PY{p}{(}\PY{k+kp}{paste}\PY{p}{(}\PY{l+s}{\PYZsq{}}\PY{l+s}{Interval Estimate of Population Mean\PYZsq{}}\PY{p}{,} \PY{p}{(}\PY{l+m}{1}\PY{o}{\PYZhy{}}alpha\PY{p}{)}\PY{o}{*}\PY{l+m}{100}\PY{p}{,} \PY{l+s}{\PYZsq{}}\PY{l+s}{\PYZpc{}\PYZsq{}}\PY{p}{)}\PY{p}{)}
              \PY{k+kp}{print}\PY{p}{(}mean.interval\PY{p}{(}samples\PY{p}{,} alpha\PY{p}{)}\PY{p}{)}
              plot.samples\PY{p}{(}samples\PY{p}{,} spc\PY{p}{)}
              \PY{k+kp}{print}\PY{p}{(}\PY{l+s}{\PYZsq{}}\PY{l+s}{\PYZsq{}}\PY{p}{)}    
          \PY{p}{\PYZcb{}}
\end{Verbatim}


    \begin{Verbatim}[commandchars=\\\{\}]
[1] "Interval Estimation for 331801C"
[1] "Population size: 1739"
[1] "n = 50"
[1] "mean = 11.22"
[1] "stdev =  16.4481746397463"
[1] "Interval Estimate of Population Mean 95 \%"
[1]  6.660882 15.779118

    \end{Verbatim}

    \begin{center}
    \adjustimage{max size={0.9\linewidth}{0.9\paperheight}}{output_39_1.png}
    \end{center}
    { \hspace*{\fill} \\}
    
    \begin{Verbatim}[commandchars=\\\{\}]
[1] ""
[1] "Interval Estimation for 332320B"
[1] "Population size: 71"
[1] "n = 50"
[1] "mean = 19.92"
[1] "stdev =  20.1097397460301"
[1] "Interval Estimate of Population Mean 95 \%"
[1] 14.34597 25.49403

    \end{Verbatim}

    \begin{center}
    \adjustimage{max size={0.9\linewidth}{0.9\paperheight}}{output_39_3.png}
    \end{center}
    { \hspace*{\fill} \\}
    
    \begin{center}
    \adjustimage{max size={0.9\linewidth}{0.9\paperheight}}{output_39_4.png}
    \end{center}
    { \hspace*{\fill} \\}
    
    \begin{Verbatim}[commandchars=\\\{\}]
[1] ""
[1] "Interval Estimation for 332320S"
[1] "Population size: 60"
[1] "n = 50"
[1] "mean = 14.12"
[1] "stdev =  14.968865647587"
[1] "Interval Estimate of Population Mean 95 \%"
[1]  9.970918 18.269082

    \end{Verbatim}

    \begin{center}
    \adjustimage{max size={0.9\linewidth}{0.9\paperheight}}{output_39_6.png}
    \end{center}
    { \hspace*{\fill} \\}
    
    \begin{center}
    \adjustimage{max size={0.9\linewidth}{0.9\paperheight}}{output_39_7.png}
    \end{center}
    { \hspace*{\fill} \\}
    
    \begin{Verbatim}[commandchars=\\\{\}]
[1] ""
[1] "Interval Estimation for 332323"
[1] "Population size: 646"
[1] "n = 50"
[1] "mean = 7.6"
[1] "stdev =  10.6291858501362"
[1] "Interval Estimate of Population Mean 95 \%"
[1]  4.653794 10.546206

    \end{Verbatim}

    \begin{center}
    \adjustimage{max size={0.9\linewidth}{0.9\paperheight}}{output_39_9.png}
    \end{center}
    { \hspace*{\fill} \\}
    
    \begin{center}
    \adjustimage{max size={0.9\linewidth}{0.9\paperheight}}{output_39_10.png}
    \end{center}
    { \hspace*{\fill} \\}
    
    \begin{Verbatim}[commandchars=\\\{\}]
[1] ""
[1] "Interval Estimation for 332330"
[1] "Population size: 475"
[1] "n = 50"
[1] "mean = 19.52"
[1] "stdev =  34.705901397842"
[1] "Interval Estimate of Population Mean 95 \%"
[1]  9.900192 29.139808

    \end{Verbatim}

    \begin{center}
    \adjustimage{max size={0.9\linewidth}{0.9\paperheight}}{output_39_12.png}
    \end{center}
    { \hspace*{\fill} \\}
    
    \begin{center}
    \adjustimage{max size={0.9\linewidth}{0.9\paperheight}}{output_39_13.png}
    \end{center}
    { \hspace*{\fill} \\}
    
    \begin{Verbatim}[commandchars=\\\{\}]
[1] ""

    \end{Verbatim}

    \begin{center}
    \adjustimage{max size={0.9\linewidth}{0.9\paperheight}}{output_39_15.png}
    \end{center}
    { \hspace*{\fill} \\}
    
    \hypertarget{hoe-zijn-de-operaties-gedistribueerd-per-dag}{%
\subsubsection{Hoe zijn de operaties gedistribueerd per
dag?}\label{hoe-zijn-de-operaties-gedistribueerd-per-dag}}

    \begin{Verbatim}[commandchars=\\\{\}]
{\color{incolor}In [{\color{incolor}115}]:} plot\PY{p}{(}dff\PY{o}{\PYZdl{}}Service.Date.Day.Of.Week\PY{p}{)}
\end{Verbatim}


    \begin{center}
    \adjustimage{max size={0.9\linewidth}{0.9\paperheight}}{output_41_0.png}
    \end{center}
    { \hspace*{\fill} \\}
    

    % Add a bibliography block to the postdoc
    
    
    
    \end{document}
