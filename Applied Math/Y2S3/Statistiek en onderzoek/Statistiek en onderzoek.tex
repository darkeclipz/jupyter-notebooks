
% Default to the notebook output style

    


% Inherit from the specified cell style.




    
\documentclass[11pt]{article}

    
    
    \usepackage[T1]{fontenc}
    % Nicer default font (+ math font) than Computer Modern for most use cases
    \usepackage{mathpazo}

    % Basic figure setup, for now with no caption control since it's done
    % automatically by Pandoc (which extracts ![](path) syntax from Markdown).
    \usepackage{graphicx}
    % We will generate all images so they have a width \maxwidth. This means
    % that they will get their normal width if they fit onto the page, but
    % are scaled down if they would overflow the margins.
    \makeatletter
    \def\maxwidth{\ifdim\Gin@nat@width>\linewidth\linewidth
    \else\Gin@nat@width\fi}
    \makeatother
    \let\Oldincludegraphics\includegraphics
    % Set max figure width to be 80% of text width, for now hardcoded.
    \renewcommand{\includegraphics}[1]{\Oldincludegraphics[width=.8\maxwidth]{#1}}
    % Ensure that by default, figures have no caption (until we provide a
    % proper Figure object with a Caption API and a way to capture that
    % in the conversion process - todo).
    \usepackage{caption}
    \DeclareCaptionLabelFormat{nolabel}{}
    \captionsetup{labelformat=nolabel}

    \usepackage{adjustbox} % Used to constrain images to a maximum size 
    \usepackage{xcolor} % Allow colors to be defined
    \usepackage{enumerate} % Needed for markdown enumerations to work
    \usepackage{geometry} % Used to adjust the document margins
    \usepackage{amsmath} % Equations
    \usepackage{amssymb} % Equations
    \usepackage{textcomp} % defines textquotesingle
    % Hack from http://tex.stackexchange.com/a/47451/13684:
    \AtBeginDocument{%
        \def\PYZsq{\textquotesingle}% Upright quotes in Pygmentized code
    }
    \usepackage{upquote} % Upright quotes for verbatim code
    \usepackage{eurosym} % defines \euro
    \usepackage[mathletters]{ucs} % Extended unicode (utf-8) support
    \usepackage[utf8x]{inputenc} % Allow utf-8 characters in the tex document
    \usepackage{fancyvrb} % verbatim replacement that allows latex
    \usepackage{grffile} % extends the file name processing of package graphics 
                         % to support a larger range 
    % The hyperref package gives us a pdf with properly built
    % internal navigation ('pdf bookmarks' for the table of contents,
    % internal cross-reference links, web links for URLs, etc.)
    \usepackage{hyperref}
    \usepackage{longtable} % longtable support required by pandoc >1.10
    \usepackage{booktabs}  % table support for pandoc > 1.12.2
    \usepackage[inline]{enumitem} % IRkernel/repr support (it uses the enumerate* environment)
    \usepackage[normalem]{ulem} % ulem is needed to support strikethroughs (\sout)
                                % normalem makes italics be italics, not underlines
    \usepackage{mathrsfs}
    

    
    
    % Colors for the hyperref package
    \definecolor{urlcolor}{rgb}{0,.145,.698}
    \definecolor{linkcolor}{rgb}{.71,0.21,0.01}
    \definecolor{citecolor}{rgb}{.12,.54,.11}

    % ANSI colors
    \definecolor{ansi-black}{HTML}{3E424D}
    \definecolor{ansi-black-intense}{HTML}{282C36}
    \definecolor{ansi-red}{HTML}{E75C58}
    \definecolor{ansi-red-intense}{HTML}{B22B31}
    \definecolor{ansi-green}{HTML}{00A250}
    \definecolor{ansi-green-intense}{HTML}{007427}
    \definecolor{ansi-yellow}{HTML}{DDB62B}
    \definecolor{ansi-yellow-intense}{HTML}{B27D12}
    \definecolor{ansi-blue}{HTML}{208FFB}
    \definecolor{ansi-blue-intense}{HTML}{0065CA}
    \definecolor{ansi-magenta}{HTML}{D160C4}
    \definecolor{ansi-magenta-intense}{HTML}{A03196}
    \definecolor{ansi-cyan}{HTML}{60C6C8}
    \definecolor{ansi-cyan-intense}{HTML}{258F8F}
    \definecolor{ansi-white}{HTML}{C5C1B4}
    \definecolor{ansi-white-intense}{HTML}{A1A6B2}
    \definecolor{ansi-default-inverse-fg}{HTML}{FFFFFF}
    \definecolor{ansi-default-inverse-bg}{HTML}{000000}

    % commands and environments needed by pandoc snippets
    % extracted from the output of `pandoc -s`
    \providecommand{\tightlist}{%
      \setlength{\itemsep}{0pt}\setlength{\parskip}{0pt}}
    \DefineVerbatimEnvironment{Highlighting}{Verbatim}{commandchars=\\\{\}}
    % Add ',fontsize=\small' for more characters per line
    \newenvironment{Shaded}{}{}
    \newcommand{\KeywordTok}[1]{\textcolor[rgb]{0.00,0.44,0.13}{\textbf{{#1}}}}
    \newcommand{\DataTypeTok}[1]{\textcolor[rgb]{0.56,0.13,0.00}{{#1}}}
    \newcommand{\DecValTok}[1]{\textcolor[rgb]{0.25,0.63,0.44}{{#1}}}
    \newcommand{\BaseNTok}[1]{\textcolor[rgb]{0.25,0.63,0.44}{{#1}}}
    \newcommand{\FloatTok}[1]{\textcolor[rgb]{0.25,0.63,0.44}{{#1}}}
    \newcommand{\CharTok}[1]{\textcolor[rgb]{0.25,0.44,0.63}{{#1}}}
    \newcommand{\StringTok}[1]{\textcolor[rgb]{0.25,0.44,0.63}{{#1}}}
    \newcommand{\CommentTok}[1]{\textcolor[rgb]{0.38,0.63,0.69}{\textit{{#1}}}}
    \newcommand{\OtherTok}[1]{\textcolor[rgb]{0.00,0.44,0.13}{{#1}}}
    \newcommand{\AlertTok}[1]{\textcolor[rgb]{1.00,0.00,0.00}{\textbf{{#1}}}}
    \newcommand{\FunctionTok}[1]{\textcolor[rgb]{0.02,0.16,0.49}{{#1}}}
    \newcommand{\RegionMarkerTok}[1]{{#1}}
    \newcommand{\ErrorTok}[1]{\textcolor[rgb]{1.00,0.00,0.00}{\textbf{{#1}}}}
    \newcommand{\NormalTok}[1]{{#1}}
    
    % Additional commands for more recent versions of Pandoc
    \newcommand{\ConstantTok}[1]{\textcolor[rgb]{0.53,0.00,0.00}{{#1}}}
    \newcommand{\SpecialCharTok}[1]{\textcolor[rgb]{0.25,0.44,0.63}{{#1}}}
    \newcommand{\VerbatimStringTok}[1]{\textcolor[rgb]{0.25,0.44,0.63}{{#1}}}
    \newcommand{\SpecialStringTok}[1]{\textcolor[rgb]{0.73,0.40,0.53}{{#1}}}
    \newcommand{\ImportTok}[1]{{#1}}
    \newcommand{\DocumentationTok}[1]{\textcolor[rgb]{0.73,0.13,0.13}{\textit{{#1}}}}
    \newcommand{\AnnotationTok}[1]{\textcolor[rgb]{0.38,0.63,0.69}{\textbf{\textit{{#1}}}}}
    \newcommand{\CommentVarTok}[1]{\textcolor[rgb]{0.38,0.63,0.69}{\textbf{\textit{{#1}}}}}
    \newcommand{\VariableTok}[1]{\textcolor[rgb]{0.10,0.09,0.49}{{#1}}}
    \newcommand{\ControlFlowTok}[1]{\textcolor[rgb]{0.00,0.44,0.13}{\textbf{{#1}}}}
    \newcommand{\OperatorTok}[1]{\textcolor[rgb]{0.40,0.40,0.40}{{#1}}}
    \newcommand{\BuiltInTok}[1]{{#1}}
    \newcommand{\ExtensionTok}[1]{{#1}}
    \newcommand{\PreprocessorTok}[1]{\textcolor[rgb]{0.74,0.48,0.00}{{#1}}}
    \newcommand{\AttributeTok}[1]{\textcolor[rgb]{0.49,0.56,0.16}{{#1}}}
    \newcommand{\InformationTok}[1]{\textcolor[rgb]{0.38,0.63,0.69}{\textbf{\textit{{#1}}}}}
    \newcommand{\WarningTok}[1]{\textcolor[rgb]{0.38,0.63,0.69}{\textbf{\textit{{#1}}}}}
    
    
    % Define a nice break command that doesn't care if a line doesn't already
    % exist.
    \def\br{\hspace*{\fill} \\* }
    % Math Jax compatibility definitions
    \def\gt{>}
    \def\lt{<}
    \let\Oldtex\TeX
    \let\Oldlatex\LaTeX
    \renewcommand{\TeX}{\textrm{\Oldtex}}
    \renewcommand{\LaTeX}{\textrm{\Oldlatex}}
    % Document parameters
    % Document title
    \title{Statistiek en onderzoek}
    
    
    
    
    

    % Pygments definitions
    
\makeatletter
\def\PY@reset{\let\PY@it=\relax \let\PY@bf=\relax%
    \let\PY@ul=\relax \let\PY@tc=\relax%
    \let\PY@bc=\relax \let\PY@ff=\relax}
\def\PY@tok#1{\csname PY@tok@#1\endcsname}
\def\PY@toks#1+{\ifx\relax#1\empty\else%
    \PY@tok{#1}\expandafter\PY@toks\fi}
\def\PY@do#1{\PY@bc{\PY@tc{\PY@ul{%
    \PY@it{\PY@bf{\PY@ff{#1}}}}}}}
\def\PY#1#2{\PY@reset\PY@toks#1+\relax+\PY@do{#2}}

\expandafter\def\csname PY@tok@w\endcsname{\def\PY@tc##1{\textcolor[rgb]{0.73,0.73,0.73}{##1}}}
\expandafter\def\csname PY@tok@c\endcsname{\let\PY@it=\textit\def\PY@tc##1{\textcolor[rgb]{0.25,0.50,0.50}{##1}}}
\expandafter\def\csname PY@tok@cp\endcsname{\def\PY@tc##1{\textcolor[rgb]{0.74,0.48,0.00}{##1}}}
\expandafter\def\csname PY@tok@k\endcsname{\let\PY@bf=\textbf\def\PY@tc##1{\textcolor[rgb]{0.00,0.50,0.00}{##1}}}
\expandafter\def\csname PY@tok@kp\endcsname{\def\PY@tc##1{\textcolor[rgb]{0.00,0.50,0.00}{##1}}}
\expandafter\def\csname PY@tok@kt\endcsname{\def\PY@tc##1{\textcolor[rgb]{0.69,0.00,0.25}{##1}}}
\expandafter\def\csname PY@tok@o\endcsname{\def\PY@tc##1{\textcolor[rgb]{0.40,0.40,0.40}{##1}}}
\expandafter\def\csname PY@tok@ow\endcsname{\let\PY@bf=\textbf\def\PY@tc##1{\textcolor[rgb]{0.67,0.13,1.00}{##1}}}
\expandafter\def\csname PY@tok@nb\endcsname{\def\PY@tc##1{\textcolor[rgb]{0.00,0.50,0.00}{##1}}}
\expandafter\def\csname PY@tok@nf\endcsname{\def\PY@tc##1{\textcolor[rgb]{0.00,0.00,1.00}{##1}}}
\expandafter\def\csname PY@tok@nc\endcsname{\let\PY@bf=\textbf\def\PY@tc##1{\textcolor[rgb]{0.00,0.00,1.00}{##1}}}
\expandafter\def\csname PY@tok@nn\endcsname{\let\PY@bf=\textbf\def\PY@tc##1{\textcolor[rgb]{0.00,0.00,1.00}{##1}}}
\expandafter\def\csname PY@tok@ne\endcsname{\let\PY@bf=\textbf\def\PY@tc##1{\textcolor[rgb]{0.82,0.25,0.23}{##1}}}
\expandafter\def\csname PY@tok@nv\endcsname{\def\PY@tc##1{\textcolor[rgb]{0.10,0.09,0.49}{##1}}}
\expandafter\def\csname PY@tok@no\endcsname{\def\PY@tc##1{\textcolor[rgb]{0.53,0.00,0.00}{##1}}}
\expandafter\def\csname PY@tok@nl\endcsname{\def\PY@tc##1{\textcolor[rgb]{0.63,0.63,0.00}{##1}}}
\expandafter\def\csname PY@tok@ni\endcsname{\let\PY@bf=\textbf\def\PY@tc##1{\textcolor[rgb]{0.60,0.60,0.60}{##1}}}
\expandafter\def\csname PY@tok@na\endcsname{\def\PY@tc##1{\textcolor[rgb]{0.49,0.56,0.16}{##1}}}
\expandafter\def\csname PY@tok@nt\endcsname{\let\PY@bf=\textbf\def\PY@tc##1{\textcolor[rgb]{0.00,0.50,0.00}{##1}}}
\expandafter\def\csname PY@tok@nd\endcsname{\def\PY@tc##1{\textcolor[rgb]{0.67,0.13,1.00}{##1}}}
\expandafter\def\csname PY@tok@s\endcsname{\def\PY@tc##1{\textcolor[rgb]{0.73,0.13,0.13}{##1}}}
\expandafter\def\csname PY@tok@sd\endcsname{\let\PY@it=\textit\def\PY@tc##1{\textcolor[rgb]{0.73,0.13,0.13}{##1}}}
\expandafter\def\csname PY@tok@si\endcsname{\let\PY@bf=\textbf\def\PY@tc##1{\textcolor[rgb]{0.73,0.40,0.53}{##1}}}
\expandafter\def\csname PY@tok@se\endcsname{\let\PY@bf=\textbf\def\PY@tc##1{\textcolor[rgb]{0.73,0.40,0.13}{##1}}}
\expandafter\def\csname PY@tok@sr\endcsname{\def\PY@tc##1{\textcolor[rgb]{0.73,0.40,0.53}{##1}}}
\expandafter\def\csname PY@tok@ss\endcsname{\def\PY@tc##1{\textcolor[rgb]{0.10,0.09,0.49}{##1}}}
\expandafter\def\csname PY@tok@sx\endcsname{\def\PY@tc##1{\textcolor[rgb]{0.00,0.50,0.00}{##1}}}
\expandafter\def\csname PY@tok@m\endcsname{\def\PY@tc##1{\textcolor[rgb]{0.40,0.40,0.40}{##1}}}
\expandafter\def\csname PY@tok@gh\endcsname{\let\PY@bf=\textbf\def\PY@tc##1{\textcolor[rgb]{0.00,0.00,0.50}{##1}}}
\expandafter\def\csname PY@tok@gu\endcsname{\let\PY@bf=\textbf\def\PY@tc##1{\textcolor[rgb]{0.50,0.00,0.50}{##1}}}
\expandafter\def\csname PY@tok@gd\endcsname{\def\PY@tc##1{\textcolor[rgb]{0.63,0.00,0.00}{##1}}}
\expandafter\def\csname PY@tok@gi\endcsname{\def\PY@tc##1{\textcolor[rgb]{0.00,0.63,0.00}{##1}}}
\expandafter\def\csname PY@tok@gr\endcsname{\def\PY@tc##1{\textcolor[rgb]{1.00,0.00,0.00}{##1}}}
\expandafter\def\csname PY@tok@ge\endcsname{\let\PY@it=\textit}
\expandafter\def\csname PY@tok@gs\endcsname{\let\PY@bf=\textbf}
\expandafter\def\csname PY@tok@gp\endcsname{\let\PY@bf=\textbf\def\PY@tc##1{\textcolor[rgb]{0.00,0.00,0.50}{##1}}}
\expandafter\def\csname PY@tok@go\endcsname{\def\PY@tc##1{\textcolor[rgb]{0.53,0.53,0.53}{##1}}}
\expandafter\def\csname PY@tok@gt\endcsname{\def\PY@tc##1{\textcolor[rgb]{0.00,0.27,0.87}{##1}}}
\expandafter\def\csname PY@tok@err\endcsname{\def\PY@bc##1{\setlength{\fboxsep}{0pt}\fcolorbox[rgb]{1.00,0.00,0.00}{1,1,1}{\strut ##1}}}
\expandafter\def\csname PY@tok@kc\endcsname{\let\PY@bf=\textbf\def\PY@tc##1{\textcolor[rgb]{0.00,0.50,0.00}{##1}}}
\expandafter\def\csname PY@tok@kd\endcsname{\let\PY@bf=\textbf\def\PY@tc##1{\textcolor[rgb]{0.00,0.50,0.00}{##1}}}
\expandafter\def\csname PY@tok@kn\endcsname{\let\PY@bf=\textbf\def\PY@tc##1{\textcolor[rgb]{0.00,0.50,0.00}{##1}}}
\expandafter\def\csname PY@tok@kr\endcsname{\let\PY@bf=\textbf\def\PY@tc##1{\textcolor[rgb]{0.00,0.50,0.00}{##1}}}
\expandafter\def\csname PY@tok@bp\endcsname{\def\PY@tc##1{\textcolor[rgb]{0.00,0.50,0.00}{##1}}}
\expandafter\def\csname PY@tok@fm\endcsname{\def\PY@tc##1{\textcolor[rgb]{0.00,0.00,1.00}{##1}}}
\expandafter\def\csname PY@tok@vc\endcsname{\def\PY@tc##1{\textcolor[rgb]{0.10,0.09,0.49}{##1}}}
\expandafter\def\csname PY@tok@vg\endcsname{\def\PY@tc##1{\textcolor[rgb]{0.10,0.09,0.49}{##1}}}
\expandafter\def\csname PY@tok@vi\endcsname{\def\PY@tc##1{\textcolor[rgb]{0.10,0.09,0.49}{##1}}}
\expandafter\def\csname PY@tok@vm\endcsname{\def\PY@tc##1{\textcolor[rgb]{0.10,0.09,0.49}{##1}}}
\expandafter\def\csname PY@tok@sa\endcsname{\def\PY@tc##1{\textcolor[rgb]{0.73,0.13,0.13}{##1}}}
\expandafter\def\csname PY@tok@sb\endcsname{\def\PY@tc##1{\textcolor[rgb]{0.73,0.13,0.13}{##1}}}
\expandafter\def\csname PY@tok@sc\endcsname{\def\PY@tc##1{\textcolor[rgb]{0.73,0.13,0.13}{##1}}}
\expandafter\def\csname PY@tok@dl\endcsname{\def\PY@tc##1{\textcolor[rgb]{0.73,0.13,0.13}{##1}}}
\expandafter\def\csname PY@tok@s2\endcsname{\def\PY@tc##1{\textcolor[rgb]{0.73,0.13,0.13}{##1}}}
\expandafter\def\csname PY@tok@sh\endcsname{\def\PY@tc##1{\textcolor[rgb]{0.73,0.13,0.13}{##1}}}
\expandafter\def\csname PY@tok@s1\endcsname{\def\PY@tc##1{\textcolor[rgb]{0.73,0.13,0.13}{##1}}}
\expandafter\def\csname PY@tok@mb\endcsname{\def\PY@tc##1{\textcolor[rgb]{0.40,0.40,0.40}{##1}}}
\expandafter\def\csname PY@tok@mf\endcsname{\def\PY@tc##1{\textcolor[rgb]{0.40,0.40,0.40}{##1}}}
\expandafter\def\csname PY@tok@mh\endcsname{\def\PY@tc##1{\textcolor[rgb]{0.40,0.40,0.40}{##1}}}
\expandafter\def\csname PY@tok@mi\endcsname{\def\PY@tc##1{\textcolor[rgb]{0.40,0.40,0.40}{##1}}}
\expandafter\def\csname PY@tok@il\endcsname{\def\PY@tc##1{\textcolor[rgb]{0.40,0.40,0.40}{##1}}}
\expandafter\def\csname PY@tok@mo\endcsname{\def\PY@tc##1{\textcolor[rgb]{0.40,0.40,0.40}{##1}}}
\expandafter\def\csname PY@tok@ch\endcsname{\let\PY@it=\textit\def\PY@tc##1{\textcolor[rgb]{0.25,0.50,0.50}{##1}}}
\expandafter\def\csname PY@tok@cm\endcsname{\let\PY@it=\textit\def\PY@tc##1{\textcolor[rgb]{0.25,0.50,0.50}{##1}}}
\expandafter\def\csname PY@tok@cpf\endcsname{\let\PY@it=\textit\def\PY@tc##1{\textcolor[rgb]{0.25,0.50,0.50}{##1}}}
\expandafter\def\csname PY@tok@c1\endcsname{\let\PY@it=\textit\def\PY@tc##1{\textcolor[rgb]{0.25,0.50,0.50}{##1}}}
\expandafter\def\csname PY@tok@cs\endcsname{\let\PY@it=\textit\def\PY@tc##1{\textcolor[rgb]{0.25,0.50,0.50}{##1}}}

\def\PYZbs{\char`\\}
\def\PYZus{\char`\_}
\def\PYZob{\char`\{}
\def\PYZcb{\char`\}}
\def\PYZca{\char`\^}
\def\PYZam{\char`\&}
\def\PYZlt{\char`\<}
\def\PYZgt{\char`\>}
\def\PYZsh{\char`\#}
\def\PYZpc{\char`\%}
\def\PYZdl{\char`\$}
\def\PYZhy{\char`\-}
\def\PYZsq{\char`\'}
\def\PYZdq{\char`\"}
\def\PYZti{\char`\~}
% for compatibility with earlier versions
\def\PYZat{@}
\def\PYZlb{[}
\def\PYZrb{]}
\makeatother


    % Exact colors from NB
    \definecolor{incolor}{rgb}{0.0, 0.0, 0.5}
    \definecolor{outcolor}{rgb}{0.545, 0.0, 0.0}



    
    % Prevent overflowing lines due to hard-to-break entities
    \sloppy 
    % Setup hyperref package
    \hypersetup{
      breaklinks=true,  % so long urls are correctly broken across lines
      colorlinks=true,
      urlcolor=urlcolor,
      linkcolor=linkcolor,
      citecolor=citecolor,
      }
    % Slightly bigger margins than the latex defaults
    
    \geometry{verbose,tmargin=1in,bmargin=1in,lmargin=1in,rmargin=1in}
    
    

    \begin{document}
    
    
    \maketitle
    
    \newpage
    \tableofcontents
    \newpage

    
    \begin{Verbatim}[commandchars=\\\{\}]
{\color{incolor}In [{\color{incolor}1}]:} \PY{k+kp}{options}\PY{p}{(}repr.plot.width\PY{o}{=}\PY{l+m}{5}\PY{p}{,} repr.plot.height\PY{o}{=}\PY{l+m}{5}\PY{p}{)}
\end{Verbatim}

    \hypertarget{schatten-van-de-variantie}{%
\section{Schatten van de variantie}\label{schatten-van-de-variantie}}

    We gebruiken hier als schatter voor \(\sigma^2\) de grootheid \(s^2\).
De formule luidt:

\[s^2 = \dfrac{\sum\left(x_i - \bar{x}\right)^2}{n-1} = \dfrac{SSE}{n-1} \]

    \hypertarget{chikwadraatverdeling}{%
\section{Chikwadraatverdeling}\label{chikwadraatverdeling}}

    Stel \(\underline{x}_1, \ldots, \underline{x}_n\) zijn alle
kansvariabelen die standaardnormaal verdeeld zijn. Deze \(x\) gaan we
kwadrateren zodat we zeker weten dat deze altijd positief is. Voor een
enkele trekking \(\underline{x}_1\) krijgen we een Chikwadraat
verdeling, aangeduidt met \(\chi^2[1]\), met één vrijheidsgraad. Met
meerdere trekkingen ontstaat de grootheid chikwadraat:

\[ \underline{\chi}^2[\nu] = \underline{x}_1 + \ldots + \underline{x}_n \]

waarbij \(\nu\) het aantal vrijheidsgraden is.

Enkele eigenschappen zijn dat:

\begin{enumerate}
\def\labelenumi{\arabic{enumi}.}
\tightlist
\item
  \(\textrm{E}(\underline{\chi}^2[\nu]) = \nu\).
\item
  \(\textrm{Var}(\underline{\chi}^2[\nu]) = 2 \nu\).
\end{enumerate}

    \textbf{Verdelingsfunctie plot}

    \begin{Verbatim}[commandchars=\\\{\}]
{\color{incolor}In [{\color{incolor}2}]:} curve\PY{p}{(}dchisq\PY{p}{(}x\PY{p}{,} df\PY{o}{=}\PY{l+m}{9}\PY{p}{)}\PY{p}{,} col\PY{o}{=}\PY{l+s}{\PYZsq{}}\PY{l+s}{red\PYZsq{}}\PY{p}{,} main \PY{o}{=} \PY{l+s}{\PYZdq{}}\PY{l+s}{Chi\PYZhy{}Square Density Graph\PYZdq{}}\PY{p}{,} 
              from\PY{o}{=}\PY{l+m}{0}\PY{p}{,}to\PY{o}{=}\PY{l+m}{60}\PY{p}{)}
\end{Verbatim}

    \begin{center}
    \adjustimage{max size={0.9\linewidth}{0.9\paperheight}}{output_6_0.png}
    \end{center}
    { \hspace*{\fill} \\}
    
    \textbf{Centrale limietstelling}

    De chikwadraat verdeling kan normaal benaderd worden als
\(\nu \rightarrow \infty\). De vuistregel hiervoor is dat \(\nu > 30\).
Er geldt dan dat:

\[ \chi^2[\nu] \sim N\left(\mu = \nu, \sigma = \sqrt{2\nu}\right) \]

Bij benadering gebruiken we:

\[ P(\chi^2[\nu] \leq k\ |\ \nu) \approx P(\underline{x}_{nor} \leq k\ |\ \mu = \nu, \sigma = \sqrt{2\nu} ).\]

    \hypertarget{betrouwbaarheidsinterval-voor-sigma2}{%
\section{\texorpdfstring{Betrouwbaarheidsinterval voor
\(\sigma^2\)}{Betrouwbaarheidsinterval voor \textbackslash{}sigma\^{}2}}\label{betrouwbaarheidsinterval-voor-sigma2}}

    Een vereiste om een betrouwbaarheidsinterval op te stellen voor
\(\sigma^2\) is dat de onderzochte kansvariabele \(\underline{x}\) een
normale verdeling volgt, of bij benadering normaal verdeeld is. De toets
hiervoor staat in de volgende sectie.

Het betrouwbaarheidsinterval voor \(\sigma^2\) luidt:

    \[ \dfrac{\sum\left(x_i - \bar{x} \right)^2}{g_R} < \sigma^2 < \dfrac{\sum\left(x_i - \bar{x} \right)^2}{g_L} \]

met andere woorden:

\[ \dfrac{SSE}{g_{1-\frac{\alpha}{2}}} < \sigma^2 < \dfrac{SSE}{g_{\frac{\alpha}{2}}}. \]

    \textbf{Voorbeeld}

    \begin{Verbatim}[commandchars=\\\{\}]
{\color{incolor}In [{\color{incolor}3}]:} SSE \PY{o}{=} \PY{l+m}{45}
        xbar \PY{o}{=} \PY{l+m}{24}
        df \PY{o}{=} \PY{l+m}{9}
        interval \PY{o}{=} \PY{k+kt}{c}\PY{p}{(}SSE \PY{o}{/} qchisq\PY{p}{(}\PY{l+m}{0.975}\PY{p}{,} df\PY{o}{=}\PY{l+m}{9}\PY{p}{)}\PY{p}{,} SSE \PY{o}{/} qchisq\PY{p}{(}\PY{l+m}{0.025}\PY{p}{,} df\PY{o}{=}\PY{l+m}{9}\PY{p}{)}\PY{p}{)}
        interval
\end{Verbatim}

    \begin{enumerate*}
\item 2.3655863582172
\item 16.6642626925958
\end{enumerate*}


    
    \hypertarget{toets-voor-sigma2}{%
\section{\texorpdfstring{Toets voor
\(\sigma^2\)}{Toets voor \textbackslash{}sigma\^{}2}}\label{toets-voor-sigma2}}

    De verzamelde uitkomsten \(x_1, x_2, \ldots, x_n\) van een variabele
\(\underline{x}\) dienen een spreiding \(s^2\) te vertonen die in
redelijke mate overeen komt met \(\sigma^2\). Als dat niet het geval is,
dan moet de nulhypothese \(\textrm{Var}(\underline{x})=\sigma^2\) worden
verworpen.

    \textbf{Toetsingsgrootheid}

Gegeven de nulhypothese volgt de toetsingsgrootheid:

\[ \dfrac{(n-1)s^2}{\sigma^2} = \dfrac{SSE}{\sigma^2}\]

een \(\chi^2[n-1]\)-verdeling.

    \textbf{Voorbeeld}

    Stel \(n=10\), \(\sigma^2=15\), \(SSE=45\) en \(\alpha=0.05\), dan:

    \begin{Verbatim}[commandchars=\\\{\}]
{\color{incolor}In [{\color{incolor}4}]:} n \PY{o}{=} \PY{l+m}{10}
        sigma2 \PY{o}{=} \PY{l+m}{15}
        SSE \PY{o}{=} \PY{l+m}{45}
        alpha \PY{o}{=} \PY{l+m}{0.05}
        SSE \PY{o}{/} sigma2
        pchisq\PY{p}{(}SSE \PY{o}{/} sigma2\PY{p}{,} n\PY{l+m}{\PYZhy{}1}\PY{p}{)}
\end{Verbatim}

    3

    
    0.0357050273149108

    
    \emph{Toetsprocedure}

\begin{enumerate}
\def\labelenumi{\arabic{enumi}.}
\tightlist
\item
  \(H_0\) : \(\sigma^2 = 15\) en \(H_A\) : \(\sigma^2 \not= 15\).
\item
  Toetsingsgrootheid: \(\dfrac{SSE}{\sigma^2} = 3\) met \(\nu=n-1=9\).
\item
  Overschrijdingskans:
  \(p = P(\chi^2[9] \leq 3) = \textrm{pchisq}(3,9) = 0.0357\).
\item
  Beslissing: \(p > \frac{\alpha}{2}\) met \(\alpha=0.05\), dus \(H_0\)
  wordt niet verworpen.
\item
  Conclusie: De waarneming valt binnen het voorspellingsinterval van
  \(\sigma^2\).
\end{enumerate}

    \hypertarget{chikwadraat-toets-voor-passing}{%
\section{Chikwadraat toets voor
passing}\label{chikwadraat-toets-voor-passing}}

    Hiermee kunnen we toetsen of de uitkomsten een patroon vertonen dat
overeenkomt met een gegeven kansverdeling. Hiervoor gaan we de
waargenomen en de theoretische (vanuit de verwachte kansverdeling)
uitkomsten met elkaar vergelijken.

\textbf{Toetsingsgrootheid}

Het vergelijken van de waargenomen en de theoretische frequenties
gebeurt met de toetsingsgrootheid:

\[ \chi^2 = \sum \dfrac{(O_i-E_i)^2}{E_i} \]

dit wordt benaderd met \(\chi^2[\nu]\) waarbij \(\nu = n - 1\).

    \textbf{Detail 1}

In het algemeent geldt er er voor alle \(E_i \geq 5\). Indien dit niet
het geval is, dan worden er klassen samemgevoegd zodanig dat dit wel het
geval is.

\textbf{Detail 2}

Stel \(\mu\) is onbekend, dan wordt \(\mu\) bepaald door het
steekproefgemiddelde (of een schatting hiervan). In dit geval gaat er
een extra vrijheidsgraad verloren. Dus dan geldt er dat \(\nu = n-2\).

    \textbf{Voorbeeld}

    We willen nagaan wat het ziekteverzuim is op verschillende dagen in de
week. Hiervoor hebben we de volgende gegevens:

    \begin{Verbatim}[commandchars=\\\{\}]
{\color{incolor}In [{\color{incolor}5}]:} dagen \PY{o}{=} \PY{k+kt}{c}\PY{p}{(}\PY{l+s}{\PYZsq{}}\PY{l+s}{ma\PYZsq{}}\PY{p}{,} \PY{l+s}{\PYZsq{}}\PY{l+s}{di\PYZsq{}}\PY{p}{,} \PY{l+s}{\PYZsq{}}\PY{l+s}{wo\PYZsq{}}\PY{p}{,} \PY{l+s}{\PYZsq{}}\PY{l+s}{do\PYZsq{}}\PY{p}{,} \PY{l+s}{\PYZsq{}}\PY{l+s}{vr\PYZsq{}}\PY{p}{)}
        zieken \PY{o}{=} \PY{k+kt}{c}\PY{p}{(}\PY{l+m}{20}\PY{p}{,} \PY{l+m}{14}\PY{p}{,} \PY{l+m}{14}\PY{p}{,} \PY{l+m}{12}\PY{p}{,} \PY{l+m}{20}\PY{p}{)}
        df \PY{o}{=} \PY{k+kt}{data.frame}\PY{p}{(}dagen\PY{p}{,} zieken\PY{p}{)}
        df
\end{Verbatim}

    \begin{tabular}{r|ll}
 dagen & zieken\\
\hline
	 ma & 20\\
	 di & 14\\
	 wo & 14\\
	 do & 12\\
	 vr & 20\\
\end{tabular}


    
    Men wil toetsen of het aantal ziektedagen gelijkmatig over de dagen van
de week is verdeeld. Voor vijf dagen in de week zou dat betekenen dat er
een kans is van \(0.20\) voor elke dag.

    \begin{Verbatim}[commandchars=\\\{\}]
{\color{incolor}In [{\color{incolor}6}]:} O \PY{o}{=} zieken
        n \PY{o}{=} \PY{k+kp}{sum}\PY{p}{(}O\PY{p}{)}
        E \PY{o}{=} \PY{k+kp}{rep}\PY{p}{(}\PY{l+m}{0.2} \PY{o}{*} n\PY{p}{,} \PY{l+m}{5}\PY{p}{)}
        diff \PY{o}{=} O \PY{o}{\PYZhy{}} E
        diffsq \PY{o}{=} \PY{k+kp}{diff}\PY{o}{*}\PY{k+kp}{diff}
        diffsqdiv \PY{o}{=} diffsq \PY{o}{/} E
        \PY{k+kt}{data.frame}\PY{p}{(}dagen\PY{p}{,} O\PY{p}{,} E\PY{p}{,} \PY{k+kp}{diff}\PY{p}{,} diffsq\PY{p}{,} diffsqdiv\PY{p}{)}
        \PY{k+kp}{sum}\PY{p}{(}diffsqdiv\PY{p}{)}
\end{Verbatim}

    \begin{tabular}{r|llllll}
 dagen & O & E & diff & diffsq & diffsqdiv\\
\hline
	 ma   & 20   & 16   &  4   & 16   & 1.00\\
	 di   & 14   & 16   & -2   &  4   & 0.25\\
	 wo   & 14   & 16   & -2   &  4   & 0.25\\
	 do   & 12   & 16   & -4   & 16   & 1.00\\
	 vr   & 20   & 16   &  4   & 16   & 1.00\\
\end{tabular}


    
    3.5

    
    \emph{Toetsprocedure}

\begin{enumerate}
\def\labelenumi{\arabic{enumi}.}
\tightlist
\item
  \(H_0\) : de ziektemeldingen zijn gelijkmatig verdeeld over de
  werkdagen. \(H_A\) : ziektemeldingen zijn niet uniform verdeeld.
\item
  Toetsingsgrootheid: \[\sum{\dfrac{(O_i-E_i)^2}{E_i}} = 3.5\] met een
  \(\chi^2[4]\)-verdeling.
\item
  Overschrijdingskans:
  \(p = P(\chi^2[4] \geq 3.5) = \chi^2_{cdf}(3.5, 10^{99}, 4) = 0.4779\).
\item
  Beslissing: \(p > \alpha\) met \(\alpha=0.05\), dus \(H_0\) wordt niet
  verworpen.
\item
  Conclusie: de geconstateerde afwijkingen kunnen heel goed aan het
  toeval worden toegeschreven.
\end{enumerate}

    \hypertarget{fishers-exacte-toets}{%
\section{Fisher's exacte toets}\label{fishers-exacte-toets}}

    In het geval dat \(n < 20\), dan moet Fisher's exacte toets worden
gebruikt. In het geval dat \(20 < n < 40\), gebruiken we alleen de Yates
correctie. In het laatste geval, als \(n >40\), dan wordt er helemaal
geen correctie toegepast.

Bij Fisher's exacte toets gaan we uit van een \(2 \times 2\) kruistabel.

\textbf{Voorbeeld}

We willen kijken of er een verschil is in ziekte tussen docenten die
wel/geen griepprik hebben gekregen.

    \begin{Verbatim}[commandchars=\\\{\}]
{\color{incolor}In [{\color{incolor}7}]:} griep \PY{o}{=} \PY{k+kt}{c}\PY{p}{(}\PY{l+m}{1}\PY{p}{,} \PY{l+m}{4}\PY{p}{)}
        geen.griep \PY{o}{=} \PY{k+kt}{c}\PY{p}{(}\PY{l+m}{11}\PY{p}{,} \PY{l+m}{2}\PY{p}{)}
        df \PY{o}{=} \PY{k+kp}{t}\PY{p}{(}\PY{k+kt}{data.frame}\PY{p}{(}griep\PY{p}{,} geen.griep\PY{p}{)}\PY{p}{)}
        \PY{k+kp}{colnames}\PY{p}{(}df\PY{p}{)} \PY{o}{=} \PY{k+kt}{c}\PY{p}{(}\PY{l+s}{\PYZsq{}}\PY{l+s}{prik\PYZsq{}}\PY{p}{,} \PY{l+s}{\PYZsq{}}\PY{l+s}{geen.prik\PYZsq{}}\PY{p}{)}
        df
\end{Verbatim}

    \begin{tabular}{r|ll}
  & prik & geen.prik\\
\hline
	griep &  1 & 4 \\
	geen.griep & 11 & 2 \\
\end{tabular}


    
    We kiezen hier het laagste getal als basis, in dit geval is dat \(1\).
Hier stellen we dat \(\underline{k}\) het aantal docenten is met
griepprik die de griep krijgen. In dit geval is \(\underline{k}\)
hypergeometrisch verdeeld. Het steekproefresultaat is \(1\), en we gaan
de kans bepalen dat dit voorkomt.

\(p = P(\underline{k} \leq 1) = P(\underline{k} = 0) + P(\underline k = 1) = \dfrac{\begin{pmatrix}12\\0\end{pmatrix}\begin{pmatrix}6\\5\end{pmatrix}}{\begin{pmatrix}18\\5\end{pmatrix}} + \dfrac{\begin{pmatrix}12\\1\end{pmatrix}\begin{pmatrix}6\\4\end{pmatrix}}{\begin{pmatrix}18\\5\end{pmatrix}} = 0.0217.\)

    We kunnen dit ook met R bepalen:

    \begin{Verbatim}[commandchars=\\\{\}]
{\color{incolor}In [{\color{incolor}8}]:} fisher.test\PY{p}{(}df\PY{p}{)}
\end{Verbatim}

    
    \begin{verbatim}

	Fisher's Exact Test for Count Data

data:  df
p-value = 0.02171
alternative hypothesis: true odds ratio is not equal to 1
95 percent confidence interval:
 0.0008583686 0.9414807659
sample estimates:
odds ratio 
0.05873784 

    \end{verbatim}

    
    \emph{Toetsprocedure}

\begin{enumerate}
\def\labelenumi{\arabic{enumi}.}
\tightlist
\item
  \(H_0\) : Er is geen verschil in de griep krijgen tussen docenten
  met/zonder griepprik en \(H_A\) : Er is wel een verschil.
\item
  Toetsingsgrootheid: \(\underline{k} \sim \textrm{hypergeometrisch}\).
\item
  Overschrijdingskans: \(p = 0.0217\).
\item
  Beslissing: \(p \leq \alpha\) (\(\alpha=0.05\)) dus \(H_0\) verwerpen.
\item
  Conclusie: De docenten met de griepprik krijgen significant minder
  griep.
\end{enumerate}

    \hypertarget{yates-correctie}{%
\section{Yates correctie}\label{yates-correctie}}

    Indien er wordt voldaan aan de volgende drie voorwaarden:

\begin{enumerate}
\def\labelenumi{\arabic{enumi}.}
\tightlist
\item
  \(20 < n < 40\).
\item
  Voor alle \(E_{ij}\)'s geldt dat \(E_{ij} > 1\).
\item
  \(80\%\) van alle \(E_{ij}\)'s is groter dan \(5\).
\end{enumerate}

dan wordt de Yates correctie toegepast. Dit geeft een nieuwe
toetsingsgrootheid, namelijk:

\[ \sum\limits_{i=1}^r \sum\limits_{j=1}^c \dfrac{\left(\left| O_{ij} - E_{ij}\right| - 0.5\right)^2 }{E_{ij}}. \]

Doormiddel van de Yates correctie, wordt de toets convervatiever.
Hierdoor wordt de nulhypthese minder snel verworpen en is de toets iets
veiliger.

    \hypertarget{t-verdeling}{%
\section{T-verdeling}\label{t-verdeling}}

    De t-verdeling kent een geschatte standaarddeviatie, terwijl de normale
verdeling een exacte bekende standaarddeviatie heeft.

\[ \underline{t} = \dfrac{\overline{\underline x} - \mu}{\frac{\underline{s}}{\sqrt{n}}} \]

waarbij \(\underline s\) een schatting is, en bepaald door:

\[ \underline{s} = \sqrt{  \dfrac{\sum\left(X_i - \underline{\overline X}\right)^2}{n-1}  } \]

Bij het opstellen van een betrouwbaarheidsinterval voor \(\mu\) voor een
continue variabele hebben we de volgende mogelijkheden:

\begin{enumerate}
\def\labelenumi{\arabic{enumi}.}
\tightlist
\item
  Normale verdeling + gegeven \(\sigma\rightarrow\) de normale verdeling
  gebruiken.
\item
  Normale verdeling + onbekende \(\sigma \rightarrow\) de t-verdeling
  gebruiken als \(n\leq 30\) en anders de normale verdeling.
\item
  Willekeurige verdeling + onbekende \(\sigma \rightarrow\) de
  t-verdeling gebruiken.
\end{enumerate}

    \textbf{Verdelingsfunctie plot}

    \begin{Verbatim}[commandchars=\\\{\}]
{\color{incolor}In [{\color{incolor}9}]:} curve\PY{p}{(}dt\PY{p}{(}x\PY{p}{,}\PY{l+m}{2}\PY{p}{,}\PY{l+m}{0.02}\PY{p}{)}\PY{p}{,} col\PY{o}{=}\PY{l+s}{\PYZsq{}}\PY{l+s}{red\PYZsq{}}\PY{p}{,} from\PY{o}{=}\PY{l+m}{\PYZhy{}5}\PY{p}{,} to\PY{o}{=}\PY{l+m}{5}\PY{p}{)}
\end{Verbatim}

    \begin{center}
    \adjustimage{max size={0.9\linewidth}{0.9\paperheight}}{output_42_0.png}
    \end{center}
    { \hspace*{\fill} \\}
    
    De t-verdeling lijkt bijna op de normale verdeling. Het verschil is dat
de t-verdeling een dikkere staart heeft.

    \hypertarget{betrouwbaarheidsinterval-voor-mu-bij-onbekende-sigma}{%
\section{\texorpdfstring{Betrouwbaarheidsinterval voor \(\mu\) bij
onbekende
\(\sigma\)}{Betrouwbaarheidsinterval voor \textbackslash{}mu bij onbekende \textbackslash{}sigma}}\label{betrouwbaarheidsinterval-voor-mu-bij-onbekende-sigma}}

    \[ \bar{x} - t \cdot \dfrac{s}{\sqrt n} < \mu < \bar{x} + t \cdot \dfrac{s}{\sqrt n} \]

    \textbf{Voorbeeld}

    \begin{Verbatim}[commandchars=\\\{\}]
{\color{incolor}In [{\color{incolor}10}]:} x \PY{o}{=} \PY{l+m}{16}
         s \PY{o}{=} \PY{l+m}{1.34}
         n \PY{o}{=} \PY{l+m}{10}
         alpha \PY{o}{=} \PY{l+m}{0.05}
         t \PY{o}{=} \PY{k+kp}{abs}\PY{p}{(}qt\PY{p}{(}alpha\PY{o}{/}\PY{l+m}{2}\PY{p}{,} n\PY{l+m}{\PYZhy{}1}\PY{p}{)}\PY{p}{)}
         interval \PY{o}{=} \PY{k+kt}{c}\PY{p}{(}x \PY{o}{\PYZhy{}} t \PY{o}{*} s \PY{o}{/} \PY{k+kp}{sqrt}\PY{p}{(}n\PY{p}{)}\PY{p}{,} x \PY{o}{+} t \PY{o}{*} s \PY{o}{/} \PY{k+kp}{sqrt}\PY{p}{(}n\PY{p}{)}\PY{p}{)}
         interval
\end{Verbatim}

    \begin{enumerate*}
\item 15.0414217459993
\item 16.9585782540007
\end{enumerate*}


    
    \hypertarget{t-toets-voor-uxe9uxe9n-gemiddelde}{%
\section{t-Toets voor één
gemiddelde}\label{t-toets-voor-uxe9uxe9n-gemiddelde}}

    Bij veel toepassingen in de praktijk is de standaarddeviatie niet
gegeven en dus moet deze worden geschat. Zoals we hebben gezien leidt
het schatten tot het gebruik van de t-verdeling met \(\nu = n-1\)
vrijheidsgraden.

\textbf{Toetsingsgrootheid}

Hiervoor berekenen we de \(t\)-waarde met:

\[t^* = \dfrac{\bar{x} - \mu}{\frac{s}{\sqrt{n}}}\]

    \textbf{Voorbeeld}

    Bij een visverwerkende industrie wordt het afvalwater voortdurend
gecontroleerd op verontreiniging. Afhankelijk van de hoeveelheid
geloosde vervuilingseenheden (VE's) moet een mileuheffing worden
afgedragen aan de overheid. Het bedrijf stelt dat gemiddeld hoogstens
200 VE's per dag worden geloosd. Toets of de uitspraak van het bedrijf
klopt.

    \begin{Verbatim}[commandchars=\\\{\}]
{\color{incolor}In [{\color{incolor}11}]:} dag \PY{o}{=} \PY{l+m}{1}\PY{o}{:}\PY{l+m}{10}
         VE \PY{o}{=} \PY{k+kt}{c}\PY{p}{(}\PY{l+m}{190}\PY{p}{,} \PY{l+m}{250}\PY{p}{,} \PY{l+m}{320}\PY{p}{,} \PY{l+m}{410}\PY{p}{,} \PY{l+m}{310}\PY{p}{,} \PY{l+m}{280}\PY{p}{,} \PY{l+m}{230}\PY{p}{,} \PY{l+m}{370}\PY{p}{,} \PY{l+m}{350}\PY{p}{,} \PY{l+m}{290}\PY{p}{)}
         n \PY{o}{=} \PY{k+kp}{length}\PY{p}{(}VE\PY{p}{)}
         df \PY{o}{=} \PY{k+kp}{t}\PY{p}{(}\PY{k+kt}{data.frame}\PY{p}{(}dag\PY{p}{,} VE\PY{p}{)}\PY{p}{)}
         df
\end{Verbatim}

    \begin{tabular}{r|llllllllll}
	dag &   1 &   2 &   3 &   4 &   5 &   6 &   7 &   8 &   9 &  10\\
	VE & 190 & 250 & 320 & 410 & 310 & 280 & 230 & 370 & 350 & 290\\
\end{tabular}


    
    \begin{Verbatim}[commandchars=\\\{\}]
{\color{incolor}In [{\color{incolor}12}]:} \PY{k+kp}{mean}\PY{p}{(}VE\PY{p}{)}
\end{Verbatim}

    300

    
    \begin{Verbatim}[commandchars=\\\{\}]
{\color{incolor}In [{\color{incolor}13}]:} t \PY{o}{=} \PY{p}{(}\PY{k+kp}{mean}\PY{p}{(}VE\PY{p}{)} \PY{o}{\PYZhy{}} \PY{l+m}{200}\PY{p}{)} \PY{o}{/} \PY{p}{(}sd\PY{p}{(}VE\PY{p}{)} \PY{o}{/} \PY{k+kp}{sqrt}\PY{p}{(}n\PY{p}{)}\PY{p}{)}
         \PY{k+kp}{t}
\end{Verbatim}

    4.74341649025257

    
    \begin{Verbatim}[commandchars=\\\{\}]
{\color{incolor}In [{\color{incolor}14}]:} qt\PY{p}{(}\PY{l+m}{0.975}\PY{p}{,} \PY{l+m}{9}\PY{p}{)}
\end{Verbatim}

    2.2621571627982

    
    Het is duidelijk dat \(t\) in het kritieke gebied
\(Z = \left[2.2621; \rightarrow\right>\) ligt.

    \begin{Verbatim}[commandchars=\\\{\}]
{\color{incolor}In [{\color{incolor}15}]:} \PY{l+m}{1} \PY{o}{\PYZhy{}} pt\PY{p}{(}\PY{l+m}{4.7434}\PY{p}{,} \PY{l+m}{9}\PY{p}{)}
\end{Verbatim}

    0.000526947811761658

    
    \emph{Toetsprocedure}

\begin{enumerate}
\def\labelenumi{\arabic{enumi}.}
\tightlist
\item
  \(H_0\) : \(\mu \leq 200\) en \(H_A\) : \(\mu > 200\).
\item
  Toetsingsgrootheid: \(t^* = 4.7434 \sim t[9]\).
\item
  Overschrijdingskans:
  \(p = P(t[9] \geq 4.7434) = 1 - P(t[9] \leq 4.7434) = 1 - \textrm{pt}(4.7434, 9) = 0.0005269\).
\item
  Beslissing: \(p \leq \alpha\) dus \(H_0\) verwerpen.
\item
  Conclusie: De uitspraak van het bedrijf wordt niet juist bevonden.
\end{enumerate}

    \hypertarget{verschiltoets-voor-mu-met-bekende-sigma}{%
\section{\texorpdfstring{Verschiltoets voor \(\mu\) met bekende
\(\sigma\)}{Verschiltoets voor \textbackslash{}mu met bekende \textbackslash{}sigma}}\label{verschiltoets-voor-mu-met-bekende-sigma}}

    In de praktijk komt het nogal eens voor dat er steekproeven worden
genomen uit twee populaties, waarbij de vraag naar voren komt of een
kansvariabele voor beide populaties dezelfde waarde voor de
verwachtingswaarde \(\mu\) heeft. In de eerste situatie kijken we naar
twee normaal verdeelde kansvariabelen \(\underline{x}\) en
\(\underline{y}\) waarvoor de varianties \(\sigma^2_x\) en
\(\sigma^2_y\) gegeven zijn.

\textbf{Hypothese}

Voor de verschiltoets werken we vaak met de hypothese dat
\(H_0 : \mu_x = \mu_y\) en \(H_A : \mu_x \not= \mu_y\).

\textbf{Toetsingsgrootheid}

Om de toets uit te voeren definiëren we de verschilvariabele
\(\nu = \underline{\overline x} - \underline{\overline y}\). De
bijbehorende toetsingsgrootheid is:

\[ \underline \nu \sim N\left(\mu_\nu = 0; \sigma_\nu = \sqrt{\dfrac{\sigma^2_x}{n} + \dfrac{\sigma^2_y}{m}} \right) \]

\textbf{Betrouwbaarheidsinterval}

Een betrouwbaarheidsinterval voor \(\mu_x - \mu_y\) kan worden bepaald
met:

\[ \left( \underline{\overline x} - \underline{\overline y} \right) - z_{1-\frac{1}{2}\alpha} \cdot \sqrt{\dfrac{\sigma^2_x}{n} + \dfrac{\sigma^2_y}{m}} < \mu_x - \mu_y < \left( \underline{\overline x} - \underline{\overline y} \right) + z_{1-\frac{1}{2}\alpha} \cdot \sqrt{\dfrac{\sigma^2_x}{n} + \dfrac{\sigma^2_y}{m}}  \]

    \textbf{Voorbeeld}

Gegeven is dat \(n_x = 10\), \(\bar{X} = 50\), \(n_y = 5\), en
\(\bar{Y} = 55\). Toets of beide dezelfde verwachtingswaarde kunnen
hebben.

    \begin{Verbatim}[commandchars=\\\{\}]
{\color{incolor}In [{\color{incolor}16}]:} n \PY{o}{=} \PY{l+m}{10}\PY{p}{;} m \PY{o}{=} \PY{l+m}{5}
         xbar \PY{o}{=} \PY{l+m}{50}\PY{p}{;} ybar \PY{o}{=} \PY{l+m}{55}
         xvar \PY{o}{=} \PY{l+m}{100}\PY{p}{;} yvar \PY{o}{=} \PY{l+m}{30}
         sigma \PY{o}{=} \PY{k+kp}{sqrt}\PY{p}{(}xvar \PY{o}{/} n \PY{o}{+} yvar \PY{o}{/} m\PY{p}{)}
         v \PY{o}{=} xbar \PY{o}{\PYZhy{}} ybar
         pnorm\PY{p}{(}v\PY{p}{,} \PY{l+m}{0}\PY{p}{,} sigma\PY{p}{)}
\end{Verbatim}

    0.105649773666855

    
    \emph{Toetsprocedure}

\begin{enumerate}
\def\labelenumi{\arabic{enumi}.}
\tightlist
\item
  \(H_0\) : \(\mu_x = \mu_y\) en \(H_A\) : \(\mu_x \not= \mu_y\).
\item
  Toetsingsgrootheid:
  \(\nu = \underline{\overline x} - \underline{\overline y} = -5 \sim N(\mu_\nu = 0, \sigma_\nu = 4)\).
\item
  Overschrijdingskans:
  \(p = P(\underline{v} \leq -5) = \textrm{normal}_{cdf}(-10^{99}, -5, 0, 4) = 0.1056\).
\item
  Beslissing: \(p \leq \alpha\) (\(\alpha=0.05\)) dus \(H_0\) wordt niet
  verworpen.
\item
  Conclusie: De steekproefgemiddelden zijn mogelijk gelijk aan elkaar.
\end{enumerate}

    \hypertarget{verschiltoets-voor-mu-met-een-onbekende-sigma}{%
\section{\texorpdfstring{Verschiltoets voor \(\mu\) met een onbekende
\(\sigma\)}{Verschiltoets voor \textbackslash{}mu met een onbekende \textbackslash{}sigma}}\label{verschiltoets-voor-mu-met-een-onbekende-sigma}}

    Als de variantie van de kansvariabele niet is gegeven, kunnen we niet
meteen de toets uitvoeren, maar moeten we eerst een schatting maken van
\(\sigma^2\).

\textbf{Variant A (gelijke \(\sigma\))}

Veronderstel dat we aan mogen nemen dat de onbekende varianties van de
variabelen aan elkaar gelijk zijn. (In de praktijk moeten we dit eerst
toetsen met de F-verdeling.) We bepalen voor beide variabelen de
variantie en deze kunnen dan worden gecombineerd tot één schatter, die
als volgt wordt gedefineerd:

\[s_p^2 = \dfrac{(n-1)s_x^2 + (m-1)s_y^2}{(n-1)+(m-1)} \quad \textrm{(pooled variance principe)}\]

waarbij de bijbehorende \(t\)-verdeling \(df=n+m-2\) vrijheidsgraden
heeft.

    \textbf{Variant B (ongelijke \(\sigma\))}

Als de variantieschattingen onderling sterk verschillen, mag de pooled
variance methode niet worden toegepast. In dit geval gebruiken we de
schatting:

\[ s_\nu = \sqrt{\dfrac{s^2_x}{n} + \dfrac{s^2_y}{m}} \]

waarbij de \(t\)-verdeling \(df=\min(m,n)-1\) vrijheidsgraden heeft.
Deze formule is een veilige methode om het aantal vrijheidsgraden te
bepalen. SPSS gebruikt een nauwkeurigere methode om het aantal
vrijheidsgraden te bepalen.

    \textbf{Betrouwbaarheidsinterval}

Een betrouwbaarheidsinterval voor \(\mu_x - \mu_y\) kan worden bepaald
met:

\[ \nu - ts_\nu < \mu_x - \mu_y < \nu + ts_\nu. \]

We hoeven hier niet te delen door \(\sqrt{n}\) omdat dit effect al zit
verwerkt in \(s_\nu\).

    \hypertarget{gepaarde-t-toets}{%
\section{Gepaarde t-toets}\label{gepaarde-t-toets}}

    Bij toepassing van de \(t\)-toets als verschiltoets wordt er wél
aandacht geschonken aan de grootte van de waargenomen verschillen.
Gegeven zijn \(n\) waargenomen getallenparen \((x_i, y_i)\). Deze paren
worden met \(v_i = y_i - x_i\) omgezet in \(n\) waargenomen verschillen.

\textbf{Toetsingsgrootheid}

De toetsingsgrootheid voor de gepaarde t-toets is het gemiddelde
verschil tussen de paren:

\[\underline{\overline v} : \textrm{gemiddelde verschil}\]

vervolgens bepalen we de \(t\)-waarde:

\[t^* = \dfrac{\bar{v} - \mu}{\frac{s}{\sqrt{n}}} \sim t[n-1].\]

    \textbf{Voorbeeld}

Een farmaceutische onderneming neemt een vermageringspil te hebben
ontdekt waarbij de proefpersonen niet van eetgewoonte hoeven te
veranderen. De resultaten zien we in de tabel:

    \begin{Verbatim}[commandchars=\\\{\}]
{\color{incolor}In [{\color{incolor}17}]:} gewicht.voor \PY{o}{=} \PY{k+kt}{c}\PY{p}{(}\PY{l+m}{80}\PY{p}{,} \PY{l+m}{86}\PY{p}{,} \PY{l+m}{66}\PY{p}{,} \PY{l+m}{75}\PY{p}{,} \PY{l+m}{90}\PY{p}{,} \PY{l+m}{68}\PY{p}{,} \PY{l+m}{78}\PY{p}{,} \PY{l+m}{70}\PY{p}{)}
         gewicht.na \PY{o}{=} \PY{k+kt}{c}\PY{p}{(}\PY{l+m}{75}\PY{p}{,} \PY{l+m}{78}\PY{p}{,} \PY{l+m}{68}\PY{p}{,} \PY{l+m}{71}\PY{p}{,} \PY{l+m}{84}\PY{p}{,} \PY{l+m}{65}\PY{p}{,} \PY{l+m}{73}\PY{p}{,} \PY{l+m}{67}\PY{p}{)}
         gewicht.verschil \PY{o}{=} gewicht.na \PY{o}{\PYZhy{}} gewicht.voor
         df \PY{o}{=} \PY{k+kp}{t}\PY{p}{(}\PY{k+kt}{data.frame}\PY{p}{(}gewicht.voor\PY{p}{,} gewicht.na\PY{p}{,} gewicht.verschil\PY{p}{)}\PY{p}{)}
         df
\end{Verbatim}

    \begin{tabular}{r|llllllll}
	gewicht.voor & 80 & 86 & 66 & 75 & 90 & 68 & 78 & 70\\
	gewicht.na & 75 & 78 & 68 & 71 & 84 & 65 & 73 & 67\\
	gewicht.verschil & -5 & -8 &  2 & -4 & -6 & -3 & -5 & -3\\
\end{tabular}


    
    \begin{Verbatim}[commandchars=\\\{\}]
{\color{incolor}In [{\color{incolor}18}]:} vbar \PY{o}{=} \PY{k+kp}{mean}\PY{p}{(}gewicht.verschil\PY{p}{)}
         t \PY{o}{=} vbar \PY{o}{/} \PY{p}{(}sd\PY{p}{(}gewicht.verschil\PY{p}{)} \PY{o}{/} \PY{k+kp}{sqrt}\PY{p}{(}\PY{k+kp}{length}\PY{p}{(}gewicht.verschil\PY{p}{)}\PY{p}{)}\PY{p}{)}
         vbar
         \PY{k+kp}{t}
         pt\PY{p}{(}\PY{k+kp}{t}\PY{p}{,} \PY{k+kp}{length}\PY{p}{(}gewicht.verschil\PY{p}{)} \PY{o}{\PYZhy{}} \PY{l+m}{1}\PY{p}{)}
\end{Verbatim}

    -4

    
    -3.86436713231718

    
    0.00308825163998739

    
    \emph{Toetsprocedure}

\begin{enumerate}
\def\labelenumi{\arabic{enumi}.}
\tightlist
\item
  \(H_0\) : \(\mu_v = 0\) en \(H_A\) : \(\mu_v \not= 0\).
\item
  Toetsingsgrootheid: \(\overline{\underline v} = -4\) met
  \(t^*=-3.8644 \sim t[7]\).
\item
  Overschrijdingskans:
  \(p = P(t[7] \leq -3.8644) = \textrm{pt}(-3.8644, 7) = 0.00308\).
\item
  Beslissing: \(p \leq \frac{\alpha}{2}\) (\(\alpha=0.05\)) dus \(H_0\)
  wordt verworpen.
\item
  Conclusie: het middel heeft een overtuigend effect.
\end{enumerate}

    \hypertarget{f-verdeling}{%
\section{F-verdeling}\label{f-verdeling}}

    De F-verdeling is een kansverdeling die betrekking heeft op twee groepen
waarnemingsuitkomsten. Als beide series waarnemingen willekeurig uit
dezelfde kansverdeling zijn getrokken, dan mag je verwachten dat de
grootheden \(s_1^2\) en \(s_2^2\) niet al te veel van elkaar
verschillen.

Bij de F-verdeling houden we ons bezig met de verhouding:

\[ F_{[\nu_1, \nu_2]} = \dfrac{s_1^2}{s_2^2} \]

waarbij \(\nu_1\) het aantal vrijheidsgraden van de teller is en
\(\nu_2\) het aantal van noemer.

    \textbf{Verdelingsfunctie plot}

    \begin{Verbatim}[commandchars=\\\{\}]
{\color{incolor}In [{\color{incolor}19}]:} curve\PY{p}{(}df\PY{p}{(}x\PY{p}{,}\PY{l+m}{8}\PY{p}{,}\PY{l+m}{6}\PY{p}{)}\PY{p}{,} col\PY{o}{=}\PY{l+s}{\PYZsq{}}\PY{l+s}{red\PYZsq{}}\PY{p}{,} from\PY{o}{=}\PY{l+m}{0}\PY{p}{,} to\PY{o}{=}\PY{l+m}{5}\PY{p}{)}
\end{Verbatim}

    \begin{center}
    \adjustimage{max size={0.9\linewidth}{0.9\paperheight}}{output_77_0.png}
    \end{center}
    { \hspace*{\fill} \\}
    
    \hypertarget{f-toets}{%
\section{F-toets}\label{f-toets}}

    Met de F-toets kunnen we nagaan of de varianties van de twee populaties
aan elkaar gelijk zouden kunnen zijn.

\textbf{Hypothese}

Als hypothese stellen we dat \(H_0 : \sigma_A^2 = \sigma_B^2\) en
\(H_A : \sigma_A^2 \not= \sigma_B^2\).

\textbf{Toetsingsgrootheid}

Als toetsingsgrootheid gebruiken we:

\[ \dfrac{s_1^2}{s_2^2} \sim F_{[\nu_1, \nu_2]}\]

    \textbf{Voorbeeld}

Stel \(s_A=6.2\) en \(s_B=2.7\), toets of de twee populaties waaruit de
steekproef is getrokken een verschillende waarde van variantie heeft.

    \begin{Verbatim}[commandchars=\\\{\}]
{\color{incolor}In [{\color{incolor}33}]:} s.a \PY{o}{=} \PY{l+m}{6.2}\PY{p}{;} s.b \PY{o}{=} \PY{l+m}{2.7}
         n.a \PY{o}{=} \PY{l+m}{10}\PY{p}{;}  n.b \PY{o}{=} \PY{l+m}{12}
         \PY{n+nb+bp}{F} \PY{o}{=} s.a\PY{o}{\PYZca{}}\PY{l+m}{2} \PY{o}{/} s.b\PY{o}{\PYZca{}}\PY{l+m}{2}
         \PY{n+nb+bp}{F}
         \PY{l+m}{1} \PY{o}{\PYZhy{}} pf\PY{p}{(}\PY{n+nb+bp}{F}\PY{p}{,} df1\PY{o}{=}n.a\PY{l+m}{\PYZhy{}1}\PY{p}{,} df2\PY{o}{=}n.b\PY{l+m}{\PYZhy{}1}\PY{p}{)}
\end{Verbatim}

    5.27297668038409

    
    0.00606775480766442

    
    \emph{Toetsprocedure}

\begin{enumerate}
\def\labelenumi{\arabic{enumi}.}
\tightlist
\item
  \(H_0\) : \(\sigma_A^2 = \sigma_B^2\) en \(H_A\) :
  \(\sigma_A^2 \not= \sigma_B^2\).
\item
  Toetsingsgrootheid:
  \(F = \dfrac{s_A^2}{s_B^2} = 5.2730 \sim F[9,11]\).
\item
  Overschrijdingskans:
  \(p = P(F[9,11] \geq 5.2730) = 1 - \textrm{pf}(5.2730, 9, 11) = 0.006068\).
\item
  Beslissing: \(p \leq \frac{\alpha}{2}\) (\(\alpha=0.05\)) dus \(H_0\)
  wordt verworpen.
\item
  Conclusie: De varianties van beide populaties kunnen aan elkaar gelijk
  zijn.
\end{enumerate}

    \hypertarget{harmonisch-gemiddelde}{%
\section{Harmonisch gemiddelde}\label{harmonisch-gemiddelde}}

    Het harmonisch gemiddelde is een speciaal gemiddelde, van toepassing bij
het berekenen van gemiddelden van verhoudingsgetallen
{[}\href{https://nl.wikipedia.org/wiki/Harmonisch_gemiddelde}{Wikipedia}{]}.

\[\bar{n}_h = \dfrac{c}{\frac{1}{n_1} + \frac{1}{n_2} + \ldots + \frac{1}{n_c}} = c \left( \sum\limits_{j=1}^c\dfrac{1}{n_j}\right)^{-1}\]

    \begin{Verbatim}[commandchars=\\\{\}]
{\color{incolor}In [{\color{incolor}20}]:} X \PY{o}{=} \PY{k+kt}{c}\PY{p}{(}\PY{l+m}{2}\PY{p}{,}\PY{l+m}{6}\PY{p}{,}\PY{l+m}{2}\PY{p}{,}\PY{l+m}{5}\PY{p}{,}\PY{l+m}{3}\PY{p}{,}\PY{l+m}{6}\PY{p}{,}\PY{l+m}{3}\PY{p}{)}
         \PY{k+kp}{mean}\PY{p}{(}X\PY{p}{)}               \PY{c+c1}{\PYZsh{} arithmetic mean}
         \PY{k+kp}{length}\PY{p}{(}X\PY{p}{)} \PY{o}{/} \PY{k+kp}{sum}\PY{p}{(}\PY{l+m}{1}\PY{o}{/}X\PY{p}{)}  \PY{c+c1}{\PYZsh{} harmonic mean}
\end{Verbatim}

    3.85714285714286

    
    3.18181818181818

    
    \hypertarget{pooled-variance-principe}{%
\section{Pooled-variance principe}\label{pooled-variance-principe}}

    Het is mogelijk om varianties samen te voegen met het pooled-variance
principe. De formule hiervoor is als volgt:

    \[ s_2 = \dfrac{(n_1 - 1)s_1^2 + \ldots + (n_k - 1)s_k^2}{(n_1 - 1) + \ldots + (n_k - 1)} \]

    waarbij \(k\) het aantal varianties zijn.

    \hypertarget{anova-uxe9uxe9n-factor-model}{%
\section{ANOVA (één-factor model)}\label{anova-uxe9uxe9n-factor-model}}

    \textbf{Hypothese}

De hypothese die we toetsen is \(H_0 : \mu_1 = \mu_2 = \mu_3\), en \$
H\_A \$ : de populatiegemiddelden zijn niet gelijk aan elkaar.

    \textbf{Toetsingsgrootheid}

Om de ANOVA analyse te doen berekenen we de volgende grootheden:

\textbf{Sum of Squares Total}:

\[SST = \sum\limits_{j=1}^c \sum\limits_{i=1}^{n_j}\left( X_{ij} - \bar{X}_{\cdot \cdot}\right)^2; MST = \dfrac{SST}{n-1}\]

\textbf{Sum of Squares Error}:

\[SSE = \sum\limits_{j=1}^c \sum\limits_{i=1}^{n_j}\left( X_{ij} - \bar{X}_{\cdot j}\right)^2; MSE = \dfrac{SSE}{n-c}\]

\textbf{Sum of Squares Groups}:

\[SSG = \sum\limits_{j=1}^c n_j \left( \bar{X}_{\cdot j} - \bar{X}_{\cdot \cdot}\right)^2; MSG = \dfrac{SSG}{c-1}\]

waarbij \(n\) het totaal aantal waarnemingen is en \(c\) het aantal
groepen. Hiervoor geldt er dat \(SST = SSG + SSE\).

\textbf{ANOVA tabel}

Met de berekende waarden kunnen we de volgende tabel opstellen:

\begin{longtable}[]{@{}lllll@{}}
\toprule
ANOVA & vrijheidsgraden (\(df\)) & Sum of Squares & Mean of Squares &
F-ratio\tabularnewline
\midrule
\endhead
Groepen (G) & \(c-1\) & \(SSG\) & \(MSG\) &
\(\frac{MSG}{MSE}\)\tabularnewline
Binnen (E) & \(n-c\) & \(SSE\) & \(MSE\) &\tabularnewline
Totaal (T) & \(n-1\) & \(SST\) & &\tabularnewline
\bottomrule
\end{longtable}

    \textbf{Voorbeeld}

    \begin{Verbatim}[commandchars=\\\{\}]
{\color{incolor}In [{\color{incolor}21}]:} economie \PY{o}{=} \PY{k+kt}{c}\PY{p}{(}\PY{l+m}{35}\PY{p}{,} \PY{l+m}{10}\PY{p}{,} \PY{l+m}{47}\PY{p}{,} \PY{l+m}{23}\PY{p}{,} \PY{l+m}{35}\PY{p}{,} \PY{l+m}{59}\PY{p}{,} \PY{l+m}{32}\PY{p}{,} \PY{l+m}{44}\PY{p}{,} \PY{l+m}{18}\PY{p}{,} \PY{l+m}{27}\PY{p}{)}
         techniek \PY{o}{=} \PY{k+kt}{c}\PY{p}{(}\PY{l+m}{52}\PY{p}{,} \PY{l+m}{38}\PY{p}{,} \PY{l+m}{45}\PY{p}{,} \PY{l+m}{37}\PY{p}{,} \PY{l+m}{60}\PY{p}{,} \PY{l+m}{48}\PY{p}{,} \PY{l+m}{54}\PY{p}{,} \PY{l+m}{66}\PY{p}{)}
         gezondheidszorg \PY{o}{=} \PY{k+kt}{c}\PY{p}{(}\PY{l+m}{18}\PY{p}{,} \PY{l+m}{24}\PY{p}{,} \PY{l+m}{10}\PY{p}{,} \PY{l+m}{48}\PY{p}{,} \PY{l+m}{20}\PY{p}{,} \PY{l+m}{12}\PY{p}{,} \PY{l+m}{36}\PY{p}{,} \PY{l+m}{24}\PY{p}{)}
         n1 \PY{o}{=} \PY{k+kp}{length}\PY{p}{(}economie\PY{p}{)}\PY{p}{;} n2 \PY{o}{=} \PY{k+kp}{length}\PY{p}{(}techniek\PY{p}{)}\PY{p}{;} n3 \PY{o}{=} \PY{k+kp}{length}\PY{p}{(}gezondheidszorg\PY{p}{)}
         Y \PY{o}{=} \PY{k+kt}{c}\PY{p}{(}economie\PY{p}{,} techniek\PY{p}{,} gezondheidszorg\PY{p}{)}
\end{Verbatim}

    \begin{Verbatim}[commandchars=\\\{\}]
{\color{incolor}In [{\color{incolor}22}]:} SST \PY{o}{=} \PY{k+kp}{sum}\PY{p}{(}\PY{p}{(}economie \PY{o}{\PYZhy{}} \PY{k+kp}{mean}\PY{p}{(}Y\PY{p}{)}\PY{p}{)}\PY{o}{\PYZca{}}\PY{l+m}{2}\PY{p}{)} \PY{o}{+} 
               \PY{k+kp}{sum}\PY{p}{(}\PY{p}{(}techniek \PY{o}{\PYZhy{}} \PY{k+kp}{mean}\PY{p}{(}Y\PY{p}{)}\PY{p}{)}\PY{o}{\PYZca{}}\PY{l+m}{2}\PY{p}{)} \PY{o}{+} 
               \PY{k+kp}{sum}\PY{p}{(}\PY{p}{(}gezondheidszorg \PY{o}{\PYZhy{}} \PY{k+kp}{mean}\PY{p}{(}Y\PY{p}{)}\PY{p}{)}\PY{o}{\PYZca{}}\PY{l+m}{2}\PY{p}{)}
         
         SSE \PY{o}{=} \PY{k+kp}{sum}\PY{p}{(}\PY{p}{(}economie \PY{o}{\PYZhy{}} \PY{k+kp}{mean}\PY{p}{(}economie\PY{p}{)}\PY{p}{)}\PY{o}{\PYZca{}}\PY{l+m}{2}\PY{p}{)} \PY{o}{+} 
               \PY{k+kp}{sum}\PY{p}{(}\PY{p}{(}techniek \PY{o}{\PYZhy{}} \PY{k+kp}{mean}\PY{p}{(}techniek\PY{p}{)}\PY{p}{)}\PY{o}{\PYZca{}}\PY{l+m}{2}\PY{p}{)} \PY{o}{+} 
               \PY{k+kp}{sum}\PY{p}{(}\PY{p}{(}gezondheidszorg \PY{o}{\PYZhy{}} \PY{k+kp}{mean}\PY{p}{(}gezondheidszorg\PY{p}{)}\PY{p}{)}\PY{o}{\PYZca{}}\PY{l+m}{2}\PY{p}{)}
                   
         SSG \PY{o}{=} n1 \PY{o}{*} \PY{p}{(}\PY{k+kp}{mean}\PY{p}{(}economie\PY{p}{)} \PY{o}{\PYZhy{}} \PY{k+kp}{mean}\PY{p}{(}Y\PY{p}{)}\PY{p}{)}\PY{o}{\PYZca{}}\PY{l+m}{2} \PY{o}{+} 
               n2 \PY{o}{*} \PY{p}{(}\PY{k+kp}{mean}\PY{p}{(}techniek\PY{p}{)} \PY{o}{\PYZhy{}} \PY{k+kp}{mean}\PY{p}{(}Y\PY{p}{)}\PY{p}{)}\PY{o}{\PYZca{}}\PY{l+m}{2} \PY{o}{+} 
               n3 \PY{o}{*} \PY{p}{(}\PY{k+kp}{mean}\PY{p}{(}gezondheidszorg\PY{p}{)} \PY{o}{\PYZhy{}} \PY{k+kp}{mean}\PY{p}{(}Y\PY{p}{)}\PY{p}{)}\PY{o}{\PYZca{}}\PY{l+m}{2}
         
         MST \PY{o}{=} SST \PY{o}{/} \PY{p}{(}\PY{k+kp}{length}\PY{p}{(}Y\PY{p}{)} \PY{o}{\PYZhy{}} \PY{l+m}{1}\PY{p}{)}
         MSE \PY{o}{=} SSE \PY{o}{/} \PY{p}{(}\PY{k+kp}{length}\PY{p}{(}Y\PY{p}{)} \PY{o}{\PYZhy{}} \PY{l+m}{3}\PY{p}{)}
         MSG \PY{o}{=} SSG \PY{o}{/} \PY{l+m}{2}
         \PY{n+nb+bp}{F} \PY{o}{=} MSG \PY{o}{/} MSE
         p \PY{o}{=} df\PY{p}{(}\PY{n+nb+bp}{F}\PY{p}{,} df1\PY{o}{=}\PY{l+m}{2}\PY{p}{,} df2\PY{o}{=}\PY{l+m}{23}\PY{p}{)} \PY{c+c1}{\PYZsh{} met df1 = c\PYZhy{}1 en df2 = n\PYZhy{}c}
         \PY{k+kp}{print}\PY{p}{(}\PY{k+kp}{paste}\PY{p}{(}\PY{l+s}{\PYZsq{}}\PY{l+s}{De f\PYZhy{}waarde is\PYZsq{}}\PY{p}{,} \PY{n+nb+bp}{F}\PY{p}{,} \PY{l+s}{\PYZsq{}}\PY{l+s}{met een p\PYZhy{}waarde van\PYZsq{}}\PY{p}{,} p\PY{p}{)}\PY{p}{)}
\end{Verbatim}

    \begin{Verbatim}[commandchars=\\\{\}]
[1] "De f-waarde is 8.65886826768073 met een p-waarde van 0.000897214262039324"

    \end{Verbatim}

    Het kritieke gebied kunnen we bepalen met:

    \begin{Verbatim}[commandchars=\\\{\}]
{\color{incolor}In [{\color{incolor}23}]:} qf\PY{p}{(}\PY{l+m}{.95}\PY{p}{,} df1\PY{o}{=}\PY{l+m}{2}\PY{p}{,} df2\PY{o}{=}\PY{l+m}{23}\PY{p}{)}
\end{Verbatim}

    3.42213220786118

    
    In dit geval is het kritieke gebied:
\(Z = \left[3.42; \rightarrow\right>\). Hier wordt \(H_0\) duidelijk
verworpen.

    \emph{Toetsprocedure}

\begin{enumerate}
\def\labelenumi{\arabic{enumi}.}
\tightlist
\item
  \(H_0\) : \(\mu_1 = \mu_2 = \mu_3\) en \(H_A\) : de
  populatiegemiddelden zijn niet gelijk aan elkaar.
\item
  Toetsingsgrootheid:
  \(F = \dfrac{MSG}{MSE} = 8.6589 \sim F_{[\nu_1, \nu_2]}\) met
  \(\nu_1=2\) en \(\nu_2=23\).
\item
  Overschrijdingskans: \(p = P(F_{[2,23]}\geq 8.6589) = 0.0008972\).
\item
  Beslissing: \(p \leq \alpha\) (\(\alpha=0.05\)) dus \(H_0\) verwerpen.
\item
  Conclusie: De drie populaties studenten hebben niet alle drie
  hetzelfde gemiddelde aantal uren internetgebruik.
\end{enumerate}

    \hypertarget{ongelijke-steekproefgroottes}{%
\section{Ongelijke
steekproefgroottes}\label{ongelijke-steekproefgroottes}}

    Bij ongelijke steekproefgroottes is \(E(\bar{X}_{\cdot \cdot})\) geen
zuivere schatter voor \(\mu\). De oplossing hiervoor is als volgt:

\[SSG = \bar{n}_h \sum\limits_{j=1}^c\left( \bar{X}_{\cdot j} - \bar{G}_{\cdot \cdot}\right)^2 \]

waarbij \(\bar{n}_h\) het harmonisch gemiddelde is en

\[ \bar{G}_{\cdot \cdot} = \dfrac{\sum\limits_{j=1}^c\left( \bar{X}_j \right)}{c}.\]

In het geval van ongelijke steekproefgroottes geldt dat
\(SST \not = SSG + SSE\).

    \hypertarget{levenes-toets}{%
\section{Levene's toets}\label{levenes-toets}}

    Met Levene's toets kunnen we toetsen of populatievarianties mogelijk
gelijk zijn aan elkaar. Dit is een vereiste voor de ANOVA analyse.
Hierbij stellen we als \(H_0\) dat de populatievarianties gelijk zijn
aan elkaar en willen we dit niet verwerpen.

\textbf{Toetsingsgrootheid}

\[ F_L = \dfrac{\left( \sum\limits_{j=1}^c n_j \left( \bar{Z}_{\cdot j} - \bar{Z}_{\cdot \cdot}\right)^2\right) / (c-1)}{\left( \sum\limits_{j=1}^c \sum\limits_{i=1}^{n_j} \left( \bar{Z}_{ij} - \bar{Z}_{\cdot j} \right)^2\right) / (n-c)} = \dfrac{MSG}{MSE} \sim F_{(\nu_1, \nu_2)} \]

waarbij \(Z_{ij} = \left|\ X_{ij} - \bar{X}_{\cdot j}\ \right|\).

\textbf{Beslissing}

Als \(p > \alpha\) dan wordt \(H_0\) niet verworpen. De
populatievarianties kunnen aan elkaar gelijk zijn.

    \hypertarget{bonferoni-methode}{%
\section{Bonferoni methode}\label{bonferoni-methode}}

    Een post-hoc analyse kan worden toegepast zodra \(H_0\) wordt verworpen
bij de variantieanalyse. Een methode hiervoor is de Bonferoni methode.
De vraag die willen beantwoorden is welke \(\mu_j\)'s niet samenvallen.

Dit kunnen we doen door het paarsgewijs vergelijken van de
steekproefgemiddelden m.b.v. de t-toets. Als er drie groepen zijn dan
moeten we \(k=\begin{pmatrix}c\\2\end{pmatrix}\) paren die we moeten
toetsen. Voor een significantie van \(\alpha=0.05\) krijgen we bij
herhalen van de t-toets:

    \begin{Verbatim}[commandchars=\\\{\}]
{\color{incolor}In [{\color{incolor}24}]:} \PY{l+m}{1}\PY{o}{\PYZhy{}}\PY{p}{(}\PY{l+m}{1}\PY{l+m}{\PYZhy{}0.05}\PY{p}{)}\PY{o}{*}\PY{o}{*}\PY{l+m}{6}
\end{Verbatim}

    0.264908109375

    
    Er is dus een kans van \(26\%\) dat er een fout van eerste soort
optreedt. Om dit op te lossen toetsen we met \(\frac{\alpha}{2\cdot k}\)
als we tweezijdige toetsen en \(\frac{\alpha}{k}\) bij een enkelzijdige
toets. Volgens BK is deze methode te rigoreus, zij komen met de Tukey's
HSD methode.

    \textbf{Hypothese}

    De hypothese die we toetsen is \(H_0 : \mu_i = \mu_j\) en
\(H_A : \mu_i \not= \mu_j\).

    \textbf{Toetsingsgrootheid}

    De te gebruiken toetsingsgrootheid voor de Bonferoni methode is als
volgt:

\[ \dfrac{\left( \overline{\underline X}_{\cdot i} - \overline{\underline X}_{\cdot j}\right) - \left( \mu_i - \mu_j \right)}{\sqrt{MSE} \cdot \sqrt{\frac{1}{n_i} + \frac{1}{n_j}}} \sim t[\nu] \]

waarbij \(\nu = n-c\).

    \textbf{Beslisregel}

In dit geval is het makkelijker om te toetsen met het kritieke gebied
i.p.v. de overschijdingskans. Het kritieke gebied kunnen we bepalen met:

\[ Z = \left<\leftarrow; P(\underline{t}[\nu] \leq L) = \frac{\alpha}{2k} \right] \cup \left[P(\underline{t}[\nu] \geq R) = \frac{\alpha}{2k}; \rightarrow \right> \]

    \textbf{Conclusie}

Zodra \(H_0\) wordt verworpen is er een significant verschil aangetoond.
Dus \(\mu_i < \mu_j\), of andersom. Dit wordt bepaalt door te kijken
welk gemiddelde het laagst/hoogst is.

    \hypertarget{tukeys-hsd-methode}{%
\section{Tukey's HSD methode}\label{tukeys-hsd-methode}}

    De Tuckey's HSD (honestly significant difference) gebruikt een waarde
die is af te lezen vanuit een tabel. Dit is tabel 3, de verdeling van
\(V\) voor Tukey's HSD procedure.

\emph{Toetsprocedure}

\begin{enumerate}
\def\labelenumi{\arabic{enumi}.}
\tightlist
\item
  \(H_0\) : \(\mu_i = \mu_j\) en \(H_A\) : \(\mu_i \not= \mu_j\).
\item
  Toetsingsgrootheid: \(V\).
\item
  Beslisregel: \(Z = \left[r;\rightarrow\right>\) met
  \(P(v\geq r\ |\ c; \nu_t)\). Hier geldt dat \(a=c\) het aantal groepen
  is. Deze waarde wordt afgelezen uit tabel 3.
\item
  Beslissing: Als \(x \in Z\) dan wordt \(H_0\) verworpen, als
  \(x \not\in Z\) dan wordt \(H_0\) niet verworpen.
\end{enumerate}

    \hypertarget{maat-voor-effectgrootte}{%
\section{Maat voor effectgrootte}\label{maat-voor-effectgrootte}}

    Een maat voor effectgrootte \(\eta^2\), eta kwadraat, geeft het
percentage van de variantie v/d scores op de afhankelijke variabele dat
door de onafhankelijke variabele wordt verklaard.

\[ \eta^2 = \dfrac{SSG}{SST} : \textrm{effectgrootte} \]

Dit geldt voor een gelijke steekproefgrootte.

    \hypertarget{rangtekentoets-wilcoxon-signed-rank-test}{%
\section{Rangtekentoets (Wilcoxon Signed Rank
test)}\label{rangtekentoets-wilcoxon-signed-rank-test}}

    Het doel van de analyse is een verschil aantonen tussen twee
\emph{afhankelijke/gekoppelde} groepen op een ordinale variabele.

\textbf{Stappenplan}

\begin{enumerate}
\def\labelenumi{\arabic{enumi}.}
\tightlist
\item
  Maak een tabel met de experimentele groep \(E\) en de controle groep
  \(C\).
\item
  Bepaal het verschil \(D_i = E_i - C_i\).
\item
  Bepaal hat absolute verschil \(|\ D_i\ |\).
\item
  Bepaal de rangnummers \(R_i\) van het absolute verschil.
\item
  Bepaal het teken voor \(R_i\), dus \(\textrm{sign}(D_i) \cdot R_i\).
\item
  Bepaal \(W_+\) wat de som is van alle \emph{positieve} rangnummers.
\item
  Zodra \(W_+\) en \(n\) bepaald zijn kunnen we de \(p\)-waarde opzoeken
  in tabel 10.
\end{enumerate}

Alle verschilscores waarvoor geldt dat \(D_i = 0\) moeten worden
verwijderd. Indien dit het geval is dat wordt ook de \(n\) aangepast. In
het geval van gelijke rangnummers, m.a.w. knopen, nemen we het
gemiddelde van de rangnummers.

    \textbf{Toetsingsgrootheid}

De toetsingsgrootheid is \(W_+\). Dit is de som is van alle positieve
rangnummers. Voor \(W_+\) geldt dat:

\begin{itemize}
\tightlist
\item
  \(\textrm{E}(W_+) = \dfrac{n(n-1)}{4}\)
\item
  \(\textrm{Var}(W_+) = \dfrac{n(n+1)(2n+1)}{24}\)
\end{itemize}

    \textbf{Centrale limietstelling}

    Het blijkt dat we \(W_+\) normaal kunnen benaderen als
\(n \rightarrow \infty\). De vuistregel hiervoor is \(n > 15\). In dit
geval geldt dat:

\[ W_+ \sim N\left(\mu=\dfrac{n(n-1)}{4}; \sigma=\sqrt{\dfrac{n(n+1)(2n+1)}{24}}\right)\]

Bij benadering gebruiken we:

\[ P(W_+ \leq k) \approx P\left(X_{nor} \leq k + \frac{1}{2}\ |\ \mu, \sigma\right) \]

    \textbf{Voorbeeld}

Voor patienten met depressieve klachten is er een experiment opgezet.
Hiervoor is \(X\) de score op de vragenlijst (\(0\leq X \leq 100\)). Hoe
hoger de score, hoe erger de klachten. Er zijn 10 gematchte paren
gemaakt (personen met dezelfde score op de vragenlijst).

De experimentele groep \(E\) krijgt een behandeling met hardlopen en de
controle groep \(C\) krijgt een behandeling zonder hardlopen. Toets of
het hardlopen effect heeft.

    \begin{Verbatim}[commandchars=\\\{\}]
{\color{incolor}In [{\color{incolor}25}]:} E \PY{o}{=} \PY{k+kt}{c}\PY{p}{(}\PY{l+m}{70}\PY{p}{,} \PY{l+m}{60}\PY{p}{,} \PY{l+m}{55}\PY{p}{,} \PY{l+m}{80}\PY{p}{,} \PY{l+m}{40}\PY{p}{,} \PY{l+m}{68}\PY{p}{,} \PY{l+m}{54}\PY{p}{,} \PY{l+m}{71}\PY{p}{,} \PY{l+m}{70}\PY{p}{,} \PY{l+m}{40}\PY{p}{)}
         C \PY{o}{=} \PY{k+kt}{c}\PY{p}{(}\PY{l+m}{75}\PY{p}{,} \PY{l+m}{62}\PY{p}{,} \PY{l+m}{65}\PY{p}{,} \PY{l+m}{77}\PY{p}{,} \PY{l+m}{49}\PY{p}{,} \PY{l+m}{60}\PY{p}{,} \PY{l+m}{63}\PY{p}{,} \PY{l+m}{78}\PY{p}{,} \PY{l+m}{59}\PY{p}{,} \PY{l+m}{55}\PY{p}{)}
         D \PY{o}{=} E\PY{o}{\PYZhy{}}C
         D.abs \PY{o}{=} \PY{k+kp}{abs}\PY{p}{(}D\PY{p}{)}
         R \PY{o}{=} \PY{k+kt}{c}\PY{p}{(}\PY{l+m}{4}\PY{p}{,} \PY{l+m}{1}\PY{p}{,} \PY{l+m}{8}\PY{p}{,} \PY{l+m}{2}\PY{p}{,} \PY{l+m}{7}\PY{p}{,} \PY{l+m}{6}\PY{p}{,} \PY{l+m}{3}\PY{p}{,} \PY{l+m}{5}\PY{p}{,} \PY{l+m}{9}\PY{p}{,} \PY{l+m}{10}\PY{p}{)}
         R.sign \PY{o}{=} \PY{k+kp}{sign}\PY{p}{(}D\PY{p}{)} \PY{o}{*} R
         df \PY{o}{=} \PY{k+kp}{t}\PY{p}{(}\PY{k+kt}{data.frame}\PY{p}{(}E\PY{p}{,} C\PY{p}{,} D\PY{p}{,} D.abs\PY{p}{,} R\PY{p}{,} R.sign\PY{p}{)}\PY{p}{)}
         df
\end{Verbatim}

    \begin{tabular}{r|llllllllll}
	E & 70  & 60  &  55 & 80  & 40  & 68  & 54  & 71  & 70  &  40\\
	C & 75  & 62  &  65 & 77  & 49  & 60  & 63  & 78  & 59  &  55\\
	D & -5  & -2  & -10 &  3  & -9  &  8  & -9  & -7  & 11  & -15\\
	D.abs &  5  &  2  &  10 &  3  &  9  &  8  &  9  &  7  & 11  &  15\\
	R &  4  &  1  &   8 &  2  &  7  &  6  &  3  &  5  &  9  &  10\\
	R.sign & -4  & -1  &  -8 &  2  & -7  &  6  & -3  & -5  &  9  & -10\\
\end{tabular}


    
    \begin{Verbatim}[commandchars=\\\{\}]
{\color{incolor}In [{\color{incolor}26}]:} W\PYZus{}pos \PY{o}{=} \PY{k+kp}{sum}\PY{p}{(}R.sign\PY{p}{[}R.sign \PY{o}{\PYZgt{}} \PY{l+m}{0}\PY{p}{]}\PY{p}{)}
         W\PYZus{}pos
\end{Verbatim}

    17

    
    \emph{Toetsprocedure}

\begin{enumerate}
\def\labelenumi{\arabic{enumi}.}
\tightlist
\item
  \(H_0\) : \(\eta_E = \eta_C\) en \(H_A\) : \(\eta_E < \eta_C\).
\item
  Toetsingsgrootheid: \(W_+ = 17\) met \(n=10\).
\item
  Overschrijdingskans:
  \(p = P(W_+\leq 17) \stackrel{\textrm{Tabel 10}}{=} 0.161\).
\item
  Beslissing: \(p \leq \alpha\) (\(0.10\)) dus \(H_0\) niet verwerpen.
\item
  Conclusie: Er is geen significant verschil in toetsscores.
\end{enumerate}

    \hypertarget{wilcoxon-som-toets-mann-whitney-toets}{%
\section{Wilcoxon som toets (Mann-Whitney
toets)}\label{wilcoxon-som-toets-mann-whitney-toets}}

    Het doel van de analyse is een verschil aantonen tussen twee
\emph{onafhankelijke} groepen op een ordinale variabele.

\begin{itemize}
\tightlist
\item
  Groepsvariabele : middel (groep 1: geneesmiddel, groep 2: placebo)
\item
  Testvariabele : mate van aggresief gedrag (testscore 1-100)
\end{itemize}

\textbf{Stappenplan}

\begin{enumerate}
\def\labelenumi{\arabic{enumi}.}
\tightlist
\item
  Sorteer beide testscores, zowel \(X\) als \(Y\).
\item
  Bepaal de rangnummers voor \(X\) en \(Y\).
\item
  Bepaal de som voor elke groep, dus \(S_x\) en \(S_y\) waarbij de \(S\)
  staat voor som.
\item
  Zodra \(m\), \(n\) en \(S_x\) bepaald zijn kunnen we de \(p\)-waarde
  opzoeken in tabel 11.
\end{enumerate}

Let er op dat geldt dat \(m \leq n\), indien dit niet het geval is
kunnen \(X\) en \(Y\) worden omgewisselt.

\textbf{Toetsingsgrootheid}

De toetsingsgrootheid is \(S_x\) waarbij \(S\) staat voor de som. Dit is
de som van alle rangnummers van \(X\). Voor \(S_x\) geldt er dat:

\begin{itemize}
\tightlist
\item
  \(\textrm E(S_x) = \frac{m(m+n+1)}{2}\)
\item
  \(\textrm{Var}(S_x) = \frac{nm(n+m+1)}{12}\)
\end{itemize}

\textbf{Centrale limietstelling}

Het blijkt dat we \(S_x\) normaal kunnen benaderen als
\(n, m \rightarrow \infty\). Hiervoor geldt dat:

\[S_x \sim N\left(\mu=\frac{m(m+n+1)}{2}; \sigma=\sqrt{\frac{nm(n+m+1)}{12}}\right)\]

De vuistregel om te mogen benaderen eist dat \(m, n > 10\). Hiervoor
gebruiken we:

\[ P(S_x \leq k) \approx P\left(X_{nor} \leq k + \frac{1}{2}\ |\ \mu, \sigma\right) \]

\textbf{Voorbeeld}

    Voor twee groepen \(A\) en \(B\) wordt er een vitaminepil gegeven. Groep
\(A\) is vertelt dat het een prestatieverhogend middel is en voor groep
\(B\) een slaapmiddel.

    \begin{Verbatim}[commandchars=\\\{\}]
{\color{incolor}In [{\color{incolor}27}]:} groep.a \PY{o}{=} \PY{k+kt}{c}\PY{p}{(}\PY{l+m}{36}\PY{p}{,} \PY{l+m}{41}\PY{p}{,} \PY{l+m}{44}\PY{p}{,} \PY{l+m}{45}\PY{p}{,} \PY{l+m}{52}\PY{p}{,} \PY{l+m}{53}\PY{p}{,} \PY{l+m}{54}\PY{p}{,} \PY{l+m}{57}\PY{p}{,} \PY{l+m}{58}\PY{p}{,} \PY{l+m}{77}\PY{p}{)}
         groep.b \PY{o}{=} \PY{k+kt}{c}\PY{p}{(}\PY{l+m}{26}\PY{p}{,} \PY{l+m}{46}\PY{p}{,} \PY{l+m}{55}\PY{p}{,} \PY{l+m}{59}\PY{p}{,} \PY{l+m}{64}\PY{p}{,} \PY{l+m}{65}\PY{p}{,} \PY{l+m}{67}\PY{p}{,} \PY{l+m}{79}\PY{p}{,} \PY{l+m}{81}\PY{p}{,} \PY{l+m}{83}\PY{p}{)}
         rank.a \PY{o}{=} \PY{k+kt}{c}\PY{p}{(}\PY{l+m}{2}\PY{p}{,} \PY{l+m}{3}\PY{p}{,} \PY{l+m}{4}\PY{p}{,} \PY{l+m}{5}\PY{p}{,} \PY{l+m}{7}\PY{p}{,} \PY{l+m}{8}\PY{p}{,} \PY{l+m}{9}\PY{p}{,} \PY{l+m}{11}\PY{p}{,} \PY{l+m}{12}\PY{p}{,} \PY{l+m}{17}\PY{p}{)}
         rank.b \PY{o}{=} \PY{k+kt}{c}\PY{p}{(}\PY{l+m}{1}\PY{p}{,} \PY{l+m}{6}\PY{p}{,} \PY{l+m}{10}\PY{p}{,} \PY{l+m}{13}\PY{p}{,} \PY{l+m}{14}\PY{p}{,} \PY{l+m}{15}\PY{p}{,} \PY{l+m}{16}\PY{p}{,} \PY{l+m}{18}\PY{p}{,} \PY{l+m}{19}\PY{p}{,} \PY{l+m}{20}\PY{p}{)}
         \PY{k+kp}{t}\PY{p}{(}\PY{k+kt}{data.frame}\PY{p}{(}groep.a\PY{p}{,} groep.b\PY{p}{,} rank.a\PY{p}{,} rank.b\PY{p}{)}\PY{p}{)}
\end{Verbatim}

    \begin{tabular}{r|llllllllll}
	groep.a & 36 & 41 & 44 & 45 & 52 & 53 & 54 & 57 & 58 & 77\\
	groep.b & 26 & 46 & 55 & 59 & 64 & 65 & 67 & 79 & 81 & 83\\
	rank.a &  2 &  3 &  4 &  5 &  7 &  8 &  9 & 11 & 12 & 17\\
	rank.b &  1 &  6 & 10 & 13 & 14 & 15 & 16 & 18 & 19 & 20\\
\end{tabular}


    
    \begin{Verbatim}[commandchars=\\\{\}]
{\color{incolor}In [{\color{incolor}28}]:} Sa \PY{o}{=} \PY{k+kp}{sum}\PY{p}{(}rank.a\PY{p}{)}
         Sa
\end{Verbatim}

    78

    
    \emph{Toetsprocedure}

\begin{enumerate}
\def\labelenumi{\arabic{enumi}.}
\tightlist
\item
  \(H_0\) : \(\eta_A = \eta_B\) en \(H_A\) : \(\eta_A < \eta_B\).
\item
  Toetsingsgrootheid: \(S_A = 78\) met \(m=n=10\).
\item
  Overschrijdingskans:
  \(p = P(S_A \leq 78) \stackrel{\textrm{Tabel 11}}{=} 0.022\).
\item
  Beslissing: \(p \leq \alpha\) dus \(H_0\) verwerpen.
\item
  Conclusie: De suggestie doet zijn werk.
\end{enumerate}

    \begin{Verbatim}[commandchars=\\\{\}]
{\color{incolor}In [{\color{incolor} }]:} 
\end{Verbatim}


    % Add a bibliography block to the postdoc
    
    
    
    \end{document}
